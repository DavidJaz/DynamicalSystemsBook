\documentclass[DynamicalBook]{subfiles}
\begin{document}
%


\setcounter{chapter}{6}%Just finished 5.

\chapter{Paradigms of Composition}\label{chapter.6} 

\section{Introduction}


Throughout this book so far, we seen dynamical systems modeled by state spaces exposing
variables and updating according to external parameters. This sort of dynamical
system is \emph{lens-based} --- systems are themselves lenses, and they compose by
lens composition. We might describe them as \emph{parameter-setting systems},
since we compose these systems by setting the parameters of some according to
the exposed state variables of others.

But this is not the only way to present a system. Consider, for example, a
circuit diagram:
\begin{equation}\label{eqn:circuit.diagram.2}
    \begin{circuitikz}
      \draw (0,0)
      to[V,v=$V$] (0,2) % The voltage source
      to (2,2)
      to[R=$R$] (2,0) % The resistor
      to[L=$L$] (0,0);
    \end{circuitikz}
    \end{equation}
In \cref{ex.RL_circuit_trajectory} we saw how we could use Kirchoff's laws to
interpret this circuit as a differential system
  \[
\lens{\update{RL}}{\id} : \lens{\rr}{\rr} \fromto \lens{\rr^2 \times \rr^{\ast}}{\rr}
  \]
  where
  \[
\update{RL}\left( I, \begin{bmatrix} V \\ R \\ L \end{bmatrix} \right) \coloneqq
\frac{V - RI}{L}.
\]
But why not consider the circuit \emph{itself} as a system? This is a different
way of thinking about systems: the circuit is a diagram, it doesn't have a set
of states, exposed variables of state, and it doesn't update according to
parameters. Nevertheless, we can compose circuits together to get more complex
circuits. For example, we can think of the circuit (\ref{eqn:circuit.diagram.2})
as the composite of two smaller circuits:
\begin{equation}\label{eqn:circut.diagram.compose}
  \begin{circuitikz}[baseline=(center)]
      \draw (0,0)
      to[V,v=$V$] (0,2) % The voltage source
      to (2,2);
      \draw (2, 0) to[L=$L$] (0,0);
      \node[circle,fill=black, inner sep = 1pt] (term1) at (2,0) {};
      \node[circle,fill=black, inner sep = 1pt] (term2) at (2,2) {};

      \node[circle,fill=black, inner sep = 1pt] (term3) at (3,0) {};
      \node[circle,fill=black, inner sep = 1pt] (term4) at (3,2) {};
      \draw (3, 2) to[R=$R$] (3,0);
      \node (center) at (1,1) {};
    \end{circuitikz}
\quad\mapsto\quad
  \begin{circuitikz}[baseline=(center)]
      \draw (0,0)
      to[V,v=$V$] (0,2) % The voltage source
      to (2,2)
      to[R=$R$] (2,0) % The resistor
      to[L=$L$] (0,0);
      
      \node (center) at (1,1) {};
      \node[circle,fill=black, inner sep = 1pt] (term1) at (2,0) {};
      \node[circle,fill=black, inner sep = 1pt] (term2) at (2,2) {};
    \end{circuitikz}
\end{equation}
We compose circuit diagrams by gluing their wires together --- just like we might
actually solder two physical circuits together. Another example of a system like
circuit diagrams is a population flow graph (as, for example, in
\cref{defn:population.flow.graph.sir}). A simple population flow graph consists
of a graph whose vertices are \emph{places} and whose edges are paths between
places, each labeled by its flow rate.
\begin{equation}\label{eqn:pop.flow.graph.1}
\begin{tikzpicture}
	\node[draw] {
  \begin{tikzcd}[column sep=small]
    \Sys{Boston} \ar[rr, bend left=5, "r_1"]\ar[rr, bend right=5, leftarrow, "r_2"'] \ar[dr, "r_3"'] &  & \ar[dl, "r_4"]\Sys{NYC}\\
& \Sys{Tallahassee} &
  \end{tikzcd}
  };
\end{tikzpicture}
\end{equation}
We can compose population flow graphs by gluing places together. Fore example,
we can think of population flow graph (\ref{eqn:pop.flow.graph.1}) as the
composite of two smaller population flow graphs:
\begin{equation}\label{eqn:pop.flow.graph.comp}
\begin{tikzpicture}[baseline=(center)]
	\node[draw] (center) {
  \begin{tikzcd}[column sep=small]
    \Sys{Boston} \ar[rr, bend left=5, "r_1"]\ar[rr, bend right=5, leftarrow, "r_2"'] \ar[dr, "r_3"'] &  & \Sys{NYC}\\
& \Sys{Tallahassee} &
  \end{tikzcd}
  };
\end{tikzpicture}
\quad
\begin{tikzpicture}[baseline=(center)]
	\node[draw] (center) {
  \begin{tikzcd}[column sep=small]
    & \ar[dl, "r_4"]\Sys{NYC}\\
\Sys{Tallahassee} &
  \end{tikzcd}
  };
\end{tikzpicture}
\quad\mapsto\quad
\begin{tikzpicture}[baseline=(center)]
	\node[draw] (center) {
  \begin{tikzcd}[column sep=small]
    \Sys{Boston} \ar[rr, bend left=5, "r_1"]\ar[rr, bend right=5, leftarrow, "r_2"'] \ar[dr, "r_3"'] &  & \ar[dl, "r_4"]\Sys{NYC}\\
& \Sys{Tallahassee} &
  \end{tikzcd}
  };
\end{tikzpicture}
\end{equation}
We have added in the connection between New York and Tallahassee by gluing
together the places in these two population flow graphs. 


How can we describe this kind of
composition in general? Instead of exposing variables of state, systems like
circuit diagrams and population flow graphs expose
certain parts of themselves (the ends of wires, some of their places) to their environment. We can refer to these parts of
a circuit-diagram like system as its \emph{ports}. The ports form the
\emph{interface} of this sort of system.

For now, let's suppose that a system $\Sys{S}$
has a finite set of ports $\Ports{S}$, which acts as its interface. For example,
we can see the ports of the open circuit diagram on the left of
(\ref{eqn:circut.diagram.compose}):
\begin{equation}\label{eqn:circuit.diagram.ports}
  \begin{circuitikz}[baseline=(center)]
      \node (left) at (-1,0) {};
      \node (bottom) at (-1, -.5) {};
      \draw (0,0)
      to[V,v=$V$] (0,2) % The voltage source
      to (2,2);
      \draw (2, 0) to[L=$L$] (0,0);
      \node[circle,fill=red, inner sep = 1pt] (term1) at (2,0) {};
      \node[circle,fill=red, inner sep = 1pt] (term2) at (2,2) {};
      \node[draw, fit={(left) (bottom) (term2)}] (S) {};
      \node[above = .3 cm of S] (syslabel) {$\Sys{S}$};


      \node[circle,fill=red, inner sep = 1pt] (term3) at (4,0) {};
      \node[circle,fill=red, inner sep = 1pt] (term4) at (4,2) {};
      \draw [|-to, shorten <=.3cm, shorten >=.3cm] (term3) to[] (term1);
      \draw [|-to, shorten <=.cm, shorten >=.3cm] (term4) to[] (term2);
      \node[draw, fit=(term3)(term4)] (ports) {};
      \node[above = .3 cm of ports] (portslabel) {$\Ports{S}$};
    \end{circuitikz}
\end{equation}

We can see the set of ports as a trivial sort of circuit diagram --- one with no
interesting components of any kind --- which has been included into the circuit
diagram $\Sys{S}$. That is, the way we describe an interface is by a map
$\partial_{\Sys{S}} : L\Ports{S} \to \Sys{S}$ which picks out the ports in the
system $\Sys{S}$, and where $L$ is some operation that takes a finite set into a
particularly simple sort of system.

Suppose we want to describe the composition
of \cref{eqn:circut.diagram.compose}. This will compose system $\Sys{S}$ with
the system $\Sys{T}$:
\begin{equation}
  \begin{circuitikz}[baseline=(center)]
      \draw (2,2) to[R=$R$] (2, 0);
      \node[circle,fill=red, inner sep = 1pt] (term1) at (2,0) {};
      \node[circle,fill=red, inner sep = 1pt] (term2) at (2,2) {};
      \node[draw, fit={($(term1) - (.5,0)$) ($ (term2) + (.5,0)$)}] (S) {};
      \node[above = .3 cm of S] (syslabel) {$\Sys{T}$};


      \node[circle,fill=red, inner sep = 1pt] (term3) at (4,0) {};
      \node[circle,fill=red, inner sep = 1pt] (term4) at (4,2) {};
      \draw [|-to, shorten <=.3cm, shorten >=.3cm] (term3) to[] (term1);
      \draw [|-to, shorten <=.cm, shorten >=.3cm] (term4) to[] (term2);
      \node[draw, fit=(term3)(term4)] (ports) {};
      \node[above = .3 cm of ports] (portslabel) {$\Ports{T}$};
    \end{circuitikz}
\end{equation}
Just like with the parameter setting systems we've been composing the whole
book, we will compose these two systems first by considering them together as a
joint system, and then composing them according to a composition pattern. Here,
the composition pattern should tell us which ports get glued together, and then
which of the resulting things should be re-exposed as ports. For this, we will use
a \emph{cospan}:
\begin{equation}
  \begin{tikzpicture}
      \node[circle,fill=red, inner sep = 1pt] (term1) at (0,0) {};
      \node[circle,fill=red, inner sep = 1pt] (term2) at (0,1) {};
      \node[draw, fit=(term1)(term2)] (ports) {};
      \node[below = .3 cm of ports] (portslabel) {$\Ports{S}$};

      \node[circle,fill=red, inner sep = 1pt] (term3) at (2,0) {};
      \node[circle,fill=red, inner sep = 1pt] (term4) at (2,1) {};
      \node[draw, fit=(term3)(term4)] (ports) {};
      \node[below = .3 cm of ports] (portslabel) {$\Ports{T}$};

      \node[circle,fill=red, inner sep = 1pt] (mid1) at (4,2) {};
      \node[circle,fill=red, inner sep = 1pt] (mid2) at (4,3) {};
      \node[draw, fit={(mid1) (mid2)}] (mid) {};
      \node[below = .3 cm of mid] (midlabel) {$M$};

     \node (end1) at (6,0) {};
      \node (end2) at (6,1) {};
      \node[draw, fit={(end1)(end2)}] (outerports) {};
      \node[below = .3 cm of outerports] (portslabel) {$\Ports{S \circ_M T}$};

      
      \draw [|->, shorten <= .3 cm, shorten >= .5 cm] (term1) to[bend left = 14] (mid1);
      \draw [|->, shorten <= .3 cm, shorten >= .5 cm] (term3) to[bend left = 10] (mid1);
      \draw [|->, shorten <= .3 cm, shorten >= .5 cm] (term2) to[bend left = 10] (mid2);
      \draw [|->, shorten <= .3 cm, shorten >= .5 cm] (term4) to[bend left = 10] (mid2);
  \end{tikzpicture}
\end{equation}

Here, the composite system exposes no ports, so we leave its set of ports empty.
But with the map on the left, we show how we want to glue the ports of $\Sys{S}$
and $\Sys{T}$ together. To actually get the composite system, we actually glue
these ports along this plan. Gluing objects together in a category means taking
a \emph{pushout}:
\[
\begin{tikzcd}
	& {\Sys{S} \circ_M T} \\
	{\Sys{S} + \Sys{T}} && M \\
	& {\Ports{S} + \Ports{T}} && {\Ports{S \circ_M T}}
	\arrow["{\partial_{\Sys{S}} + \partial{\Sys{T}}}", from=3-2, to=2-1]
	\arrow[from=3-2, to=2-3]
	\arrow[from=3-4, to=2-3]
	\arrow[from=2-3, to=1-2]
	\arrow[from=2-1, to=1-2]
	\arrow["\ulcorner"{anchor=center, pos=0.125, rotate=-45}, draw=none, from=1-2, to=3-2]
	\arrow["{\partial_{\Sys{S} \circ_{M} \Sys{T}}}"', bend right = 30, dashed, from=3-4, to=1-2]
\end{tikzcd}
\]

The symbol $+$ here is denoting the coproduct, or disjoint union, which lets us
put our circuit diagrams side by side.

We can describe systems like circuit diagrams and population flow graphs which
are composed using pushouts in the above way \emph{port-plugging
  systems}. The idea with these systems is that they expose some ports, and we
compose them by plugging the ports of one system into the ports of another ---
gluing them together using pushouts. But parameter-setting (lens-based) and
port-plugging (pushout based) ways of thinking about systems still do not exhaust all the ways we can
think about systems. In fact, we've missed a crucially important perspective:
the \emph{behavioral approach} to systems theory.

The parameter setting (lens-based) and port-plugging (pushout-based) ways of thinking about systems are very useful for the \emph{design} of systems; we give
a minimal set of data ($\expose{}$ and $\update{}$ in the case of parameter
setting systems, and something like a circuit diagram or flow graph with exposed
ports for port-plugging systems) which in principle determines all behaviors, though it
might take some work to understand what behaviors are actually in the system
once we have set it up. But for the \emph{analysis} of dynamical systems, we
seek to prove properties about how systems behave. It helps if we
already know how a system behaves.

In the \emph{behavioral} approach to systems theory, pioneered by Jan Willems,
we take ``behavior'' as a primitive. In its most basic formulation, the behavioral
approach to systems theory considers a system $\Sys{S}$ to have a set
$\Set{B}(S)$ of ``behaviors''. The system also exposes some variables of these
behaviors in a function $\expose{S} : \Set{B}_{\Sys{S}} \to \Set{V}_{\Sys{S}}$
to a set
$\Set{V}_{\Sys{S}}$ of possible values for these exposed variables. For example
consider water flowing through the a pipe $\Sys{P}_1$. We could model the
behaviors of this system by the changing inflow and outflow over time. That is,
we could take
\[
\Set{B}_{\Sys{P}_1} = \{(i, o) \mid i, o: \rr \to \rr\}
\]
to be the set of pairs of functions representing the flows through time.




We can see \cref{thm.representable_discrete} as
showing us that we can apply the behavioral approach to dynamical systems in any
doctrine; the doubly indexed functoriality of the behaviors says that the
process of taking behaviors respects the composition of systems.









\end{document}