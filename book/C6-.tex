\documentclass[DynamicalBook]{subfiles}
\begin{document}
%


\setcounter{chapter}{5}%Just finished 5.


%------------ Chapter ------------%
\chapter{}\label{chapter.6}



\paragraph{Databases}

Discrete opfibrations will be important in our story, so let's spend a bit of time on them. It turns out that discrete opfibrations on a category have a close relationship with databases. Let's start with a definition.


\begin{definition}[Discrete opfibration]
A functor $F\colon\cat{C}\to\cat{D}$ is called a \emph{discrete opfibration} if, for every object $c\in\cat{C}$, object $d'\in\cat{D}$, and morphism $g\colon F(c)\to d'$ there exists a unique object $c'\in\cat{C}$ and morphism $f\colon c\to c'$ such that $F(c')=d'$ and $F(f)=g$.
\[
\begin{tikzcd}
  c\ar[r, dashed, "f"]\ar[d, |->, "F"']&
  c'\ar[d, |->, "F"]\\
  F(c)\ar[r, "g"']&
  d'
\end{tikzcd}
\]
\end{definition}


\begin{proposition}\label{prop.dopf_copresheaf}
Given a functor $\cat{C}\to\smset$, its category of elements is a discrete opfibration over $\cat{C}$, and this is functorial. Moreover this functor is an equivalence of categories.
\end{proposition}


\begin{example}[Database]
\end{example}

\begin{definition}[Database schema and instance]
Let $\cat{C}$ be a category. An \emph{attribute structure} on $\cat{C}$ is a pair $(V, \alpha)$ where
\begin{enumerate}
	\item $V$ is a set, elements of which are called \emph{attribute values} and
	\item $\alpha\colon V\to \Ob(\cat{C})$ is a function, called the \emph{attribute assignment}.
\end{enumerate}
If $(V,\alpha)$ is an attribute structure on $\cat{C}$, we call the triple $(\cat{C},V,\alpha)$ a \emph{database schema} and refer to $\cat{C}$ as the \emph{entity category} of the schema.

An \emph{instance} on $(\cat{C},V,\alpha)$ consists of a functor $I\colon\cat{C}\to\smset$ together with a function $v_c\colon I(c)\to\alpha\inv(c)$ for each $c\in\Ob(\cat{C})$.
\end{definition}

\begin{proposition}
Let $\cat{C}$ be a category and $\eta_\cat{C}\colon\cat{C}\to\Ob(\cat{C})$ the unit cofunctor. An attribute structure on $\cat{C}$ can be identified with a $(\Ob(\cat{C}),0)$-bimodule $(\Ob(\cat{C})\bimodfrom[V]0$. An instance on $(\cat{C},V,\alpha)$ can be identified with a square of the form
\[
\begin{tikzcd}[background color=theoremcolor]
	\cat{C}\ar[r, bimr-biml, "I"]\ar[d, "\cofun" marking, "\eta"']&
	0\ar[d, equal]\\
	\Ob(\cat{C})\ar[r, bimr-biml, "V"']&
	0\ar[ul, phantom, "\hphantom{\scriptstyle\alpha}\Downarrow\scriptstyle\alpha"]
\end{tikzcd}
\]
\end{proposition}



\begin{proposition}\label{prop.cart_com_dopf}
The category of polynomial comonoids and cartesian maps is equivalent to the category of categories and discrete opfibrations
\[
\Cat{Comon}(\poly)_{\text{cart}}\cong \smcat_{\text{dopf}}.
\]
\end{proposition}
\begin{proof}
**
\end{proof}

\begin{example}
Let $S$ be a set and $\com{S}\coloneqq(S\yon^S,\epsilon,\delta)$ the state comonoid. Then by \cref{prop.dopf_copresheaf,prop.cart_com_dopf}, a comonoid $\com{C}$ equipped with a cartesian map $f\colon\com{C}\to\com{S}$ is the category of elements for a functor $\cat{S}\to\smset$. But $\cat{S}$ is terminal in $\smcat$, so every functor $F\colon \cat{S}\to\smset$ is constant on some set $X\in\smset$, and hence there is an isomorphism of categories $\cat{C}\cong X\times \cat{S}$.
\end{example}


%-------- Section --------%
\section{Bimodules}

\begin{definition}
**
\end{definition}
\[
\begin{tikzpicture}
	\node (p1) {
  \begin{tikzpicture}[polybox, tos]
  	\node[poly, dom, "$m$" left] (m) {};
  	\node[poly, right= of m.south, yshift=-1ex, "\tiny$D$" below] (D) {};
  	\node[poly, above=of D, "\tiny$m$" above] (mm) {};
  	\node[poly, cod, right= of D.south, yshift=-1ex, "$D$" right] (DD) {};
  	\node[poly, cod, above=of DD, "$m$" right] (mmm) {};
  	\node[poly, cod, above=of mmm, "$C$" right] (C) {};
%
		\draw (m_pos) to[first] (D_pos);
		\draw (D_dir) to[climb] (mm_pos);
		\draw (mm_dir) to[last] (m_dir);
		\draw[double] (D_pos) to[first] (DD_pos);
		\draw[double] (DD_dir) to[last] (D_dir);
		\draw[double] (mm_pos) to[first] (mmm_pos);
		\draw (mmm_dir) to[climb] (C_pos);
		\draw (C_dir) to[last] (mm_dir);
	\end{tikzpicture}
	};
%
	\node (p2) [below=of p1] {
  \begin{tikzpicture}[polybox, tos]
  	\node[poly, dom, "$m$" left] (m') {};
  	\node[poly, right= of m'.south, yshift=-1ex, "\tiny$m$" below] (mm') {};
  	\node[poly, above=of mm', "\tiny$C$" above] (C') {};
  	\node[poly, cod, right= of mm'.south, yshift=-1ex, "$D$" right] (D') {};
  	\node[poly, cod, above=of D', "$m$" right] (mmm') {};
  	\node[poly, cod, above=of mmm, "$C$" right] (CC') {};
%
		\draw[double] (m'_pos) to[first] (mm'_pos);
		\draw (mm'_dir) to[climb] (C'_pos);
		\draw (C'_dir) to[last] (m'_dir);
		\draw (mm'_pos) to[first] (D'_pos);
		\draw (D'_dir) to[climb] (mmm'_pos);
		\draw (mmm'_dir) to[last] (mm'_dir);
		\draw[double] (C'_pos) to[first] (CC'_pos);
		\draw[double] (CC'_dir) to[last] (C'_dir);
	\end{tikzpicture}
	};	
	\node at ($(p1.south)!.5!(p2.north)$) {$=$};
\end{tikzpicture}
\]

\[
\begin{tikzpicture}[polybox, tos]
	\node[poly, dom, "$m$" left] (m) {};
	\node[poly, cod, right=of m, "$m$" right] (mm) {};
	\node[poly, cod, above=of mm, "$C$" right] (C) {};
	\node[poly, cod, below=of mm, "$D$" right] (D) {};
%
	\draw (m_pos) to[out=0, in=180] (D_pos);
	\draw (D_dir) to[climb] (mm_pos);
	\draw (mm_dir) to[climb] (C_pos);
	\draw (C_dir) to[last] (m_dir);
\end{tikzpicture}
\]

Recall from \cref{prop.basechange} that for any function $f\colon A\to B$, we have a base-change functor $f^*\colon B\poly\to A\poly$ and a cartesian morphism $f^*p\to p$ for any polynomial $p$ and isomorphism $p(\1)\cong B$.

\begin{proposition}\label{prop.right_modules_as_sums}
Let $\com{C}=(\ema{c},\epsilon,\delta)$ be a comonoid and suppose that $\rho\colon m\to m\tri\ema{c}$ is a right $\com{C}$-module. Then for any set $A$ and function $f\colon A\to m\tri\1$, the polynomial $f^*m$ has an induced right module structure $\rho_f$ fitting into the commutative square below:
\[
\begin{tikzcd}
  f^*m\ar[d]\ar[r, "\rho_f"]&
  f^*m\tri\ema{c}\ar[d]\\
  m\ar[r, "\rho"']&
  m\tri\ema{c}
\end{tikzcd}
\]
\end{proposition}
\begin{proof}
The pullback diagram to the left defines $f^*(m)$ and that to the right is its composition with $\ema{c}$
\[
\begin{tikzcd}
	f^*m\ar[r]\ar[d]&
	m\ar[d]\\
	A\ar[r, "f"']&
	m\tri\1\ar[ul, phantom, very near end, "\lrcorner"]
\end{tikzcd}
\hspace{.7in}
\begin{tikzcd}
	f^*m\tri\ema{c}\ar[r]\ar[d]&
	m\tri\ema{c}\ar[d]\\
	A\ar[r, "f"']&
	m\tri\1\ar[ul, phantom, very near end, "\lrcorner"]
\end{tikzcd}
\]
which is again a pullback by \cref{thm.connected_limits,exc.composing_with_constants}. Now the map $\rho\colon m\to m\tri\ema{c}$ induces a map $\rho_f\colon f^*(m)\to f^*(m)\tri c$; we claim it is a right module. It suffices to check that $\rho_f$ interacts properly with $\epsilon$ and $\delta$, which we leave to \cref{exc.right_modules_as_sums}.
\end{proof}

\begin{exercise}\label{exc.right_modules_as_sums}
Let $\ema{c},\epsilon,\delta)$, $\rho\colon m\to m\tri\ema{c}$, and $f\colon A\to m\tri\1$ be as in \cref{prop.right_modules_as_sums}. Complete the proof of that proposition as follows:
\begin{enumerate}
	\item Show that $\rho_f\then\epsilon=\id_m$
	\item Show that $\rho_f\then(f^*m\tri\delta)=\rho_f\then(\rho_f\tri\ema{c})$.
\qedhere
\end{enumerate}
\end{exercise}


\begin{theorem}\label{thm.tfae_c_sets}
For a comonoid $\com{C}=(\ema{c},\epsilon,\delta)$ (category $\cat{C}$), the following are equivalent:
\begin{enumerate}
	\item functors $\cat{C}\to\smset$
	\item discrete opfibrations over $\cat{C}$
	\item cartesian cofunctors to $\com{C}$
	\item linear left $\com{C}$-modules
	\item constant left $\com{C}$-modules
	\item $(\com{C},0)$-bimodules
	\item representable right $\com{C}$-modules
	\item $\com{C}$-coalgebras (sets with a coaction by $\com{C}$)
	\item dynamical systems with comonoid interface $\com{C}$
\end{enumerate}
\end{theorem}
\begin{proof}
\begin{description}
	\item[$1\to 2$:] Category of elements.
	\item[$2\to 1$:] Fibers.
\\	\item[$4\to 3$:] Given a linear left module, we can factor the underlying morphism $A\yon\to\ema{c}\tri A\yon$ as a vertical followed by a cartesian using \cref{prop.vert_cart_factorization}. The intermediate object has the structure of a category.
\end{description}
\end{proof}

Let $\cat{C}$ be a category. Under the above correspondence, the terminal functor $\cat{C}\to\smset$ corresponds to the identity discrete opfibration $\cat{C}\to\cat{C}$, the identity cofunctor $\com{C}\to\com{C}$, a certain left $\com{C}$ module with carrier $\com{C}(\1)\yon$ which we call the \emph{canonical left $\com{C}$-module}, a certain constant left $\com{C}$ module with carrier $\com{C}(\1)$ which we call the \emph{canonical $(\com{C},0)$-bimodule}, and a certain representable right $\com{C}$-module with carrier $\yon^{\com{C}(\1)}$ which we call the \emph{canonical right $\cat{C}$-module}.

\begin{exercise}
For any object $c\in \cat{C}$, consider the representable functor $\cat{C}(c,-)\colon\cat{C}\to\smset$. What does it correspond to as a
\begin{enumerate}
	\item discrete opfibration over $\cat{C}$?
	\item cartesian cofunctor to $\com{C}$?
	\item linear left $\com{C}$-module?
	\item constant left $\com{C}$-module?
	\item $(\com{C},0)$-bimodule?
	\item representable right $\com{C}$-module?
	\item dynamical system with comonoid interface $\com{C}$?
\qedhere
\end{enumerate}
\end{exercise}

\begin{proposition}[Niu]
The composite of a linear left $\cat{C}$-module and a representable right $\cat{C}$-module is the set of natural transformations between the corresponding copresheaves.
\end{proposition}

\begin{proposition}\label{prop.all_free_modules}
Let $\com{C}=(\ema{c},\epsilon,\delta)$ be a comonoid in $\poly$. For any set $G$, the polynomial $\yon^G\tri\ema{c}$ has a natural right $\com{C}$-module structure.
\end{proposition}
\begin{proof}
We use the map $(\yon^G\tri\delta)\colon(\yon^G\tri\ema{c})\to(\yon^G\tri\ema{c}\tri\ema{c})$. It satisfies the unitality and associativity laws because $\ema{c}$ does.
\end{proof}

We can think of elements of $G$ as ``generators''. Then if $i'\colon G\to\ema{c}\tri\1$ assigns to every generator an object of a category $\cat{C}$, then we should be able to find the free $\cat{C}$-set that $i'$ generates.

\begin{proposition}
Functions $i'\colon G\to\ema{c}\tri\1$ are in bijection with positions $i\in\yon^G\tri\ema{c}\tri\1$. Let $m\coloneqq i^*(\yon^G\tri\ema{c})$ and let $\rho_i$ be the induced right $\com{C}$-module structure from \cref{prop.right_modules_as_sums}. Then $\rho_i$ corresponds to the free $\cat{C}$-set generated by $i'$. 
\end{proposition}
\begin{proof}
The polynomial $m$ has the following form:
\[
m\cong\yon^{\sum_{g\in G}\ema{c}[i'(g)]}
\]
In particular $\rho_i$ is a representable right $\com{C}$-module, and we can identify it with a $\cat{C}$-set by \cref{thm.tfae_c_sets}. The elements of this $\cat{C}$-set are pairs $(g, f)$, where $g\in G$ is a generator and $f\colon i'(g)\to\cod(f)$ is a morphism in $\cat{C}$ emanating from $i'(g)$. It is easy to see that the module structure induced by \cref{prop.all_free_modules} is indeed the free one.
\end{proof}


\begin{proposition}
Let $\com{C}$ be a comonoid. The category of left $\com{C}$ modules is equivalent to the category of functors $\cat{C}\to\poly$.

Moreover, for any functor $F\colon\cat{C}\to\poly$, the limit polynomial $\lim_{c\in\cat{C}}F(c)$ is obtained by composing with the canonical right bimodule $\yon^{\Ob(\cat{C})}$
\[
\begin{tikzcd}
  \yon\ar[r,biml-bimr, "F"]\ar[rr, biml-bimr, bend right=20pt, "\lim F"']&
  \com{C}\ar[r,biml-bimr, "\yon^{\Ob(\cat{C})}"]&[5pt]
  \yon
\end{tikzcd}
\]
\end{proposition}

\begin{proposition}\label{prop.break_up_right_mods}
Let $\com{C}$ be a comonoid. For any set $I$ and right $\com{C}$-modules $(m_i)_{i\in I}$, the coproduct $m\coloneqq \sum_{i\in I}m_i$ has a natural right-module structure. Moreover, each representable summand in the carrier $m$ of a right $\com{C}$-module is itself a right-$\com{C}$ module and $m$ is their sum.
\end{proposition}
\begin{proof}
**
\end{proof}
%
%Let $\sum_{I\in\smset}\prod_{i\in I}\bimod{}{\com{C}}$ denote the category whose objects are pairs $(I,(m_i)_{i\in I})$ where $I$ is a set and $m_i$ is a right $\com{C}$-module for each $i\in I$. A morphism $(I,(m_i)_{i\in I})\to(J,(n_j)_{j\in J})$ is a function $f\colon I\to J$ and, for each $i\in I$ a morphism $m_i\to n_{f(i)}$ of right-$\com{C}$ modules.
%
%\begin{proposition}
%For any comonoid $\com{C}$ there is an adjunction
%\[
%\adj[40pt]{\bimod{}{\com{C}}}{\sum_{i\in I}m_i}{{(n(\1),n[-])}}{\sum_{I\in\smset}\prod_{i\in I}\bimod{}{\com{C}}}
%\]
%with functors labeled by where they send $n\in\bimod{}{\com{C}}$ and $(I,(m_i)_{i\in I})\in\sum_{I\in\smset}\prod_{i\in I}\bimod{}{\com{C}}$. 
%
%Moreover, the left adjoint is fully faithful.
%\end{proposition}

\begin{proposition}
If $m\in\poly$ is equipped with both a right $\com{C}$-module and a right $\com{D}$-module structure, we can naturally equip $m$ with a $(\com{C}\times\com{D})$-module structure.
\end{proposition}
\begin{proof}
It suffices by \cref{prop.break_up_right_mods} to assume that $m=\yon^M$ is representable. But a right $\com{C}$-module with carrier $\yon^M$ can be identified with a cofunctor $M\yon^M\to\com{C}$.

Thus if $\yon^M$ is both a right-$\com{C}$ module and a right-$\com{D}$ module, then we have comonoid morphisms $\com{C}\from M\yon^M\to\com{D}$. This induces a unique comonoid morphism $M\yon^M\to(\com{C}\times\com{D})$ to the product, and we identify it with a right-$(\com{C}\times\com{D})$ module on $\yon^M$.
\end{proof}



\section{Questions}

In this section, we lay out some questions that whose answers may or may not be known, but which were not known to us at the time of writing. They vary from concrete to open-ended, they are not organized in any particular way, and are in no sense complete. Still we hope they may be useful to some readers.

\begin{enumerate}
  \item What can you say about the internal logic for the topos $\cofree{p}\smset$ of dynamical systems with interface $p$, in terms of $p$?
  \item How does the logic of the topos $\cofree{p}$ help us talk about issues that might be useful in studying dynamical systems?
  \item Morphisms $p\to q$ in $\poly$ give rise to left adjoints $\cofree{p}\to\cofree{q}$ that preserve connected limits. These are not geometric morphisms in general; in some sense they are worse and in some sense they are better. They are worse in that they do not preserve the terminal object, but they are better in that they preserve every connected limit not just finite ones. How do these left adjoints translate statements from the internal language of $p$ to that of $q$?
  \item Consider the $\times$-monoids and $\otimes$-monoids in three categories: $\poly$, $\smcat^\sharp$, and $\bimod{}{}$. Find examples of these comonoids, and perhaps characterize them or create a theory of them.
  \item Is there a functor $\poly$ has pullbacks, so one can consider the bicategory of spans in $\poly$. Is there a functor from that to $\bimod{}{}$ that sends $p\mapsto\cofree{p}$?
  \item Databases are static things, whereas dynamical systems are dynamic; yet we see them both in terms of $\poly$. How do they interact? Can a dynamical system read from or write to a database in any sense?
  \item Can we do database aggregation in a nice dynamic way?
  \item In the theory of polynomial functors, sums of representable functors $\smset\to\smset$, what happens if we replace sets with homotopy types: how much goes through? Is anything improved?
  \item Are there any functors $\smset\to\smset$ that aren't polynomial, but which admit a comonoid structure with respect to composition $(\yon,\tri)$?
  \item Characterize the monads in poly? They're generalizations of one-object operads (which are the Cartesian ones), but how can we think about them?
  \item Both functors and cofunctors give left adjoint bimodules: for functors $F\colon\cat{D}\to\cat{C}$ we use the pullback $\Delta_F$ and for cofunctors $G\colon\cat{C}\cof\cat{D}$ we use the companion as in \cref{**}. Can we characterize left adjoint bimodules in general?
  \item What limits exist in $\smcat^\sharp$? Describe them combinatorially.
  \item Since the forgetful functor $U\colon\smcat^\sharp\to\poly$ is faithful, it reflects monomorphisms: if $f\colon\cat{C}\cof\cat{D}$ is a cofunctor whose underlying map on emanation polynomials is monic, then it is monic. Are all monomorphisms in $\cat^\sharp$ of this form?
  \item At first blush it appears that $\poly$ may be suitable as the semantics of a language for protocols. Develop such a language or showcase the limitations that make it impossible or inconvenient.
\end{enumerate}

\end{document}
