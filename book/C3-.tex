\documentclass[DynamicalBook]{subfiles}
\begin{document}
%


\setcounter{chapter}{2}%Just finished 2.


%------------ Chapter ------------%
\chapter{Behaviors of the whole from behaviors of the parts}\label{chapter.3}

%-------- Section --------%
\section{Introduction}

\jaz{WRITE-ME}

Let's take stock of where we've been so far in the past couple chapters.
\begin{itemize}
  \item In \cref{sec.deterministic_system}, we saw the definition of a
    \emph{deterministic system}.
  \item In \cref{sec.wiring_sytems_discrete}, we learned about \emph{lenses}. We saw how systems can be
    interpreted as special sorts of lenses, and how we can wire together systems
    using lens composition.
  \item In \cref{sec.behavior_discrete}, we learned about behaviors and \emph{charts}. We saw
    how to define behaviors of systems using the notion of chart. Finally, we
    saw how the steady states of wired together systems can be calculated from their
    component systems with matrix arithmetic.
\end{itemize}

The two sorts of composition we have seen so far --- lens composition and chart
composition --- mirror the two sorts of composition at play in systems theory:
\begin{itemize}
  \item We can compose \emph{systems} by wiring them together. This uses lens composition.
  \item We can compose \emph{behaviors} of systems like we compose functions.
    This uses chart composition.
\end{itemize}

In this section, we will see how these two sorts of composition interact.
Because there are two sorts of composition involved, we will use the notion of a
\emph{double category}. A double category is like a category, but there are two
sorts of map between the objects, and there is a notion of interaction between
these two sorts of map. We'll get to see that behaviors $\phi : \Sys{T} \to
\Sys{S}$ of systems
(\cref{def.behavior_discrete}) are squares
\[
  \begin{tikzcd}
    \lens{\State{T}}{\State{T}} \ar[r, shift left, "\lens{\phi}{\phi}"] \ar[r,
    shift right]
    \ar[d, shift right, "\lens{\update{T}}{\expose{T}}"'] \ar[d, shift left, leftarrow] &
    \lens{\State{S}}{\State{S}} \ar[d, shift left, leftarrow,
    "\lens{\update{S}}{\expose{S}}"] \ar[d, shift right]\\
    \lens{\In{T}}{\Out{T}} \ar[r, shift right, "\lens{f^{\sharp}}{f}"'] \ar[r,
    shift left] & \lens{\In{S}}{\Out{S}}
  \end{tikzcd}
\]
in the double category of \emph{arenas}, \cref{prop.behavior_as_square_of_arenas_discrete}.

Along the way, we'll see a new interpretation of the categories of lenses and
charts which explain why they seem so eerily familiar. We will see them as
instances of a very general construction that forms a category out of an
\emph{indexed category}: the \emph{Grothendieck construction}. We take this
detour into the abstract because when we move to different doctrines in later
chapters, the particular notion of lens and chart might change, but they will
always be built out of an indexed category in the way we will see in this section.

%---- Section ----%
\section{Steady states compose according to the laws of matrix arithmetic}\label{sec.steady_states_matrix_arithmetic}


We have seen how we can compose systems, and we have seen how systems behave. We
have seen a certain composition of behaviors, a form of transitivity that says
that if we have a $\Sys{T}$-shaped behavior in $\Sys{S}$ and a $\Sys{S}$-shaped
behavior in $\Sys{U}$, then we get a $\Sys{T}$-shaped behavior in $\Sys{U}$. But what's the relationship between composing systems and composing their behaviors?

This question will be a major theme of this book. In this section we will give a
taste by showing how steady states compose. Later, in \cref{sec.representables}, we will see a very abstract
theorem that generalizes what we do here for steady states to something that works for \emph{all behaviors}.
But in order for that abstract theorem to make sense, we should first see the concrete
case of steady states in detail.  

Recall that the chart of a steady state $s \in \State{S}$ is the pair
$\lens{i}{o}$ with $o = \expose{S}(s)$ and $\update{S}(s, i) = s$. The set of all
possible charts for steady states is therefore $\In{S} \times \Out{S}$, and for
every chart $\lens{i}{o}$ we have the set $\Set{Steady}_{Sys{S}}\lens{i}{o}$ of
steady states for this chart. 

We can see this function $\Set{Steady}_{\Sys{S}} : \In{S} \times \Out{S} \to \smset$ as a
\emph{matrix of sets} with $\Set{Steady}_{\Sys{S}}\lens{i}{o}$ in the row $i$
and column $o$. For example, consider system $\Sys{S_1}$ of
\cref{ex.wiring_transition_diagrams}:
\begin{equation}\label{eqn.wiring_transition_diagrams_steady_diag1}
\Sys{S_1} \coloneqq \begin{tikzpicture}[baseline=(Center)]
  \coordinate (Center) at (0,0);
	\node[draw] {
  \begin{tikzcd}[column sep=small]
    \LMOO{\const{s_{11}}}{\Blue} \ar[loop left, "\const{false}"] \ar[rr, bend left, "\const{true}"] \ar[dd, leftarrow, bend right, "\true"'] &  & \LMOO{\const{s_{12}}}{\Red} \ar[loop right, "\true"] \ar[dd, bend left, "\const{false}" ]\\
    & & \\
    \LMOO{\const{s_{13}}}{\Blue} \ar[loop left, "\false"] \ar[rr, leftarrow, bend left, "\false"] \ar[rr, leftarrow, bend right, "\true"'] & & \LMOO{\const{s_{14}}}{\Green}
  \end{tikzcd}
  };
\end{tikzpicture}
\end{equation}
This has output value set $\Set{Colors} = \{\Blue, \Red, \Green\}$ and input
parameter set $\Set{Bool} = \{\true, \false\}$. Here is its $(\Set{Colors}
\times \Set{Bool})$ steady state matrix:
\begin{equation}\label{eqn.steady_state_matrix1}
 \Set{Steady}_{\Sys{S_1}} =
  \kbordermatrix{
    & \Blue & \Red & \Green \\
    \true & \emptyset & \left\{ \begin{tikzcd} \LMOO{\const{s_{12}}}{\Red} \ar[loop right,
        "\true"]\end{tikzcd} \right\}  & \emptyset \\
    \false & \left\{ \begin{tikzcd} \LMOO{\const{s_{11}}}{\Blue} \ar[loop left,
        "\false"] \end{tikzcd}, \begin{tikzcd} \LMOO{\const{s_{13}}}{\Blue} \ar[loop left,
        "\false"] \end{tikzcd} \right\} & \emptyset & \emptyset
}    
\end{equation}
If we just want to know how many $\lens{i}{o}$-steady states there are, and not
precisely which states they are, we can always take the cardinality of the sets
in our matrix of sets to get a bona-fide matrix of numbers. Doing this to the
above matrix gives us the matrix
 \[\kbordermatrix{
    & \Blue & \Red & \Green \\
    \true & 0 & 1 & 0 \\
    \false & 2 & 0 & 0
}    
\]

Now, let's take a look at system $\Sys{S_2}$ from the same exercise:
\[
\Sys{S_2} \coloneqq \begin{tikzpicture}[baseline=(bl)]
	\node[draw] (bl) {
  \begin{tikzcd}[column sep=small]
    \LMOO{\const{s_{21}}}{\true} \ar[out=120, in=90, loop, red] \ar[in=210, out=250, loop, blue] \ar[rr, bend left = 10, dgreen] \ar[rr, leftarrow, red, bend right= 10] \ar[ddr, leftarrow, dgreen, bend right= 10] &  & \LMOO{\const{s_{22}}}{\false} \ar[loop right, dgreen] \ar[ddl, blue, bend left= 10] \ar[ddl,red, leftarrow, bend right= 10]  \\
    & & \\
    & \LMOO{\const{s_{23}}}{\true} \ar[out=300, in=240, loop, blue] & 
  \end{tikzcd}
  };
\end{tikzpicture}
\]

This has steady state matrix:
\begin{equation}\label{eqn.steady_state_matrix2}
  \Set{Steady}_{\Sys{S_2}} = \kbordermatrix{
    & \true & \false \\
    \Blue & \left\{ \begin{tikzcd} \LMOO{\const{s_{21}}}{\true}  \ar[ loop left,
        blue] \end{tikzcd}, \begin{tikzcd}\LMOO{\const{s_{23}}}{\true} \ar[loop left,
        blue] \end{tikzcd}\right\} & \emptyset \\
    \Red & \left\{ \begin{tikzcd} \LMOO{\const{s_{21}}}{\true} \ar[loop left,
        red] \end{tikzcd} \right\}& \emptyset \\
    \Green & \emptyset & \left\{ \begin{tikzcd} \LMOO{\const{s_{22}}}{\false} \ar[loop left,
        dgreen]
      \end{tikzcd} \right\}
}
\end{equation}
Or, again, if we just want to know how many steady states there are for each
chart:
\[
  \Set{Steady}_{\Sys{S_2}} = \kbordermatrix{
    & \true & \false \\
    \Blue & 2 & 0 \\
    \Red & 1 & 0 \\
    \Green & 0 & 1
}
\]

We can wire these systems together to get a system $\Sys{S}$:
\[
\Sys{S} \coloneqq 
\begin{tikzpicture}[oriented WD, bbx = .3cm, bby =.3cm, bb min width=.5cm, bb port length=2pt, bb port sep=1, every fit/.style={inner xsep=\bbx, inner ysep=\bby}
, baseline=(Outer.center)]
  \node[bb={1}{1}, fill=blue!10] (S1) {$\Sys{S_1}$};
  \node[bb={1}{1}, fill=blue!10, right= of S1] (S2) {$\Sys{S_2}$};

  \node[bb={1}{1}, fit={(S1) (S2)}] (Outer) {};

  \draw (Outer_in1) to (S1_in1);
  \draw (S1_out1) to (S2_in1);
  \draw (S2_out1) to (Outer_out1);
\end{tikzpicture}
\]

With just a bit of thought, we can find the steady states of this systems without fully calculating its
dynamics. A state of $\Sys{S}$ is a pair of states $s_1 \in  \State{S_1}$ and
$s_2 \in \State{S_2}$, so for it to be steady both its constituent states must be steady.
So let $\lens{i}{o} : \lens{\ord{1}}{\ord{1}} \tto
\lens{\Set{Bool}}{\Set{Bool}}$ be a chart for $\Sys{S}$ --- a pair of booleans.
We need $s_1$ and $s_2$ to both be steady, so in particular $s_1$ must be steady
at the input $i$, and $s_2$ must expose $o$; but, most importantly, $s_2$ must then be steady at the input
$\expose{S_1}(s_1)$ which $s_1$ exposes.

So, to find the set of
$\lens{\true}{\true}$-steady states of $\Sys{S}$, we must a state of
$\Sys{S_1}$ which is steady for the input $\true$ and then a steady state of
$\Sys{S_2}$ whose input is what that state outputs and whose output is $\true$.
There are three pieces of data here: the state of $\Sys{S_1}$, the state of
$\Sys{S_2}$, and the intermediate value expose by the first state and input into
the second state. We can therefore describe the set of $\lens{\true}{\true}$-steady states of
$\Sys{S}$ like this:
\begin{align*}
  \Set{Steady}_{\Sys{S}}\lens{\true}{\true} &= \left\{ (m, s_1, s_2)
  \middle| \begin{aligned}
    s_1 &\in \Set{Steady}_{\Sys{S_1}}\lens{\true}{m},
    s_2 &\in \Set{Steady}_{\Sys{S_2}}\lens{m}{\true}
  \end{aligned}\right\} \\
  &= \sum_{m \in \Set{Colors}} \Set{Steady}_{\Sys{S_1}} \lens{\true}{m} \times \Set{Steady}_{\Sys{S_2}}\lens{m}{\true}.
\end{align*}

This formula looks very suspiciously like matrix multiplication! Indeed, if we
multiply the matrices of numbers of steady states from $\Sys{S_1}$ and
$\Sys{S_2}$, we get:
\[\kbordermatrix{
    &  &  &  \\
    \true & 0 & 1 & 0 \\
    \false & 2 & 0 & 0
}   
\kbordermatrix{
    & \true & \false \\
     & 2 & 0 \\
     & 1 & 0 \\
     & 0 & 1
}
= \kbordermatrix{
  & \true & \false \\
  \true & 1 & 0 \\
  \false & 4 & 0 
} 
\]
which is the matrix of how many steady states $\Sys{S}$ has! What's even more
suspicious is that our wiring diagram for $\Sys{S}$ looks a lot like the string
diagram we would use to describe the multiplication of matrices:
\[
\begin{tikzpicture}[oriented WD, bbx = .3cm, bby =.3cm, bb min width=.5cm, bb port length=2pt, bb port sep=1, every fit/.style={inner xsep=\bbx, inner ysep=\bby}
, baseline=(Outer.center)]
  \node[bb={1}{1}, fill=blue!10] (S1) {$\Sys{S_1}$};
  \node[bb={1}{1}, fill=blue!10, right= of S1] (S2) {$\Sys{S_2}$};

  \node[bb={1}{1}, fit={(S1) (S2)}] (Outer) {};

  \draw (Outer_in1) to (S1_in1);
  \draw (S1_out1) to (S2_in1);
  \draw (S2_out1) to (Outer_out1);
\end{tikzpicture} \quad\quad\quad\quad
\begin{tikzpicture}[oriented WD, bb small, bb port length=5pt, baseline=(f)]
	\node[bb={1}{1}, rounded corners=5pt, draw=dgreen] (f) {$\Set{Steady}_{\Sys{S_1}}$};
	\node[bb={1}{1}, rounded corners=5pt, draw=dgreen, right=3 of f] (g) {$\Set{Steady}_{\Sys{S_2}}$};
	\node[left=0 of f_in1] {$\Set{Bool}$};
	\node[right=0 of g_out1] {$\Set{Bool}$};
	\draw (f_out1) to node[above, font=\scriptsize] {$\Set{Colors}$} (g_in1);
\end{tikzpicture}
\]
This can't just be a coincidence. Luckily for our sanity, it isn't. In the
remainder of this section, we will show how various things one can do with
matrices --- multiply them, trace them, Kronecker product them --- can be done
for matrices of sets, and how if your wiring diagram looks like its telling you
to that thing, then you can do that thing to the steady states of your internal
systems to get the steady states of the whole wired system

\paragraph{Matrices of sets}\label{sec.matrix_of_sets}

We'll be working with matrices of sets --- now and in the coming section ---
quite a bit, so we should really nail them down. Matrices of sets work a lot
like matrices of numbers, especially when the sets are finite; then they are
very nearly the same thing as matrices of whole numbers. But the matrix
arithmetic of infinite sets works just the same as with finite sets, so we'll do
everything in that generality.\footnote{This will help us later when we deal
  with behaviors that have more complicated charts. For example, even finite
  systems can have infinitely many different trajectories, so we really need the
  infinite sets.}

\begin{definition}\label{def.matrix_of_sets}
  Let $A$ and $B$ be two sets. $B \times A$ \emph{matrix of sets} is a dependent
  set $M : B \times A \to \smset$. For $a \in A$ and $b \in B$, we write
  $M_{ba}$ or $M_{(b, a)}$ for set indexed by $a$ and $b$, and call this the
  $(b,a)$-entry of the matrix $M$.


  We draw of matrix of sets with the following string diagram:
  \[
\begin{tikzpicture}[oriented WD, bb small, bb port length=5pt, baseline=(f)]
	\node[bb={1}{1}, rounded corners=5pt, draw=dgreen] (f) {$M$};
	\node[left=0 of f_in1] {$A$};
  \node[right=0 of f_out1] {$B$};
\end{tikzpicture}
  \]
\end{definition}
\begin{remark}
  We can see a dependent set $X_{-} : A \to \smset$ through the matrix of sets
  point of view as a \emph{vector of sets}. This is because $X_{-}$ is
  equivalently given by $X_{-} : A \times \ord{1} \to \smset$, which we see is a
  $A \times \ord{1}$ matrix of sets. A $n \times 1$ matrix is equivalently a
  column vector.
\end{remark}

Now we'll go through and define the basic operations of matrix arithmetic:
mutliplication, Kronecker product (also known as the tensor product), and
partial trace.

\begin{definition}\label{def.matrix_of_sets_multiplication}
  Given an $B \times A$ matrix of sets $M$ and a  $C \times B$ matrix of sets
  $N$, their \emph{product} $NM$ (or $M \times_B N$ for emphasis) is the $C
  \times A$ matrix of sets with entries
  $$NM_{ca} = \sum_{b \in B}  N_{cb}\times M_{ba}.$$

  We draw the multiplication of matrices of sets with the following string
  diagram:
  \[
\begin{tikzpicture}[oriented WD, bb small, bb port length=5pt, baseline=(f)]
	\node[bb={1}{1}, rounded corners=5pt, draw=dgreen] (f) {$M$};
	\node[bb={1}{1}, rounded corners=5pt, draw=dgreen, right=1.5 of f] (g) {$N$};
	\node[left=0 of f_in1] {$A$};
	\node[right=0 of g_out1] {$C$};
	\draw (f_out1) to node[above, font=\scriptsize] {$B$} (g_in1);
\end{tikzpicture}
  \]
   
  The identity matrix $I_A$ is an $A \times A$ matrix with entries
  $$I_{aa'} = \begin{cases} \ord{1} &\mbox{if $a = a'$} \\ \emptyset &\mbox{if
      $a \neq a'$} \end{cases}.$$

  We draw the identity matrix as a string with no beads on it.
  \[
\begin{tikzpicture}[oriented WD, bb small, bb port length=5pt, baseline=(Left)]
  \node (Left) (Left){};
  \node[right = 3 of Left](Right) {};
  \draw (Left.center) -- (Right.center);
  \node[left=0 of Left] {$A$};
  \node[right=0 of Right] {$A$};
\end{tikzpicture}
  \]
  
\end{definition}

\begin{exercise}\label{ex.matrix_of_sets_mult_laws}
  Multiplication of matrices of sets satisfies the usual properties of
  associativity and unity, but only up to isomorphism. Let $M$ be a $B \times A$
  matrix, $N$ a $C \times B$ matrix, and $L$ a $D \times C$ of sets. Show that
  \begin{enumerate}
   \item  For all $a \in A$ and $d \in D$, $((LN)M)_{da} \cong (L(NM))_{da}$.
   \item For all $a \in A$ and $b \in B$, $(MI_A)_{ba} \cong M_{ba} \cong (I_BM)_{ba}$.
  \end{enumerate}
\end{exercise}

\begin{remark}
  The isomorphisms you defined in \cref{ex.matrix_of_sets_mult_laws} are
  \emph{coherent}, much in the way the associativity and unity isomorphisms of a
  monoidal category are. Together, this means that there is a \emph{bicategory}
  of sets and matrices of sets between them. 
\end{remark}

\begin{definition}\label{def.matrix_of_sets_tensor}
  Let $M$ be a $B \times A$ matrix and $N$ a $C \times D$ matrix of sets. Their
  \emph{Kronecker product} or \emph{tensor product} $M \otimes N$ is a $(B
  \times C) \times (A \times D)$ matrix of sets with entries:
  $$(M \otimes N)_{(b, c)(a, d)} = M_{ba} \times N_{cd}.$$

  We draw the tensor product $M \otimes M$ of matrices as:
  \[
\begin{tikzpicture}[oriented WD, bb small, bb port length=5pt, baseline=(f)]
	\node[bb={1}{1}, rounded corners=5pt, draw=dgreen] (f) {$M$};
	\node[bb={1}{1}, rounded corners=5pt, draw=dgreen, below= of f] (g) {$N$};
	\node[left=0 of f_in1] {$A$};
	\node[right=0 of f_out1] {$B$};
	\node[right=0 of g_out1] {$D$};
	\node[left=0 of g_in1] {$C$};
\end{tikzpicture}
  \]
\end{definition}

Finally, we need to define the partial trace of a matrix of sets.
\begin{definition}
  Suppose that $M$ is a $(A \times C) \times (A \times B)$ matrix of sets. Its
  \emph{partial trace} $\fun{tr}_{A} M$ is a $C \times B$ matrix of sets with
  entries:
  $$(\fun{tr}_{A})M_{cb} = \sum_{a \in A} M_{(a,c)(a,b)}.$$

    We draw the partial trace of a matrix of sets as:
\[
\begin{tikzpicture}[oriented WD, bb small, bb port length=5pt, baseline=(M)]
	\node[bb={2}{2}, rounded corners=5pt, draw=dgreen] (M) {$M$};
	\node[left=1.5 of M_in2] (B) {$B$};
	\node[right=1.5 of M_out2] (C) {$C$};
  \draw let \p1=(M.north east), \p2=(M.north west), \n1={\y2+\bby}, \n2=\bbportlen in
          (M_out1) to[in=0] (\x1+\n2,\n1) -- node[above, font=\scriptsize] {$A$} (\x2-\n2,\n1) to[out=180] (M_in1);
  \draw (B) to (M_in2);
  \draw (M_out2) to (C);
\end{tikzpicture}
\]
\end{definition}

\begin{exercise}\label{ex.matrix_of_sets_sanity_check}
Here's an important sanity check we should do about our string diagrams for
matrices of sets. The following two diagrams should describe the same matrix,
even though they describe it in different ways:
    \[
\begin{tikzpicture}[oriented WD, bb small, bb port length=5pt, baseline=(f)]
	\node[bb={1}{1}, rounded corners=5pt, draw=dgreen] (f) {$M$};
	\node[bb={1}{1}, rounded corners=5pt, draw=dgreen, right=1.5 of f] (g) {$N$};
	\node[left=0 of f_in1] {$A$};
	\node[right=0 of g_out1] {$C$};
	\draw (f_out1) to node[above, font=\scriptsize] {$B$} (g_in1);
\end{tikzpicture}
\quad\quad\quad\quad
\begin{tikzpicture}[oriented WD, bb small, bb port length=5pt, baseline=(Q)]
  \node[bb={1}{1}, rounded corners=5pt, draw=dgreen] (M) {$M$};
  \node[bb={1}{1}, rounded corners=5pt, draw=dgreen, below= of M] (N) {$N$};
  \node[bb={0}{0}, rounded corners=5pt, draw=dgreen, dashed, fit={(N) (M)}] (Q) {};

  \node[left = 3 of Q.center] (Left) {$A$};
  \node[right = 3 of Q.center] (Right) {$C$};

  \draw let \p1=(Q.north east), \p2=(Q.south west), \n1={\y2-\bby}, \n2=\bbportlen in    (M_out1-|Q.east) to[in=0] (\x1 +\n2, \n1) -- node[below, font=\scriptsize] {$B$} (\x2-\n2, \n1) to[out=180] (N_in1-|Q.west);
  \draw (Left) to[out=0, in=180] (M_in1-|Q.west);
  \draw (N_out1-|Q.east) to[out=0, in=180] (Right);
\end{tikzpicture}
    \]
The diagram on the left says ``multiply $M$ and $N$'', while the diagram on the
right says ``tensor $M$ and $N$, and then partially trace them.''. Show that
these two diagrams do describe the same matrix:
$$NM \cong \fun{tr}_{B}(M \otimes N).$$
Compare this to \cref{ex.ClockWithDisplay}, where we say that wiring an input of
a system to an output of another can be seen as first taking their parallel
product, and then forming a loop.
\end{exercise}

\paragraph{Steady states and matrix arithmetic}

For the remainder of this section, we will show that we can calculate the steady
state matrix of a wired together system in terms of its component system in a
very simple way:
\begin{itemize}
  \item First, take the steady state matrices of the component systems.
  \item Then consider the wiring diagram as a string diagram for multiplying,
tensoring, and tracing matrices.
  \item Finally, finish by doing all those operations to the matrix.
\end{itemize}

In \cref{sec.representables}, we will see that this method --- or something a
lot like it --- works calculating the behaviors of a composite system out of the
behaviors of its components, as long as the representative of that behavior
exposes its entire state. That result will be nicely packaged in a beautiful
categorical way: we'll make an \emph{doubly indexed functor}.

But for now, let's just show that tensoring and partially tracing steady state
matrices correponds to taking the parallel product and wiring an input to an
output, respectively, of systems.

\begin{proposition}\label{prop.steady_state_matrix_parallel_tensor}
  Let $\Sys{S_1}$ and $\Sys{S_2}$ be systems. Then the steady state matrix of the
  parallel product $\Sys{S_1 \otimes S_2}$ is the tensor of their steady state
  matrices:
  $$\Set{Steady}_{\Sys{S_1 \otimes S_2}} \cong \Set{Steady}_{\Sys{S_1}} \otimes \Set{Steady}_{\Sys{S_2}}.$$
\end{proposition}
\begin{proof}
  First, we note that these are both $(\Out{S_1} \times \Out{S_2}) \times
  (\In{S_1} \times \In{S_2})$-matrices of sets. Now, on a chart $\lens{(i_1,
    i_2)}{(o_1, o_2)}$, a steady state in $\Sys{S_1} \otimes \Sys{S_2}$ will be
  a pair $(s_1, s_2) \in \State{S_1} \times \State{S_2}$ such that
  $\update{S_j}(s_j, i_j) = s_j$ and $\expose{S_j}(s_j) = o_j$ for $j = 1,\, 2$.
  In other words, its just a pair of steady states, one in $\Sys{S_1}$ and one
  in $\Sys{S_2}$. This is precisely the $\lens{(i_1, i_2)}{(o_1, o_2) }$-entry
  of the right hand side above. 
\end{proof}


\begin{remark}
  \cref{prop.steady_state_matrix_parallel_tensor} is our motiviation for using
  the symbol ``$\otimes$'' for the parallel product of systems.
\end{remark}

\begin{proposition}\label{prop.steady_state_matrix_trace}
Let $\Sys{S}$ be a system with $\In{S} = A \times B$ and $\Out{S} = A \times C$.
Let $\Sys{S'}$ be the system formed by wiring the $A$ output into the $A$ input
of $\Sys{S}$:
\[\Sys{S'} \coloneqq
\begin{tikzpicture}[oriented WD, every fit/.style={inner xsep=\bbx, inner ysep=\bby}, bbx = .3cm, bby =.3cm, bb min width=.5cm, bb port length=2pt, bb port sep=1, baseline=(S'.center)]
	\node[bb={2}{2}, fill=blue!10] (S) {$\Sys{S}$};

  \node[bb={0}{0}, fit={($(S.north east) + (1,1)$) ($(S.south west) - (1,0)$)}] (S') {};
  
  \draw (S_out2) to (S_out2-|S'.east);
  \draw (S_in2-|S'.west) to (S_in2);

  \draw let \p1=(S.north east), \p2=(S.north west), \n1={\y2+\bby}, \n2=\bbportlen in    (S_out1) to[in=0] (\x1 +\n2, \n1) -- (\x2-\n2, \n1) to[out=180] (S_in1);
\end{tikzpicture}
\]
Then the steady state matrix of $\Sys{S'}$ is given by partially tracing out $A$
in the steady state matrix of $\Sys{S}$:
\[
  \Set{Steady}_{\Sys{S'}} = 
\begin{tikzpicture}[oriented WD, bb small, bb port length=5pt, baseline=(M)]
	\node[bb={2}{2}, rounded corners=5pt, draw=dgreen] (M) {$\Set{Steady}_{\Sys{S}}$};
	\node[left=1.5 of M_in2] (B) {};
	\node[right=1.5 of M_out2] (C) {};
  \draw let \p1=(M.north east), \p2=(M.north west), \n1={\y2+\bby}, \n2=\bbportlen in
          (M_out1) to[in=0] (\x1+\n2,\n1) -- node[above, font=\scriptsize] {$A$} (\x2-\n2,\n1) to[out=180] (M_in1);
  \draw (B) to (M_in2);
  \draw (M_out2) to (C);
\end{tikzpicture} = \fun{tr}_{A}\left(\Set{Steady}_{\Sys{S}}\right)
\]
\end{proposition}
\begin{proof}
  Let's first see what a steady state of $\Sys{S'}$ would be. Since $\Sys{S'}$
  is just a rewiring of $\Sys{S}$, it has the same states; so, a steady state $s$ of
  $\Sys{S'}$ is in particular a state of $\Sys{S}$. Now,
  $$\update{S'}(s,b) = \update{S}(s, (\pi_1\expose{S}(s), b))$$
  by definition, so if $\update{S'}(s, b) = s$, then $\update{S}(s, (\pi_1\expose{S}(s),
  b)) = s$. If also $\expose{S'}(s) = c$ (so that $s$ is a
  $\lens{b}{c}$-steady state of $\Sys{S'}$), then $\pi_2\expose{S}(s) =
  \expose{S'}(s) = c$ as well. In total then, starting with a
  $\lens{b}{c}$-steady state $s$ of $\Sys{S'}$, we get a
  $\lens{(\pi_1\expose{S}(s), b)}{(\pi_1\expose{S}(s), c)}$-steady state of
  $\Sys{S}$.  That is, we have a function
  $$s \mapsto (\pi_1 \expose{S}(s), s) : \Set{Steady}_{\Sys{S'}}\lens{b}{c} \to
  (\fun{tr}_{A} \Set{Steady}_{\Sys{S}}) \lens{b}{c}.$$

  It remains to show that this function is a bijection. So, suppose we have a
  pair $(a, s) \in \fun{tr}_A \Set{Steady}_{\Sys{S}}\lens{b}{c}$ of an $a \in A$ and a
  $\lens{(a, b)}{(a, c)}$ steady state of $\Sys{S}$. Then
  \begin{align*}
    \update{S'}(s, b) &= \update{S}(s, (\pi_1\expose{S}(s), b)) \\
                      &= \update{S}(s, (a, b)) &\mbox{since $\expose{S}(s) = (a, c)$.}\\
                      &= s &\mbox{since $s$ is a $\lens{(a, b)}{(a, c)}$-steady state.}\\
    \expose{S'}(s) &= \pi_2\expose{S}(s) = c.
  \end{align*}
  This shows that $s$ is also a $\lens{b}{c}$ steady state of $\Sys{S'}$, giving
  us a function
  $(a, s) \mapsto s : (\fun{tr}_A \Set{Steady}_{\Sys{S}}) \to \Set{Steady}_{\Sys{S'}}.$
  These two functions are plainly inverse.
\end{proof}

We can summarize \cref{prop.steady_state_matrix_trace} in the following
commutative diagram:
\begin{equation}\label{eqn.steady_state_matrix_compose}
\begin{tikzpicture}
\node[draw] (Topleft) {
\begin{tikzpicture}[oriented WD, every fit/.style={inner xsep=\bbx, inner ysep=\bby}, bbx = .3cm, bby =.3cm, bb min width=.5cm, bb port length=2pt, bb port sep=1, baseline=(S)]
	\node[bb={2}{2}, fill=blue!10] (S) {$\Sys{S}$};
\end{tikzpicture}

};


\node[draw, below = 4 of Topleft] (Botleft) {
\begin{tikzpicture}[oriented WD, every fit/.style={inner xsep=\bbx, inner ysep=\bby}, bbx = .3cm, bby =.3cm, bb min width=.5cm, bb port length=2pt, bb port sep=1, baseline=(S'.center)]
	\node[bb={2}{2}, fill=blue!10] (S) {$\Sys{S}$};

  \node[bb={0}{0}, fit={($(S.north east) + (1,1)$) ($(S.south west) - (1,0)$)}] (S') {};
  
  \draw (S_out2) to (S_out2-|S'.east);
  \draw (S_in2-|S'.west) to (S_in2);

  \draw let \p1=(S.north east), \p2=(S.north west), \n1={\y2+\bby}, \n2=\bbportlen in    (S_out1) to[in=0] (\x1 +\n2, \n1) -- (\x2-\n2, \n1) to[out=180] (S_in1);
\end{tikzpicture}
};



\node[draw, right = 4 of Botleft]  (Botright) {
\begin{tikzpicture}[oriented WD, bb small, bb port length=5pt, baseline=(M)]
	\node[bb={2}{2}, rounded corners=5pt, draw=dgreen] (M) {$\Set{Steady}_{\Sys{S}}$};
	\node[left=1.5 of M_in2] (B) {};
	\node[right=1.5 of M_out2] (C) {};
  \draw let \p1=(M.north east), \p2=(M.north west), \n1={\y2+\bby}, \n2=\bbportlen in
          (M_out1) to[in=0] (\x1+\n2,\n1) -- (\x2-\n2,\n1) to[out=180] (M_in1);
  \draw (B) to (M_in2);
  \draw (M_out2) to (C);
\end{tikzpicture}
};

\node[draw] at (Botright|-Topleft)(Topright) {
\begin{tikzpicture}[oriented WD, bb small, bb port length=5pt, baseline=(f)]
	\node[bb={2}{2}, rounded corners=5pt, draw=dgreen] (f) {$\Set{Steady}_{\Sys{S}}$};
\end{tikzpicture}
};

\draw[->, shorten <= 5pt, shorten >= 5pt] (Topleft) to node[above] {$\Set{Steady}$} (Topright);
\draw[->, shorten <= 5pt, shorten >= 5pt] (Botleft) to node[below] {$\Set{Steady}$} (Botright);
\draw[->, shorten <= 5pt, shorten >= 5pt] (Topleft) to coordinate[left] (Leftlabel) (Botleft);
\draw[->, shorten <= 5pt, shorten >= 5pt] (Topright) to coordinate[right] (Rightlabel) (Botright);

\node[left = 0 of Leftlabel] {
\begin{tikzpicture}[oriented WD, every fit/.style={inner xsep=\bbx, inner ysep=\bby}, bbx = .3cm, bby =.3cm, bb min width=.5cm, bb port length=2pt, bb port sep=1, baseline=(S'.center)]
	\node[bb={2}{2}, fill=blue!10, dashed] (S) {};

  \node[bb={0}{0}, fit={($(S.north east) + (1,1)$) ($(S.south west) - (1,0)$)}] (S') {};
  
  \draw (S_out2) to (S_out2-|S'.east);
  \draw (S_in2-|S'.west) to (S_in2);

  \draw let \p1=(S.north east), \p2=(S.north west), \n1={\y2+\bby}, \n2=\bbportlen in    (S_out1) to[in=0] (\x1 +\n2, \n1) -- (\x2-\n2, \n1) to[out=180] (S_in1);
\end{tikzpicture}
};
\node[right= 0 of Rightlabel] {
\begin{tikzpicture}[oriented WD, bb small, bb port length=5pt, baseline=(M)]
	\node[bb={2}{2}, rounded corners=5pt, draw=dgreen, dashed] (M) {$\phantom{\Set{Steady}_{\Sys{S}}}$};
	\node[left=1.5 of M_in2] (B) {};
	\node[right=1.5 of M_out2] (C) {};
  \draw let \p1=(M.north east), \p2=(M.north west), \n1={\y2+\bby}, \n2=\bbportlen in
          (M_out1) to[in=0] (\x1+\n2,\n1) -- (\x2-\n2,\n1) to[out=180] (M_in1);
  \draw (B) to (M_in2);
  \draw (M_out2) to (C);
\end{tikzpicture}
};
\end{tikzpicture}
\end{equation}

The horizontal maps take the steady states of a system, while the vertical map
on the left wires together the system with that wiring diagram, and the vertical
map on the right applies that transformation of the matrix. In the next section,
we will see how this square can be interepreted as a naturality condition in a
\emph{doubly indexed functor}.

One thing to notice here is that taking the partial trace (the right vertical
arrow in the diagram) is itself given by multiplying by a certain matrix.
\begin{proposition}\label{prop.trace_multiplying_by_matrix}
  Let $M$ be a $(A \times C) \times (A \times B)$ matrix of sets. Let
  $\Set{Tr}^A$ be the $\big(C \times B \big) \times \big((A \times C) \times (A
  \times B)\big)$ matrix of sets with entries:
  \[
    \Set{Tr}^A{A}_{(c, b)((a,c'),(a', b'))} \coloneqq \begin{cases} 
      \ord{1} &\mbox{if $a = a'$, $b = b'$, and $c = c'$.} \\
      \emptyset &\mbox{otherwise.}
    \end{cases} 
\]
  Then, considering $M$ as a $\big((A \times C) \times (A \times B)\big) \times
  \ord{1}$ matrix of sets, taking its trace is given by multiplying by $\Set{Tr}^A$:
$$\fun{tr}_{A}M \cong \Set{Tr}^A M$$
\end{proposition}
\begin{proof}
  Let's calculate that matrix product on the right.
  \begin{align*}
    (\Set{Tr}^A M)_{(c, b)} &= \sum_{((a, c'), (a', b')) \in (A \times C) \times (A \times B)} \Set{Tr}^A_{(c,b)((a, c'),(a', b'))} \times M_{(a,c')(a',b')} \\
  \end{align*}
  Now, since $\Set{Tr}^A_{(c,b)((a,c'),(a',b'))}$ is a one element set (if $a =
  a'$, $c = c'$, and $b = b'$) and is empty otherwise, the inner expression has
  the elements of $M_{(a,c')(a', b')}$ if and only if $a = a'$, $b = b'$, and $c
  = c'$ and is otherwise empty. So, we conclude that
  \[
\sum_{((a, c'), (a', b')) \in (A \times C) \times (A \times B)}
\Set{Tr}^A_{(c,b)((a, c'),(a', b'))} \times M_{(a,c')(a',b')} \cong M_{(a, c)(a,
  b)}.
\]
As desired.
\end{proof}




%---- Section ----%
\section{Indexed categories and the Grothendieck construction}\label{sec.indexed_categories}

An \emph{indexed category} is a category which varies functorially over the
objects of another category.
%%
%% :CUSTOM-ID:cite-this.pseudo_functors
%%

\begin{definition}
  An \emph{indexed category} $\cat{A} : \cat{C}\op \to \Cat{Cat}$ is a
  contravariant (pseudo-) functor\footnote{A pseudo-functor is like a functor,
    but it only satisfies the functoriality conditions up to isomorphism, and
    these isomorphisms must satisfy some laws that make them ``cohere''. The
    indexed categories we will see in \cref{chapter.1,chapter.2,chapter.3}
    will all be actual functors, however.}. We call the category $\cat{C}$ the
  \emph{base} of the indexed category $\cat{A}$. Explicitly, an indexed
  category $\cat{A}$ has:
  \begin{itemize}
  \item A base category $\cat{C}$.
  \item For every object $C \in \cat{C}$ of the base, a category $\cat{A}(C)$.
  \item For every map $f : C \to C'$ in the base, a \emph{pullback} functor
    $f^{\ast} : \cat{A}(C') \to \cat{A}(C)$, which we think of as ``reindexing''
    the objects of $\cat{A}(C')$ so that they live over $\cat{A}(C)$.
  \item (Psuedo-functoriality) For any two functions $f : C \to C'$ and $g : C' \to C''$, we need a
    natural isomorphism $f^{\ast} \circ g^{\ast} \xequals{\gamma_{g, f}} (g \circ
    f)^{\ast}$. And for any object $C$, we need a natural isomorphism $:
    \id_{\cat{A}(C)} \xequals{\iota_C } (\id_C)^{\ast}$. These are required to satisfy a
    few \emph{coherence conditions}:
\begin{align}
  \gamma_{h,gf} \circ (\gamma_{g, f}h^{\ast}) &= \gamma_{hg, f} \circ (f^{\ast}\gamma_{h, g}) \\
  \gamma_{\id_C, f} \circ (f^{\ast}\iota_C) &= \gamma_{f, \id_{C'}} \circ (\iota_{C'} f^{\ast}).
\end{align}
   Often, $\gamma$ and $\iota$ will both be identities, and these equalities
   will follow trivially.
  \end{itemize}
\end{definition}

\begin{remark}\label{rmk.coherence_condition_indexed_cat}
  The coherence conditions are almost always trivial --- if not strictly
  identities, then they are probably just a shuffling of parentheses. For this
  reason, and because keeping track of them is a headache, we will omit
  discussion of the coherences. We will write a coherence with a big equals,
  like this $f^{\ast} \circ g^{\ast} \xequals{\phantom{ \gamma_{g, f} }} (g \circ
    f)^{\ast}$ and leave the name implicit. The coherence conditions ensure that
    working with these isomorphisms as though they really were equalities works well.
\end{remark}

Indexed categories are quite common throughout mathematics. We will construct a
particular example for our own purposes in \cref{sec.context_indexed_cat}, and
more throughout the book.

\begin{example}\label{ex.indexed_cat_of_dependent_sets}
  Recall that a \emph{dependent set} is a function $X : A \to \smset$ from a set
  into the category of sets. We have an indexed category of dependent sets
  $$\smset^{(-)} : \smset\op \to \Cat{Cat}$$
  which is defined as follows:
  \begin{itemize}
    \item To each set $A$, we assign the category $\smset^A$ of sets indexed by
      $A$. The objects of $\smset^A$ are the sets $X : A \to \smset$ indexed by
      $A$, and a map $f : X \to Y$ is a family of maps $f_a : X_a \to Y_a$
      indexed by the elements $a \in A$. Composition is given componentwise: $(g
      \circ f)_a = g_a \circ f_a$.
    \item To every function $f : A' \to A$, we get a reindexing functor
$$f^{\ast} : \smset^A \to \smset^{A'}$$
   Given by precomposition: $X \mapsto X \circ f$. The indexed set $X \circ f :
   A' \to \smset$ is the set $X_{f(a')}$ on the index $a' \in A'$. The families
   of functions get reindexed the same way.
     \item Since our reindexing is just given by precomposition, it is clearly
       functorial on the nose (that is, not just up to isomorphism).
  \end{itemize}
  We will return to this example in much greater detail in \cref{chapter.4}.
\end{example} 


If we have an family of sets $A : I \to \Cat{Set}$ indexed by a set $I$, we can
form the disjoint union $\sum_{i \in I} A_i$, together with the projection $\pi
: \sum_{i \in I} A_i \to I$ sending each $a \in A_i$ to $i$. The Grothendieck
construction is a generalization of this construction to indexed categories. Namely, we will take an indexed category $\cat{A}
: \cat{C} \to \Cat{Cat}$ and form a new category

$$\int^{C : \cat{C}} \cat{A}(C)$$
which we think of as a ``union'' off all the categories $\cat{A}(C)$. But this
``union'' will not be
disjoint since there will be morphisms from objects in $\cat{A}(C)$ to objects
in $\cat{A}(C')$. This is why we use the integral notation; we want to suggest
that the Grothendieck construction is a sort of sum.\footnote{The Grothendieck
  construction is an example of a \emph{lax colimit} in 2-category theory,
  another sense in which it is a `sort of sum'.}
\begin{definition}\label{def.grothendieck_construction}
  Let $\cat{A} : \cat{C}\op \to \Cat{Cat}$ be an indexed category. The
  \emph{Grothendieck construction} of $\cat{A}$
  $$\int^{C : \cat{C}} \cat{A}(C)$$
  is the category with:
  \begin{itemize}
    \item Objects pairs $\lens{A}{C}$ of objects $C \in \cat{C}$ and $A \in
      \cat{A}(C)$. We say that $A$ ``sits over'' $C$.
    \item Maps $\lens{f_{\flat}}{f} : \lens{A}{C} \rightrightarrows
      \lens{A'}{C'}$ pairs of $f : C \to C'$ in $\cat{C}$ and $f_{\flat} :
      A \to f^{\ast}A'$ in $\cat{A}(C)$.
    \item Given $\lens{f_{\flat}}{f} : \lens{A}{C} \rightrightarrows
      \lens{A'}{C'}$ and $\lens{g_{\flat}}{g} : \lens{A'}{C'} \rightrightarrows
      \lens{A''}{C''}$, their composite is given by
      $$\lens{g_{\flat}}{g} \circ \lens{f_{\flat}}{f} \coloneqq \lens{f^{\ast}g_{\flat}
      \circ f_{\flat}}{g \circ f}$$
    Written with the signatures, this looks like
    $$\lens{A \xto{f_{\flat}} f^{\ast}A' \xto{f^{\ast}g_{\flat}}
      f^{\ast}g^{\ast}A'' \xequals{\phantom{\gamma}} (g \circ f)^\ast A''}{C \xto{f} C' \xto{g} C''}$$
    \item The identity is given by $\lens{\id_A}{\id_C} : \lens{A}{C}
      \rightrightarrows \lens{A}{C}$
  \end{itemize}
\end{definition}

\begin{exercise}
  Check that \cref{def.grothendieck_construction} does indeed make $\int^{C :
    \cat{C}} \cat{A}(C)$ into a category. That is, check that composition as
  defined above is associative and unital.
\end{exercise}

The confluence of notation with that of charts is no accident. Indeed, we will
see in the next section that the category of charts can be described as the
Grothendieck construction of a certain indexed category.

%---- Section ----%
\section{Sets with context, charts, and lenses}\label{sec.context_indexed_cat}

In this section, we will see how both the category $\Cat{Chart}$ and
$\Cat{Lens}$ of charts and lenses in the category of sets can be described using
the Grothendieck construction. To do this, we need some other categories named
after their maps (rather than their objects): 
\emph{category of sets and functions with context $C$} for some a given set $C$. 

\begin{definition}
  Let $C$ be a set. The \emph{category $\smctx{C}$ of sets and functions with context $C$}
  is the category defined by:
  \begin{itemize}
    \item Objects are sets.
    \item Maps $f : X \ctxto Y$ are functions $f : C \times X \to Y$.
    \item The composite $g \circ f$ of $f : X \ctxto Y$ and $g : Y \ctxto Z$ is
      the function
       $$(c, x) \mapsto g(c, f(c, x)) : C \times X \to Z.$$
       Diagrammatically, this is the composite:
       $$C \times X \xto{\Delta_C \times X} C \times C \times X \xto{C \times f}
       C \times Y \xto{g} Z.$$
    \item The identity $\id : X \ctxto X$ is the second projection $\pi_2 : C
      \times X \to X$.
  \end{itemize}
\end{definition}

\begin{exercise}
  Check that $\smctx{C}$, as defined, really is a category. That is,
  \begin{enumerate}
    \item For $f : X \ctxto Y$, $g : Y \ctxto Z$, and $h : Z \ctxto W$, check
      that $h \circ (g \circ f) = (h \circ g) \circ f$.
    \item For $f : X \ctxto Y$, check that $f \circ \id_X = f = \id_Y \circ f$.
  \end{enumerate}
\end{exercise}

Together, we can arrange the categories of sets and functions with context into
an indexed category.
\begin{definition}
  The \emph{indexed category of sets and functions with context}
  $$\smctx{-} : \smset\op \to \Cat{Cat}$$
  is defined by:
  \begin{itemize}
    \item For a set $C$, we have the category $\smctx{C}$.
    \item For a function $r : C' \to C$, we get a functor
      $$r^{\ast} : \smctx{C} \to \smctx{C'}$$
      given by sending each object to itself, but each morphism $f : C \times X
      \to Y$ in $\smctx{C}$ to the morphism $r^{\ast}f \coloneqq f \circ (r \times X)$:
      $$C' \times X \xto{r \times X} C \times X \xto{f} Y.$$
  \end{itemize}
  We note that this is evidently functorial.
\end{definition}

Now we can see why the category of charts and the category of lenses are so
eerily similar: one will be the Grothendieck construction of $\smctx{-}$, and
the other will be the Grothendieck construction of its dual $\smctx{-}\op$.
\begin{proposition}\label{prop.charts_as_groth_construction}
  The category $\Cat{Chart}$ of charts in the category of sets
  (\cref{def.category_of_charts}) is the Grothendieck construction of $\smctx{-}
  : \smset\op \to \Cat{Cat}$:
  $$\Cat{Chart} = \int^{C \in \smset} \smctx{C}.$$
\end{proposition}
\begin{proof}
  We will expand the definition of the Grothendieck construction, and see that
  it gives us precisely \cref{def.category_of_charts}.

  First, the objects of $\int^{C \in \smset} \smctx{C}$ are pairs of sets
  $\lens{A^-}{A^+}$, since the objects of $\smctx{A^+}$ are just sets
  themselves. So far so good.

  Next, a map $\lens{f_{\flat}}{f} : \lens{A^-}{A^+} \tto \lens{B^-}{B^+}$
  in $\int^{C \in \smset} \smctx{C}$ is a pair of maps $f : A^+ \to B^+$ and
  $f_{\flat} : A^- \ctxto f^{\ast} B^-$ in $\smctx{A^+}$. But $f^{\ast}B^- =
  B^-$ by the definition of the reindexing functor $f^{\ast}$, so by the
  definition of map with context $A^+$, we see that $f_{\flat} : A^+ \times A^-
  \to B^-$. In other words, the maps in $\int^{C \in \smset} \smctx{C}$ are
  precisely the charts. We note that the identity map in the Grothendieck
  construction is the identity chart.

  Finally, we should check that composition of charts is given by composition in
  the Grothendieck construction. Suppose that $\lens{f_{\flat}}{f} :
  \lens{A^-}{A^+} \tto \lens{B^-}{B^+}$ and $\lens{g_{\flat}}{g} :
  \lens{B^-}{B^+} \tto \lens{C^-}{C^+}$ are charts. Then their composite
  in $\int^{C \in \smset} \smctx{C}$ is given by
  $$\lens{f^{\ast}g_{\flat} \circ f_{\flat}}{g \circ f}.$$
  Well, the bottom is all good, what about the top? Expanding the definition, we
  see that $f^{\ast}g_{\flat} = g_{\flat} \circ (f \times B^-)$, so that
  $f^{\ast}g_{\flat} \circ f_{\flat}$ is the map given by
  $$(a^+, a^-) \mapsto g_{\flat}(f(a^+), f_{\flat}(a^+, a^-)),$$
  which is precisely how we defined composition of charts!
\end{proof}

  
The only difference between the category of charts and the category of lenses is
that for lenses, the maps on top go backwards.
\begin{proposition}\label{prop.lenses_as_groth_construction}
  The category $\Cat{Lens}$ of lenses in the category of sets is the
  Grothendieck construction of the indexed category of \emph{opposites} of the
  categories of sets and functions with context:
  $$\Cat{Lens} = \int^{C \in \smset} \smctx{C}\op.$$
\end{proposition}
\begin{proof}
As with \cref{prop.charts_as_groth_construction}, we will simply expand the
definition of the right hand side, and see that it is precisely the category of lenses.

The objects of $\int^{C \in \smset} \smctx{C}\op$ are pairs $\lens{A^-}{A^+}$ of
sets. All good so far.

A map in $\int^{C \in \smset} \smctx{C}\op$ is a pair $\lens{f^{\sharp}}{f}$
with $f : A^+ \to B^+$ and $f^{\sharp} : A^- \ctxto f^{\ast} B^-$ in
$\smctx{A^+}\op$. Now, $f^{\ast}B^- = B^-$ so $f^{\sharp}$ has signature $A^-
\ctxto B^-$ in $\smctx{A^+}\op$, which means $f^{\sharp}$ has signature $B^-
\ctxto A^{-}$ in $\smctx{A^+}$, which means that $f^{\sharp}$ is a really a
function $A^+ \times B^- \to A^{-}$. In other words, a map in $\int^{C \in \smset}
\smctx{C}\op$ is precisely a lens. We note that the identity map is the identity lens.

Finally, we need to check that composition in $\int^{C \in \smset} \smctx{C}\op$
is lens composition. Suppose that $\lens{f^{\sharp}}{f} :
  \lens{A^-}{A^+} \fromto \lens{B^-}{B^+}$ and $\lens{g^{\sharp}}{g} :
  \lens{B^-}{B^+} \from \lens{C^-}{C^+}$ are lenses. In $\int^{C \in \smset}
  \smctx{C}\op$, their composite is
$$\lens{f^{\ast}g^{\sharp} \circ f^{\sharp}}{g \circ f}.$$
  The bottom is all good, we just need to check that the top --- which,
  remember, lives in $\smctx{A^+}\op$ --- is correct. Since the composite up top
  is in
  the opposite, we are really calculating $f^{\sharp} \circ f^{\ast} g^{\sharp}$
  in $\smctx{A^+}$. By definition, this is
$$(a^+, c^-) \mapsto f^{\sharp}(a^+, g^{\sharp}(f(a^+), c^-))$$
which is precisely their composite as lenses!
\end{proof}

\begin{exercise}\label{ex.really_understand_charts_as_groth_construction}
  Make sure you \emph{really} understand \cref{prop.charts_as_groth_construction,prop.lenses_as_groth_construction}.
\end{exercise}

Inspired by these two propositions, we will make the following general
definitions.
\begin{definition}
 Let $\cat{A} : \cat{C}\op \to \Cat{Cat}$ be an indexed category.
 \begin{enumerate}
   \item The category of $\cat{A}$-charts is the Grothendieck construction of
     $\cat{A}$:
$$\Cat{Chart}_{\cat{A}} = \int^{C \in \cat{C}} \cat{A}(C).$$ 
\label{def.chart_general}
   \item The category of $\cat{A}$-lenses is the Grothendieck construction of
     $\cat{A}\op$:
$$\Cat{Lens}_{\cat{A}} = \int^{C \in \cat{C}} \cat{A}(C)\op.$$
\label{def.lens_general}
 \end{enumerate}
\end{definition}

\paragraph{Sections of indexed categories.}

Any function $f : A \to B$ gives rise to a chart $\lens{f \circ \pi_2}{f} :
\lens{A}{A} \tto \lens{B}{B}$ by simply ignoring the context $A$. This gives us
a functor $\smset \to \Cat{Chart}$ which we call a \emph{section} of the
indexed category $\smctx{-}$. 
\begin{definition}\label{def.section_of_indexed_category}
Let $\cat{A} : \cat{C}\op \to \Cat{Cat}$ be an indexed category. A \emph{section} $T$ of
$\cat{A}$ is a functor $\lens{T-}{-} : \cat{C} \to \int^{C \in
  \cat{C}}\cat{A}(C)$ so that the bottom component of the image $\lens{Tf}{f}$
of any map $f$ in $\cat{C}$ is $f$ itself.
\end{definition}

\begin{proposition}\label{prop.section_charts_discrete}
The functor $\lens{-\circ \pi_2}{-} : \smset \to \Cat{Chart}$ is a section of
the indexed category $\smctx{-}$.
\end{proposition}
\begin{proof}
By definition, the bottom component of $\lens{f \circ \pi_2}{f}$ is $f$, so we
are really just checking that this is a functor. We note that it sends
identities to identities.

Let $f :A \to B$ and $g : B \to C$ be functions. We need to show that
$$\lens{g \circ \pi_2}{g} \circ \lens{f \circ \pi_2}{f} = \lens{(g \circ f)
  \circ \pi_2}{g \circ f}.$$ 
The chart composite of $g \circ \pi_2$ with $f \circ \pi_2$ sends $(a, a')$ to
$g(\pi_2(f(a), f(\pi_2(a, a'))))$, which we can quickly see equals $g(f(a'))$ which
is $(g \circ f)(\pi_2(a, a'))$.
\end{proof}

\begin{remark}
  Sections of indexed categories are very useful in the theory of dynamical
  systems because they give us a notion of ``changes possible in a given
  state''. The section of \cref{prop.section_charts_deterministic} is a way of
  telling us that in a deterministic system, a system can transition to any
  state from any state. We'll see more of this point of view when we formally
  define dynamical system doctrines in \cref{chapter.2}.
\end{remark}


\paragraph{Pure and cartesian maps.}\label{sec.pure_and_cartesian_maps}

A map in a Grothendieck construction is a pair $\lens{f_{\flat}}{f} :
\lens{A}{C} \tto \lens{A'}{C'}$ of maps $f : C \to C'$ and $f_{\flat}: A \to
f^{\ast}A'$. It is not too hard to see that a map is an isomorphism in a Grothendieck
construction if and only if both its constituent maps are isomorphisms in their
respective categories.

\begin{proposition}\label{prop.isomorphism_in_groth_construction}
Let $\cat{A} : \cat{C}\op \to \Cat{Cat}$ be an indexed category and let $\lens{f_{\flat}}{f} :
\lens{A}{C} \tto \lens{A'}{C'}$ be a map in its Grothendieck construction. Then
$\lens{f_{\flat}}{f}$ is an isomorphism if and only if $f$ is an isomorphism in
$\cat{C}$ and $f_{\flat}$ is an isomorphism in $\cat{A}(C)$.
\end{proposition}
\begin{proof}
First, let's show that if both $f$ and $f_{\flat}$ are isomorphisms, then
$\lens{f_{\flat}}{f}$ is an isomorphism. We then have $f\inv : C' \to C$ and
$f_{\flat}\inv : f^{\ast}A' \to A$. From $f_{\flat}\inv$, we can form
$(f\inv)^{\ast} (f_{\flat}\inv) : (f\inv)^{\ast}f^{\ast} A' \to
(f\inv)^{\ast}A$, which we can pre-compose with some of the coherences to have
the signature $A' \to (f\inv)^{\ast} A$:
$$A' \xequals{\,} (f \circ f\inv)^{\ast} A' \xequals{\,} 
(f\inv)^{\ast} f^{\ast} A' \xto{(f\inv)^{\ast} (f_{\flat}\inv)} (f\inv)^{\ast} A.$$
Now, consider the map $\lens{(f\inv)^{\ast}f_{\flat}\inv}{f\inv} : \lens{A'}{C'}
\tto \lens{A}{C}$. We'll show that this is an
inverse to $\lens{f_{\flat}}{f}$. Certainly, the bottom components will work
out; we just need to worry about the top. That is, we need to show that
$f^{\ast}((f\inv)^{\ast} f_{\flat}\inv) \circ f_{\flat} = \id$ and
$(f\inv)^{\ast}(f_{\flat}) \circ (f\inv)^{\ast}(f_{\flat}\inv) = \id$. Both of
these follow quickly by functoriality.

On the other hand, suppose that $\lens{f_{\flat}}{f}$ is an isomorphism with
inverse $\lens{g_{\flat}}{g}$. Then $gf = \id$ and $fg = \id$, so $f$ is an
isomorphism. We can focus on $f_{\flat}$. We know that $f^{\ast}g_{\flat} \circ
f_{\flat} = \id$ and $g^{\ast}f_{\flat} \circ g_{\flat} = \id$. Applying
$f^{\ast}$ to the second equation, we find that $f_{\flat} \circ
f^{\ast}g_{\flat} = \id$, so that $f_{\flat}$ is an isomorphism with inverse $f^{\ast}g_{\flat}$.
\end{proof}  

\begin{remark}
  \cref{prop.isomorphism_in_groth_construction} gives a general solution to
  \cref{ex.isomorphism_in_category_of_charts}, since the category of charts is a
  Grothendieck construction.
\end{remark}

This proposition suggests two interesting classes of maps in a Grothendieck
construction: the maps $\lens{f_{\flat}}{f}$ for which $f$ is an isomorphism, and
those for which $f_{\flat}$ is an isomorphism.
\begin{definition}\label{def.pure_and_cartesian}
Let $\cat{A} : \cat{C}\op \to \Cat{Cat}$ be an indexed category and let
$\lens{f_{\flat}}{f}$ be a map in its Grothendieck construction. We say that
$\lens{f_{\flat}}{f}$ is
\begin{itemize}
\item \emph{pure} if $f$ is an isomorphism, and
 \item \emph{cartesian} if $f_{\flat}$ is an isomorphism.
\end{itemize}
\end{definition}

The pure maps correspond essentially to the maps in the categories $\cat{A}(C)$
at a given index $C$, while the cartesian maps correspond essentially to the
maps in $\cat{C}$.

\begin{remark}
  The name ``pure'' is non-standard. The usual name is ``vertical''. But we are
  about to talk about ``vertical'' maps in a technical sense when we come to
  double categories, so we've renamed the concept here to avoid confusion later.
\end{remark}

\begin{example}
  We have often seen systems that expose their entire state, like $\Sys{Time}$ 
  of \cref{ex.trajectory_as_behavior_discrete}. Considered as lenses, these are
  \emph{pure} in the sense that their $\expose{}$ function is an isomorphism.
\end{example}

\begin{exercise}\label{ex.2-of-3_for_pure_cartesian}
  Let $\lens{f_{\flat}}{f}$ and $\lens{g_{\flat}}{g}$ be composable maps in a Grothendieck construction,
  \begin{enumerate}
    \item Suppose that $\lens{g_{\flat}}{g}$ is cartesian.  Show that
      $\lens{f_{\flat}}{f}$ is cartesian if and only if their composite is
      cartesian. Is the same true for pure maps?
    \item Suppose that $\lens{f_{\flat}}{f}$ is pure. Show that
      $\lens{g_{\flat}}{g}$ is pure if and only if their composite is pure. Is
      the same true for cartesian maps?
  \end{enumerate}
\end{exercise}


%---- Section ----%
\section{Arranging categories along two kinds of composition: Doubly indexed categories}
\label{sec.indexed_double_category_of_systems}

While we described a category of systems and behaviors in
\cref{prop.category_of_systems_discrete}, we haven't been thinking of systems in
quite this way. We have been organizing our systems a bit more particularly than
just throwing them into one large category. We've made the following observations:
\begin{itemize}
  \item Each system has an interface, and many different systems can have the
    same interface. From this observation, we defined the categories
    $\Cat{Sys}\lens{I}{O}$ of systems with the interface $\lens{I}{O}$ in \cref{def.cat_of_systems_discrete}.
  \item Every wiring diagram, or more generally lens, gives us an operation that
    changes the interface of a system by wiring things together. We formalized
    this observation into a functor $\lens{w^{\sharp}}{w} : \Cat{Sys}\lens{I}{O}
    \to \Cat{Sys}\lens{I'}{O'}$ in \cref{prop.lens_comp_functor_discrete}.
  \item To describe the behavior of a system, first we have to chart out how it
    will look on its interface. We formalized this observation by giving a
    profunctor $\Cat{Sys}\lens{f_{\flat}}{f} : \Cat{Sys}\lens{I}{O} \tickar
    \Cat{Sys}\lens{I'}{O'}$ for each chart in \cref{ex.profunctor_from_chart}.
  \item If we wire together a chart for one interface into a chart for the wired
    interface, then every behavior for that chart gives rise to a behavior for
    the wired together chart. We formalized this observation as a morphism of
    profunctors 
\[
\Cat{Sys}(\alpha) : \Cat{Sys}\lens{f_{\flat}}{f} \to
\Cat{Sys}\lens{g_{\flat}}{g}\left( \Cat{Sys}\littlelens{j^{\sharp}}{j}, \Cat{Sys}\littlelens{k^{\sharp}}{k} \right)
\]
in \cref{ex.prof_square_from_arena_square}.
\end{itemize}

Now comes the time to organize all these observations. In this section, we will
see that collectively, these observations are telling us that there is an
\emph{doubly indexed category} of dynamical systems. We will also see that
matrices of sets give rise to a doubly indexed category which we will call the
doubly indexed category of vectors of sets.

\begin{definition}\label{def.doubly_indexed_category}
A \emph{doubly indexed category} $\cat{A} : \cat{D} \to \Cat{Cat}$ consists of
the following:\footnote{This is what an expert would call a \emph{unital (or
    normal) lax double functor}, but we won't need this concept in any other
  setting.} 
\begin{itemize}
  \item A double category $\cat{D}$ called the \emph{indexing base}.
  \item For every object $D \in \cat{D}$, we have a category $\cat{A}(D)$.
  \item For every vertical arrow $j : D \to D'$, we have a functor $\cat{A}(j) :
    \cat{A}(D)
    \to \cat{A}(D')$.
  \item For every horizontal arrow $f : D \to D'$, we have a profunctor
$\cat{A}(f) : \cat{A}(D) \tickar \cat{A}(D')$.
  \item For every square 
\[
        \begin{tikzcd}[sep=tiny]
          A \ar[dd, "j"'] \ar[rr, "f"] & & B \ar[dd, "k"] \\
           & \alpha & \\
          C \ar[rr, "g"'] & & D
        \end{tikzcd}
\]
in $\cat{D}$, a square
\[
        \begin{tikzcd}[sep=tiny]
          \cat{A}(A) \ar[dd, "\cat{A}(j)"'] \ar[rr, "\cat{A}(f)"] & & \cat{A}(B) \ar[dd, "\cat{A}(k)"] \\
           & \cat{A}( \alpha ) & \\
          \cat{A}( C ) \ar[rr, "\cat{A}( g )"'] & & \cat{A}(D)
        \end{tikzcd}
\]
in $\Cat{Cat}$.
\item For any two horizontal maps $f : A \to B$ and $g : B \to E$ in
  $\cat{D}$, we have a square $\mu_{f, g} : \cat{A}(f) \odot \cat{A}(g) \to
  \cat{A}(f \mid g)$ called the \emph{compositor}:
\begin{equation}\label{eqn.compositor}
        \begin{tikzcd}[sep=tiny]
          \cat{A}(A) \ar[dd, equals] \ar[r, "\cat{A}(f)"] & \cat{A}(B) \ar[r, "\cat{A}(f)"] & \cat{A}(E) \ar[dd, equals] \\
           & \mu_{f, g} & \\
          \cat{A}( C ) \ar[rr, "\cat{A}(f \mid g)"'] & & \cat{A}(F)
        \end{tikzcd}
\end{equation}
\end{itemize}
This data is required to satisfy the following laws:
\begin{itemize}
  \item (Vertical Functoriality) For vertical maps $j : D \to D'$ and $k : D' \to D''$, we have
    that $$\cat{A}(k \circ j) = \cat{A}(k) \circ \cat{A}(j)$$
and that $\cat{A}(\id_D) = \id_{\cat{A}(D)}$.\footnote{Here, we are hiding some
  coherence issues. While our doubly indexed category of deterministic systems
  will satisfy this functoriality condition on the nose, we will soon see a
  doubly indexed category of matrices of sets for which this law only holds up
  to a coherence isomorphism. Again, the issue invovles shuffling parentheses
  around, and we will sweep it under the rug.}
  \item (Horizontal Lax Functoriality) For horizontal maps $f : D_1 \to D_2$, $g :
    D_2 \to D_3$ and $h : D_3 \to D_4$, the compositors $\mu$ satisfy the
    following associativity and unitality conditions:
\begin{itemize}
\item (Associativity) $$\frac{\mu_{f, g} | \cat{A}(h)}{\mu_{g \circ f, h}} =
  \frac{\cat{A}(f) | \mu_{g, h}}{\mu_{f, h \circ g}}.$$
\item (Unitality) The profunctor $\cat{A}(\id_{D_1}) : \cat{A}(D_1) \tickar
  \cat{A}(D_1)$ is the identity profunctor, $\cat{A}(\id_{D_1}) = \cat{A}(D_1)$.
  Furthermore, $\mu_{\id_{D_1}, f}$ and $\mu_{f, \id_{D_2}}$ are equal to the
  isomorphisms of \cref{ex.identity_profunctor} given by the naturality of
  $\cat{A}(f)$ on the left and right respectively. We may sumarize this may
  saying that 
$$\mu_{\id, f} = \id_{\cat{A}(f)} = \mu_{f, \id}.$$
\end{itemize}
\item (Naturality of Compositors) For any horizontally composable squares
  $\alpha$ and $\beta$ with bottom horizontal maps $f$ and $g$ respectively, 
\[
\frac{\cat{A}( \alpha ) \mid \cat{A}( \beta )}{\mu_{f, g}} = \frac{\mu_{f} \mid \mu_g}{\cat{A}(\alpha \mid \beta)}.
\] 
\end{itemize}
\end{definition}

That's another big definition! It seems like it will be a slog to actually ever
prove that something is a doubly indexed category. Luckily, in our cases, these
proofs will go quite smoothly. This is because each of the three laws of a
doubly indexed category has a sort of sister law from the definition of a double
category which will help us prove it.

\begin{itemize}
  \item The Vertical Functoriality law will often involve the vertical
    associativity and unitality of squares in the indexing base.
  \item The Horizontal Lax Functoriality law will often involve the horizontal
    associativity and unitality of squares in the indexing base.
  \item The Naturality of Compositors law will often involve the interchange law
    in the indexing base.
\end{itemize}

We'll see how these sisterhoods play out in practice as we define the doubly
indexed categories of deterministic systems and vectors of sets.

\paragraph{The doubly indexed category of deterministic systems}

Let's show that deterministic systems do indeed form a doubly indexed category
$$\Cat{Sys} : \Cat{Arena} \to \Cat{Cat}.$$



\begin{definition}
  The doubly indexed category $\Cat{Sys} : \Cat{Arena} \to \Cat{Cat}$ is defined
  as follows:
\begin{itemize}
\item Our indexing base is the double category $\Cat{Arena}$ of arenas, since we
  will arrange our systems according to their interface.
\item To every arena $\lens{I}{O}$, we associate the category $\Cat{Sys}\lens{I}{O}$
of systems with interface $\lens{I}{O}$ and behaviors whose chart is the
identity chart on $\lens{I}{O}$ (\cref{def.cat_of_systems_discrete}).
\item To every lens $\lens{w^{\sharp}}{w} : \lens{I}{O} \fromto \lens{I'}{O'}$, we associate the functor
$\Cat{Sys}\lens{w^{\sharp}}{w} : \Cat{Sys}\lens{I}{O} \to \Cat{Sys}\lens{I}{O}$ given by wiring according to
$\lens{w^{\sharp}}{w}$:
$$\Cat{Sys}\lens{w^{\sharp}}{w}(\Sys{S}) = \frac{\Sys{S}}{\littlelens{w^{\sharp}}{w}}.$$
\item To every chart $\lens{f_{\flat}}{f} : \lens{I}{O} \tto \lens{I'}{O'}$, we
  associate the profunctor $\Cat{Sys}\lens{f_{\flat}}{f} : \Cat{Sys}\lens{I}{O}
  \tickar \Cat{Sys}\lens{I'}{O'}$ which sends the $\lens{I}{O}$-system $\Sys{T}$
  and the $\lens{I'}{O'}$-system $\Sys{S}$ to the set of behaviors $\Sys{T} \to
  \Sys{S}$ with chart $\lens{f_{\flat}}{f}$:
\begin{align*}
  \Cat{Sys}\lens{f_{\flat}}{f}(\Sys{T}, \Sys{S}) &= \left\{ \phi : \State{T} \to
                                                   \State{S}\, \middle| \mbox{ $\phi$ is a behavior with chart $\lens{f_{\flat}}{f}$} \right\}\\
  &= \left\{  
    {
    \begin{tikzcd}[ampersand replacement = \&]
      \lens{\State{T}}{\State{T}} \ar[r, dashed, shift left, "\lens{\phi \circ
        \pi_2}{\phi}"] \ar[r, dashed, shift right] \ar[d, shift right,
      "\lens{\update{T}}{\expose{T}}"'] \ar[d, shift left, leftarrow] \&
      \lens{\State{S}}{\State{S}} \ar[d, shift left, leftarrow,
      "\lens{\update{S}}{\expose{S}}"] \ar[d, shift right]\\
      \lens{\In{T}}{\Out{T}} \ar[r, shift right, "\lens{f^{\sharp}}{f}"'] \ar[r,
      shift left] \& \lens{\In{S}}{\Out{S}}
    \end{tikzcd}
                    }
                    \right\}
\end{align*}
We saw this profunctor in \cref{ex.profunctor_from_chart}.
\item To every square $\alpha$, we assign the morphism of profunctors given by
  composing vertically with $\alpha$ in $\Cat{Arena}$:
$$\Cat{Sys}(\alpha)(\phi) = \frac{\phi}{\alpha}.$$
We saw in \cref{ex.prof_square_from_arena_square_naturality} that this was a
natural transformation.
\item The compositor is given by horizontal composition in the double category
  of arenas:
  \begin{align*}
    \mu_{\littlelens{f_{\flat}}{f},\littlelens{g_{\flat}}{g}} : \Cat{Sys}\littlelens{f_{\flat}}{f} \odot \Cat{Sys}\littlelens{g_{\flat}}{g} &\to \Cat{Sys}\left( \littlelens{f_{\flat}}{f} \then \littlelens{g_{\flat}}{g} \right) \\
(\phi, \psi) &\mapsto \phi \mid \psi
  \end{align*}
\end{itemize}
\end{definition}

Let's check now that this does indeed satisfy the laws of a doubly indexed
category. The task may appear to loom over us; there are quite a few laws, and
there is a lot of data involved. But nicely, they all follow quickly from a bit of fiddling in the double
category of arenas.
\begin{itemize}
  \item (Vertical Functoriality) We show that $\Cat{Sys}\left(
      \lens{k^{\sharp}}{k} \circ \lens{j^{\sharp}}{j} \right) =
    \Cat{Sys}\lens{k^{\sharp}}{k} \circ \Cat{Sys}\lens{j^{\sharp}}{j}$ by
    vertical associativity:
\begin{align*}
  \Cat{Sys}\left(\lens{k^{\sharp}}{k} \circ \lens{j^{\sharp}}{j} \right)(\phi) &= \frac{\phi}{\left( \frac{\littlelens{j^{\sharp}}{j}}{\littlelens{k^{\sharp}}{k}} \right)} 
= \frac{\left( \frac{\phi}{\littlelens{j^{\sharp}}{j}} \right)}{\littlelens{k^{\sharp}}{k}} \\
&= \Cat{Sys}\lens{k^{\sharp}}{k} \circ \Cat{Sys}\lens{j^{\sharp}}{j}(\phi).
\end{align*}

\item (Horizontal Lax Functoriality) This law follows from horizontal
  associativity in $\Cat{Arena}$.
\begin{align}
  \mu(\mu(\phi, \psi), \xi) = (\phi \mid \psi ) \mid \xi = \phi \mid (\psi \mid \xi) = \mu(\phi, \mu(\psi, \xi)).
\end{align}
\item (Naturality of Compositor) This law follows from interchange in
  $\Cat{Arena}$.
\begin{align*}
  \left( \frac{\Cat{Sys}(\alpha) \mid \Cat{Sys}(\beta)}{\mu} \right)(\phi, \psi) &= \left. \frac{\phi}{\alpha} \middle| \frac{\psi}{\beta} \right. 
= \frac{\phi \mid \psi}{\alpha \mid \beta} \\
&= \left(  \frac{\mu}{\Cat{Sys}(\alpha \mid \beta)}\right)(\phi, \psi).
\end{align*}
\end{itemize}


\paragraph{The doubly indexed category of vectors of sets}

In addition to our doubly indexed category of systems, we have a doubly indexed
category of ``vectors of sets''. 

Classically, an $m \times n$ matrix $M$ can act on a vector $v$ of length $n$ by multiplication to get
another vector $Mv$ of length $m$. We can generalize this to matrices of sets if
we define a vector of sets of length $A$ to be a dependent set $V : A \to
\smset$. 
\begin{definition}\label{def.linear_functor}
  For a set $A$, we define the category of \emph{vectors of sets of length} $A$
  to be $$\Cat{Vec}(A) \coloneqq \smset^A$$
the category of sets depending on $A$. 

Given a $(B \times A)$-matrix $M$, we can treat a $A$-vector $V$ as a $A \times
\ord{1}$ matrix and form the $B \times \ord{1}$ matrix $MV$. This gives us a
functor
\begin{align*}
\Cat{Vec}(M) : \Cat{Vec}(A) &\to \Cat{Vec}(B)\\
               V &\mapsto (MV)_b = \sum_{a \in A} M_{ba} \times V_a \\
           f : V \to W &\mapsto ( (a, m, v) \mapsto (a, m, f(v)) )
\end{align*}
which we refer to as the linear functor given by $M$.
\end{definition}

\begin{definition}
The doubly indexed category $\Cat{Vec} : \Cat{Matrix} \to \Cat{Cat}$ of vectors
of sets is defined by:
\begin{itemize}
  \item Its indexing base is the double category of matrices of sets.
  \item To every set $A$, we assign the category $\Cat{Vec}(A) = \smset^A$ of
    vectors of length $A$.
  \item To every $(B \times A)$-matrix $M : A \to B$, we assign the linear
    functor $\Cat{Vec}(M) : \Cat{Vec}(A) \to \Cat{Vec}(B)$ given by $M$ (\cref{def.linear_functor}).
  \item To every function $f : A \to B$, we associate the profunctor
    $\Cat{Vec}(f) : \Cat{Vec}(A) \tickar \Cat{Vec}(B)$ defined by
$$\Cat{Vec}(f)(V, W) = \{ F : (a \in A) \to V_a \to W_{f(a)} \}.$$
That is, $F \in \Cat{Vec}(f)(V, W)$ is a family of functions $F(a,-) : V_a \to
W_{f(a)}$ indexed by $a \in A$. This is natural by index-wise composition.
\item To every square
  \[
        \begin{tikzcd}[sep=tiny]
          A \ar[dd, "M"'] \ar[rr, "f"] & & B \ar[dd, "N"] \\
           & \alpha & \\
          C \ar[rr, "g"'] & & D
        \end{tikzcd}
  \]
  that is, family of functions $\alpha_{ca} : M_{ca} \to N_{g(c)f(a)}$, we
  associate the square
  \[
        \begin{tikzcd}[sep=tiny]
          \Cat{Vec}( A ) \ar[dd, "\Cat{Vec}(M)"'] \ar[rr, "\Cat{Vec}(f)"] & & \Cat{Vec}( B ) \ar[dd, "\Cat{Vec}(N)"] \\
           & \Cat{Vec}(\alpha) & \\
          \Cat{Vec}(C) \ar[rr, "\Cat{Vec}(g)"'] & & \Cat{Vec}(D)
        \end{tikzcd}
  \]
  defined by sending a family of functions $F : (a \in A) \to V_{a} \to
  W_{f(a)}$ in $\Cat{Vec}(f)(V, W)$ to the family 
  \begin{align*}
\Cat{Vec}(\alpha)(F) : (c \in C) \to MV_c &\to MW_{g(c)} \\ 
  \Cat{Vec}(\alpha)(F)(c, (a, m, v)) &= (f(a), \alpha(m), F(a, v))
  \end{align*}
  That is, $\Cat{Vec}(\alpha)(F)(c, -)$ takes an element $(a, m, v) \in MV_{c} =
  \sum_{a \in A} M_{ca} \times V_a$ and gives the elements $(f(a), \alpha(m),
  F(a, v))$ of $MW_{g(c)} = \sum_{b \in B} N_{g(c)b} \times W_b$.
\item The compositor is given by componentwise composition: If $f : A \to B$ and
  $g : B \to C$ and $F \in \Cat{Vec}(f)(V, W)$ and $G \in \Cat{Vec}(g)(W, U)$,
  then 
\begin{align*}
  \mu_{f, g}(F, G) : (a \in A) \to V_{a} &\to U_{gf(a)} \\
  \mu_{f, g}(F, G)(a, v) &\coloneqq G(f(a), F(a, v)).
\end{align*}
\end{itemize}
\end{definition}

It might seem like it will turn out to be a big hassle to show that this
definition satisfies all the laws of a doubly indexed category. Like with the
doubly indexed category of arenas, we will find that all the laws follow for
matrices by fiddling around in the double category of matrices.

Let's first rephrase the above definition in terms of the category of matrices.
We note that a vector of sets $V \in \Cat{Vec}(A)$ is equivalently a matrix $V :
\ord{1} \to A$. Then the linear functor $\Cat{Vec}(M) : \Cat{Vec}(A) \to
\Cat{Vec}(B)$ is given by matrix multiplication, or in double category notation:
$$\Cat{Vec}(M)(V) = \frac{V}{M}.$$
This means that the Vertical Functoriality law follows by vertical associativity
in the double category of matrices, which is to say associativity of matrix
multiplication.

Similarly, we can interpret the profunctor $\Cat{Vec}(f)$ for $f : A \to B$ in
terms of the double category $\Cat{Matrix}$. An element $F \in \Cat{Vec}(f)(V, W)$ is
equivalently a square of the following form in $\Cat{Matrix}$:
\[
        \begin{tikzcd}[sep=tiny]
          \ord{1} \ar[dd, "V"'] \ar[rr, equals] & & \ord{1} \ar[dd, "W"] \\
           & F & \\
          A \ar[rr, "f"'] & & B
        \end{tikzcd}
      \]
      Therefore, we can describe $\Cat{Vec}(f)(V, W)$ as the following set:
\[
\Cat{Vec}(f)(V, W) = \left\{ F \,\middle|
        \begin{tikzcd}[sep=tiny]
          \ord{1} \ar[dd, "V"'] \ar[rr, equals] & & \ord{1} \ar[dd, "W"] \\
           & F & \\
          A \ar[rr, "f"'] & & B
        \end{tikzcd}
  \right\}
\]
Then the Horizontal Lax Functoriality laws follow from associativity and unitality of
horizontal composition of squares in $\Cat{Matrix}$! 


Finally, we need to interpret the rather fiddly transformation
$\Cat{Vec}(\alpha)$ in terms of the double category of matrices. Its a matter of
unfolding the definitions to see that
$\Cat{Vec}(\alpha)(F) = \frac{F}{\alpha}$
in $\Cat{Matrix}$, and therefore that the Naturality of Compositors law follows
by the interchange law.

\begin{remark}
  If this argument seemed wholly too similar to the one we gave for the doubly
  indexed category of systems, your suspicions are not misplaced. We will see in
  \cref{chapter.2} that both are instances of a very general \emph{vertical
    slice construction}.
\end{remark}

%---- Section ----%
\section{The big theorem: representable doubly indexed functors}

We have now introduced all the characters in our play: the double categories of
arenas and matrices, and doubly indexed categories of systems and vectors. In
this section, we will put the plot in motion. 

In \cref{sec.steady_states_matrix_arithmetic}, we saw that the steady states of
dynamical systems with interface $\lens{I}{O}$ compose like an $I \times O$
matrix. We proved a few propositions to this effect, namely
\cref{prop.steady_state_matrix_parallel_tensor} and
\cref{prop.steady_state_matrix_trace}, but we didn't precisely mark out the
scope of these results, or describe the full range of laws that are satisfied.

In this section, we will generalize the results of that section to \emph{all
  behaviors} of systems, not just steady states. We will precisely state all the
ways that behaviors can be composed by systems, and we will give a condition on
the kinds of behaviors for which we can calculate the behavior of a wired
together system entirely from the behavior of its component systems. All of this will be organized into a \emph{doubly indexed functor}
$\Fun{Behave}_{\Sys{T}} : \Cat{Sys} \to \Cat{Vec}$
which will send a system $\Sys{S}$ to its set of
$\Sys{T}$-shaped behaviors. 


\subsection{Turning lenses into matrices: Double Functors}

In \cref{sec.steady_states_matrix_arithmetic}, we saw how we could re-interpret
a wiring diagram as a schematic for multiplying, tensoring, and tracing
matrices. At the very end, in \cref{prop.trace_multiplying_by_matrix}, we saw
that we can take the trace $\fun{tr}_A M$ of a $(A \times C) \times (A \times B)$-matrix $M$ by
considering it as a $(A \times C) \times (B \times C)$ length vector and then
multiplying it by a big but very sparse $(C \times B) \times ((A \times C) \times (B \times
C))$-matrix $\Fun{Tr}^A$. Taking the trace of a matrix corresponded to the
wiring diagram
\[
\begin{tikzpicture}[oriented WD, every fit/.style={inner xsep=\bbx, inner ysep=\bby}, bbx = .3cm, bby =.3cm, bb min width=.5cm, bb port length=2pt, bb port sep=1, baseline=(S'.center)]
	\node[bb={2}{2}, fill=blue!10, dashed] (S) {};

  \node[bb={0}{0}, fit={($(S.north east) + (1,1)$) ($(S.south west) - (1,0)$)}] (S') {};
  
  \draw (S_out2) to (S_out2-|S'.east);
  \draw (S_in2-|S'.west) to (S_in2);

  \draw let \p1=(S.north east), \p2=(S.north west), \n1={\y2+\bby}, \n2=\bbportlen in    (S_out1) to[in=0] (\x1 +\n2, \n1) -- (\x2-\n2, \n1) to[out=180] (S_in1);
\end{tikzpicture}
\]
In this section, we will see a general formula for taking an arbitrary lens and
turning it into a matrix. Mutliplying by the matrix will then correspond to
wiring according to that lens.

This process of turning a lens into a matrix will give us a functor $\Cat{Lens}
\to \Cat{Matrix}$ from the category of lenses to the category of matrices of
sets.

The resulting matrices will have entries that are either $\ord{1}$ or
$\emptyset$; we can think of this as telling us whether ($\ord{1}$) or not ($\emptyset$) the two charts
are to be wired together. As we saw in
\cref{ex.understanding_squares_in_double_cat_of_arenas}, we can see a square in the double
category of arenas as telling us whether how a chart can be wired together along
a lens into another chart. Therefore, we will take the entries of our matrices
to be the sets of appropriate squares in arena --- but there is either a single
square (if the appropriate equations hold) or no square (if they don't), so we
will end up with a matrix whose entries either have a single element or are empty.


\begin{proposition}\label{prop.lens_to_matrix_functor_discrete}
  For any arena $\lens{I}{O}$, there is a functor 
$$\Cat{Chart}\left( \littlelens{I}{O},\, - \right) : \Cat{Lens} \to \Cat{Matrix}$$
from the category of lenses to the category of matrices of sets which sends an
arena $\lens{A^-}{A^+}$ to the set $\Cat{Chart}\left( \lens{I}{O},
  \lens{A^-}{A^+} \right)$ of charts from $\lens{I}{O}$ to $\lens{A^-}{A^+}$,
and which sends a lens $\lens{w^{\sharp}}{w} : \lens{A^-}{A^+} \fromto
\lens{B^-}{B^+}$ to the $\Cat{Chart}\left( \lens{I}{O}, \lens{B^-}{B^+} \right)
\times \Cat{Chart}\left( \lens{I}{O}, \lens{A^-}{A} \right)$ matrix of sets
\begin{align*}
 \Cat{Chart}\left( \littlelens{I}{O}, \littlelens{w^{\sharp}}{w} \right) &: \Cat{Chart}\left( \littlelens{I}{O}, \littlelens{B^-}{B^+} \right)
\times \Cat{Chart}\left( \littlelens{I}{O}, \littlelens{A^-}{A} \right) \to \smset \\
\left( \littlelens{f_{\flat}}{f}, \littlelens{g_{\flat}}{g} \right) &\mapsto \left\{ \mbox{ The set of squares }
      \begin{tikzcd}[ampersand replacement = \&]
        \lens{I}{O} \ar[r, shift left, "\lens{f_{\flat}}{f}"] \ar[r, shift
        right] \ar[d, shift right, equals] \ar[d, shift left,
        leftarrow, equals] \& \lens{A^-}{A^+} \ar[d, shift left, leftarrow,
        "\lens{w^{\sharp}}{w}"] \ar[d, shift right]\\
        \lens{I}{O} \ar[r, shift right, "\lens{g^{\sharp}}{g}"'] \ar[r,
        shift left] \& \lens{B^-}{B^+}
      \end{tikzcd}  \mbox{ in $\Cat{Arena}$}   \right\} \\
&=\begin{cases} \ord{1} &\mbox{ if \(\begin{cases} g(o) = w(f(o)) &\mbox{ for all $o \in O$,}\\ f_{\flat}(i, o) = w^{\sharp}(f(o), g_{\flat}(i, o)) &\mbox{for all $i \in I$ and $o \in O$.}\end{cases}\)} \\ \emptyset &\mbox{otherwise} \end{cases}
\end{align*}
\end{proposition}
\begin{proof}
By vertical composition of squares,
\[
  \begin{tikzcd}
    \lens{I}{O} \ar[r, shift left, "\lens{f_{\flat}}{f}"] \ar[r, shift right] \ar[d, shift right,
    equals] \ar[d, shift left, equals] &
    \lens{A^-}{A^+} \ar[d, shift left, leftarrow,
    "\lens{w^{\sharp}}{w}"] \ar[d, shift right]\\
    \lens{I}{O} \ar[d, shift right, equals] \ar[d, shift left,
        equals] \ar[r, shift right, "\lens{g_{\flat}}{g}"']
    \ar[r, shift left] & \lens{B^-}{B^+} \ar[d, shift left, leftarrow,
        "\lens{v^{\sharp}}{v}"] \ar[d, shift right]\\
    \lens{I}{O} \ar[r, shift right, "\lens{h_{\flat}}{h}"']
    \ar[r, shift left] & \lens{I'}{O'} 
  \end{tikzcd} \xequals{\quad}
  \begin{tikzcd}
    \lens{I}{O} \ar[r, shift left, "\lens{f_{\flat}}{f}"] \ar[r, shift right] \ar[d, shift right,
    equals] \ar[d, shift left, equals] &
    \lens{A^-}{A^+} \ar[d, shift left, leftarrow,
    "\lens{w^{\sharp}}{w} \then \lens{v^{\sharp}}{v}"] \ar[d, shift right]\\
    \lens{I}{O} \ar[r, shift right, "\lens{h_{\flat}}{h}"']
    \ar[r, shift left] & \lens{I}{O'}
  \end{tikzcd}
\]
there is always a map from the composite of two of these matrices to the matrix
described by the composite. It is not, however, obvious that this map is a
bijection --- which is what we need to prove functoriality.

Suppose we have a square as on the left hand side; let's see that we can factor
it into two squares as on the right hand side. We need to construct the middle
chart $\lens{g_{\flat}}{g} : \lens{I}{O} \tto \lens{B^-}{B^+}$ from
$\lens{f_{\flat}}{f}$ and $\lens{h_{\flat}}{h}$. For the bottom of top square to commute,
we see that $g$ must equal $w \circ f$, so we can define $g \coloneqq w \circ
f$. On the other hand, for the top of the bottom square to commute, we must have that
$g_{\flat}(i, o) = v^{\sharp}(g(o), h_{\flat}(i, o))$; again, we can take this
as a definition. It remains to show that the other half of each square commutes.
For the top of the top square to commute means that
$$f_{\flat}(i, o) = w^{\sharp}(f(o), g_{\flat}(i, o))$$
which we can see holds by
\begin{align*}
  w^{\sharp}(f(o), g_{\flat}(i, o)) &= w^{\sharp}(f(o), v^{\sharp}(g(o), h_{\flat}(i, o))) \\
                                    &= w^{\sharp}(f(o), v^{\sharp}(wf(o), h_{\flat}(i, o))) \\
  &= f_{\flat}(i, o) &\mbox{by the commutativity of the square on the right.}
\end{align*}

On the other hand, to show that the bottom of the bottoms square commutes, we
need that $h =  v \circ g$. But by hypothesis, $h = v \circ w \circ f$, and we
defined $g = w \circ f$.
\end{proof}
%%%%%%%%%%%%%%%%%%%%%%%%%%%%%%%%%%%%%%%%%%%%%%%%%%%%%%%%%
%% Decided to remove this proof, but keeping the text for later, in case.
\iffalse
\begin{proof}
  To prove that this is a functor, it will help to reinterpret the matrix above
  in terms of diagrams in the category of charts. We can crack the lens
  $\lens{w^{\sharp}}{w} : \lens{A^-}{A^+} \fromto \lens{B^-}{B^+}$ into two
  charts: 
\begin{itemize}
  \item $\lens{w^{\sharp}}{\id} : \lens{B^-}{A^+} \tto \lens{A^-}{A+}$, and
  \item $\lens{\pi_2}{w} : \lens{B^-}{A^+} \tto \lens{B^-}{B^+}$.
\end{itemize}
Note that we've cracked the lens into two \emph{charts}. Now, we can interpret
the matrix $\Cat{Chart}\left( \littlelens{I}{O}, \littlelens{w^{\sharp}}{w}
\right)$ as sending charts $\lens{f_{\flat}}{f} : \lens{I}{O} \tto
\lens{A^-}{A^+}$ and $\lens{g_{\flat}}{g} : \lens{I}{O} \tto \lens{B^-}{B^+}$ to
the set of charts $\lens{h_{\flat}}{h} : \lens{I}{O} \tto \lens{B^-}{A^+}$ so
that the following diagram in $\Cat{Arena}$ commutes:
\[
\begin{tikzcd}
  \lens{I}{O} \ar[d, equals, shift left]\ar[d, equals, shift right] \ar[r, shift
  left, "\littlelens{f_{\flat}}{f}"]\ar[r, shift right]& \lens{A^-}{A^+} \ar[d,
  shift left, leftarrow, "\lens{w^{\sharp}}{\id}"] \ar[d, leftarrow, shift right] \\
  \lens{I}{O} \ar[d, equals, shift left]\ar[d, equals, shift right] \ar[r,
  dashed, shift left, "\littlelens{h_{\flat}}{h}"]\ar[r, dashed, shift right]& \lens{B^-}{A^+} \ar[d,
  shift left, "\lens{\pi_2}{w}"] \ar[d, shift right] \\
  \lens{I}{O} \ar[r, shift
  left]\ar[r, shift right, "\littlelens{g_{\flat}}{g}"']& \lens{B^-}{B^+}
\end{tikzcd}
\]
This isn't obvious, so let's take a moment to understand it. First off, remember
that everything in this diagram is a chart; it is \emph{not} a diagram in the
double category of arenas, but in the category $\Cat{Chart}$ of charts. Now, we
can ask what it means for this diagram to commute. First, the bottom part of the
top square must commute: but this just means that $h = f$! Then the bottom part
of the bottom square means that $g = w \circ h$, or $g = w \circ f$. Now, the
top part of the bottom square says that $h_{\flat}(f(i), o) = g_{\flat}(i, o)$,
and the top part of the top square says that 

\end{proof}
\fi
%%%%%%%%%%%%%%%%%%%%%%%%%%%%%%%%%%%%%%%%%%%%%%%%%%%%%%%%%%%%%%%%

\begin{example}
  Let's see what happens when we take the functor
  $\Cat{Chart}\left(  \littlelens{I}{O}, -\right)$ for the arena
  $\littlelens{\ord{1}}{\ord{1}}$. A chart $\lens{a^-}{a^+} :
  \lens{\ord{1}}{\ord{1}} \tto \lens{A^-}{A^+}$ is just a pair of elements $a^-
  \in A^-$ and $a^+ \in A^+$, so
$$\Cat{Chart}\left( \littlelens{\ord{1}}{\ord{1}}, \littlelens{A^-}{A^+} \right) = A^-
\times A^+.$$
Now, if we have a lens $\lens{w^{\sharp}}{w} : \lens{A^-}{A^+} \fromto
\lens{B^-}{B^+}$, we have a square 
\[
    \begin{tikzcd}
      \lens{\ord{1}}{\ord{1}} \ar[r, shift left, "\lens{a^-}{a^+}"] \ar[r, shift
      right] \ar[d, shift right, equals] \ar[d, shift left, equals] &
      \lens{A^-}{A^+} \ar[d, shift left, leftarrow,
      "\lens{w^{\sharp}}{w}"] \ar[d, shift right]\\
      \lens{\ord{1}}{\ord{1}} \ar[r, shift right, "\lens{b^-}{b^+}"'] \ar[r,
      shift left] & \lens{B^-}{B^+}
    \end{tikzcd}
\]
if and only if $w(a^+) = b^+$ and $w^{\sharp}(a^+, b^-) = a^-$. Thinking of
$\lens{w^{\sharp}}{w}$ as a wiring diagram, this would mean that $b^+$ is that
part of $a^+$ which is passed forward on the outgoing wires, and $a^-$ is the
inner input which comes from the inner output $a^+$ and outer input $b^-$.

To take a concrete example, suppose that $\lens{w^{\sharp}}{w}$ were the
following wiring diagarm:
\[
\begin{tikzpicture}[oriented WD, every fit/.style={inner xsep=\bbx, inner ysep=\bby}, bbx = .3cm, bby =.3cm, bb min width=.5cm, bb port length=2pt, bb port sep=1, baseline=(S'.center)]
	\node[bb={2}{2}, fill=blue!10, dashed] (S) {};

  \node[bb={0}{0}, fit={($(S.north east) + (1,1)$) ($(S.south west) - (1,0)$)}] (S') {};
  
  \draw (S_out2) to (S_out2-|S'.east);
  \draw (S_in2-|S'.west) to (S_in2);

  \draw let \p1=(S.north east), \p2=(S.north west), \n1={\y2+\bby}, \n2=\bbportlen in    (S_out1) to[in=0] (\x1 +\n2, \n1) -- (\x2-\n2, \n1) to[out=180] (S_in1);
\end{tikzpicture}
\]
That is, let's take $A^+ = X \times Y$ and $A^- = X \times Z$, and $B^+ = Y$
and $B^- = Z$, and
\begin{align*}
w(x, y) &= y \\
w^{\sharp}((x, y), z) &= (x, z).
\end{align*}
Using the definition above, we can calculate the resulting matrix $\Cat{Chart}\left( \lens{I}{O},
  \lens{w^{\sharp}}{w} \right)$ as having $( ((x, y), (x', z)), (y', z')
)$-entry
\[\begin{cases} \ord{1} &\mbox{if $w(x, y) = y'$ and $w^{\sharp}((x, y), z) =
    (x', z')$} \\ \emptyset &\mbox{otherwise.} \end{cases}
\]
or, by the definition of $\lens{w^{\sharp}}{w}$, 
\[\begin{cases} \ord{1} &\mbox{if $x = x'$, $y = y'$, and $z = z'$} \\ \emptyset
    &\mbox{otherwise.}\end{cases}
\]
  which was the definition of $\Fun{Tr}^{X}$ given in \cref{prop.trace_multiplying_by_matrix}!
\end{example}

\begin{exercise}
Let $\lens{w^{\sharp}}{w} : \lens{A \times B}{B \times C} \fromto \lens{A}{C}$
be the wiring diagram 
\[
\begin{tikzpicture}[oriented WD, bbx = .3cm, bby =.3cm, bb min width=.5cm, bb port length=2pt, bb port sep=1, every fit/.style={inner xsep=\bbx, inner ysep=\bby}
, baseline=(Outer.center)]
  \node[bb={1}{1}, fill=blue!10] (S1) {};
  \node[bb={1}{1}, fill=blue!10, right= of S1] (S2) {};

  \node[bb={1}{1}, fit={(S1) (S2)}] (Outer) {};

  \draw (Outer_in1) to (S1_in1);
  \draw (S1_out1) to (S2_in1);
  \draw (S2_out1) to (Outer_out1);
\end{tikzpicture}
\] 
Calculate the entries of the matrix $\Cat{Chart}\left( \lens{\ord{1}}{\ord{1}}, \lens{w^{\sharp}}{w} \right)$.
\end{exercise}

By the functoriality of \cref{prop.lens_to_matrix_functor_discrete}, we can
calculate the matrix of a big wiring diagram by expressing it in terms of a
series of traces, and mutliplying the resulting matrices together. This means
that the process of multiplying, tensoring, and tracing matrices described by a
wiring diagram is well described by the matrix we constructed in
\cref{prop.lens_to_matrix_functor_discrete}, since we already know that it
interprets the basic wiring diagrams correctly.

 But we are also interested in charts, since we have to chart out our
behaviors. So we will give a \emph{double functor} $\Cat{Arena} \to
\Cat{Matrix}$ that tells us not only how to turn a lens into a matrix, but also
how this operation interacts with charts.

A double functor is, unsurprisingly, a functor between double categories. Just
as a double category has a bit more than twice the information involved in a
category, a double functor has a bit more than twice the information involved in
a functor.
\begin{definition}\label{def.double_functor}
 Let $\cat{D}_1$ and $\cat{D}_2$ be double categories. A \emph{double functor}
 $\Fun{F} : \cat{D}_1 \to \cat{D}_2$ consists of:
\begin{itemize}
  \item An object assignment $F : \Ob \cat{D}_1 \to \Ob \cat{D}_2$ which assigns an object $F D$
    in $\cat{D}_2$ to each object $D$ in $\cat{D}_1$.
  \item A vertical functor $F : v\cat{D}_1 \to v\cat{D}_2$ on the vertical
    categories, which acts the same as the object assignment on objects.
  \item A horizontal functor $F : h\cat{D}_1 \to h\cat{D}_2$ on the horizontal
    categories, which acts the same as the object assignment on objects.
  \item For every square
    \[
        \begin{tikzcd}[sep=tiny]
          A \ar[dd, "j"'] \ar[rr, "f"] & & B \ar[dd, "k"] \\
           & \alpha & \\
          C \ar[rr, "g"'] & & D
        \end{tikzcd}
    \]
    in $\cat{D}_1$, a square
    \[
        \begin{tikzcd}[sep=tiny]
          FA \ar[dd, "Fj"'] \ar[rr, "Ff"] & & FB \ar[dd, "Fk"] \\
           & F\alpha & \\
          FC \ar[rr, "Fg"'] & & FD
        \end{tikzcd}
    \]
    such that the following laws hold:
    \begin{itemize}
    \item $F$ commutes with horizontal compostition: $F(\alpha \mid \beta) = F\alpha \mid F\beta$.
    \item $F$ commutes with vertical comopsition: $F\left( \frac{\alpha}{\beta} \right) = \frac{F\alpha}{F\beta}$.
    \item $F$ sends horizontal identities to horizontal identities, and vertical
      identities to vertical identities.
    \end{itemize}
\end{itemize}
\end{definition}

We are now ready to define the double functor $\Cat{Chart}\left( \lens{I}{O}, -
\right) : \Cat{Arena} \to \Cat{Matrix}$ represented by an arena $\lens{I}{O}$.

\begin{proposition}\label{prop.representable_double_functor}
  There is a double functor
  $$\Cat{Chart}\left( \littlelens{I}{O}, -
\right) : \Cat{Chart} \to \Cat{Matrix}$$
which acts in the following way:
\begin{itemize}
  \item An arena $\lens{A^-}{A^+}$ gets sent to the set $\Cat{Chart}\left(
      \lens{I}{O}, \lens{A^-}{A^+} \right)$ of charts from $\lens{I}{O}$ to $\lens{A^-}{A^+}$.
  \item The vertical functor is $\Cat{Chart}\left( \lens{I}{O}, - \right) :
    \Cat{Lens} \to \Cat{Matrix}$ from \cref{prop.lens_to_matrix_functor_discrete}.
  \item The horizontal functor is the representable functor $\Cat{Chart}\left(
      \lens{I}{O}, - \right) : \Cat{Arena} \to \smset$ which acts on a chart
    $\lens{f_{\flat}}{f} : \lens{A^-}{A^+} \tto \lens{B^-}{B^+}$ by
    post-composition.
  \item To a square
    \[ \alpha = 
      \begin{tikzcd}
        \littlelens{A^-}{A^+} \ar[r, shift left, "\littlelens{f_{\flat}}{f}"] \ar[r, shift
        right] \ar[d, shift right, "\littlelens{j^{\sharp}}{j}"'] \ar[d, shift left,
        leftarrow] & \littlelens{B^-}{B^+} \ar[d, shift left, leftarrow,
        "\littlelens{k^{\sharp}}{k}"] \ar[d, shift right]\\
        \littlelens{C^-}{C^+} \ar[r, shift right, "\littlelens{g^{\sharp}}{g}"'] \ar[r,
        shift left] & \littlelens{D^-}{D^+}
      \end{tikzcd}
    \]
    in the double category of arenas, we give the square
    \[\Cat{Chart}\left(\littlelens{I}{O}, \alpha \right)  =
      \begin{tikzcd}
        \Cat{Chart}\left(\littlelens{I}{O}, \littlelens{A^-}{A^+} \right) \ar[r, "{\Cat{Chart}\left(\littlelens{I}{O}, \littlelens{f_{\flat}}{f} \right)}"]  \ar[d, "{\Cat{Chart}\left(\littlelens{I}{O}, \littlelens{j^{\sharp}}{j} \right)}"'] 
         &  \Cat{Chart}\left(\littlelens{I}{O}, \littlelens{B^-}{B^+} \right)\ar[d,
        "{ \Cat{Chart}\left(\littlelens{I}{O}, \littlelens{k^{\sharp}}{k} \right) }"] \\
   \Cat{Chart}\left(\littlelens{I}{O}, \littlelens{C^-}{C^+} \right)      \ar[r, "{\Cat{Chart}\left(\littlelens{I}{O}, \littlelens{g_{\flat}}{g_{\flat}} \right)}"']  & \Cat{Chart}\left(\littlelens{I}{O}, \littlelens{D^-}{D^+} \right)
      \end{tikzcd}
    \]
    in the double category of matrices defined by horizontal composition of
    squares in $\Cat{Arena}$ (remember that the entries of these matrices are
    sets of squares in $\Cat{Arena}$, even though that means they either have a
    single element or no elements).
    \begin{align*}
    \Cat{Chart}\left(\littlelens{I}{O}, \alpha \right)(\beta) = \beta \mid \alpha.
\end{align*}
\end{itemize}
\end{proposition}
\begin{proof}
 

We can write the double functor $\Cat{Chart}\left(\littlelens{I}{O}, - \right)$
entirely in terms of the double category $\Cat{Arena}$:
\begin{itemize}
  \item It sends an arena $\lens{A^-}{A^+}$ to the set of charts (horizontal
    maps) $\lens{f_{\flat}}{f} : \lens{I}{O}  \tto \lens{A^-}{A^+}$.
  \item It sends a chart $\lens{g_{\flat}}{g}$ to the map $\lens{f_{\flat}}{f}
    \mapsto \left. \lens{f_{\flat}}{f} \middle| \lens{g_{\flat}}{g} \right.$.
  \item It sends a lens $\lens{w^{\sharp}}{w}$ to the set of squares $\beta :
    \lens{I}{O} \to \lens{w^{\sharp}}{w}$, indexed by their top and bottom
    boundaries.
  \item It sends a square $\alpha$ to the map given by horizontal compostion
    $\beta \mapsto \beta \mid \alpha$.
\end{itemize}

We can see that this double functor (let's call it $F$, for short) takes seriously the idea that ``squares are
charts between lenses'' from
\cref{ex.understanding_squares_in_double_cat_of_arenas}. From this description,
and the functoriality of \cref{prop.lens_to_matrix_functor_discrete}, we can see
that the assignments above satisfy the double functor laws. 
\begin{itemize}
  \item Horizontal functoriality follows from horizontal associativity in
    $\Cat{Arena}$:
$$F(\alpha\mid \beta)(\gamma) = \gamma \mid (\alpha \mid \beta) = (\gamma \mid
\alpha) \mid \beta = F(\alpha) \mid F(\beta) (\gamma).$$
\item Vertical functoriality follows straight from the definitions:
  $$F\left( \frac{\alpha}{\beta} \right)(\_, \gamma, \delta) = (\_, \gamma \mid
  \alpha, \delta \mid \beta) = \frac{F(\alpha)(\gamma)}{F(\beta)(\delta)}.$$
\item It's pretty straightforward to check that identities get sent to identities.
\end{itemize}
\end{proof}



\subsection{How behaviors of systems wire together: doubly indexed functors}

We now come to the mountaintop. It's been quite a climb, and we're almost there.

We can now describe all the ways that behaviors of systems get put together when
we wire systems together. There are a bunch of laws governing how behaviors get
put together, and we organize them all into the notion of a \emph{lax doubly
  indexed functor}. To any system $\Sys{T}$, we will give a lax doubly indexed
functor
$$\Fun{Behave}_{\Sys{T}} : \Cat{Sys} \to \Cat{Vec}.$$
Here's what that means.
\begin{definition}\label{def.lax_doubly_indexed_functor}
  Let $\cat{A} : \cat{D}_1 \to \Cat{Cat}$ and $\cat{B} : \cat{D}_2 \to
  \Cat{Cat}$ be doubly indexed categories. A \emph{lax doubly indexed functor}
  $(F^0, F) : \cat{A} \rightharpoonup \cat{B}$ consists of:
\begin{itemize}
  \item A double functor $$F^0 : \cat{D}_1 \to \cat{D}_2.$$
  \item For each object $D \in \cat{D}_1$, a functor $$F^D : \cat{A}(D)
    \to \cat{B}(F^0D).$$
  \item For every vertical map $j : D \to D'$ in $\cat{D}_1$, a natural
    transformation $$F^{j} : \cat{B}(F^0 j) \circ F^D \to F^{D'} \circ \cat{A}(j).$$
    We ask that $F^{\id_{D}} = \id$.
   \item For every horizontal map $f : D \to D'$, a square
\[
\begin{tikzcd}[sep=tiny]
\cat{A}(D) \ar[rr, tick,"\cat{A}(f)"] \ar[dd, "F^D"'] & & \cat{A}(D') \ar[dd, "F^{D'}"]\\
& F^{j} & \\
\cat{B}(F^0D) \ar[rr, tick, "\cat{B}(F^0 f)"'] &  &\cat{B}(F^0 D')
\end{tikzcd}
\]
in $\Cat{Cat}$. We ask that $F^{\id_{D}} = \id$.
\end{itemize}

This data is required to satisfy the following laws:
\begin{itemize}
  \item (Vertical Lax Functoriality) For composable vertical arrows $j : D \to D'$ and $k :
    D' \to D''$, 
    $$F^{\frac{j}{k}} = (F^k\cat{A}(j)) \circ (\cat{B}(k)F^j).$$
  \item (Horizontal functoriality) For composable horizontal arrows $f : D \to
    D'$ and $g : D' \to D''$, 
$$\frac{\mu^{\cat{A}}_{f, g}}{F^{f \mid g}} = \frac{F^f \mid
  F^g}{\mu^{\cat{B}}_{F^0f, F^0g}}.$$ 
\item (Functorial Interchange) For any square
\[
\begin{tikzcd}[sep=tiny]
D_1 \ar[rr, "f"] \ar[dd, "j"'] & & D_2 \ar[dd, "k"] \\
 & \alpha & \\
D_3 \ar[rr, "g"'] & & D_4
\end{tikzcd}
\]
in $\cat{D}_1$, we have that
\[
  \left. F^j \,\middle|\, \frac{\cat{A}(\alpha)}{F^g} \right. = \left.
    \frac{F^f}{\cat{B}(\alpha)} \,\middle|\, F^k \right. .
\]
\end{itemize}
\end{definition}

\begin{theorem}\label{thm.representable_discrete}
  Let $\Sys{T}$ be a deterministic system. There is a lax doubly indexed functor 
$\Fun{Behave}_{\Sys{T}} : \Cat{Sys} \to \Cat{Vec}$
which sends systems to their sets of $\Sys{T}$-shaped behaviors.
\end{theorem}

Let's see what this theorem is really asking for while we construct it. As with
many of the constructions we have been seeing, the hard part is understanding
what we are supposed to be constructing; once we do that, the answer will always
be ``compose in the appropriate way in the appropriate double category''.
\begin{itemize}
  \item First, we need $\Fun{Behave}_{\Sys{T}}^0 : \Cat{Arena} \to
    \Cat{Matrix}$ which send an arena to the set of charts from
    $\lens{\In{T}}{\Out{T}}$ to that arena. It will send a chart to the function
    given by composing with that chart, and it will send a lens to a matrix that
    describes the wiring pattern in the lens. We've seen how to do this in \cref{prop.representable_double_functor}:
    $$\Fun{Behave}^0_{\Sys{T}} = \Cat{Chart}\left( \littlelens{\In{T}}{\Out{T}},
      -\right)$$
This is the blueprint for how our systems will compose.
  \item Next, for any arena $\lens{I}{O}$, we need a functor
    \[\Fun{Behave}_{\Sys{T}}^{\littlelens{I}{O}} :
    \Cat{Sys}\littlelens{I}{O} \to \Cat{Vec}\left( \Cat{Chart}\left( \littlelens{\In{T}}{\Out{T}},
      \littlelens{I}{O}\right) \right)\]
    which will send a system $\Sys{S}$ with interface $\lens{I}{O}$ to its
    set of behaviors of shape $\Sys{T}$, indexed by their chart. That is, we
    make the following definition:
   \[
\Fun{Behave}_{\Sys{T}}^{\littlelens{I}{O}}(\Sys{S})_{\littlelens{f_{\flat}}{f}}
\coloneqq \Cat{Sys}\littlelens{f_{\flat}}{f}(\Sys{T}, \Sys{S}). 
\]
This is quickly shown to be functorial by horizontal associativity of squares in $\Cat{Arena}$.
\item For any lens $\lens{w^{\sharp}}{w} : \lens{I}{O} \fromto \lens{I'}{O'}$,
  we need a natural transformation
  \[
    \begin{tikzcd}
\Cat{Sys}\lens{I}{O} \ar[dd, "{\Cat{Sys}\littlelens{w^{\sharp}}{w}}"'] \ar[rr, "{\Fun{Behave}_{\Sys{T}}^{\littlelens{I}{O}}}"] & & \Cat{Vec}\left( \Cat{Chart}\left(
    \littlelens{\In{T}}{\Out{T}}, \littlelens{I}{O} \right) \right) \ar[dd, "{\Cat{Vec}\left(
  \Cat{Chart}\left( \littlelens{\In{T}}{\Out{T}}, \littlelens{w^{\sharp}}{w}
  \right)\right)}"] \ar[ddll, Rightarrow, "{\Fun{Behave}_{\Sys{T}}^{\littlelens{w^{\sharp}}{w}}}"']\\
 & & \\
\Cat{Sys}\lens{I'}{O'}  \ar[rr, "{\Fun{Behave}_{\Sys{T}}^{\littlelens{B^-}{B^+}} }"']& & \Cat{Vec}\left( \Cat{Chart}\left(
    \littlelens{\In{T}}{\Out{T}}, \littlelens{I}{O} \right) \right) \\
    \end{tikzcd}
\]
This will take any behaviors of component systems whose charts compatible
according to the wiring pattern of $\lens{w^{\sharp}}{w}$ and wire them together
into a behavior of the wired together systems. In other words, this will be
given by vertical composition of squares in $\Cat{Arena}$. To see how that
works, we need follow a $\lens{I}{O}$-system $\Sys{S}$ around this diagram and see
how this natural transformation can be described so simply. Following $\Sys{S}$
around the top path of the diagram gives us the following vector of sets, we
first send $\Sys{S}$ to the vector of sets
\begin{align*}
\littlelens{f_{\flat}}{f} : \littlelens{\In{T}}{\Out{T}} \tto \littlelens{I}{O}
&\mapsto \Cat{Sys}\littlelens{f_{\flat}}{f}(\Sys{T}, \Sys{S}) \\
&= \left\{  
    \begin{tikzcd}[ampersand replacement = \&]
      \lens{\State{T}}{\State{T}} \ar[r, shift left, dashed, "\lens{\phi \circ
        \pi_2}{\phi}"] \ar[r, dashed, shift right] \ar[d, shift right,
      "\lens{\update{T}}{\expose{T}}"'] \ar[d, shift left, leftarrow] \&
      \lens{\State{S}}{\State{S}} \ar[d, shift left, leftarrow,
      "\lens{\update{S}}{\expose{S}}"] \ar[d, shift right]\\
      \lens{\In{T}}{\Out{T}} \ar[r, shift right, "\lens{f^{\sharp}}{f}"'] \ar[r,
      shift left] \& \lens{I}{O}
    \end{tikzcd}
\right\}
\end{align*}
We then multiply this by the matrix \(\Cat{Chart}\left(
  \littlelens{\In{T}}{\Out{T}}, \littlelens{w^{\sharp}}{w} \right)\) to get the
vector of sets whose entries are pairs of the following form:
\begin{align*}
\littlelens{g_{\flat}}{g} : \littlelens{\In{T}}{\Out{T}} \tto \littlelens{I'}{O'}
&\mapsto \left\{    
    \begin{tikzcd}[ampersand replacement = \&]
      \lens{\In{T}}{\Out{T}} \ar[r, shift left, dashed, "\lens{f_{\flat}}{f}"] \ar[r, dashed, shift right] \ar[d, shift right,
      equals] \ar[d, shift left, equals] \&
      \lens{I}{O} \ar[d, shift left, leftarrow,
      "\lens{w^{\sharp}}{w}"] \ar[d, shift right]\\
      \lens{\In{T}}{\Out{T}} \ar[r, shift right, "\lens{g_{\flat}}{g}"'] \ar[r,
      shift left] \& \lens{I'}{O'}
    \end{tikzcd}, \quad
    \begin{tikzcd}[ampersand replacement = \&]
      \lens{\State{T}}{\State{T}} \ar[r, shift left, dashed, "\lens{\phi \circ
        \pi_2}{\phi}"] \ar[r, dashed, shift right] \ar[d, shift right,
      "\lens{\update{T}}{\expose{T}}"'] \ar[d, shift left, leftarrow] \&
      \lens{\State{S}}{\State{S}} \ar[d, shift left, leftarrow,
      "\lens{\update{S}}{\expose{S}}"] \ar[d, shift right]\\
      \lens{\In{T}}{\Out{T}} \ar[r, shift right, "\lens{f_{\flat}}{f}"'] \ar[r,
      shift left] \& \lens{I}{O}
    \end{tikzcd}
\right\}
\end{align*}
On the other hand, following $\Sys{S}$ along the bottom path has us first composing it vertically with $\lens{w^{\sharp}}{w}$ and then finding the behaviors in it:
\[
\littlelens{g_{\flat}}{g} : \littlelens{\In{T}}{\Out{T}} \tto \littlelens{I'}{O'}
\mapsto \left\{  
    \begin{tikzcd}[ampersand replacement = \&]
      \lens{\State{T}}{\State{T}} \ar[r, shift left, dashed, "\lens{\phi \circ
        \pi_2}{\phi}"] \ar[r, dashed, shift right] \ar[d, shift right,
      "\lens{\update{T}}{\expose{T}}"'] \ar[d, shift left, leftarrow] \&
      \lens{\State{S}}{\State{S}} \ar[d, shift left, leftarrow,
      "\lens{\update{S}}{\expose{S}}\then \lens{w^{\sharp}}{w}"] \ar[d, shift right]\\
      \lens{\In{T}}{\Out{T}} \ar[r, shift right, "\lens{g^{\sharp}}{g}"'] \ar[r,
      shift left] \& \lens{I'}{O'}
    \end{tikzcd} 
\right\}
\]

  Finally, we are ready to define our natural transformation from the virst
  vector of sets to the second using vertical composition:
  \[
\Fun{Behave}_{\Sys{T}}^{\littlelens{w^{\sharp}}{w}}(\Sys{S})_{\littlelens{g_{\flat}}{g}}(\square_{w}, \phi) = \frac{\phi}{\square_w}.
  \]
  That this is natural for behaviors $\psi : \Sys{S} \to \Sys{U}$ in $\Cat{Sys}\lens{I}{O}$ follows quickly from the horizontal identity and interchange laws in $\Cat{Arena}$:
  \begin{align*}
    \frac{\phi \mid \psi}{\square_{w}} &= \frac{\phi \mid \psi}{\square_{w} \mid \littlelens{w^{\sharp}}{w}} \\
                                  &= \left. \frac{\phi}{\square_{w}} \,\middle|\, \frac{\psi}{\littlelens{w^{\sharp}}{w}} \right. .
  \end{align*}


\item For any chart $\lens{g_{\flat}}{g} : \lens{I}{O} \tto
  \lens{I'}{O'}$, we need a square
  \[
\begin{tikzcd}
  \Cat{Sys}\littlelens{I}{O} \ar[dd,
  "{\Fun{Behave}_{\Sys{T}}^{\littlelens{I}{O}}}"'] \ar[rr, tick,
  "{\Cat{Sys}\littlelens{g_{\flat}}{g}}"] & & \Cat{Sys}\littlelens{I'}{O'} \ar[dd, "{\Fun{Behave}_{\Sys{T}}^{\littlelens{I'}{O'}}}"]\\
  & \Fun{Behave}_{\Sys{T}}^{\littlelens{g_{\flat}}{g}} & \\
   \Cat{Vec}\left( \Cat{Chart}\left( \littlelens{\In{T}}{\Out{T}},
      \littlelens{I}{O}\right) \right) \ar[rr, tick, "{\Cat{Vec}\left( \Cat{Chart}\left( \littlelens{\In{T}}{\Out{T}},
      \littlelens{g_{\flat}}{g}\right) \right)}"']& & \Cat{Vec}\left( \Cat{Chart}\left( \littlelens{\In{T}}{\Out{T}},
      \littlelens{I'}{O'}\right) \right)
\end{tikzcd}
\]
This will take any behavior from $\Sys{S}$ to $\Sys{U}$ with chart $\lens{g_{\flat}}{g}$ and give the
function which takes behaviors of shape $\Sys{T}$ in $\Sys{S}$ and gives the
composite behavior of shape $\Sys{T}$ in $\Sys{U}$. That is,
$$\Fun{Behave}_{\Sys{T}}^{\littlelens{g_{\flat}}{g}}(\Sys{S}, \Sys{U})(\psi) =
\phi \mapsto \phi \mid \psi.$$
The naturality of this assignment follows from horizontal associativity in $\Cat{Arena}$.
\end{itemize}



Its a bit scary to see written out with all the names and symbols, but the idea
is simple enough. We are composing two sorts of things: behaviors and systems.
If we have some behaviors of shape $\Sys{T}$ in our systems and their charts are
compatible with a wiring pattern, then we get a behavior of the wired together
system. If we have a chart, then behaviors with that chart give us a way of
mapping forward behaviors of shape $\Sys{T}$. 

The lax doubly indexed functor laws now tell us some facts about how these two
sorts of composition interact.
\begin{itemize}
  \item (Vertical Lax Functoriality) This asks us to suppose that we are wiring
    our systems together in two stages. The law then says that if we take a
    bunch of behaviors whose charts are compatible for the total wiring pattern
    and wire them together into a behavior of the whole system, this is the same
    behavior we get if we first noticed that they were compatible for the first
    wiring pattern, wired them together, then noticed that the result was
    compatible for the second wiring pattern, and wired that together. This
    means that nesting of wiring diagrams commutes with finding behaviors of our systems.
  \item (Horizontal Functoriality) This asks us to suppose that we have two
    charts and a behavior of each. The law then says that composing a behavior
    of shape $\Sys{T}$ the composite of the behaviors is the same as composing
    it with the first one and then with the second one.
  \item (Functorial Interchange) This asks us to suppose that we have a pair of
    wiring patterns and compatible charts between them (a square in $\Cat{Arena}$). The law then says that if we
    take a bunch of behaviors whose charts are compatable according to the first
    wiring pattern, wire them together, and then compose with a behavior of the
    second chart, we get the same thing as if we compose them all with behaviors
    of the first chart, noted that they were compatible with the second wiring
    pattern, and then wired them together.
\end{itemize}

Though it seems like it would be a mess of symbols to check these laws, they in
fact fall right out of the laws for the double categories of arenas and
matrices, and the functoriality of \cref{prop.representable_double_functor}.
That is, we've already built up all the tools we need to prove this fact, we
just need to finish describing it.

\begin{itemize}
  \item (Vertical Lax Functoriality) Suppose we have composable lenses
    $\lens{w^{\sharp}}{w} : \lens{I_1}{O_1} \fromto \lens{I_2}{O_2}$ and
    $\lens{u^{\sharp}}{u} : \lens{I_2}{O_2} \fromto \lens{I_3}{O_3}$. We need to
    show that
    $$\Fun{Behave}_{\Sys{T}}^{\littlelens{w^{\sharp}}{w} \then
      \littlelens{u^{\sharp}}{u}} = \left( \Fun{Behave}_{\Sys{T}}^{
        \littlelens{u^{\sharp}}{u}} \Cat{Sys}\littlelens{w^{\sharp}}{w} \right)
    \circ \left(
\Cat{Vec}\Cat{Chart}\left( \littlelens{\In{T}}{\Out{T}}, \littlelens{u^{\sharp}}{u} \right)
\Fun{Behave}_{\Sys{T}}^{\littlelens{w^{\sharp}}{w}}  \right).$$
This follows immediately from vertical associativity in $\Cat{Arena}$, once both
sides have been expanded out. 
 Let $\Sys{S}$ be a $\lens{I_1}{O_1}$-system, then 
\begin{align*}
  \Fun{Behave}_{\Sys{T}}^{\littlelens{w^{\sharp}}{w} \then
      \littlelens{u^{\sharp}}{u}} (\Sys{S})(\alpha, \phi) &=   \Fun{Behave}_{\Sys{T}}^{\littlelens{w^{\sharp}}{w} \then
      \littlelens{u^{\sharp}}{u}} (\Sys{S})\left(  \frac{\beta}{\gamma}, \phi\right) &\mbox{by \cref{prop.lens_to_matrix_functor_discrete},}\\
  &= \frac{\phi}{\frac{\beta}{\gamma}} \\
&= \frac{\frac{\phi}{\beta}}{\gamma} \\
&= \left( \Fun{Behave}_{\Sys{T}}^{
        \littlelens{u^{\sharp}}{u}} \Cat{Sys}\littlelens{w^{\sharp}}{w} \right)
    \circ \left(
\Cat{Chart}\left( \littlelens{\In{T}}{\Out{T}}, \littlelens{u^{\sharp}}{u} \right)
\Fun{Behave}_{\Sys{T}}^{\littlelens{w^{\sharp}}{w}}  \right)(\gamma, \beta, \phi).
\end{align*}
\item (Horizontal Functoriality) This follows directly from horizontal associativity in $\Cat{Arena}$.
\item (Functorial Interchange) This law will follow directly from interchange in
  the double category of arenas. Let $\alpha$ be a square in $\Cat{Arena}$ of the
  following form:
\[
  \alpha = 
    \begin{tikzcd}
      \lens{A^-}{A^+} \ar[r, shift left, "\lens{f_{\flat}}{f}"] \ar[r, shift
      right] \ar[d, shift right, "\lens{j^{\sharp}}{j}"'] \ar[d, shift left,
      leftarrow] & \lens{B^-}{B^+} \ar[d, shift left, leftarrow,
      "\lens{k^{\sharp}}{k}"] \ar[d, shift right]\\
      \lens{C^-}{C^+} \ar[r, shift right, "\lens{g^{\sharp}}{g}"'] \ar[r, shift
      left] & \lens{D^-}{D^+}
    \end{tikzcd}
\]
We need to show that
\begin{equation}\label{eqn.interchange_func}
\left.\Fun{Behave}_{\Sys{T}}^{\littlelens{j^{\sharp}}{j}} \,\middle|\,
  \frac{\Cat{Sys}(\alpha)}{\Fun{Behave}_{\Sys{T}}^{\littlelens{g_{\flat}}{g}}}
\right. = \left.
  \frac{\Fun{Behave}_{\Sys{T}}^{\littlelens{f_{\flat}}{f}}}{\Cat{Vec}\Cat{Chart}\left(
        \littlelens{\In{T}}{\Out{T}}, \alpha \right)} \,
    \middle|\, \Fun{Behave}_{\Sys{T}}^{\littlelens{k^{\sharp}}{k}} \right.
\end{equation}
We can see both sides as natural transformations of the signature
\[
  \begin{tikzcd}
    \Cat{Sys}\littlelens{A^-}{A^+} \ar[d,
    "{\Fun{Behave}_{\Sys{T}}^{\littlelens{A^-}{A^+}}}"'] \ar[rr, 
    tick, "{\Cat{Sys}\littlelens{f_{\flat}}{f}}"] & & \Cat{Sys}\littlelens{B^-}{B^+} \ar[d,
    "{\Cat{Sys}\littlelens{k^{\sharp}}{k}}"] \\
    \Cat{Vec}\Cat{Chart}\left( \littlelens{\In{T}}{\Out{T}},
      \littlelens{A^-}{A^+} \right) \ar[d, "{ \Cat{Vec}\Cat{Chart}\left( \littlelens{\In{T}}{\Out{T}},
      \littlelens{j^{\sharp}}{j} \right)}"']& \ref{eqn.interchange_func} & \Cat{Sys}\littlelens{D^-}{D^+}
  \ar[d, "{\Fun{Behave}_{\Sys{T}}^{\littlelens{D^-}{D^+}}}"]\\
     \Cat{Vec}\Cat{Chart}\left( \littlelens{\In{T}}{\Out{T}},
      \littlelens{C^-}{C^+} \right) \ar[rr, tick, "{ \Cat{Vec}\Cat{Chart}\left( \littlelens{\In{T}}{\Out{T}},
      \littlelens{g_{\flat}}{g} \right) }"'] & &  \Cat{Vec}\Cat{Chart}\left( \littlelens{\In{T}}{\Out{T}},
      \littlelens{D^-}{D^+} \right) 
  \end{tikzcd}
\]

Accordingly, let $\psi \in \Cat{Sys}\littlelens{f_{\flat}}{f}(\Sys{S}, \Sys{U})$
be a behavior with chart $\littlelens{f_{\flat}}{f}$. We need to show that
passing this through the left side of \cref{eqn.interchange_func} equals the
result of passing it through the right hand side. The result is an element of
\[
\Cat{Vec}\Cat{Chart}\left( \littlelens{\In{T}}{\Out{T}},
      \littlelens{g_{\flat}}{g} \right)\left(\cdots, \cdots  \right)
\]
and is accordingly a function that takes in a pair of the following form:
\[
  (\square_j, \phi) = \left(  
    \begin{tikzcd}[ampersand replacement = \&]
      \lens{\In{T}}{\Out{T}} \ar[r, shift left, dashed, "\lens{a_{\flat}}{a}"] \ar[r, dashed, shift right] \ar[d, shift right,
      equals] \ar[d, shift left, equals] \&
      \lens{A^-}{A^+} \ar[d, shift left, leftarrow,
      "\lens{j^{\sharp}}{j}"] \ar[d, shift right]\\
      \lens{\In{T}}{\Out{T}} \ar[r, shift right, "\lens{c_{\flat}}{c}"'] \ar[r,
      shift left] \& \lens{C^-}{C^+}
    \end{tikzcd}, \quad
    \begin{tikzcd}[ampersand replacement = \&]
      \lens{\State{T}}{\State{T}} \ar[r, shift left, dashed, "\lens{\phi \circ
        \pi_2}{\phi}"] \ar[r, dashed, shift right] \ar[d, shift right,
      "\lens{\update{T}}{\expose{T}}"'] \ar[d, shift left, leftarrow] \&
      \lens{\State{S}}{\State{S}} \ar[d, shift left, leftarrow,
      "\lens{\update{S}}{\expose{S}}"] \ar[d, shift right]\\
      \lens{\In{T}}{\Out{T}} \ar[r, shift right, "\lens{a^{\sharp}}{a}"'] \ar[r,
      shift left] \& \lens{A^-}{A^+}
    \end{tikzcd}
\right)
\]
The left hand side sends this pair to 
\[
\Fun{Behave}_{\Sys{T}}^{\littlelens{g_{\flat}}{g}}
  \big( \Cat{Sys}(\alpha)(\psi) \big)\left(
    \Fun{Behave}_{\Sys{T}}^{\littlelens{j^{\sharp}}{j}}(\square_j, \phi) \right) 
\]
which equals, rather simply:
\[
\left. \frac{\phi}{\square_j} \,\middle|\, \frac{\psi}{\alpha} \right. .
\]
The right hand side sends the pair to
\[
\Fun{Behave}_{\Sys{T}}^{\littlelens{k^{\sharp}}{k}}\left(
  \Cat{Vec}\Cat{Chart}\left( \littlelens{\In{T}}{\Out{T}}, \alpha \right)\left(
    \square_j, \Fun{Behave}_{\Sys{T}}^{\littlelens{f_{\flat}}{f}}(\psi)(\phi) \right) \right)
\]
which equals, rather simply:
\[
\frac{\phi \mid \psi}{\square_j \mid \alpha}.
\]



\end{itemize}



%---- Section ----%
\section{Change of doctine}


\paragraph{Functoriality of the Grothendieck construction}
As with any categorical construction, the Grothendieck construction is
functorial in its argument. To see how this works, we need to know what an
indexed functor between indexed categories is. 

\begin{definition}
  Let $\cat{A} : \cat{C}\op \to \Cat{Cat}$ and $\cat{B} : \cat{D}\op \to
  \Cat{Cat}$ be indexed categories. An indexed functor $(F, \overline{F}) :
  \cat{A} \to \cat{B}$ consists of a functor $F : \cat{C} \to \cat{D}$ together
  with a pseudo-natural transformation\footnote{As with the psuedo-functoriality
  of each indexed category, pseudo-naturality means naturality up to coherent
  isomorphism. In many of our cases, we will have bona-fide naturality.} $\overline{F} : \cat{A} \Rightarrow
  \cat{B} \circ F\op$. Explicitly, this is:
  \begin{itemize}
  \item A functor $F : \cat{C} \to \cat{D}$.
  \item For each $C \in \cat{C}$, a functor $\overline{F}_C : \cat{A}(C) \to \cat{B}(FC)$.
  \item For each $f : C \to C'$ in $\cat{C}$, a natural isomorphism $\phi_f :
    \overline{F}_{C} \circ f^{\ast} \cong f^{\ast} \circ \overline{F}_{C'}$.
  \item (Psuedo-naturality) These naturality isomorphisms are required to satisfy a coherence condition: For $g : C' \to C''$,
    we need that
    \[
\begin{tikzcd}
\cat{A}(C'') \arrow[d, "\overline{F}_{C''}"']  \arrow[rr, "(g
\circ f)^\ast"{name=Top, below}, bend left=49]  &  & \cat{A}(C) \arrow[d,
"\overline{F}_C"] \arrow[lld, "\phi_{gf}"', Rightarrow] \\
\cat{A}(FC'')  \arrow[rr, "(g \circ
f)^{\ast}"'{name=Bottom, above}, bend right=49]                          &
                                                                          & \cat{A}(FC)                                                             
\end{tikzcd}
=
\begin{tikzcd}
\cat{A}(C'') \arrow[d, "\overline{F}_{C''}"'] \arrow[r, "g^\ast"] \arrow[rr, "(g
\circ f)^\ast"{name=Top, below}, bend left=49] \ar[r, Rightarrow, from=Top,
"\mu_{g, f}\inv"] & \cat{A}(C')  \arrow[r, "f^\ast"] \arrow[d, "\overline{F}_{C'}" description] \arrow[ld, "\phi_g"', Rightarrow] & \cat{A}(C) \arrow[d, "\overline{F}_C"] \arrow[ld, "\phi_f"', Rightarrow] \\
\cat{A}(FC'') \arrow[r, "g^{\ast}"'] \arrow[rr, "(g \circ
f)^{\ast}"'{name=Bottom, above}, bend right=49]                          &
\cat{A}(FC') \arrow[Rightarrow, to=Bottom, "\mu_{g, f}"] \arrow[r, "f^{\ast}"']                                                                          & \cat{A}(FC)                                                             
\end{tikzcd}
    \]
  \end{itemize}
\end{definition}

\begin{proposition}[Functoriality of the Grothendieck construction]\label{prop.groth_construction_functoriality}
  Let $(F, \overline{F}) : \cat{A} \to \cat{B}$ be an indexed double functor.
  Then there is a functor
  $$\lens{\overline{F}}{F} : \int^{C : \cat{C}} \cat{A}(C) \to \int^{D :
    \cat{D}} \cat{B}(D)$$
  between their Grothendieck constructions given on objects by
  $$\lens{\overline{F}}{F} \lens{A}{C} \coloneqq \lens{\overline{F}A}{FC}.$$
\end{proposition}
\begin{proof}

\end{proof}


\end{document}

