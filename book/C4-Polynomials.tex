\documentclass[DynamicalBook]{subfiles}
\begin{document}
%


\setcounter{chapter}{3}%Just finished 3.

\tableofcontents*

%------------ Chapter ------------%
\chapter{Polynomial functors and dynamics}

\slogan{
The category of polynomial functors is a jackpot. Its beauty flows without bound.}

%-------- Section --------%
\section{Introduction}


In this chapter we will investigate a remarkable category called $\poly$. We will see its intimate relationship with both dynamic processes, data storage and transformations, and decision-making. We will see its intimate relationship with safety and decision-making. But our story begins with something quite humble: middle school algebra.
\begin{align}\label{eqn.polynomial1892}
\yon^\2+\yon+\1 \quad&\quad
\textit{polynomial}
\intertext{
All our polynomials will involve one variable, $\yon$, chosen for reasons we'll explain soon. Polynomials in one variable can be depicted as a set of mini-trees:
}
\label{eqn.poly1892}
\begin{tikzpicture}[trees, grow'=up]
  \node (1) {$\bullet$} 
    child {}
    child {};
  \node[right=.5 of 1] (2) {$\bullet$} 
    child {};
  \node[right=.5 of 2] (3) {$\bullet$};
\end{tikzpicture}
\quad&\quad\textit{poly}
\end{align}
More technically, mini-trees are called \emph{corollas}, so what we're calling a \emph{poly} is a set of corollas. 

Intuitively, one might think of each corolla as representing a \emph{decision}. Associated to every decision is a set of options, or what we will call \emph{possibilities}. The three decisions we exhibit in \cref{eqn.poly1892} are particularly interesting; they respectively have two possibilities, one possibility, and no possibilities. Having two possibilities is familiar from life---it's the classic yes/no decision---as well as from Claude Shannon's Information Theory. Having one possibility is also familiar theoretically and in life: ``sorry, ya just gotta go through it.'' Having no possibilities is when you actually don't get through it: it's ``the end''. While the corollas $\1,\yon,$ and $\yon^\2$ are each interesting as decisions, their sum $\yon^\2+\yon+\1$ has very little theoretical interest; it's just a first example.

In a polynomial like $\4\2\yon^\3+\1\7\yon$, the pure power term $\yon^\3$ has a coefficient of \4\2 next to it, so the corolla with three leaves shows up 42 times%
\footnote{
In standard font, 42 represents the usual natural number. In sans serif font \4\2 represents the set $\4\2=\{1,2,\ldots,42\}$ with 42 elements.
}
in the associated poly. Intuitively, there are 42 situations in which you have a decision with three possibilities. When you put all those decisions together, you have the \emph{wheelhouse} associated to $p$. 

Here is a table of terminology. The first column is book-wide terminology, the second is algebraic terminology about polynomials as in \eqref{eqn.polynomial1892}, the third is about the pictures of corollas as in \eqref{eqn.poly1892}, and the fourth column is intuitive terminology.

\begin{equation}\label{eqn.table_terminology}
\small
\begin{tabular}{llll}
\multicolumn{4}{c}{\normalsize Terminology}\\
\textbf{Interface} & \textbf{Polynomial} $p$ & \textbf{Poly} & \textbf{Wheelhouse}\\\hline
Set of positions & $p(\1)$ & Set of roots & Set of situations\\
Interface in position $\fun{i}$ & Pure power summand $\yon^{p_\fun{i}}$ & Corolla & Decision\\
Distinction $c\in p_\fun{i}$ & Element of $p_\fun{i}$ & Leaf & Possibility
\end{tabular}
\end{equation}

If you're wondering about the upright $\fun{i}$, we'll get to all our notational conventions in \cref{sec.poly}.

\begin{exercise}
Consider the polynomial $p\coloneqq\2\yon^\3+\2\yon+\1$ and the associated poly.
\begin{enumerate}
	\item Draw the poly whose name is $p$.
	\item How many roots does this poly have?
	\item How many decisions does this represent?
	\item For each corolla in the poly, say how many leaves it has.
	\item For each decision, how many possibilities does it have?
\end{enumerate}
Now referring instead to the polynomial $q\coloneqq\yon^\nn+\4\yon$:
\begin{enumerate}[resume]
	\item Does the polynomial $q$ have a pure-power summand $\yon^\2$?
	\item Does the polynomial $q$ have a pure-power summand $\yon$?
	\item Does the polynomial $q$ have a pure-power summand $\4\yon$?
	\qedhere
\end{enumerate}

\end{exercise}

Mathematically and notationally we do not make a distinction between a poly and its name $p$; they are two different syntaxes for the same object. 


\begin{exercise}
If you were a suitor choosing the poly you love, aesthetically speaking, which would strike your interest? Answer by circling the associated polynomial:
\begin{enumerate}
	\item $\yon^\2+\yon+\1$
	\item $\yon^\2+\3\yon^\2+\3\yon+\1$
	\item $\yon^\2$
	\item $\yon+\1$
	\item $(\nn\yon)^\nn$
	\item $S\yon^S$
	\item $\yon^{\1\0\0}+\yon^\2+\3\yon$
	\item Your poly's name $p$ here.
\end{enumerate}
Any reason for your choice?
\end{exercise}

Before we can really get into this story, let's summarize where we're going: polynomials are going to have really surprising applications to dynamics, data, and decision. Remember, we spoke superlatively of $\poly$ at the beginning of this chapter,
\slogan{
The category of polynomial functors is a jackpot. Its beauty flows without bound}
and we have not yet begun to deliver. So let's introduce some of the applications and mathematics to come.

%---- Subsection ----%
\subsection{Dynamical systems}

Throughout this book, we've seen dynamical systems---machines of various sorts---which have an internal state that can be read out to other systems, as well as updated based on input received from other systems. In the context of this chapter, we'll be looking mainly at deterministic systems, but with a lot of interesting new possibilities:
\begin{enumerate}
	\item The interface of the system can change shape through time.
	\item The wiring diagram connecting a bunch of systems can change in time.
	\item One can speed up the dynamics of a system.
	\item One can introduce ``effects'', i.e.\ as defined by monads on $\smset$.
	\item The dynamical systems on any interface form a topos.
\end{enumerate}

To give some intuition for the first two, imagine yourself as a system, wired up to other systems. You have some input ports: your eyes, your ears, etc., and you have some output ports: your speech, your gestures, etc. And you connect with other systems: your family, your colleagues, the GPS of your phone, etc.
\begin{equation}\label{eqn.wired_forever}
\begin{tikzpicture}[oriented WD, bb min width =.5cm, bbx=.5cm, bb port sep =1,bb port length=0, bby=.15cm]
	\node[bb={2}{2}, green!25!black] (X11) {\tiny Alice};
	\node[bb={3}{3}, green!25!black, below right=of X11] (X12) {\tiny you};
	\node[bb={2}{1}, green!25!black, above right=of X12] (X13) {\tiny Bob};
	\node[bb={2}{2}, green!25!black, below right = -1 and 1.5 of X12] (X21) {\tiny GPS};
	\node[bb={1}{2}, green!25!black, above right=-1 and 1 of X21] (X22) {\tiny me};
  \node[bb={2}{2}, fit = {($(X11.north east)+(-1,4)$) (X12) (X13) ($(X21.south)$) ($(X22.east)+(.5,0)$)}, bb name = {\small Wired together like this forever?}] (Z) {};
	\draw (X21_out1) to (X22_in1);
	\draw let \p1=(X22.north east), \p2=(X21.north west), \n1={\y1+\bby}, \n2=\bbportlen in
          (X22_out1) to[in=0] (\x1+\n2,\n1) -- (\x2-\n2,\n1) to[out=180] (X21_in1);
	\draw (X11_out1) to (X13_in1);
	\draw (X11_out2) to (X12_in1);
	\draw (X12_out1) to (X13_in2);
	\draw (Z_in1'|-X11_in2) to (X11_in2);	
	\draw (Z_in2'|-X12_in2) to (X12_in2);
	\draw (X12_out2) to (X21_in2);
	\draw (X21_out2) to (Z_out2'|-X21_out2);
	 \draw let \p1=(X12.south east), \p2=(X12.south west), \n1={\y1-\bby}, \n2=\bbportlen in
	  (X12_out3) to[in=0] (\x1+\n2,\n1) -- (\x2-\n2,\n1) to[out=180] (X12_in3);
	\draw let \p1=(X22.north east), \p2=(X11.north west), \n1={\y2+\bby}, \n2=\bbportlen in
          (X22_out2) to[in=0] (\x1+\n2,\n1) -- (\x2-\n2,\n1) to[out=180] (X11_in1);
	\draw (X13_out1) to (Z_out1'|-X13_out1);
\end{tikzpicture}
\end{equation}
We wrote a little question for you at the top of the diagram. Isn't there something a little funny about wiring diagrams? Maybe for old-fashioned machines, you would wire things together once and they'd stay like that for the life of the machine. But my phone connects to different wifi stations at different times, I drop my connection to Alice for weeks at a time, etc. So wiring diagrams should be able to change in time; $\poly$ will let us do that.

\begin{example}
Here are some familiar circumstances where we see wiring diagrams changing in time.
\begin{enumerate}[itemsep=0pt]
%	\item Airplanes only communicate when they get near enough;
%	\item A phone is connected to 4G or to wifi depending on circumstances;
%	\item A person can choose when to open (receive input through) their eyes and when to speak (produce output);\goodbreak
	\item When too much force is applied to a material, bonds can break;
\end{enumerate}
\[
\begin{tikzpicture}[oriented WD, bb small, bb port length=0]
	\foreach \i in {0,...,4} {
		\node[bb={1}{1}, fill=blue!10] at (1.7*\i,0) (X\i) {};
	}
%	\node[bb={1}{1}, fit=(X0) (X4)] (X) {};
	\foreach \i in {0,...,3} {
		\draw[thick] (X\i_out1) -- (X\the\numexpr\i+1\relax_in1);
	};
	\draw[thick, ->] (X0_in1) -- node[above, font=\tiny] {Force} +(-2.5,0);
	\draw[thick, ->] (X4_out1) -- node[above, font=\tiny] {Force} +(2.5,0) node (R) {};
%
\def\x{21};
	\foreach \i in {0,...,2} {
		\node[bb={1}{1}, fill=blue!10] at (\x+1.7*\i,0) (Y\i) {};
	}
	\foreach \i in {3,...,4} {
		\node[bb={1}{1}, fill=blue!10] at (\x+1.3+1.7*\i,0) (Y\i) {};
	}
%	\node[bb={1}{1}, fit=(Y0) (Y4)] (Y) {};
	\foreach \i in {0,1,3} {
		\draw[thick] (Y\i_out1) -- (Y\the\numexpr\i+1\relax_in1);
	};
	\draw[thick, ->] (Y0_in1) -- node[above, font=\tiny] {Force} +(-2.5,0) node (L) {};
	\draw[thick, ->] (Y4_out1) -- node[above, font=\tiny] {Force} +(2.5,0);
	\node[starburst, draw, minimum width=2cm, minimum height=1.5cm,red,fill=orange,line width=1.5pt] at ($(L)!.5!(R)$)
{Snap!};
\end{tikzpicture}
\]
\begin{quote}
In materials science the Young's modulus accounts for how much force can be transferred across a material as its endpoints are pulled apart. When the material breaks, the two sides can no longer feel evidence of each other. Thinking of pulling as sending a signal (a signal of force), we might say that the ability of internal entities to send signals to each other---the connectivity of the wiring diagram---is being measured by Young's modulus. It will also be visible within $\poly$.
\end{quote}
\begin{enumerate}[resume]
	\item A company may change its supplier at any time;
\end{enumerate}
\begin{equation*}%\label{eqn.supplier}
\begin{tikzpicture}[oriented WD, font=\ttfamily, every node/.style={fill=blue!10}, baseline=(c)]
	\node[bb={0}{1}] (s1) {Supplier 1};
	\node[bb={0}{1}, below=of s1] (s2) {Supplier 2};
	\coordinate (helper) at ($(s1)!.5!(s2)$);
	\node[bb={1}{0}, right=1.5 of helper] (c) {Company};
	\draw (s1_out1) to (c_in1);
	\draw (s2_out1) to +(5pt,0) node[fill=none] {$\bullet$};
\begin{scope}[xshift=3.5in]
	\node[bb={0}{1}] (s1') {Supplier 1};
	\node[bb={0}{1}, below=of s1'] (s2') {Supplier 2};
	\coordinate (helper') at ($(s1')!.5!(s2')$);
	\node[bb={1}{0}, right=1.5 of helper'] (c') {Company};
	\draw (s2'_out1) to (c'_in1);
	\draw (s1'_out1) to +(5pt,0) node[fill=none] {$\bullet$};
\end{scope}
	\node[starburst, draw, minimum width=2cm, minimum height=2cm,align=center,fill=green!10, font=\small, fill=white, line width=1.5pt] at ($(c)!.5!(helper')$)
{Change\\supplier!};
\end{tikzpicture}
\end{equation*}
\begin{quote}
The company can get widgets either from supplier 1 or supplier 2; we could imagine this choice is completely up to the company. The company can decide based on the quality of widgets it has received in the past, i.e.\ when the company gets a bad widget, it updates an internal variable, and sometimes that variable passes a threshold making the company switch states. Whatever its strategy for deciding, we should be able to encode it in $\poly$.
\end{quote}
\begin{enumerate}[resume]
	\item When someone assembles a machine, their own outputs dictate the connection pattern of the machine's components.
\end{enumerate}
\begin{equation*}%\label{eqn.someone}
\begin{tikzpicture}[oriented WD, font=\ttfamily, bb port length=0, every node/.style={fill=blue!10}, baseline=(someone.north)]
	\node[bb port sep=.5, bb={0}{1}] (A) {unit A};
	\node[bb port sep=.5, bb={1}{0}, right=of A] (B) {unit B};
	\coordinate (helper) at ($(A)!.5!(B)$);
	\node[bb={1}{1}, below=2 of helper] (someone) {\tikzsymStrichmaxerl[3]};
	\draw[->, dashed, blue] (someone_in1) to[out=180, in=270] (A.270);
	\draw[->, dashed, blue] (someone_out1) to[out=0, in=270] (B.270);
	\draw (A_out1) -- +(10pt,0);
	\draw (B_in1) -- +(-10pt,0);
%
\begin{scope}[xshift=3.5in]
	\node[bb port sep=.5, bb={0}{1}] (A') {unit A};
	\node[bb port sep=.5, bb={1}{0}, right=.5of A'] (B') {unit B};
	\coordinate (helper') at ($(A')!.5!(B')$);
	\node[bb={1}{1}, below=2 of helper'] (someone') {\tikzsymStrichmaxerl[3]};
	\draw[->, dashed, blue] (someone'_in1) to[out=180, in=270] (A'.270);
	\draw[->, dashed, blue] (someone'_out1) to[out=0, in=270] (B'.270);
	\draw (A'_out1) -- (B'_in1);
\end{scope}
%
	\node[starburst, draw, minimum width=2cm, minimum height=2cm,fill=blue!50,line width=1.5pt, align=center, font=\upshape] at ($(B)!.5!(A')-(0,.6cm)$)
{Attach!};
\end{tikzpicture}
\end{equation*}
\begin{quote}
Have you ever assembled something? Your internal states dictate the connection pattern of some other things. We can say this in $\poly$.
\end{quote}

All of the examples discussed here will be presented in some detail once we have the requisite mathematical theory \cref{?,?,?}.
\end{example}

\begin{exercise}
Think of another example where systems are sending each other information, but where the sort of information or who it's being sent to or received from can change based on the states of the systems involved. You might have more than two, say $\rr$-many, different wiring patterns in your situation.
\end{exercise}

But there's more that's intuitively wrong or limiting about the picture in \eqref{eqn.wired_forever}. Ever notice how you can change how you interface with the world? Sometimes I close my eyes, which makes that particular way of sending me information inaccessible: that port vanishes and you need to use your words. Sometimes I'm in a tunnel and my phone can't receive GPS. Sometimes I extend my hand to give or receive an object from another person, but sometimes I don't. My ports themselves change in time. Sometimes I even use my output ports to determine the wiring pattern of other machines. We will be able to say all this using $\poly$.

And there's even more that's wrong with the above description. Namely, when I move my eyes, that's actually something you can see---e.g.\ whether I'm looking at you. When I turn around, I see different things, and \emph{you can notice I'm turned around}! When I use my muscles or mouth to express things, my very position changes: my tongue moves, my body moves. So my output is actually achieved by changing position. The model in $\poly$ will be able to express this too.

\begin{example}
Imagine a million little eyeballs, each of which has a tiny brain inside it, all together in a pond of eyeballs. All that an individual eyeball $e$ can do is open and close. When $e$ is open, it can make some distinction about all the rest of the eyeballs in view: maybe it can count how many are open, or maybe it can see whether just one certain eyeball $e'$ is open or closed. But when $e$ is closed, it can't see anything; whatever else is happening, it's all the same to $e$. All it can do in that state is process previous information.

Each eyeball in this system will correspond to the polynomial $\yon^\ord{n}+\yon$, which intuitively consists of two situations, one with $n$-many possibilities, and the other with only one possibility.

The point is, however, is that the other eyeballs can tell if $e$ is opened or closed. We can imagine some interesting dynamics in this system, e.g.\ waves of openings or closings sweeping over the group, a ripple of openings expanding through the pond.

Talk about real-world applications!
\end{example}

Hopefully you now have an idea of what we mean by mode-dependence: interfaces and wiring diagrams changing in time, based on the states of all the systems involved. We'll see that $\poly$ speaks about mode-dependent systems and wiring diagrams in this sense. 

But $\poly$ is very versatile in its applications. Next we'll show how it relates to information storage and translation.

%---- Subsection ----%
\subsection{Data}

Data is information, maybe thought of as quantized into atomic pieces, but where these atomic pieces are somehow linked together according to a conceptual structure. When a person or organization uses certain data repeatedly, they often find it useful to put their data in a database. This requires organizing the little pieces into a conceptual structure. So when you hear ``database'', just think of it as a conceptual structure filled with examples.

To fix a mental image, let's say that you need to constantly look up employees, what department they're in, who the admin person is for that department, who their manager is, etc. Here's an associated database
\begin{equation}\label{eqn.mytables}
\begin{tabular}{ c | c  c  c}
  \textbf{Employee}&\textbf{FirstName}&\textbf{WorksIn}&\textbf{Mngr}\\\hline
  1&Alan&101&2\\
  2&Ruth&101&2\\
  3&Carla&102&3
\end{tabular}
\hspace{.3in}
\begin{tabular}{ c | c  c}
  \textbf{Department}&\textbf{Name}&\textbf{Admin}\\\hline
  101&Sales&1\\
  102&IT&3\\~
\end{tabular}
%\hspace{.3in}
%\begin{tabular}{ c |}
%	\textbf{String}\\\hline
%	Alan\\
%	IT\\[-3pt]
%	\resizebox{!}{10pt}{$\vdots$}
%\end{tabular}
\end{equation}
We can see it as being associated to the following conceptual scheme, also called a \emph{schema}:
\begin{equation}\label{eqn.myschema}
\tiny\cat{C}\coloneqq
\boxCD{
\begin{tikzcd}[row sep=large, ampersand replacement = \&]
 	\LTO{Employee}\ar[rr, shift left, "\text{WorksIn}"]\ar[dr, bend right, "\text{FirstName}"']\ar[loop left, "\text{Mngr}"]\&\&
  \LTO{Department}\ar[ll, shift left, "\text{Admin}"]\ar[dl, bend left, "\text{Name}"]\\
  \&\LTO[\circ]{String}
\end{tikzcd}
\\~\\\tiny
  Department.Admin.WorksIn = Department
}
\end{equation}
The equation at the bottom says that for any department $d$, if you ask for the admin person and see which department they work in, it's required to be $d$.

What's called $\cat{C}$ in \eqref{eqn.myschema} is a \emph{finitely presented category}, and the database instance presented in \eqref{eqn.mytables} correspond to a functor $I\colon\cat{C}\to\smset$; it sends the object $\text{Employee}\in\cat{C}$ to the set $I(\text{Employee})\coloneqq\{1,2,3\}$. It sends the morphism $\text{Mngr}\colon\text{Employee}\to\text{Employee}$ to the function $I(\text{Mngr})\colon\{1,2,3\}\to\{1,2,3\}$ given by $I(1)=2$, $I(2)=2$, and $I(3)=3$. A functor $\cat{C}\to\smset$ is called a \emph{copresheaf on $\cat{C}$}. So the story of database schemas and their data can be based on the story of categories and their copresheaves.

There's a very important thing that we do with databases: we query them. We ask them questions like ``tell me everyone that's either the admin person of the Math department or their manager''.
\begin{minted}{mysql}
 FOR    d: Department, e: Employee
 WHERE  Name(d)="Math" AND
        (e=Admin(d) OR e=Mngr(Admin(d)))
 RETURN FirstName(e)
\end{minted}
This sort of question is formally called a ``union of conjunctive queries''. We will see this sort of query is intimately connected with $\poly$.

We will also see how databases can be conceived in terms of dynamical systems.

%---- Subsection ----%
\subsection{Decisions}

Consider the following three trees, the first two are infinite (though that's hard to draw):
\[
\begin{tikzpicture}[trees]
\begin{scope}[
  level 1/.style={sibling distance=20mm},
  level 2/.style={sibling distance=10mm},
  level 3/.style={sibling distance=5mm},
  level 4/.style={sibling distance=2.5mm}]
  \node[dgreen] (a) {$\bullet$}
    child {node[dgreen] {$\bullet$}
    	child {node[dgreen] {$\bullet$}
    		child {node[dgreen] {$\bullet$}
  				child 
  				child
  			}
    		child {node[dyellow] {$\bullet$}
  				child 
  				child
  			}
    	}
    	child {node[dyellow] {$\bullet$}
    		child {node[dgreen] {$\bullet$}
  				child
  				child
  			}
    		child  {node[red] {$\bullet$}}
    	}
    }
    child {node[dyellow] {$\bullet$}
    	child {node[dgreen] {$\bullet$}
    		child {node[dgreen] {$\bullet$}
  				child
  				child
  			}
    		child {node[dyellow] {$\bullet$}
  				child
  				child
  			}
  		}
  		child {node[red] {$\bullet$}
  		}
  	}
  ;
\end{scope}
\begin{scope}[
  level 1/.style={sibling distance=13mm},
  level 2/.style={sibling distance=10mm},
  level 3/.style={sibling distance=5mm},
  level 4/.style={sibling distance=2.5mm}]
\node (b) [right=4 of a, dyellow] {$\bullet$}
  child {node[dgreen] {$\bullet$}
  	child {node[dgreen] {$\bullet$}
  		child {node[dgreen] {$\bullet$}
				child
				child
			}
  		child {node[dyellow] {$\bullet$}
				child
				child
			}
		}
  	child {node[dyellow] {$\bullet$}
  		child {node[dgreen] {$\bullet$}
				child
				child
			}
  		child  {node[red] {$\bullet$}}
  	}
	}
	child {node[red] {$\bullet$}}	
;
\end{scope}
\node (c) [red, right=2 of b] {$\bullet$};
\end{tikzpicture}
\]
These are patterned examples---and we'll understand what this pattern is more clearly in \cref{???}---of what we will call \emph{decision streams}. Decision streams form the objects in a category that also includes the following level-3 abbreviations of binary decision streams (the third of which is a finite stream):
\[
\begin{tikzpicture}[trees]
\begin{scope}[
  level 1/.style={sibling distance=20mm},
  level 2/.style={sibling distance=10mm},
  level 3/.style={sibling distance=5mm},
  level 4/.style={sibling distance=2.5mm}]
  \node (a) {$\bullet$}
    child {node {$\bullet$}
    	child {node {$\bullet$}
    		child {node {$\bullet$}
  			}
    		child {node {$\bullet$}
  				child 
  				child
  			}
    	}
    	child {node {$\bullet$}
  			}
    }
    child {node {$\bullet$}
    	child {node {$\bullet$}
    		child {node {$\bullet$}
  				child
  				child
  			}
    		child {node {$\bullet$}
  				child
  				child
  			}
  		}
  		child {node {$\bullet$}
    		child {node {$\bullet$}
  			}
    		child {node {$\bullet$}
  			}
  		}
  	}
  ;
\end{scope}
\begin{scope}[
  level 1/.style={sibling distance=13mm},
  level 2/.style={sibling distance=10mm},
  level 3/.style={sibling distance=5mm},
  level 4/.style={sibling distance=2.5mm}]
  \node (b) [right=4 of a] {$\bullet$}
    child {node {$\bullet$}
    }
    child {node {$\bullet$}
    	child {node {$\bullet$}
    		child {node {$\bullet$}
  			}
    		child {node {$\bullet$}
  				child
  				child
  			}
  		}
  		child {node {$\bullet$}
    		child {node {$\bullet$}
  				child
  				child
  			}
    		child {node {$\bullet$}
  				child
  				child
  			}
  		}
  	}
  ;
\end{scope}
\begin{scope}[
  level 1/.style={sibling distance=13mm},
  level 2/.style={sibling distance=8mm},
  level 3/.style={sibling distance=5mm},
  level 4/.style={sibling distance=2.5mm}]
  \node (c) [right=4 of b] {$\bullet$}
    child {node {$\bullet$}
    	child {node {$\bullet$}
  		}
  		child {node {$\bullet$}
    		child {node {$\bullet$}
  			}
    		child {node {$\bullet$}
  			}
  		}
    }
    child {node {$\bullet$}
  	}
  ;
\end{scope}
\end{tikzpicture}
\]

The set of such decision streams in fact form the objects of a category with very nice properties (it's a topos), which we call $\sys(\yon^\2+\1)$. The idea is that every corolla in these diagrams has either two possibilities, corresponding to $\yon^\2$, or no possibilities, corresponding to $\1=\yon^\0$.

\begin{exercise}
\begin{enumerate}
	\item Draw a level-3 abbreviation of a decision stream of type $\yon^\2+\yon^\0$.
	\item Draw a level-4 abbreviation of a decision stream of type $\yon$.
	\item Draw a level-3 abbreviation of a decision stream of type $\nn\yon^\2$ by labeling every node with a natural number.
	\qedhere
\end{enumerate}
\end{exercise}

But decisions aren't just about choosing; they're also about trying to accomplish something. The logic of accomplishment is exceptionally rich in this setting. We will concentrate on what we call a \emph{win condition}, which is a subgraph of the decision stream with the property that if $n$ is winning node, then any child of $n$ is also a winning node. 
\[\begin{tikzpicture}[trees,
  level 1/.style={sibling distance=20mm},
  level 2/.style={sibling distance=10mm},
  level 3/.style={sibling distance=5mm},
  level 4/.style={sibling distance=2.5mm}]
  \node (root) {$\bullet$}
    child {node {$\bullet$}
    	child {node {$\bullet$}
    		child {node {$\bullet$}
  			}
    		child {node {$\bullet$}
  				child
  				child
  			}
    	}
    	child {node {$\bullet$}
  			}
    }
    child {node {$\bullet$}
    	child {node {$\bullet$}
    		child {node {$\bullet$}
  				child
  				child
  			}
    		child {node {$\bullet$}
  				child
  				child
  			}
  		}
  		child {node {$\bullet$}
    		child {node {$\bullet$}
  				child
  				child
  			}
    		child {node {$\bullet$}
  			}
  		}
  	}
  ;
 \begin{scope}[every node/.style={circle, inner sep=3pt, blue, draw}]
  \node at (root-1-2)     {};
  \node at (root-2-1-1-1) {};
  \node at (root-2-1-1-2) {};
  \node at (root-2-1-2-1) {};
	\node at (root-2-2) 		{};
  \node at (root-2-2-1) 	{};
  \node at (root-2-2-2) 	{};
  \node at (root-2-2-1-1) {};
  \node at (root-2-2-1-2) {};
 \end{scope}
\end{tikzpicture}
\]
More formally, these are called \emph{sieves}; they form the elements of a logical system called a Heyting algebra: you can take any two sieves and form the intersection or union (which correspond to AND and OR), or even things like implication, negation, and existential and universal quantification. This will give us a calculus of win-conditions for any sort $p$ of decision stream.

%---- Subsection ----%
\subsection{Implementation}

Everything we talk about can actually be implemented in a computer without much difficulty, at least if you have access to a language that supports dependent types. We will continue to use Agda.

What we have been calling polynomials---things like $\yon^\2+\2\yon+\1$---are often called \emph{containers} in the computer science literature. A container consists of a type $S$, usually called the type of \emph{shapes}, and a type $P(s)$ for each term $s:S$, called the type of positions in shape $s$. It's mildly unfortunate that the names clash with our own: for us a container-shape is a position and a container-position is a distinction.

Luckily, the agda code is pretty easy to understand.
\begin{agda}
record Wheelhouse : Set where  -- a wheelhouse consists of 
   field                       -- two fields
     decn : Set                -- one called decn, a set, and
     posy : pos -> Set         -- one called posy,
                               -- a set that depends on decn
\end{agda}  

%---- Subsection ----%
\subsection{Mathematical theory}\label{subsec.math_theory}

The applications of $\poly$ are quite diverse and interesting, including dynamics, data, and decisions. However it is how the mathematics of $\poly$ supports these applications that is so fantastic. For experts, here are some reasons for the excitement.

\begin{proposition}
$\poly$ is a biCartesian closed category, meaning it has sums, products, and exponential objects. It supports the simply typed lambda calculus.
\end{proposition}

In fact $\poly$ has infinite coproducts, infinite products, and is completely distributive. We'll explain all of this in \cref{??}; for now we continue with the highlights.

\begin{proposition}
Beyond the coCartesian and Cartesian monoidal structures $(\0,+)$ and $(\1,\times)$, the category $\poly$ has two additional monoidal structures, denoted $(\yon,\otimes)$ and $(\yon,\circ)$, which are together duoidal. Moreover $\otimes$ is also a closed monoidal structure and it distributes over $+$.
\end{proposition}

\begin{proposition}
$\poly$ has an adjoint quadruple with $\smset$ and an adjoint pair with $\smset\op$:
\begin{equation*}%\label{eqn.adjoints_galore}
\begin{tikzcd}[column sep=60pt]
  \smset
  	\ar[r, shift left=7pt, "A" description]
		\ar[r, shift left=-21pt, "A\yon"']&
  \poly
  	\ar[l, shift right=21pt, "p(\0)"']
  	\ar[l, shift right=-7pt, "p(\1)" description]
	\ar[l, phantom, "\scriptstyle\Leftarrow"]
	\ar[l, phantom, shift left=14pt, "\scriptstyle\Rightarrow"]
	\ar[l, phantom, shift right=14pt, "\scriptstyle\Rightarrow"]
\end{tikzcd}
\hspace{.6in}
\adjr[50pt]{\smset\op}{\yon^A}{\Gamma p}{\poly}.
\end{equation*}
Each functor is labeled by where it sends $p\in\poly$ or $A\in\smset$.
\end{proposition}

There's a lot we're leaving out of this summary, just so we can hit the highlights.

\begin{proposition}
The functor $\poly\to\smset$ given by $p\mapsto p(\1)$ is a monoidal fibration.
\end{proposition}

In fact it's a monoidal $*$-bifibration in the sense of \cite{shulman2008framed}. But here's where things get really interesting.

\begin{proposition}[Ahman-Uustalu]\label{prop.ahman_uustalu1}
Comonoids in $(\poly,\yon,\circ)$ are categories (up to isomorphism).
\end{proposition}

\begin{proposition}[Niu]
The category $\comon$ has finite coproducts and products, and coproducts in $\comon$ agree with those in $\smcat$.
\end{proposition}

\begin{proposition}
Suppose that the category $\cat{C}$ corresponds under \cref{prop.ahman_uustalu1}, to the comonoid $\com{C}$. Then there is an equivalence of categories
\[
\Cat{Bimod}(\com{C},0)\cong\cat{C}\set
\]
between $(\com{C},0)$-bimodules and $\cat{C}$-copresheaves.
\end{proposition}
We will use the convention that the comonoid $\com{C}$ corresponds to category $\cat{C}$, and similarly for $\com{D}$ and $\cat{D}$, etc. 

\begin{proposition}[Garner]
For any categories $\cat{C}$ and $\cat{D}$, there is an equivalence of categories
\[
\Cat{Bimod}(\com{C},\com{D})\cong\Cat{pra}(\cat{C}\set,\cat{D}\set)
\]
between that of bimodules between comonoids in $\poly$ and parametric right adjoints between copresheaf categories.
\end{proposition}

\begin{proposition}
The functor $\comon\to\poly$ has a right adjoint
\[
\adjr{\poly}{\com{K}_-}{u_-}{\comon},
\]
called the \emph{cofree comonoid} construction. It is lax monoidal with respect to $\otimes$.
\end{proposition}

\begin{proposition}
The category $\comon$ has a third symmetric monoidal structure $(\yon,\otimes)$, and the functor $u_-\colon(\comon,\yon,\otimes)\to(\poly,\yon,\otimes)$ is strong monoidal.
\end{proposition}



\begin{proposition}[Niu]\label{prop.sysp_is_topos}
For any polynomial $p$, the category
\[\sys(p)\cong\com{K}_p\set\]
of dynamical systems on $p$ forms a topos.
\end{proposition}

\begin{proposition}\label{prop.poly_map_pregeo_topos}
A morphism $p\to q$ of polynomials induces a pre-geometric morphism between their respective toposes
\[
\adj{\sys(p)}{}{}{\sys(q)}.
\]
\end{proposition}

If you skipped over any of that---or all of that---it'll be no problem whatsoever! We will get to everything in the current section, \ref{subsec.math_theory}, over the next few chapters. To briefly summarize the results we have just seen, they say in particular that there is a full-fledged logic of dynamical systems inhabiting any interface (\cref{prop.sysp_is_topos}), that these logics can be combined and compared (\cref{prop.poly_map_pregeo_topos}), and that the whole story of dynamics carries a strong connection to database theory. There are many avenues for study, but we need to push forward.

Let's begin.


%-------- Section --------%
\section{$\poly$ as a category}\label{sec.poly}


In this section, we will set down the basic category-theoretic story of $\poly$, so that we can have a firm foundation from which to speak about dynamical systems, decisions, and data. We begin with the category $\smset$ of sets and functions, and what is arguably the fundamental theorem of category theory, the Yoneda lemma.

%---- Subsection ----%
\subsection{Representable functors and the Yoneda lemma}

\begin{definition}
Given any set $S$, we denote by $\yon^S\colon\smset\to\smset$ the functor that sends any set $X$ to the set $X^S=\smset(S,X)$, and sends any function $h\colon X\to Y$ to the function $h^S\colon X^S\to Y^S$, the one that sends $g\colon S\to X$ to $g\then h\colon S\to Y$.

We refer to $\yon^S$ as the functor \emph{represented by} $S$.
\end{definition}

The symbol $\yon$ stands for Yoneda, for reasons we will get to in \cref{lemma.yoneda}.

\begin{example}
The functor that sends every set $X$ to $X\times X$, and sends $h\colon X\to Y$ to $(h\times h)\colon (X\times X)\to(Y\times Y)$ is representable. 
\end{example}

\begin{exercise}
For each of the following functors $\smset\to\smset$, say if it is representable or not; if it is, say what the representing set is.
\begin{enumerate}
	\item The identity functor $X\mapsto X$.
	\item The constant functor $X\mapsto\2$.
	\item The constant functor $X\mapsto\1$.
	\item The constant functor $X\mapsto\0$.
	\item The functor $X\mapsto X^\nn$.
	\item The functor $X\mapsto 2^X$.
\qedhere
\end{enumerate}
\end{exercise}

\begin{proposition}\label{prop.representable_nt}
For any function $f\colon R\to S$, there is an induced natural transformation $\yon^f\colon\yon^S\to \yon^R$; on any set $X$ the $X$-component $(\yon^f)_X\colon X^S\to X^R$ is given by sending $g\colon S\to X$ to $f\then g\colon R\to X$.
\end{proposition}

\begin{exercise}
Prove that for any function $f\colon R\to S$, what we said was a natural transformation in \cref{prop.representable_nt} really is natural. That is, for any function $h\colon X\to Y$, show that the following diagram commutes:
\[
\begin{tikzcd}
	X^S\ar[r, "h^S"]\ar[d, "X^f"']&
	Y^S\ar[d, "Y^f"]\\
	X^R\ar[r, "h^R"']&
	Y^R\ar[ul, phantom, "?"]
\end{tikzcd}
\qedhere
\]
\end{exercise}

\begin{exercise}
Let $X=\{a,b,c\}$. For each of the following sets $R,S$ and functions $f\colon R\to S$, describe the $X$-component of, i.e.\ the function $X^S\to X^R$ coming from, the natural transformation $\yon^f\colon\yon^S\to\yon^R$.
\begin{enumerate}
	\item $R=\5$, $S=\5$, $f=\id$.
	\item $R=\2$, $S=\1$, $f$ is the unique function.
	\item $R=\1$, $S=\2$, $f(1)=1$.
	\item $R=\1$, $S=\2$, $f(1)=2$.
	\item $R=\0$, $S=\5$, $f$ is the unique function.
	\item $R=\nn$, $S=\nn$, $f(n)=n+1$.
\qedhere
\end{enumerate}
\end{exercise}

\begin{exercise}
Show that the construction in \cref{prop.representable_nt} is functorial
\begin{equation}
\yon^-\colon\smset\op\to\smset^\smset,
\end{equation}
as follows.
\begin{enumerate}
	\item Show that for any set $S$, we have $\yon^{\id_S}\colon\yon^S\to\yon^S$ is the identity.
	\item Show that for any functions $f\colon R\to S$ and $g\colon S\to T$, we have $\yon^g\then\yon^f=\yon^{f\then g}$.
\qedhere
\end{enumerate}
\end{exercise}

\begin{lemma}[Yoneda lemma]\label{lemma.yoneda}
Given a functor $F\colon\smset\to\smset$ and a set $S$, there is an isomorphism
\begin{equation}\label{eqn.yoneda}
F(S)\cong\nat(\yon^S,F)
\end{equation}
where $\nat$ denotes the set of natural transformations. Moreover, \eqref{eqn.yoneda} is natural in both $S$ and $F$.
\end{lemma}
\begin{proof}[Sketch of proof]
For any natural transformation $m\colon\yon^S\to F$, consider the component $m_S\colon S^S\to F(S)$. Applying it to the identity on $S$ as an element of $S^S$, we get an element $m_S(\id_S)\in F(S)$.

For any element $a\in F(S)$, there is a natural transformation $m_a\colon\yon^S\to F$ whose component on $X$ is the function $X^S\to F(X)$ given by sending $g\colon S\to X$ to $F(g)(a)$. In \cref{exc.finish_proof_yoneda} we ask you to show that this is natural in $X$ and that these two constructions are mutually inverse.
\end{proof}

\begin{exercise}\label{exc.finish_proof_yoneda}
Whoever solves this exercise can say they've proved the Yoneda Lemma.
\begin{enumerate}
	\item Show that for any $a\in F(S)$, the maps $X^S\to F(X)$ given as in \cref{lemma.yoneda} are natural in $X$.
	\item Show that the two mappings from \cref{lemma.yoneda} are mutually inverse.
	\item As a corollary of \cref{lemma.yoneda}, show that $\yon^-\colon\smset\op\to\smset^\smset$ is fully faithful, in particular that there is an isomorphism $\nat(\yon^S,\yon^T)\cong S^T$.
\qedhere
\end{enumerate}
\end{exercise}

%---- Subsection ----%
\subsection{Polynomials: sums of representables}

We've seen that for any set $A$, the symbol $\yon^A$ represents a functor $\smset\to\smset$. We will generalize this by adding representable functors together to form polynomials. In some sense the name polynomial doesn't quite fit, because in algebra polynomials are often taken to be finite sums, whereas we will use sums that may be infinite. However, we are not the first by far to use this term.

So here's the deal. All of our polynomials will be polynomials in one variable, $\yon$; every other letter or number that shows up will represent a set.%
\footnote{For those who clamor for polynomials in many variables, we will see in \cref{??} that the multivariable story falls out of the one-variable story.}
For example, in the following crazy polynomial
\begin{equation}\label{eqn.crazy_poly}
p\coloneqq\rr\yon^\zz+\3\yon^{\3}+A\yon+\sum_{i\in I}Q_i\yon^{R_i+Q_i^2},
\end{equation}
$\rr$ denotes the set of real numbers, $\zz$ denotes the set of integers, $\3$ denotes the set $\{1,2,3\}$, and $A$, $I$, $Q_i$, and $R_i$ denote some arbitrary sets that should have already been defined in order for \eqref{eqn.crazy_poly} to make sense.

The polynomials we already understand at this point are the pure-power polynomials $\yon^A$ for some set $A$; they are representable functors $\yon^A\colon\smset\to\smset.$
So to understand general polynomials like $p$, we just need to understand sums of functors. It will be useful to discuss products of functors at the same time. However, we haven't discussed sums and products of sets yet, so we need to do that first.

\paragraph{Dependent sums and products of sets.}

Let $I$ be a set, and let $X_i$ be a set for each $i\in I$. We denote this \emph{$I$-indexed collection of sets} by $X\colon I\to\smset$ or more classically as $(X_i)_{i\in I}$. Choosing one element from one set in the collection would be denoted $(i,x)$ where $x\in X_i$. Choosing an element from each set in the collection would be denoted $(x_i)_{i\in I}$, where each $x_i\in X_i$.

\begin{definition}[Dependent sums and products of sets]
Let $I$ be a set and $X\colon I\to\smset$ be an $I$-indexed collection of sets. The sum $\sum_{i\in I}X_i$ (respectively, the \emph{product} $\prod_{i\in I}X_i$) of this collection is the set
\[
\sum_{i\in I}X_i=\{(i,x)\mid i\in I\text{ and }x\in X_i\}
\qqand
\prod_{i\in I}X_i=\{(x_i)_{i\in I}\mid x_i\in X_i\}.
\]
\end{definition}

\begin{example}\label{ex.two_sums_and_prods}
If $I=\2=\{1,2\}$ then a collection $X\colon I\to \smset$ is just two sets, say $X_1=\{a,b,c\}$ and $X_2=\{c,d\}$. The sum is the disjoint union
\[\sum_{i\in \2}X_i=X_1+X_2=\{(1,a),(1,b),(1,c),(2,c),(2,d)\}.\]
Its number of elements (cardinality) will always be the sum of the numbers of elements in $X_1$ and $X_2$. Similarly,
\[\prod_{i\in \2}X_i=X_1\times X_2=\{(a,c),(a,d),(b,c),(b,d),(c,c),(c,d)\}.\]
\end{example}


\begin{exercise}\label{exc.on_sums_prods_sets}
Let $I$ be a set and let $X_i=\1$ be a one-element set for each $i\in I$. 
\begin{enumerate}
	\item Show that there is an isomorphism of sets $I\cong\sum_{i\in I}\1$.
	\item Show that there is an isomorphism of sets $\1\cong\prod_{i\in I}\1$.
\end{enumerate}
As a special case, suppose $I\coloneqq\varnothing$ and $X\colon \varnothing\to\smset$ is the unique empty collection of sets.
\begin{enumerate}
	\item Is it true that $X_i=\1$ for each $i\in I$?
	\item Show that there is an isomorphism of sets $\0\cong\sum_{i\in\varnothing}X_i$.
	\item Show that there is an isomorphism of sets $\1\cong\prod_{i\in\varnothing}X_i$.
\qedhere
\end{enumerate}
\end{exercise}

Where are we, and where are we going? We've define dependent sums and products of sets; that's where we are. Our goal is to define polynomial functors, e.g.\ $\yon^\2+\2\yon+\1$, and the maps between them. Since $\yon^2$, $\yon$, and $\1$ are functors, we just need to define sums of functors $\smset\to\smset$. But we might as well define products of functors at the same time, because they'll very much come in handy.

\paragraph{Dependent sums and products of functors $\smset\to\smset$.}

\begin{definition}[Dependent sums and products of functors $\smset\to\smset$]\label{def.sum_prod}
For any two functors $F,G\colon\smset\to\smset$, let
\[
  (F+G)\colon\smset\to\smset
  \qqand
  (F\times G)\colon\smset\to\smset
\]
denote the functors that respectively assign to each $X\in\smset$ the sets
\[
  (F+G)(X)\coloneqq F(X)+G(X)
  \qqand
	(F\times G)(X)\coloneqq F(X)\times G(X).
\]
More generally, for any set $I$ and functors $(F_i)_{i\in I}$, let
\[
\sum_{i\in I}F_i\coloneq\smset\to\smset
\qqand
\prod_{i\in I}F_i\coloneq\smset\to\smset
\]
denote the functors that respectively assign to each $X\in\smset$ the sum and product of sets
\[
	\Big(\sum_{i\in I}F_i\Big)(X)\coloneqq\sum_{i\in I} F_i(X)
	\qqand
	\Big(\prod_{i\in I}F_i\Big)(X)\coloneqq\prod_{i\in I} F_i(X).
\]
We denote by $\0,\1\colon\smset\to\smset$ the constant functors, which respectively assign $\0$ and $\1$ to each $X\in\smset$.
\end{definition}

\begin{proposition}
Referring to the notation in \cref{def.sum_prod}, the functors $\0$ and $\1$ are respectively an initial object and a terminal object in $\smset^\smset$, the operations $+$ and $\times$ are respectively a binary coproduct and a binary product in $\smset^\smset$, and the operations $\sum_{i\in I}$ and $\prod_{i\in I}$ are arbitrary coproducts and products in $\smset^\smset$.
\end{proposition}
\begin{proof}
By \cref{ex.two_sums_and_prods,exc.on_sums_prods_sets}, it suffices to show that $\sum_{i\in I}F_i$ and $\prod_{i\in I}F_i$ are a sum and product in $\smset^\smset$. This itself is a special case of a more general fact, where sums and products are replaced by arbitrary colimits and limits, and where $\smset^\smset$ is replaced by an arbitrary functor category $\cat{C}^\cat{D}$, where $\cat{C}$ is a category that (like $\smset$) has limits and colimits; see \cite[page 22 -- 23, displays (24) and (25)]{macLane1992sheaves}.
\end{proof}

We've finally arrived: we can define the category of polynomial functors!

\paragraph{$\poly$, the category of polynomial functors}

\begin{definition}[Polynomial functors]
A \emph{polynomial functor} is a functor $p\colon\smset\to\smset$ such that there exists a set $I$, sets $(p_i)_{i\in I}$, and an isomorphism
\[p\cong\sum_{i\in I}\yon^{p_i}\]
to a sum of representables.
A \emph{morphism of polynomial functors} is simply a natural transformation $p\to q$.
\end{definition}

So (up to isomorphism), a polynomial functor is just a sum of representables.

\begin{example}\label{ex.pedantic_poly_eval}
Consider the polynomial $\yon^\2+\2\yon+\1$. It denotes a functor $\smset\to\smset$; what does this functor do to the set $X\coloneqq\{a,b\}$? 
To be very precise and pedantic, let's say
\[
I\coloneqq\4\qqand
  p_1\coloneqq\2,\quad
  p_2\coloneqq\1,\quad
  p_3\coloneqq\1,\quad
  p_4\coloneqq\0\
\]
so that $p\cong\sum_{i\in I}\yon^{p_i}$. Now we have
\[
p(X)\cong
\{(1,a,a),(1,a,b),(1,b,a),(1,b,b),(2,a),(2,b),(3,a),(3,b),(4)\}.
\]
It has $(2^2+2+2+1)$-many, i.e.\ $9$, elements. The representable summand $\yon^A$ throws in all $A$-tuples from $X$, but it's indexed by the name of the summand. In particular if $A=\varnothing$ then it just records the empty tuple at that summand.
\end{example}

\begin{exercise}
In the pedantic style of \cref{ex.pedantic_poly_eval}, write out all the elements of $p(X)$ for $p$ and $X$ as follows:
\begin{enumerate}
	\item $p\coloneqq\yon^\3$ and $X\coloneqq\{4,9\}.$
	\item $p\coloneqq\3\yon^\2+\1$ and $X\coloneqq\{a\}$.
	\item $p\coloneqq\0$ and $X\coloneqq\nn$.
	\item $p\coloneqq\4$ and $X\coloneqq\nn$.
	\item $p\coloneqq\yon$ and $X\coloneqq\nn$.
\qedhere
\end{enumerate}
\end{exercise}

\begin{example}
Suppose $p\coloneqq\yon^\2+\2\yon+1$. As a functor $\smset\to\smset$ it should be able to act not only on sets but on functions. Let $X\coloneqq\{a_1,a_2,b_1\}$, $Y\coloneqq\{a,b,c\}$ and $f\colon X\to Y$ the function sending $a_1,a_2\mapsto a$ and $b_1\mapsto b$. The induced function $p(f)\colon p(X)\to p(Y)$ is shown below
\[
\begin{tikzcd}[sep=3pt, shorten <=-5pt, shorten >=-5pt, dashed]
\LMO{(1,a_1,a_1)}\ar[rrrr, blue, bend left=25pt]&\LMO{(1,a_1,a_2)}\ar[rrr, blue, bend left=25pt]&\LMO{(1,a_1,b_1)}\ar[rrr, blue, bend left=25pt]&[30pt]&
\LMO{(1,a,a)}&\LMO{(1,a,b)}&\LMO{(1,a,c)}&&
\\
\LMO{(1,a_2,a_1)}\ar[rrrru, blue, bend left=10pt]&\LMO{(1,a_2,a_2)}\ar[rrru, blue, bend left=10pt]&\LMO{(1,a_2,b_1)}\ar[rrru, blue, bend left=10pt]&&
\LMO{(1,b,a)}&\LMO{(1,b,b)}&\LMO{(1,b,c)}&&
\\
\LMO{(1,b_1,a_1)}\ar[rrrru, blue, bend left=10pt]&\LMO{(1,b_1,a_2)}\ar[rrru, blue, bend left=10pt]&\LMO{(1,b_1,b_1)}\ar[rrru, blue, bend left=10pt]&&
\LMO{(1,c,a)}&\LMO{(1,c,b)}&\LMO{(1,c,c)}&&
\\
\LMO{(2,a_1)}\ar[rrrr, blue, bend left=25pt]&\LMO{(2,a_2)}\ar[rrr, blue, bend left=25pt]&\LMO{(2,b_1)}\ar[rrr, blue, bend left=25pt]&&
\LMO{(2,a)}&\LMO{(2,b)}&\LMO{(2,c)}&&
\\
\LMO{(3,a_1)}\ar[rrrr, blue, bend left=25pt]&\LMO{(3,a_2)}\ar[rrr, blue, bend left=25pt]&\LMO{(3,b_1)}\ar[rrr, blue, bend left=25pt]&&
\LMO{(3,a)}&\LMO{(3,b)}&\LMO{(3,c)}&&
\\&
\LMO{(4)}\ar[rrrr, blue]&&&&\LMO{(4)}
\end{tikzcd}
\]
\end{example}

\begin{exercise}
Let $p=\yon^\2+\yon$. Choose a function $f\colon\1\to\2$ and write out the induced function $p(f)\colon p(\1)\to p(\2)$.
\end{exercise}

\begin{proposition}\label{prop.apply1}
Let $p\coloneqq\sum_{i\in I}\yon^{p_i}$ be an arbitrary polynomial functor. Then $I\cong p(\1)$, so there is an isomorphism of functors
\begin{equation}\label{eqn.sum_p1}
p\cong\sum_{i\in p(\1)}\yon^{p_i}.
\end{equation}
\end{proposition}
\begin{proof}
We need to show that $I\cong p(\1)$; the latter claim follows directly. In \cref{exc.on_sums_prods_sets} it was shown that $I\cong\sum_{i\in I}\1$, so we just need to show that $(\yon^A)(\1)\cong\1$ for any $A\in\smset$. But $1^A\cong 1$ because there is a unique function $A\to 1$ for any $A$.
\end{proof}

The notation in \eqref{eqn.sum_p1} will be how we denote arbitrary polynomials from now on.

\begin{exercise}\label{exc.apply0}
We saw in \cref{prop.apply1} that for any polynomial $p$, e.g.\ $p=\yon^3+3\yon^2+4$, the set $p(\1)$ gives back the set of summands, in this case $\8$. 

What does $p(\0)$ give you?
\end{exercise}

%---- Subsection ----%
\subsection{Morphisms between polynomial functors}

For a seasoned category theorist, we've already said enough to calculate the set $\poly(p,q)$ of morphisms between polynomials $p$ and $q$: it's the set of natural transformations between $p$ and $q$, which are sums of representable functors. But it is worth going over this subject at a more leisurely pace, because it helps us insert some intuition that will be very useful later.

In particular, we're going to think about polynomials like $p$ and $q$ using the intuition from the beginning of the chapter, around the table in \eqref{eqn.table_terminology}. We said that we'd think of $p$ as a wheelhouse: a set of decisions you can make and what options you have. That is, $p(\1)$ is conceived as a set of decisions, and for every decision $i\in p(\1)$, we conceive of the set $p_i$ as the options or \emph{possibilities} available in that decision. Here's an alternative way to write $p$:
\[
\begin{tikzcd}
p(\1)\ar[r, "p_-"]&\smset
\end{tikzcd}
\]
Again, every decision $i\in p(\1)$ gets a set $p_i$ of possibilities: that's the whole content of $p$.

The maps $f\colon p\to q$ of polynomials are then the ways to \emph{delegate} $p$'s decisions to $q$. Every one of $p$'s decisions, say $i\in p(\1)$, is passed forward to a decision $j\in q(\1)$ for $q$ to make, and every choice $d\in q_j$ that $q$ could make among its possibilities is passed back as some choice $c\in p_i$ among $p$'s possibilities.

\begin{agda}
record Delegation (w : Wheelhouse) (w' : Wheelhouse) : Set where
   field 
     passfwd : (decn p) -> (decn q)
     passbck : (d : decn p) -> posy q (passfwd d) -> posy p d
\end{agda}

Let's state this mathematically and prove it.

\begin{proposition}\label{prop.poly_maps_colax_prod_sum}
Let $p\coloneqq\sum_{i\in p(\1)}\yon^{p_i}$ and $q\coloneqq\sum_{j\in q(\1)}\yon^{q_j}$ be polynomials. A morphism $p\to q$ can be identified with a pair $(f_1,f^\sharp)$
\begin{equation}\label{eqn.colax_poly_map}
\begin{tikzcd}[column sep=small]
	p(\1)\ar[dr, "p_-"']\ar[rr, "f_1"]&~&
	q(\1)\ar[dl, "q_-"]\\&
	\smset\ar[u, phantom, near end, "\overset{f^\sharp}{\Leftarrow}"]
\end{tikzcd}
\end{equation}
where $f_1$ is a function (or functor between discrete categories) and $f^\sharp$ is a natural transformation: for each $i\in p(\1)$ with $j\coloneqq f_1(i)$, there is a function $f^\sharp_i\colon q_j\to p_i$.

To summarize, we have an isomorphism
\begin{equation}\label{eqn.poly_maps_prod_sum}
\poly(p,q)\cong\prod_{i\in p(\1)}\sum_{j\in q(\1)}{p_i}^{q_j}.
\end{equation}
\end{proposition}
\begin{proof}
We have the following isomorphisms
\begin{align*}
  \poly(p,q)&=
  \nat\Big(\sum_{i\in p(\1)}\yon^{p_i},\;\sum_{j\in q(\1)}\yon^{q_j}\Big)&
  	\text{Definitions}\\&\cong
  \prod_{i\in p(\1)}\nat\Big(\yon^{p_i},\sum_{j\in q(\1)}\yon^{q_j}\Big)&
  	\text{Univ. prop. of coproduct}\\&\cong
  \prod_{i\in p(\1)}\sum_{j\in q(\1)}{p_i}^{q_j}&
  	\text{Yoneda lemma}
\end{align*}
This proves what we called the summary. It says that a morphism $p\to q$ consists of: for every $i\in p(\1)$ both a choice of $j\in q(\1)$ and a function from $q_j\to p_i$; the first part determines $f_1$ (i.e.\ $f_1(i)=j$) and the second part just gives $f^\sharp_i$ directly for each $i$.
\end{proof}

\eqref{eqn.poly_maps_prod_sum} will be quite important to us; we will call it our \emph{main formula} for the set of polynomial maps.

\paragraph{Working with $\sum$ and $\prod$ formulas.}
Here's how to understand and work with a set that involves $\sum$'s and $\prod$'s, e.g.\ something like
\begin{equation}\label{eqn.misc98339}
\sum_{i\in I}\prod_{j\in J(i)}\sum_{k\in K(i,j)}X(i,j,k).
\end{equation}
To understand it, you ask ``what is involved in choosing an element of this set?'' 

The answer given by the formula \eqref{eqn.misc98339} begins by saying ``To choose an element of \eqref{eqn.misc98339}, first..." and then you begin pronouncing the formula. Every time you see $\sum_{i\in I}...$, say ``choose an element $i\in I$; then, ...''. Every time you see $\prod_{j\in J}$, say ``for each $j\in J$, ...''. At the very end when you see some $X$, say ``conclude by choosing an element of $X$''.

So the formula in \eqref{eqn.misc98339} would be pronounced as follows:
\begin{quote}
To choose an element of \eqref{eqn.misc98339}, first choose an element $i\in I$. Then, for every $j\in J(i)$, choose an element $k\in K(i,j)$; then, conclude by choosing an element of $X(i,j,k)$.
\end{quote}
Note that the choice of $k\in k(i,j)$ can depend on $i$ and $j$; it must be able to because these sets $K(i,j)$ can be different for different $i,j$.

\begin{example}[Notation for $\sum\prod$ stuff]\label{ex.notation_sum_prod}
Here we give notation for the elements of a set involving $\sum$'s and $\prod$'s such as that in \eqref{eqn.misc98339}.

Let $I=\{1,2\}$, let $J(1)=\{j\}$ and $J(2)\coloneqq\{j,j'\}$, let $K(1,j)\coloneqq\{k_1,k_2\}$, $K(2,j)\coloneqq\{k_1\}$, and $K(2,j')\coloneqq\{k'\}$, and let $X(i,j,k)=\{x,y\}$ for all $i,j,k$. Now the formula 
\[\sum_{i\in I}\prod_{j\in J(i)}\sum_{k\in K(i,j)}X(i,j,k)\]
 has been given meaning as an actual set. Here is a list of all eight of its elements:
\[
\left\{
\begin{gathered}
	(1, j\mapsto(k_1,x)),\qquad
	(1, j\mapsto(k_1,y)),\qquad
	(1, j\mapsto(k_2,x)),\qquad
	(1, j\mapsto(k_2,y)),\\
	(2, j\mapsto(k_1,x), j'\mapsto(k_1,x)),\qquad
	(2, j\mapsto(k_1,x), j'\mapsto(k_1,x)),\\
	(2, j\mapsto(k_1,x), j'\mapsto(k_1,x)),\qquad
	(2, j\mapsto(k_1,x), j'\mapsto(k_1,x))
\end{gathered}
\right\}
\]
In each case, we first chose an element $i\in I$, either 1 or 2. Then for each $j\in J(i)$ we chose an element $k\in K(i,j)$; then we concluded by choosing an element of $X(i,j,k)$.
\end{example}

\begin{exercise}
With $I$, $J$, $K$, and $X$ as in \cref{ex.notation_sum_prod}, consider the set
\[\prod_{i\in I}\sum_{j\in J(i)}\prod_{k\in K(i,j)}X(i,j,k).\]
\begin{enumerate}
	\item Pronounce it using the ``To choose an element'' formulation just below \eqref{eqn.misc98339}.
	\item How many elements are in it?
	\item Write out three of them.
\qedhere
\end{enumerate}
\end{exercise}

The following is sometimes called the \emph{type-theoretic axiom of choice} or the \emph{completely distributive property}, in this case of $\smset$.

\begin{proposition}[Pushing $\prod$ past $\sum$]
For any set $I$, collection of sets $\{J(i)\}_{i\in I}$ and collection of sets $\{X(i,j)\}_{i\in i, j\in J(i)}$, we have a bijection
\begin{equation}\label{eqn.set_completely_distributive}
\prod_{i\in I}\sum_{j\in J(i)}X(i,j)\cong\sum_{j\colon\prod_{i\in I}J(i)}\;\prod_{i\in I}X(i,j).
\end{equation}
\end{proposition}
\begin{proof}
This is almost trivial, once you understand what it's saying; for example once the statement is written, the proof is one line of agda code. An element of the left-hand side is, for each $i\in I$ a choice  of $j_i\in J(i)$ and an element $x_{i,j_i}\in X(i,j_i)$. An element of the right-hand side is a choice of, for each $i\in I$ an element $j(i)\in J(i)$, together with, for each $i$ in $I$ an element $x_{i,j(i)}\in X(i,j(i))$. These are in obvious bijection.
\end{proof}
Below, e.g.\ in \cref{exc.practice_sum_prod}, you'll often see alternating sums and products; using \cref{eqn.set_completely_distributive}, you can always put it into sums-then-products (``disjunctive normal'') form.

When $J(i)=J$ does not depend on $i\in I$, the formula in \eqref{eqn.set_completely_distributive} is much easier:
\[
\prod_{i\in I}\sum_{j\in J}X(i,j)\cong\sum_{j\colon I\to J}\;\prod_{i\in I}X(i,j).
\]

\paragraph{Back to maps of polynomials}
After that interlude, you are hopefully more comfortable with our main formula \cref{eqn.poly_maps_prod_sum} describing the set of morphisms between two arbitrary polynomial functors. Let's practice.

\begin{exercise}\label{exc.practice_sum_prod}
Let $p,q$ be polynomials as in \cref{prop.poly_maps_colax_prod_sum}. 
\begin{enumerate}
\item Is it true that the following are isomorphic?
\[
  \prod_{i\in p(\1)}\sum_{j\in q(\1)}{p_i}^{q_j}
  \cong^?
  \prod_{i\in p(\1)}\sum_{j\in q(\1)}\prod_{d\in q_j}\sum_{c\in p_i}\1
\]
\item Here's another way to think of maps in $\poly$; it'll be useful in \cref{exc.mono_epi_poly}. Show that there is an isomorphism
	\[
	\poly(p,q)\cong\sum_{f_1\colon p(\1)\to q(\1)}\prod_{j\in q(\1)}\smset\Big(q_j,\prod_{\{i\in p(\1)\mid i\mapsto j\}}p_i\Big).
	\]
	Say how any element of the right-hand side gives a way of delegating decisions from $p$ to $q$.
\end{enumerate}
\end{exercise}

\begin{example}[Constants are sent to constants]
Any morphism $p\to q$ must send constants in $p$ to constants in $q$. So if $p$ has constant terms and $q$ doesn't, e.g.\ if $p=\yon^\2+\1$ and $q=\yon^3$, then there are no maps $p\to q$. 

Indeed, let's more generally say $p\coloneqq\sum_{i\in p(\1)}\yon^{p_i}$ and $q\coloneqq\sum_{j\in q(\1)}\yon^{q_j}$ are polynomials, and that we have a map $p\to q$, i.e.\ an element of the set $\prod_{i\in p(\1)}\sum_{j\in q(\1)}{p_i}^{q_j}$. If there is some $i_0\in p(\1)$ such that $p_{i_0}=\0$, then the formula says that for $i_0$ our map has chosen a $j\in q(1)$ and a function $q_j\to p_i$. But in order for such a function to exist, if $p_i$ is empty, $q_j$ must be too.
\end{example}

\begin{exercise}
For each of the following $p,q$, count the number of morphisms $p\to q$:
\begin{enumerate}
	\item $p=\yon^3$, $q=\yon^4$.
	\item $p=\yon^3+1$, $q=\yon^4$.
	\item $p=\yon^3+1$, $q=\yon^4+1$.
	\item $p=4\yon^3+3\yon^2+\yon$, $q=\yon$.
	\item $p=4\yon^3$, $q=3\yon$.
\qedhere
\end{enumerate}
\end{exercise}

\begin{exercise}\label{exc.mono_epi_poly}
Suppose $p,q\in\poly$ are polynomials and $f=(f_1,f^\sharp)\colon p\to q$ is a morphism with notation as in \cref{eqn.colax_poly_map}.
\begin{enumerate}
	\item Under what conditions is $f$ a monomorphism?
	\item Under what conditions is $f$ an epimorphism? Hint: use \cref{exc.practice_sum_prod}, \#2.
	\item Suppose $f$ is both a mono and an epi; it is an iso? (That is, is $\poly$ \emph{balanced}?)
	\item Can every morphism in $\poly$ be factored as an epic followed by a monic?
\qedhere
\end{enumerate}
\end{exercise}

%---- Subsection ----%
\subsection{Adjunctions and special polynomials}

Discuss adjunctions with $\smset$ and $\smset\op$, two-variable adjunctions, and adjunctions with pointed sets op. Discuss constant, linear, affine, pure-power polynomials, monomials as lenses.

We saw in \cref{exc.apply0} that the constant terms of a polynomial $p$ are given by $p(\0)$. It turns out that's part of an adjunction.

\begin{proposition}
For any set $B$ there is an adjunction
\[
\adj{\smset}{A\yon^B}{p(B)}{\poly}
\]
where the functors are labeled by where they send $p\in\poly$ and $A\in\smset$.

When $B=\0$ or $B=\1$, the functor $A\yon^B$ is fully faithful.
\end{proposition}
\begin{proof}
For any sets $A,B\in\smset$ and polynomial $q\sum_{j\in q(\1)}\yon^{q_j}$, we need to check that there is a natural isomorphism
\[
\smset(A,q(B))\cong\poly(A\yon^B,q).
\]
This follows from our main formula \cref{eqn.poly_maps_prod_sum}, since setting $p\coloneqq A\yon^B$ we have $p(\1)\cong A$ and $q(B)=\sum_{j\in q(\1)}B^{q_j}$.
\end{proof}

%---- Subsection ----%
\subsection{Limits and colimits in $\poly$}


%-------- Section --------%
\section{Dynamical systems}

%-------- Section --------%
\section{Wiring together dynamical systems}

%---- Subsection ----%
\subsection{Dirichlet product}


















\end{document}
