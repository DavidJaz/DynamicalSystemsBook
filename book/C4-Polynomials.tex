\documentclass[DynamicalBook]{subfiles}
\begin{document}
%


\setcounter{chapter}{3}%Just finished 3.


%------------ Chapter ------------%
\chapter{Polynomial functors and dynamics}

\slogan{
The category of polynomial functors is a jackpot. Its beauty flows without bound.}

%-------- Section --------%
\section{Introduction}


In this chapter we will investigate a remarkable category called $\poly$. We will see its intimate relationship with both dynamic processes, data storage and transformations, and decision-making. We will see its intimate relationship with safety and decision-making. But our story begins with something quite humble: middle school algebra.
\begin{align}\label{eqn.polynomial1892}
\yon^2+\yon+1 \quad&\quad
\textit{polynomial}
\intertext{
All our polynomials will involve one variable, $\yon$, chosen for reasons we'll explain soon. Polynomials in one variable can be depicted as a set of mini-trees:
}
\label{eqn.poly1892}
\begin{tikzpicture}[trees, grow'=up]
  \node (1) {$\bullet$} 
    child {}
    child {};
  \node[right=.5 of 1] (2) {$\bullet$} 
    child {};
  \node[right=.5 of 2] (3) {$\bullet$};
\end{tikzpicture}
\quad&\quad\textit{poly}
\end{align}
More technically, mini-trees are called \emph{corollas}, so what we're calling a \emph{poly} is a set of corollas. 

Intuitively, one might think of each corolla as representing a \emph{decision}. Associated to every decision is a set of \emph{options}. The three decisions we exhibit in \cref{eqn.poly1892} are particularly interesting; they respectively have two options, one options, and no options. Having two options is familiar from life---it's the classic yes/no decision---as well as from Claude Shannon's Information Theory. Having one option is also familiar theoretically and in life: ``sorry, ya just gotta go through it.'' Having no options is when you actually don't get through it: it's ``the end''. While the corollas $1,\yon,$ and $\yon^2$ are each interesting as decisions, their sum $\yon^2+\yon+1$ has very little theoretical interest; it's just a first example.

In a polynomial like $42\yon^3+17\yon$, the pure power term $\yon^3$ has a coefficient of 42 next to it, so the corolla with three leaves shows up 42 times in the associated poly. Intuitively, there are 42 situations in which you have a decision with three options.

\[
\begin{tabular}{llll}
\textbf{Overarching} & \textbf{Polynomial} $p$ & \textbf{Poly} & \textbf{Intuitive}\\\hline
Set of positions & $p(1)$ & Set of roots & Set of situations\\
Interface in position $i$ & Pure power summand $\yon^{p_i}$ & Corolla & Decision\\
Distinction $d\in p_i$ & Element of $p_i$ & Leaf & Option
\end{tabular}
\]

\begin{exercise}
Consider the polynomial $2\yon^3+2\yon+1$ and the associated poly.
\begin{enumerate}
	\item Draw the poly.
	\item How many roots does this poly have?
	\item How many decisions does this represent?
	\item For each corolla in the poly, say how many leaves it has.
	\item For each decision, how many options does it have?
	\item Does the polynomial $\yon^\nn+4\yon$ have a representable summand $\yon^2$?
	\qedhere
\end{enumerate}

\end{exercise}

Mathematically and notationally we do not make a distinction between a poly and its name $p$; they are two different syntaxes for the same object. 


\begin{exercise}
If you were a suitor choosing the poly you love, aesthetically speaking, which would strike your interest? Answer by circling the associated polynomial:
\begin{enumerate}
	\item $\yon^2+\yon+1$
	\item $\yon^2+3\yon^2+3\yon+1$
	\item $\yon^2$
	\item $\yon+1$
	\item $(\nn\yon)^\nn$
	\item $S\yon^S$
	\item $\yon^{100}+\yon^2+3\yon$
	\item Your poly's name $p$ here.
\end{enumerate}
Any reason for your choice?
\end{exercise}

But before we get into the details, let's say where we're going. We're going to use polynomials to think about dynamics, data, and decision. We made very big promises at the beginning of this section, and we have not yet begun to deliver.

%---- Subsection ----%
\subsection{Dynamical systems}

We've seen dynamical systems throughout this book

%---- Subsection ----%
\subsection{Data}

%---- Subsection ----%
\subsection{Decision theory}


Here's a pretty thing we're getting to: the decision stream.
\begin{equation}\label{eqn.trees_comp}
\begin{tikzpicture}[trees, grow'=up]
	\node (p3) {
  \begin{tikzpicture}[trees, grow'=up]
    \node (a) {$\bullet$}[sibling distance=.75cm] 
      child {[fill]
      	node[left=-3pt] {$\bullet$}[sibling distance=.3cm]
				  child {node (x) {}} child 
			}
      child {[fill]
      	node[left=-3pt] {$\bullet$}[sibling distance=.3cm]
				  child child
  		};
    \node (b) [right=1.5 of a] {$\bullet$} 
      child {[fill]
      	node[left=-3pt] {$\bullet$}[sibling distance=.3cm]
				  child child
			}
      child {[fill]
      	node[left=-3pt] {$\bullet$}
  		};
    \node (c) [right=1 of b]{$\bullet$}
      child {[fill]
      	node[left=-3pt] {$\bullet$}
			}
      child {[fill]
      	node[left=-3pt] {$\bullet$}[sibling distance=.3cm]
				  child child
  		};
    \node (d) [right=1 of c]{$\bullet$} 
      child {[fill]
      	node[left=-3pt] {$\bullet$}
			}
      child {[fill]
      	node[left=-3pt] {$\bullet$}
  		};		
     \node (e) [right=1 of d] {$\bullet$};
    \node[draw, fill=yellow!80!black, opacity=0.2, rounded corners=5pt, inner sep=5pt, fit=(x) (e)] (y) {};	
  \end{tikzpicture}
   };
\end{tikzpicture}
\end{equation}

%---- Subsection ----%
\subsection{Implementation}

%---- Subsection ----%
\subsection{Mathematical theory}

%-------- Section --------%
\section{Dynamical systems}



\end{document}
