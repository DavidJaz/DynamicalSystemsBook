\documentclass[DynamicalBook]{subfiles}
\begin{document}
%


\setcounter{chapter}{3}%Just finished 3.


%------------ Chapter ------------%
\chapter{Polynomial functors and dynamics}

\slogan{
The category of polynomial functors is a jackpot. Its beauty flows without bound.}

%-------- Section --------%
\section{Introduction}


In this chapter we will investigate a remarkable category called $\poly$. We will see its intimate relationship with both dynamic processes, data storage and transformations, and decision-making. We will see its intimate relationship with safety and decision-making. But our story begins with something quite humble: middle school algebra.
\begin{align}\label{eqn.polynomial1892}
\yon^\2+\yon+\1 \quad&\quad
\textit{polynomial}
\intertext{
All our polynomials will involve one variable, $\yon$, chosen for reasons we'll explain soon. Polynomials in one variable can be depicted as a set of mini-trees:
}
\label{eqn.poly1892}
\begin{tikzpicture}[trees, grow'=up]
  \node (1) {$\bullet$} 
    child {}
    child {};
  \node[right=.5 of 1] (2) {$\bullet$} 
    child {};
  \node[right=.5 of 2] (3) {$\bullet$};
\end{tikzpicture}
\quad&\quad\textit{poly}
\end{align}
More technically, mini-trees are called \emph{corollas}, so what we're calling a \emph{poly} is a set of corollas. 

Intuitively, one might think of each corolla as representing a \emph{decision}. Associated to every decision is a set of \emph{possibilities}. The three decisions we exhibit in \cref{eqn.poly1892} are particularly interesting; they respectively have two possibilities, one possibility, and no possibilities. Having two possibilities is familiar from life---it's the classic yes/no decision---as well as from Claude Shannon's Information Theory. Having one possibility is also familiar theoretically and in life: ``sorry, ya just gotta go through it.'' Having no possibilities is when you actually don't get through it: it's ``the end''. While the corollas $\1,\yon,$ and $\yon^\2$ are each interesting as decisions, their sum $\yon^\2+\yon+\1$ has very little theoretical interest; it's just a first example.

In a polynomial like $\4\2\yon^\3+\1\7\yon$, the pure power term $\yon^\3$ has a coefficient of \4\2 next to it, so the corolla with three leaves shows up 42 times%
\footnote{
In standard font, 42 represents the usual natural number. In sans serif font \4\2 represents the set $\4\2=\{1,2,\ldots,42\}$ with 42 elements.
}
in the associated poly. Intuitively, there are 42 situations in which you have a decision with three possibilities.

\[
\begin{tabular}{llll}
\multicolumn{4}{c}{\large Terminology}\\
\textbf{Overarching} & \textbf{Polynomial} $p$ & \textbf{Poly} & \textbf{Chapter-intuition}\\\hline
Set of positions & $p(\1)$ & Set of roots & Set of situations\\
Interface in position $\fun{i}$ & Pure power summand $\yon^{p_\fun{i}}$ & Corolla & Decision\\
Distinction $\fun{d}\in p_\fun{i}$ & Element of $p_\fun{i}$ & Leaf & Possibility
\end{tabular}
\]

If you're wondering about the upright $\fun{i}$, we'll get to all our notational conventions in \cref{sec.poly}.

\begin{exercise}
Consider the polynomial $p\coloneqq\2\yon^\3+\2\yon+\1$ and the associated poly.
\begin{enumerate}
	\item Draw the poly whose name is $p$.
	\item How many roots does this poly have?
	\item How many decisions does this represent?
	\item For each corolla in the poly, say how many leaves it has.
	\item For each decision, how many possibilities does it have?
\end{enumerate}
Now referring instead to the polynomial $q\coloneqq\yon^\nn+\4\yon$:
\begin{enumerate}[resume]
	\item Does the polynomial $q$ have a pure-power summand $\yon^\2$?
	\item Does the polynomial $q$ have a pure-power summand $\yon$?
	\item Does the polynomial $q$ have a pure-power summand $\4\yon$?
	\qedhere
\end{enumerate}

\end{exercise}

Mathematically and notationally we do not make a distinction between a poly and its name $p$; they are two different syntaxes for the same object. 


\begin{exercise}
If you were a suitor choosing the poly you love, aesthetically speaking, which would strike your interest? Answer by circling the associated polynomial:
\begin{enumerate}
	\item $\yon^\2+\yon+\1$
	\item $\yon^\2+\3\yon^\2+\3\yon+\1$
	\item $\yon^\2$
	\item $\yon+\1$
	\item $(\nn\yon)^\nn$
	\item $S\yon^S$
	\item $\yon^{\1\0\0}+\yon^\2+\3\yon$
	\item Your poly's name $p$ here.
\end{enumerate}
Any reason for your choice?
\end{exercise}

Before we can really get into this story, let's summarize where we're going: polynomials are going to have really surprising applications to dynamics, data, and decision. Remember, we spoke superlatively of $\poly$ at the beginning of this chapter,
\slogan{
The category of polynomial functors is a jackpot. Its beauty flows without bound}
and we have not yet begun to deliver. So let's introduce some of the applications and mathematics to come.

%---- Subsection ----%
\subsection{Dynamical systems}

Throughout this book, we've seen dynamical systems---machines of various sorts---which have an internal state that can be read out to other systems, as well as updated based on input received from other systems. In the context of this chapter, we'll be looking mainly at deterministic systems, but with a lot of interesting new possibilities:
\begin{enumerate}
	\item The interface of the system can change shape through time.
	\item The wiring diagram connecting a bunch of systems can change in time.
	\item One can speed up the dynamics of a system.
	\item One can introduce ``effects'', i.e.\ as defined by monads on $\smset$.
	\item The dynamical systems on any interface form a topos.
\end{enumerate}

To give some intuition for the first two, imagine yourself as a system, wired up to other systems. You have some input ports: your eyes, your ears, etc., and you have some output ports: your speech, your gestures, etc. And you connect with other systems: your family, your colleagues, the GPS of your phone, etc.
\begin{equation}\label{eqn.wired_forever}
\begin{tikzpicture}[oriented WD, bb min width =.5cm, bbx=.5cm, bb port sep =1,bb port length=0, bby=.15cm]
	\node[bb={2}{2}, green!25!black] (X11) {\tiny Alice};
	\node[bb={3}{3}, green!25!black, below right=of X11] (X12) {\tiny you};
	\node[bb={2}{1}, green!25!black, above right=of X12] (X13) {\tiny Bob};
	\node[bb={2}{2}, green!25!black, below right = -1 and 1.5 of X12] (X21) {\tiny GPS};
	\node[bb={1}{2}, green!25!black, above right=-1 and 1 of X21] (X22) {\tiny me};
  \node[bb={2}{2}, fit = {($(X11.north east)+(-1,4)$) (X12) (X13) ($(X21.south)$) ($(X22.east)+(.5,0)$)}, bb name = {\small Wired together like this forever?}] (Z) {};
	\draw (X21_out1) to (X22_in1);
	\draw let \p1=(X22.north east), \p2=(X21.north west), \n1={\y1+\bby}, \n2=\bbportlen in
          (X22_out1) to[in=0] (\x1+\n2,\n1) -- (\x2-\n2,\n1) to[out=180] (X21_in1);
	\draw (X11_out1) to (X13_in1);
	\draw (X11_out2) to (X12_in1);
	\draw (X12_out1) to (X13_in2);
	\draw (Z_in1'|-X11_in2) to (X11_in2);	
	\draw (Z_in2'|-X12_in2) to (X12_in2);
	\draw (X12_out2) to (X21_in2);
	\draw (X21_out2) to (Z_out2'|-X21_out2);
	 \draw let \p1=(X12.south east), \p2=(X12.south west), \n1={\y1-\bby}, \n2=\bbportlen in
	  (X12_out3) to[in=0] (\x1+\n2,\n1) -- (\x2-\n2,\n1) to[out=180] (X12_in3);
	\draw let \p1=(X22.north east), \p2=(X11.north west), \n1={\y2+\bby}, \n2=\bbportlen in
          (X22_out2) to[in=0] (\x1+\n2,\n1) -- (\x2-\n2,\n1) to[out=180] (X11_in1);
	\draw (X13_out1) to (Z_out1'|-X13_out1);
\end{tikzpicture}
\end{equation}
We wrote a little question for you at the top of the diagram. Isn't there something a little funny about wiring diagrams? Maybe for old-fashioned machines, you would wire things together once and they'd stay like that for the life of the machine. But my phone connects to different wifi stations at different times, I drop my connection to Alice for weeks at a time, etc. So wiring diagrams should be able to change in time; $\poly$ will let us do that.

\begin{example}
Here are some familiar circumstances where we see wiring diagrams changing in time.
\begin{enumerate}[itemsep=0pt]
%	\item Airplanes only communicate when they get near enough;
%	\item A phone is connected to 4G or to wifi depending on circumstances;
%	\item A person can choose when to open (receive input through) their eyes and when to speak (produce output);\goodbreak
	\item When too much force is applied to a material, bonds can break;
\end{enumerate}
\[
\begin{tikzpicture}[oriented WD, bb small, bb port length=0]
	\foreach \i in {0,...,4} {
		\node[bb={1}{1}, fill=blue!10] at (1.7*\i,0) (X\i) {};
	}
%	\node[bb={1}{1}, fit=(X0) (X4)] (X) {};
	\foreach \i in {0,...,3} {
		\draw[thick] (X\i_out1) -- (X\the\numexpr\i+1\relax_in1);
	};
	\draw[thick, ->] (X0_in1) -- node[above, font=\tiny] {Force} +(-2.5,0);
	\draw[thick, ->] (X4_out1) -- node[above, font=\tiny] {Force} +(2.5,0) node (R) {};
%
\def\x{21};
	\foreach \i in {0,...,2} {
		\node[bb={1}{1}, fill=blue!10] at (\x+1.7*\i,0) (Y\i) {};
	}
	\foreach \i in {3,...,4} {
		\node[bb={1}{1}, fill=blue!10] at (\x+1.3+1.7*\i,0) (Y\i) {};
	}
%	\node[bb={1}{1}, fit=(Y0) (Y4)] (Y) {};
	\foreach \i in {0,1,3} {
		\draw[thick] (Y\i_out1) -- (Y\the\numexpr\i+1\relax_in1);
	};
	\draw[thick, ->] (Y0_in1) -- node[above, font=\tiny] {Force} +(-2.5,0) node (L) {};
	\draw[thick, ->] (Y4_out1) -- node[above, font=\tiny] {Force} +(2.5,0);
	\node[starburst, draw, minimum width=2cm, minimum height=1.5cm,red,fill=orange,line width=1.5pt] at ($(L)!.5!(R)$)
{Snap!};
\end{tikzpicture}
\]
\begin{quote}
In materials science the Young's modulus accounts for how much force can be transferred across a material as its endpoints are pulled apart. When the material breaks, the two sides can no longer feel evidence of each other. Thinking of pulling as sending a signal (a signal of force), we might say that the ability of internal entities to send signals to each other---the connectivity of the wiring diagram---is being measured by Young's modulus. It will also be visible within $\poly$.
\end{quote}
\begin{enumerate}[resume]
	\item A company may change its supplier at any time;
\end{enumerate}
\begin{equation*}%\label{eqn.supplier}
\begin{tikzpicture}[oriented WD, font=\ttfamily, every node/.style={fill=blue!10}, baseline=(c)]
	\node[bb={0}{1}] (s1) {Supplier 1};
	\node[bb={0}{1}, below=of s1] (s2) {Supplier 2};
	\coordinate (helper) at ($(s1)!.5!(s2)$);
	\node[bb={1}{0}, right=1.5 of helper] (c) {Company};
	\draw (s1_out1) to (c_in1);
	\draw (s2_out1) to +(5pt,0) node[fill=none] {$\bullet$};
\begin{scope}[xshift=3.5in]
	\node[bb={0}{1}] (s1') {Supplier 1};
	\node[bb={0}{1}, below=of s1'] (s2') {Supplier 2};
	\coordinate (helper') at ($(s1')!.5!(s2')$);
	\node[bb={1}{0}, right=1.5 of helper'] (c') {Company};
	\draw (s2'_out1) to (c'_in1);
	\draw (s1'_out1) to +(5pt,0) node[fill=none] {$\bullet$};
\end{scope}
	\node[starburst, draw, minimum width=2cm, minimum height=2cm,align=center,fill=green!10, font=\small, fill=white, line width=1.5pt] at ($(c)!.5!(helper')$)
{Change\\supplier!};
\end{tikzpicture}
\end{equation*}
\begin{quote}
The company can get widgets either from supplier 1 or supplier 2; we could imagine this choice is completely up to the company. The company can decide based on the quality of widgets it has received in the past, i.e.\ when the company gets a bad widget, it updates an internal variable, and sometimes that variable passes a threshold making the company switch states. Whatever its strategy for deciding, we should be able to encode it in $\poly$.
\end{quote}
\begin{enumerate}[resume]
	\item When someone assembles a machine, their own outputs dictate the connection pattern of the machine's components.
\end{enumerate}
\begin{equation*}%\label{eqn.someone}
\begin{tikzpicture}[oriented WD, font=\ttfamily, bb port length=0, every node/.style={fill=blue!10}, baseline=(someone.north)]
	\node[bb port sep=.5, bb={0}{1}] (A) {unit A};
	\node[bb port sep=.5, bb={1}{0}, right=of A] (B) {unit B};
	\coordinate (helper) at ($(A)!.5!(B)$);
	\node[bb={1}{1}, below=2 of helper] (someone) {\tikzsymStrichmaxerl[3]};
	\draw[->, dashed, blue] (someone_in1) to[out=180, in=270] (A.270);
	\draw[->, dashed, blue] (someone_out1) to[out=0, in=270] (B.270);
	\draw (A_out1) -- +(10pt,0);
	\draw (B_in1) -- +(-10pt,0);
%
\begin{scope}[xshift=3.5in]
	\node[bb port sep=.5, bb={0}{1}] (A') {unit A};
	\node[bb port sep=.5, bb={1}{0}, right=.5of A'] (B') {unit B};
	\coordinate (helper') at ($(A')!.5!(B')$);
	\node[bb={1}{1}, below=2 of helper'] (someone') {\tikzsymStrichmaxerl[3]};
	\draw[->, dashed, blue] (someone'_in1) to[out=180, in=270] (A'.270);
	\draw[->, dashed, blue] (someone'_out1) to[out=0, in=270] (B'.270);
	\draw (A'_out1) -- (B'_in1);
\end{scope}
%
	\node[starburst, draw, minimum width=2cm, minimum height=2cm,fill=blue!50,line width=1.5pt, align=center, font=\upshape] at ($(B)!.5!(A')-(0,.6cm)$)
{Attach!};
\end{tikzpicture}
\end{equation*}
\begin{quote}
Have you ever assembled something? Your internal states dictate the connection pattern of some other things. We can say this in $\poly$.
\end{quote}

All of the examples discussed here will be presented in some detail once we have the requisite mathematical theory \cref{?,?,?}.
\end{example}

\begin{exercise}
Think of another example where systems are sending each other information, but where the sort of information or who it's being sent to or received from can change based on the states of the systems involved. You might have more than two, say $\rr$-many, different wiring patterns in your situation.
\end{exercise}

But there's more that's intuitively wrong or limiting about the picture in \eqref{eqn.wired_forever}. Ever notice how you can change how you interface with the world? Sometimes I close my eyes, which makes that particular way of sending me information inaccessible: that port vanishes and you need to use your words. Sometimes I'm in a tunnel and my phone can't receive GPS. Sometimes I extend my hand to give or receive an object from another person, but sometimes I don't. My ports themselves change in time. Sometimes I even use my output ports to determine the wiring pattern of other machines. We will be able to say all this using $\poly$.

And there's even more that's wrong with the above description. Namely, when I move my eyes, that's actually something you can see---e.g.\ whether I'm looking at you. When I turn around, I see different things, and \emph{you can notice I'm turned around}! When I use my muscles or mouth to express things, my very position changes: my tongue moves, my body moves. So my output is actually achieved by changing position. The model in $\poly$ will be able to express this too.

\begin{example}
Imagine a million little eyeballs, each of which has a tiny brain inside it, all together in a pond of eyeballs. All that an individual eyeball $e$ can do is open and close. When $e$ is open, it can make some distinction about all the rest of the eyeballs in view: maybe it can count how many are open, or maybe it can see whether just one certain eyeball $e'$ is open or closed. But when $e$ is closed, it can't see anything; whatever else is happening, it's all the same to $e$. All it can do in that state is process previous information.

Each eyeball in this system will correspond to the polynomial $\yon^\ord{n}+\yon$, which intuitively consists of two situations, one with $n$-many possibilities, and the other with only one possibility.

The point is, however, is that the other eyeballs can tell if $e$ is opened or closed. We can imagine some interesting dynamics in this system, e.g.\ waves of openings or closings sweeping over the group, a ripple of openings expanding through the pond.

Talk about real-world applications!
\end{example}

Hopefully you now have an idea of what we mean by mode-dependence: interfaces and wiring diagrams changing in time, based on the states of all the systems involved. We'll see that $\poly$ speaks about mode-dependent systems and wiring diagrams in this sense. 

But $\poly$ is very versatile in its applications. Next we'll show how it relates to information storage and translation.

%---- Subsection ----%
\subsection{Data}

Data is information, maybe thought of as quantized into atomic pieces, but where these atomic pieces are somehow linked together according to a conceptual structure. When a person or organization uses certain data repeatedly, they often find it useful to put their data in a database. This requires organizing the little pieces into a conceptual structure. So when you hear ``database'', just think of it as a conceptual structure filled with examples.

To fix a mental image, let's say that you need to constantly look up employees, what department they're in, who the admin person is for that department, who their manager is, etc. Here's an associated database
\[
\begin{tabular}{ c | c  c  c}
  \textbf{Employee}&\textbf{FirstName}&\textbf{WorksIn}&\textbf{Mngr}\\\hline
  1&Alan&101&2\\
  2&Ruth&101&2\\
  3&Carla&102&3
\end{tabular}
\hspace{.3in}
\begin{tabular}{ c | c  c}
  \textbf{Department}&\textbf{Name}&\textbf{Admin}\\\hline
  101&Sales&1\\
  102&IT&3\\~
\end{tabular}
%\hspace{.3in}
%\begin{tabular}{ c |}
%	\textbf{String}\\\hline
%	Alan\\
%	IT\\[-3pt]
%	\resizebox{!}{10pt}{$\vdots$}
%\end{tabular}
\]
We can see it as being associated to the following conceptual scheme, also called a \emph{schema}:
\begin{equation}\label{eqn.myschema}
\tiny\cat{C}\coloneqq
\boxCD{
\begin{tikzcd}[row sep=large, ampersand replacement = \&]
 	\LTO{Employee}\ar[rr, shift left, "\text{WorksIn}"]\ar[dr, bend right, "\text{FirstName}"']\ar[loop left, "\text{Mngr}"]\&\&
  \LTO{Department}\ar[ll, shift left, "\text{Admin}"]\ar[dl, bend left, "\text{Name}"]\\
  \&\LTO[\circ]{String}
\end{tikzcd}
\\~\\\tiny
  Department.Admin.WorksIn = Department
}
\end{equation}
The equation at the bottom says that for any department $d$, if you ask for the admin person and see which department they work in, it's required to be $d$.

There's a very important thing that we do with databases: we query them. We ask them questions like ``tell me everyone that's either the admin person of the Math department or their manager''.
\begin{minted}{mysql}
 FOR    d: Department, e: Employee
 WHERE  Name(d)="Math" AND
        (e=Admin(d) OR e=Mngr(Admin(d)))
 RETURN FirstName(e)
\end{minted}
This sort of question is formally called a ``union of conjunctive queries''. We will see this sort of query is intimately connected with $\poly$.

We will also see how databases can be conceived in terms of dynamical systems.

%---- Subsection ----%
\subsection{Decisions}

Consider the following three trees, the first two are infinite (though that's hard to draw):
\[
\begin{tikzpicture}[trees]
\begin{scope}[
  level 1/.style={sibling distance=20mm},
  level 2/.style={sibling distance=10mm},
  level 3/.style={sibling distance=5mm},
  level 4/.style={sibling distance=2.5mm}]
  \node[dgreen] (a) {$\bullet$}
    child {node[dgreen] {$\bullet$}
    	child {node[dgreen] {$\bullet$}
    		child {node[dgreen] {$\bullet$}
  				child 
  				child
  			}
    		child {node[dyellow] {$\bullet$}
  				child 
  				child
  			}
    	}
    	child {node[dyellow] {$\bullet$}
    		child {node[dgreen] {$\bullet$}
  				child
  				child
  			}
    		child  {node[red] {$\bullet$}}
    	}
    }
    child {node[dyellow] {$\bullet$}
    	child {node[dgreen] {$\bullet$}
    		child {node[dgreen] {$\bullet$}
  				child
  				child
  			}
    		child {node[dyellow] {$\bullet$}
  				child
  				child
  			}
  		}
  		child {node[red] {$\bullet$}
  		}
  	}
  ;
\end{scope}
\begin{scope}[
  level 1/.style={sibling distance=13mm},
  level 2/.style={sibling distance=10mm},
  level 3/.style={sibling distance=5mm},
  level 4/.style={sibling distance=2.5mm}]
\node (b) [right=4 of a, dyellow] {$\bullet$}
  child {node[dgreen] {$\bullet$}
  	child {node[dgreen] {$\bullet$}
  		child {node[dgreen] {$\bullet$}
				child
				child
			}
  		child {node[dyellow] {$\bullet$}
				child
				child
			}
		}
  	child {node[dyellow] {$\bullet$}
  		child {node[dgreen] {$\bullet$}
				child
				child
			}
  		child  {node[red] {$\bullet$}}
  	}
	}
	child {node[red] {$\bullet$}}	
;
\end{scope}
\node (c) [red, right=2 of b] {$\bullet$};
\end{tikzpicture}
\]
These are patterned examples---and we'll understand what this pattern is more clearly in \cref{???}---of what we will call \emph{decision streams}. Decision streams form the objects in a category that also includes the following level-3 abbreviations of binary decision streams (the third of which is a finite stream):
\[
\begin{tikzpicture}[trees]
\begin{scope}[
  level 1/.style={sibling distance=20mm},
  level 2/.style={sibling distance=10mm},
  level 3/.style={sibling distance=5mm},
  level 4/.style={sibling distance=2.5mm}]
  \node (a) {$\bullet$}
    child {node {$\bullet$}
    	child {node {$\bullet$}
    		child {node {$\bullet$}
  			}
    		child {node {$\bullet$}
  				child 
  				child
  			}
    	}
    	child {node {$\bullet$}
  			}
    }
    child {node {$\bullet$}
    	child {node {$\bullet$}
    		child {node {$\bullet$}
  				child
  				child
  			}
    		child {node {$\bullet$}
  				child
  				child
  			}
  		}
  		child {node {$\bullet$}
    		child {node {$\bullet$}
  			}
    		child {node {$\bullet$}
  			}
  		}
  	}
  ;
\end{scope}
\begin{scope}[
  level 1/.style={sibling distance=13mm},
  level 2/.style={sibling distance=10mm},
  level 3/.style={sibling distance=5mm},
  level 4/.style={sibling distance=2.5mm}]
  \node (b) [right=4 of a] {$\bullet$}
    child {node {$\bullet$}
    }
    child {node {$\bullet$}
    	child {node {$\bullet$}
    		child {node {$\bullet$}
  			}
    		child {node {$\bullet$}
  				child
  				child
  			}
  		}
  		child {node {$\bullet$}
    		child {node {$\bullet$}
  				child
  				child
  			}
    		child {node {$\bullet$}
  				child
  				child
  			}
  		}
  	}
  ;
\end{scope}
\begin{scope}[
  level 1/.style={sibling distance=13mm},
  level 2/.style={sibling distance=8mm},
  level 3/.style={sibling distance=5mm},
  level 4/.style={sibling distance=2.5mm}]
  \node (c) [right=4 of b] {$\bullet$}
    child {node {$\bullet$}
    	child {node {$\bullet$}
  		}
  		child {node {$\bullet$}
    		child {node {$\bullet$}
  			}
    		child {node {$\bullet$}
  			}
  		}
    }
    child {node {$\bullet$}
  	}
  ;
\end{scope}
\end{tikzpicture}
\]

The set of such decision streams in fact form the objects of a category with very nice properties (it's a topos), which we call $\sys(\yon^\2+\1)$. The idea is that every corolla in these diagrams has either two possibilities, corresponding to $\yon^\2$, or no possibilities, corresponding to $\1=\yon^\0$.

\begin{exercise}
\begin{enumerate}
	\item Draw a level-3 abbreviation of a decision stream of type $\yon^\2+\yon^\0$.
	\item Draw a level-4 abbreviation of a decision stream of type $\yon$.
	\item Draw a level-3 abbreviation of a decision stream of type $\nn\yon^\2$ by labeling every node with a natural number.
	\qedhere
\end{enumerate}
\end{exercise}

But decisions aren't just about choosing; they're also about trying to accomplish something. The logic of accomplishment is exceptionally rich in this setting. We will concentrate on what we call a \emph{win condition}, which is a subgraph of the decision stream with the property that if $n$ is winning node, then any child of $n$ is also a winning node. 
\[\begin{tikzpicture}[trees,
  level 1/.style={sibling distance=20mm},
  level 2/.style={sibling distance=10mm},
  level 3/.style={sibling distance=5mm},
  level 4/.style={sibling distance=2.5mm}]
  \node (root) {$\bullet$}
    child {node {$\bullet$}
    	child {node {$\bullet$}
    		child {node {$\bullet$}
  			}
    		child {node {$\bullet$}
  				child
  				child
  			}
    	}
    	child {node {$\bullet$}
  			}
    }
    child {node {$\bullet$}
    	child {node {$\bullet$}
    		child {node {$\bullet$}
  				child
  				child
  			}
    		child {node {$\bullet$}
  				child
  				child
  			}
  		}
  		child {node {$\bullet$}
    		child {node {$\bullet$}
  				child
  				child
  			}
    		child {node {$\bullet$}
  			}
  		}
  	}
  ;
 \begin{scope}[every node/.style={circle, inner sep=3pt, blue, draw}]
  \node at (root-1-2)     {};
  \node at (root-2-1-1-1) {};
  \node at (root-2-1-1-2) {};
  \node at (root-2-1-2-1) {};
	\node at (root-2-2) 		{};
  \node at (root-2-2-1) 	{};
  \node at (root-2-2-2) 	{};
  \node at (root-2-2-1-1) {};
  \node at (root-2-2-1-2) {};
 \end{scope}
\end{tikzpicture}
\]
More formally, these are called \emph{sieves}; they form the elements of a logical system called a Heyting algebra: you can take any two sieves and form the intersection or union (which correspond to AND and OR), or even things like implication, negation, and existential and universal quantification. This will give us a calculus of win-conditions for any sort $p$ of decision stream.

%---- Subsection ----%
\subsection{Implementation}

Everything we talk about can actually be implemented in a computer without much difficulty, at least if you have access to a language that supports dependent types. We will continue to use Agda.

What we have been calling polynomials---things like $\yon^\2+\2\yon+\1$---are often called \emph{containers} in the computer science literature. Here is some sample Agda code that first implements interfaces in general, and then a specific one that's isomorphic to $\yon^\2+\2\yon+\1$:
\begin{agda}
record Interface : Set where
   field
     pos : Set
     dis : pos -> Set
open Interface

MyInterface : Interface
                     -- finish
\end{agda}
%---- Subsection ----%
\subsection{Mathematical theory}

%-------- Section --------%
\section{$\poly$ as a category}\label{sec.poly}

%-------- Section --------%
\section{Dynamical systems}



\end{document}
