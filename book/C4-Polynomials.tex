\documentclass[DynamicalBook]{subfiles}
\begin{document}
%


\setcounter{chapter}{3}%Just finished 3.


%------------ Chapter ------------%
\chapter{Polynomial functors and dynamics}

%-------- Section --------%
\section{Introduction}

The category of polynomial functors is a jackpot. Its beauty flows without bound. We will see its intimate relationship with both information and process. We will see its intimate relationship with safety and decision-making. But our story begins with something quite humble: middle school algebra.
\begin{align}\label{eqn.polynomial1892}
\yon^2+\yon+1 \quad&\quad
\textit{polynomial}
\intertext{
All our polynomials will involve one variable, $\yon$, chosen for reasons we'll explain soon. Polynomials in one variable can be depicted as a set of mini-trees:
}
\label{eqn.poly1892}
\begin{tikzpicture}[trees, grow'=up]
  \node (1) {$\bullet$} 
    child {}
    child {};
  \node[right=.5 of 1] (2) {$\bullet$} 
    child {};
  \node[right=.5 of 2] (3) {$\bullet$};
\end{tikzpicture}
\quad&\quad\textit{poly}
\end{align}
More technically, mini-trees are called \emph{corollas}, so what we're calling a \emph{poly} is a set of corollas. 

Intuitively, one might register each corolla as representing a \emph{decision}. Associated to every decision is a set of \emph{options}. The three decisions we exhibit in \cref{eqn.poly1892} are particularly interesting; they respectively have two options, one options, and no options. Having two options is familiar from life---it's the classic yes/no decision---as well as from Claude Shannon's Information Theory. Having one option is also familiar theoretically and in life: ``sorry, ya just gotta go through it.'' Having no options is when you actually don't get through it: it's ``the end''. While the corollas $1,\yon,$ and $\yon^2$ are each interesting as decisions, their sum $\yon^2+\yon+1$ has very little theoretical interest; it's just a first example.

\[
\begin{tabular}{llll}
\textbf{Polynomial} $p$ & \textbf{Overarching terminology} & \textbf{Poly} & \textbf{Intuitive}\\\hline
$p(1)$ & Set of positions & Set of corollas & Set of decisions\\
Pure power summand $\yon^A$& Position & Corolla & Decision\\
Distinction $d\in p_i$ & ? & Set of leaves & Set of options
\end{tabular}
\]

\begin{exercise}
Consider the polynomial $2\yon^3+2\yon+1$ and the associated poly.
\begin{enumerate}
	\item Draw the poly.
	\item How many roots does this poly have?
	\item How many decisions does this represent?
	\item For each corolla in the poly, say how many leaves it has.
	\item For each decision, how many options does it have?
	\item Does the polynomial $\yon^\nn+4\yon$ have a representable summand $\yon^2$?
\end{enumerate}

\end{exercise}

In our terminology, a polynomial $p$ consists of positions and their distinctions, so as in \eqref{eqn.polynomial1892} would be pronounced as having three positions, one with two distinctions, one with one, and one with none. The associated poly \eqref{eqn.poly1892} would be pronounced as having three decisions, one with two options, one with one, one with none. We do not make a distinction in notation between the poly and its name $p$; they are two different syntaxes for the same object. 


\begin{exercise}
If you were a suitor choosing the poly you love, aesthetically speaking, which would strike your interest? Answer by circling the associated polynomial:
\begin{enumerate}
	\item $\yon^2+\yon+1$
	\item $\yon^2+3\yon^2+3\yon+1$
	\item $\yon^2$
	\item $\yon+1$
	\item $(\nn\yon)^\nn$
	\item $S\yon^S$
	\item $\yon^{100}+\yon^2+3\yon$
	\item Your poly's nomial $p$ here.
\end{enumerate}
Any reason?
\end{exercise}

But before we get into the details, let's say where we're going: dynamical systems and data.

%---- Subsection ----%
\subsection{Dynamical systems}

%---- Subsection ----%
\subsection{Data}

%---- Subsection ----%
\subsection{Decision theory}


Here's a pretty thing we're getting to: the decision stream.
\begin{equation}\label{eqn.trees_comp}
\begin{tikzpicture}[trees, grow'=up]
	\node (p3) {
  \begin{tikzpicture}[trees, grow'=up]
    \node (a) {$\bullet$}[sibling distance=.75cm] 
      child {[fill]
      	node[left=-3pt] {$\bullet$}[sibling distance=.3cm]
				  child {node (x) {}} child 
			}
      child {[fill]
      	node[left=-3pt] {$\bullet$}[sibling distance=.3cm]
				  child child
  		};
    \node (b) [right=1.5 of a] {$\bullet$} 
      child {[fill]
      	node[left=-3pt] {$\bullet$}[sibling distance=.3cm]
				  child child
			}
      child {[fill]
      	node[left=-3pt] {$\bullet$}
  		};
    \node (c) [right=1 of b]{$\bullet$}
      child {[fill]
      	node[left=-3pt] {$\bullet$}
			}
      child {[fill]
      	node[left=-3pt] {$\bullet$}[sibling distance=.3cm]
				  child child
  		};
    \node (d) [right=1 of c]{$\bullet$} 
      child {[fill]
      	node[left=-3pt] {$\bullet$}
			}
      child {[fill]
      	node[left=-3pt] {$\bullet$}
  		};		
     \node (e) [right=1 of d] {$\bullet$};
    \node[draw, fill=yellow!80!black, opacity=0.2, rounded corners=5pt, inner sep=5pt, fit=(x) (e)] (y) {};	
  \end{tikzpicture}
   };
\end{tikzpicture}
\end{equation}

%---- Subsection ----%
\subsection{Implementation}

%---- Subsection ----%
\subsection{Mathematical theory}

%-------- Section --------%
\section{Dynamical systems}



\end{document}
