\documentclass[DynamicalBook]{subfiles}
\begin{document}
%


\setcounter{chapter}{4}%Just finished 4.


%------------ Chapter ------------%
\chapter{Polynomial `time'}\label{chapter.5}

%-------- Section --------%
\section{Introduction}

In \cref{chapter.4} we saw that the category $\poly$ of polynomial functors---otherwise known as dependent lenses---is a very well-behaved category in which to think about dynamical systems of quite a general nature.

But we left one thing---what in some sense is the most interesting part of the story---out entirely. That thing is quite simple to state, and yet has profound consequences. Namely: polynomials can be composed:
\[
\yon^\2\circ(\yon+\1)=(\yon+\1)^\2\cong\yon^\2+\2\yon+\1
\]
What could be simpler?

It turns out that this operation, which we'll see soon is a monoidal product, has a lot to do with time. There is a strong sense---made precise in \cref{?}---in which the polynomial $p\circ q$ represents ``starting at a position $i$ in $p$, choosing a direction in $p_i$, landing at a position $j$ in $q$, and choosing a direction in $q_j$.''

The composition product has many surprises up its sleeve, as we'll see. We've told many of them to you already in \cref{subsec.math_theory}. We won't amass them all here; instead, we'll take you through the story step by step. But as a preview, this chapter will get us into decision trees, databases, and more dynamics, and it's all based on $\circ$.

As in \cref{eqn.sum_p1}, we'll continue to denote polynomials with the following notation
\begin{equation}\label{eqn.sum_p1_again}
p\cong\sum_{i\in p(\1)}\yon^{p_i},
\end{equation}
and refer to $p(\1)$ as the set of positions, and for each $i\in p(\1)$ we'll refer to $p_i$ as the set of direction at position $i$.

%-------- Section --------%
\section{The composition product}

We begin with the definition of composition product.

\begin{proposition}
Suppose $p,q\in\poly$ are polynomial functors $p,q\colon\smset\to\smset$. Then their composite $p\circ q$ is again a polynomial functor and we have the following isomorphisms
\begin{equation}\label{eqn.composite_formula}
p\circ q\cong\sum_{i\in p(\1)}\prod_{d\in p_i}\sum_{j\in q(\1)}\prod_{e\in q_j}\yon.
\end{equation}
\end{proposition}
\begin{proof}
We can rewrite \cref{eqn.sum_p1_again} for $p$ and $q$ as
\[
p\cong\sum_{i\in p(\1)}\prod_{d\in p_i}\yon
\qqand
q\cong\sum_{j\in q(\1)}\prod_{e\in q_j}\yon.
\]
For any set $X$ we have $(p\circ q)(X)=p(q(X))=p(\sum_j\prod_e X)=\sum_i\prod_d\sum_j\prod_eX$, so \eqref{eqn.composite_formula} is indeed the formula for their composite. To see this is a polynomial, we use \cref{prop.completely_distributive}, which says we can rewrite the $\prod\sigma$ in \eqref{eqn.composite_formula} as a $\sigma\prod$. The result 
\begin{equation}\label{eqn.composition_formula_sums_first}
  p\circ q\cong
  \scalebox{1.3}{$\displaystyle
  \sum_{i\in p(\1)}\sum_{j_i\colon p_i\to q(\1)}\yon^{\sum_{d\in p_i}q_{j_i(d)}},$}
\end{equation}
(written slightly bigger for clarity) is clearly a polynomial.
\end{proof}

\begin{exercise}
Let's consider \eqref{eqn.composition_formula_sums_first} piece by piece, with concrete polynomials $p\coloneqq\yon^\2+\yon$ and $q\coloneqq \yon^\3+\1$. Note that $p_1=\2$ and $p_2=\1$.
\begin{enumerate}
	\item What is $q^\2$?
	\item What is $\yon^2\circ q$?
	\item What is $\yon\circ q$?
	\item What is $(\yon^\2+\yon)\circ q$? This is what $p\circ q$ ``should be''.
	\item How many functions $j_1\colon p_1\to q(\1)$ are there?
	\item For each function $j_1$ as above, what is $\sum_{d\in p_1} q_{j_1(d)}$?
	\item How many functions $j_2\colon p_2\to q(\2)$ are there?
	\item For each function $j_2$ as above, what is $\sum_{d\in p_2} q_{j_2(d)}$?
	\item Write out $\sum_{i\in p(\1)}\sum_{j_i\colon p_i\to q(\1)}\yon^{\sum_{d\in p_i}q_{j_i(d)}}$.
	\item Does the result agree with what $p\circ q$ should be?
\qedhere
\end{enumerate}
\end{exercise}

In terms of corollas, composition product $p\circ q$ is given by stacking $q$ on top of $p$ in every possible way.



\end{document}
