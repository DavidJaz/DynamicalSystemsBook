\documentclass[DynamicalBook]{subfiles}
\begin{document}
%


\setcounter{chapter}{1}%Just finished 1.


%------------ Chapter ------------%
\chapter{How systems behave: Trajectories, Steady States, and Periodic Orbits}\label{chapter.2}

\section{Introduction}

So far, we have seen how to wire up dynamical systems. But we haven't seen our
dynamical systems actually \emph{do anything}. In this section, we will begin to
study the behavior of our dynamical systems. We will see particular kinds of
behaviors our systems can have:
trajectories, steady states, and periodic orbits.

\begin{informal}
  A \emph{behavior} of a dynamical system is a particular way its states can
  change according to its dynamics.  

  There are different \emph{kinds of behavior} corresponding to the different
  sorts ways that the states of a system could evolve. Perhaps they eventually
  repeat, or they stay the same despite changing conditions.
\end{informal}

In \cref{sec.behaviors}, we will give a formal definition of behavior of
dynamical system. We will see that the different kinds of behaviors ---
trajectories, steady states, periodic orbits, etc. --- can each
be packaged up into a single system that \emph{represents} that kind of
behavior. This system will behave in exactly that kind of way, and do nothing else. Maps from it to a system of interest will exhibit that sort of behavior in the system of interest.

We will end the section by seeing how the steady states of a complex system formed by wiring together some
component systems can be calculated from the steady states of the components.
The calculation will turn out to be just a bit of matrix arithmetic!

This result will be a preview to our more general results in
\cref{sec.representables} which concern arbitrary behaviors of dynamical
systems. But that's
getting ahead of ourselves.

%---- Section ----%
\section{Trajectories}\label{sec.trajectory_discrete}

In the introduction, we saw that the $\Sys{Clock}$ system
\cref{eqn.clock_system_box} has behaves in this way if it starts at $3$ o'clock: 
$$3 \xmapsto{\fun{tick}} 4 \xmapsto{\fun{tick}} 5 \xmapsto{\fun{tick}} 6
\xmapsto{\fun{tick}} \cdots$$

This sequence of states of the clock system, each following from the last by the
dynamics of the system, is called a \emph{trajectory}. When our systems have
input parameters, we will need to choose a sequence of input parameters to feed
the system in order for the states to change.

\begin{definition}\label{def.trajectory_discrete}
 Let $$\Sys{S} = \lens{\update{S}}{\expose{S}} : \lens{\State{S}}{\State{S}}
 \leftrightarrows \lens{\In{S}}{\Out{S}}$$
 be a deterministic system. Suppose that $p : \nn \to \In{S}$ is a sequence of
 parameters for $\Sys{S}$. Then a \emph{$p$-trajectory} of $\Sys{S}$ is a sequence $s : \nn \to \State{S}$ of states so that
 $$\update{S}(s_i, p_i) = s_{i + 1}$$
 for all $i \in \nn$.

 If additionally $v : \nn \to \Out{S}$ is a sequence of output values for $\Sys{S}$, then a
 \emph{$\lens{p}{v}$-trajectory} is a sequence of states $s : \nn \to \State{S}$ so
 that
 \begin{align*}
   \update{S}(s_i, p_i) &= s_{i + 1}\\
   \expose{S}(s_i) &= v_i
 \end{align*}
 for all $i \in \nn$. We call the pair $\lens{p}{v}$ the \emph{chart} of the trajectory $s$.
\end{definition}

Its worth noting that a trajectory $s : \nn \to \State{S}$ in a deterministic system is determined
entirely by its start state $s_0$. This is what makes deterministic systems
deterministic; if you know the dynamics and you know what state the system is
in, you know how it will continue to behave. We'll relax this condition later in \cref{??}.

\begin{example}
Consider the SIR model of \cref{ex.SIR_model_discrete}. Suppose that we let our
parameters $(a, b) : \nn \to \In{\Sys{SIR}}$ be constant at $.2$ and $.3$
respectively: that is, $a_t = .2$ and $
b_t = .3$ for all $t$. Then a trajectory for $\Sys{SIR}$ with parameters $(a, b)$
is a sequence of populations $(s, i, r) : \nn \to \State{SIR}$ such that
$$\begin{bmatrix}s_{t+1}\\ i_{t+1}\\ r_{t+1} \end{bmatrix} = \begin{bmatrix}
  s_t - .2s_t i_t\\ i_t + .2s_t i_t - .3i_t \\ r_t + .3i_t\end{bmatrix}$$

Here is an example of such a trajectory with a $1000$ total people and one infected
person to start, that is $(s_0, i_0, r_0) = (999, 1, 0)$.
\[
  \begin{tikzpicture}
    horizontal axis
    \draw[->] (0,0) to (11, 0);
    \draw[label]
      node at (11, -.3) {$t$};
    \foreach \x in {1,...,10}
      \draw (\x cm,1pt) -- (\x cm,-1pt) node[anchor=north] {$\x$};

   %% vertical axis
   \draw[->] (0,0) to (0, 11);
   \draw[label] node at (-1, 11) {$(s, i, r)$};
   \foreach \y in {1,...,10}
   {
     \pgfmathtruncatemacro{\display}{100*\y};
     \draw (1pt,\y cm) -- (-1pt,\y cm) node[anchor=east] {$\display$};
   }

%   %% plot
%   \tikzmath{
%     real \s; 
%     \s = 9.99;
%   
%     function srecusion(\x){
%       return {\x-1};
%     }
%     \foreach \t in {0,1,...,10}
%     {
%       \node[draw,circle,red] at (\t, \s) {};
%       \s = srecursion(\s);
%     }
%    }
  \end{tikzpicture}
\]
\jaz{I don't know how to actually plot this...}
\end{example}


\begin{example}\label{ex.transition_diagram_discrete_traj}
If a deterministic system is written as a transition diagram, then the
trajectories in the system are paths through the diagram. Recall from
\cref{ex.transition_diagram_discrete} this system:
\[
\begin{tikzpicture}
	\node[draw] {
  \begin{tikzcd}[column sep=small]
  	\LMO{a}\ar[rr, dgreen, thick, bend left]\ar[loop left, thick, orange]&&
  	\LMO{b}\ar[ll, thick, orange, bend left]\ar[dl, bend left, thick, dgreen]\\&
  	\LMO{b} \ar[ul, thick, orange, bend left] \ar[loop left, thick, dgreen]
  \end{tikzcd}
  };
\end{tikzpicture}
\]

Suppose that $p : \nn \to \{\Green, \Orange\}$
alternates between $\Green$ and $\Orange$. Then
starting at the top right state, a trajectory quickly settles into alternating between the top two
states:
\[
\begin{tikzpicture}
	\node[draw] {
  \begin{tikzcd}[column sep=small]
    \LMO{b} \ar[r, dgreen, thick] & \LMO{b} \ar[r, orange, thick] & \LMO{a} \ar[r, dgreen,thick] & \LMO{b} \ar[r, orange, thick] & \LMO{a} \ar[r, dgreen, thick] & \cdots
  \end{tikzcd}
  };
\end{tikzpicture}
\]

\end{example}

Knowing about trajectories can show us another important role that deterministic
systems play: they are \emph{stream transformers}. From a stream $p : \nn \to
\In{S}$ of inputs and a start state $s_0 \in \State{S}$, we get a trajectory $s
: \nn \to \State{S}$ given recursively by
\begin{align*}
  s_{t+1} &:= \update{S}(s_t,\, p_t).
\end{align*}
We then get a stream $v : \nn \to \Out{S}$ of output values by defining
$$v_t := \expose{S}(s_t).$$
The system $\Sys{S}$ is a way of transforming streams of input parameters into
streams of output values.
\begin{proposition}[Deterministic systems as stream transformers]
 Let $$\Sys{S} = \lens{\update{S}}{\expose{S}} : \lens{\State{S}}{\State{S}}
 \leftrightarrows \lens{\In{S}}{\Out{S}}$$
 be a deterministic system. Then for every $s_0 \in \State{S}$, we get a stream
 transformation function
 $$\fun{transform}_{\Sys{S}} : \In{S}^{\nn} \to \Out{S}^{\nn}$$
 Given by
 \begin{align*}
   \fun{transform}_{\Sys{S}}(p)_0 &= \expose{S}(s_0)\\
   \fun{transform}_{\Sys{S}}(p)_{t + 1} &= \expose{S}(\update{S}(s_t,\, p_t))
 \end{align*}
 where $s_{t+1} = \update{S}({s_t,\, p_t})$ is the trajectory given by $s_0$.
\end{proposition}

\begin{exercise}
 Say how the system of \cref{ex.transition_diagram_discrete_traj} acts as a
 stream transformer on the following streams:
 \begin{enumerate}
   \item $p_{2t} = \Green$ and $p_{2t + 1} = \Orange$.
   \item $p_t = \Green$.
   \item $p_0 = \Green$ and $p_t = \Orange$ for
     all $t > 0$.
     \qedhere
 \end{enumerate}
\end{exercise}

Later, in \cref{sec.representables}, we will see that given trajectories of
component systems, we get a trajectory of a whole wired system. Even better,
every trajectory of the whole wired system can be calculated this way.


%---- Section ----%
\section{Steady states}

A steady state of a system is a state which does not change. Steady states are
important because they are guarantees of stability: a vase in a steady state is doing great, a heart in a steady state is in need of attention.

\begin{definition}\label{def.steady_state_discrete}
 Let $$\Sys{S} = \lens{\update{S}}{\expose{S}} : \lens{\State{S}}{\State{S}}
 \leftrightarrows \lens{\In{S}}{\Out{S}}$$
 be a deterministic system. For input parameter $i \in \In{S}$ and output value
 $o \in \Out{S}$, an \emph{$\lens{i}{o}$-steady state} is a state $s \in \State{S}$
 such that
 \begin{align*}
   \update{S}(s, i) &= s, \\
   \expose{S}(s) &= o.
 \end{align*}
 We call the pair $\lens{i}{o}$ the \emph{chart} of the steady state.
\end{definition}

\begin{remark}
Its important to note that a steady state is relative to the input parameter
chosen. For example, in \cref{ex.transition_diagram_discrete}, the top left
state is steady for the input parameter $\Orange$ but not for
the input parameter $\Green$.
\end{remark}

Unlike with trajectories, a system need not necessarily have \emph{any} steady
states. For example, the $\Sys{Clock}$ has no steady states; it always keeps
ticking to the next hour. 

In the transition diagram of a finite deterministic system, steady states will be loops
that begin and end at the same node. Since the system is finite, we can arrange
the steady states by their chart into a $\In{S} \times \Out{S}$ matrix. For example, in
\cref{ex.transition_diagram_discrete}, we get the following $\{\Green,\, \Orange\} \times \{a,\, b\}$ matrix:
\begin{equation}\label{eqn.steady_state_matrix_transition_diagram_discrete}
\kbordermatrix{
  & \Green & \Orange\\
  a & \emptyset & \left\{ \begin{tikzcd}\LMO{a} \ar[loop left, orange, thick] \end{tikzcd} \right\}  \\
  b & \left\{ \begin{tikzcd}\LMO{b} \ar[loop left, dgreen, thick]\end{tikzcd} \right\}& \emptyset  \\
}
\end{equation}

This is a ``matrix of sets'', in that the entries are the actual sets of steady
states. If we just counted how many steady states there were for each
input-output pair, we would get this matrix:
\begin{equation}\label{eqn.steady_state_matrix_transition_diagram_discrete2}
\kbordermatrix{
  & \Green & \Orange\\
  a & 0 & 1  \\
  b & 1 & 0  \\
}
\end{equation}

In \cref{sec.steady_states_matrix_arithmetic}, we'll see that each wiring
diagram gives a formula for calculating the matrix of steady states of the
composite system from the matrices of steady states of the inner systems. 

\begin{exercise}\label{ex.find_steady_states1}
  What are the steady state matrices of systems $\Sys{S_1}$ and $\Sys{S_2}$ from
  \cref{ex.wiring_transition_diagrams}? What about the combined system $\Sys{S}$?
\end{exercise}

\paragraph{Steady-looking trajectories.}

The reason we are interested in steady states is that they are highly
predictable; if we know we are in a steady state, then we know we are always
going to get the same results. But it is possible for us to always get the same
outputs for the same input even though the internal state keeps changing. These
are special trajectories, and we call them \emph{steady-looking} trajectories.

\begin{definition}\label{def.steady_looking_trajectory_discrete}
  For $i \in \In{S}$ and $o \in \Out{S}$ of a system $\Sys{S}$, a \emph{$\lens{i}{o}$-steady
    looking trajectory} is a sequence of states $s : \nn \to \State{S}$ such
  that
  \begin{align*}
    \update{S}(s_t, i) &= s_{t + 1} \\
    \expose{S}(s_t) &= o
  \end{align*}
  for all $t \in \nn$. We call the pair $\lens{i}{o}$ the \emph{chart} of the
  steady-looking trajectory $s$.
\end{definition}

While the steady states of a wired together system can be calculated from those
of its components, this is not true for steady-looking trajectories.
Intuitively, this is because the internal systems can be exposing changing
outputs between eachother even while the eventual external output remains
unchanged.

\begin{exercise}\label{ex.steady_looking_trajectories}
Consider the wiring diagram:
\[
\Sys{S} \coloneqq 
\begin{tikzpicture}[oriented WD, bbx = .3cm, bby =.3cm, bb min width=.5cm, bb port length=2pt, bb port sep=1, every fit/.style={inner xsep=\bbx, inner ysep=\bby}
, baseline=(Outer.center)]
  \node[bb={1}{1}, fill=blue!10] (S1) {$\Sys{S_1}$};
  \node[bb={1}{1}, right= of S1, fill=blue!10] (S2) {$\Sys{S_2}$};

  \node[bb={1}{1}, fit={(S1) (S2)}] (Outer) {};

  \draw (Outer_in1) to (S1_in1);
  \draw (S1_out1) to (S2_in1);
  \draw (S2_out1) to (Outer_out1);
\end{tikzpicture}
\]


Find systems $\Sys{S_1}$ and $\Sys{S_2}$ and a steady-looking trajectory of the
wired system $\Sys{S}$ which is not steady-looking on the component systems.
\qedhere
\erase{
%% Here is my start at a solution
Here are two systems, $\Sys{S_1}$ and $\Sys{S_2}$ presented in terms of
transition diagrams. 

 First, let $\Set{Colors}
= \{\Red, \Blue\}$ and let $\Set{Bool} = \{\const{true}, \const{false}\}$. Here is our first system
$\Sys{S_1}$, which has interface $\lens{\Set{Bool}}{\Set{Colors}}$:
\begin{equation}\label{eqn.steady_looking_trajectories1}
\Sys{S_1} \coloneqq \begin{tikzpicture}[baseline=center]
	\node[draw] {
  \begin{tikzcd}[column sep=small]
    \LMO{\Blue} \ar[loop left, "\false"] \ar[rr, bend left, "\true"] \ar[rr, leftarrow, bend right, "\true"'] & & \LMO{\Red} \ar[loop right, "\false"]
  \end{tikzcd}
  };
\end{tikzpicture}
\end{equation}

Here is our second system, $\Sys{S_2}$, a $\lens{\Set{Colors}}{\Set{Bool}}$-system: 
\begin{equation}\label{eqn.steady_looking_trajectories2}
\Sys{S_2} \coloneqq \begin{tikzpicture}[baseline=center]
	\node[draw] {
  \begin{tikzcd}[column sep=small]
    \LMO{\true} \ar[loop left, red] \ar[loop right, blue]
  \end{tikzcd}
  };
\end{tikzpicture}
\end{equation}

When we wire these together, we get this system:
\begin{equation}\label{eqn.steady_looking_trajectories2}
\Sys{S} \coloneqq \begin{tikzpicture}[baseline=center]
	\node[draw] {
  \begin{tikzcd}[column sep=small]
    \LMO{\true} \ar[loop left, "\false"] \ar[rr, bend left, "\true"] \ar[rr, leftarrow, bend right, "\true"'] & & \LMO{\true} \ar[loop right, "\false"]
  \end{tikzcd}
  };
  As we can see, there is a $(\true, \true)$-steady-looking trajectory which hops between these two states, but always outputs $\true$ on an input of $\true$. However, the output of component system $\Sys{S_1}$ is alternating between $\Blue$ and $\Red$ as this trajectory progresses, so it is not steady-looking when restricted to $\Sys{S_1}$.
\end{tikzpicture}
\end{equation}
}%erased
  
\end{exercise}

%---- Section ----%
\section{Periodic orbits}\label{sec.periodic_orbit_discrete}

Even if the behavior of a system isn't perfectly steady, it may continually
repeat. To a reasonable approximation, the position of the earth around the sun
follows a cycle that repeats every year. Using this as a paradigmatic example,
we call these behaviors that repeat \emph{periodic orbits}.

\begin{definition}[Periodic orbit] \label{def.periodic_orbit_discrete}
  A $\lens{p}{v}$-trajectory $s : \nn \to \State{S}$ is \emph{periodic} if there
  exists a time $t_0 \in \nn_{\geq1}$, called the \emph{period}, such that $s_{t_0} = s_0$. If the sequence of
  parameters $p : \nn \to \In{S}$ is also periodic with the same period (in that $p_{t_0} = p_0$ as well), then we say that $s$ has \emph{periodic parameters}.
  

\end{definition}

\begin{remark}
  Note that when we say that a periodic orbit has periodic parameters, we assume
  that they are periodic with the same period. This has important but subtle
  consequences for our theorems concerning the composition of behaviors in
  \cref{sec.representables}. We explain the difference between a periodic orbit
  and a periodic orbit with periodic parameters in a more precise manner
  in \cref{rmk.periodic_parameters_versus_not}.
\end{remark}

\begin{remark}
  Note that a steady state is a periodic orbit (with periodic parameters) that has
  a period of $1$. 
\end{remark}

\begin{exercise}
Describe a periodic orbit with period $1$ that does not have periodic
parameters; how are they different from steady states? Are there any of these in systems $\Sys{S_1}$ and $\Sys{S_2}$ of \cref{ex.wiring_transition_diagrams}?
\end{exercise}

\begin{example}
  The $\Sys{Clock}$ system is an exemplary periodic system with a period of 12.
  The $\Sys{ClockWithDisplay}$ of \cref{eqn.whole_clock_system_box} has period 24.
\end{example}

\begin{exercise}\label{ex.wiring_transition_diagrams_periodic_orbits}
  What are the periodic orbits in the systems $\Sys{S_1}$ and $\Sys{S_2}$ of
  \cref{ex.wiring_transition_diagrams} with periodic parameters, and what are their periods? What about
  the combined system $\Sys{S}$?
\end{exercise}

\begin{exercise}\label{ex.wiring_transition_diagrams_periodic_parameters}
  Can you think of any periodic orbits in $\Sys{S_1}$ and $\Sys{S_2}$ of
  \cref{ex.wiring_transition_diagrams} which
  don't have periodic parameters? 
\end{exercise}

A trajectory might not get back to where it started, but may still end up being
periodic. We call these trajectories \emph{eventually} periodic orbits, since
they eventually end up in a repeating cycle of states.

\begin{definition}[Eventually periodic orbit] \label{def.eventually_periodic_orbit_discrete}
A $\lens{p}{v}$-trajectory $s : \nn \to
  \State{S}$ is \emph{eventually periodic} if there are times $t_0 < t_1 \in
  \nn$ such that $s_{t_0 + t} = s_{t_1 + t}$ for all $t \in \nn$. If the sequence of
  parameters $p : \nn \to \In{S}$ is also eventually periodic with the same
  period (in that $p_{t_0 + t} = p_{t_1 + t}$ for all $t$), then we say that $s$ has \emph{eventually periodic parameters}.

The \emph{period} of an eventually periodic trajectory is
  the smallest difference $t_1 - t_0$ between times such that $s_{t_0} = s_{t_1}$.
\end{definition}

%---- Section ----%
\section{Behaviors of deterministic systems}\label{sec.behaviors}

In the previous
\cref{sec.trajectory_discrete,sec.steady_state_discrete,sec.periodic_orbit_discrete},
we saw a number of different kinds of behaviors of dynamical systems. Not only
were there a lot of definitions in those sections, each of those definitions had
slight variants (like periodic orbits versus periodic orbits with periodic
parameters, or steady states versus steady-looking trajectories). In this
section, we'll define a general notion of behavior and see that we can package each of the above sorts of behavior into a single system in its own right, one that \emph{represents} that sort of
behavior. The representative system of a certain kind of behavior behaves in
exactly that way, and does nothing else.

To get started, we will give a formal definition of behavior of deterministic
system. Before we do this, we need to define another sort of map that can go
between the interfaces of systems.
\begin{definition}[Category of charts]\label{def.chart_discrete}
  A \emph{chart} $\lens{f_{\flat}}{f} : \lens{A^-}{A^+} \rightrightarrows
  \lens{B^-}{B^+}$ in a cartesian category $\cat{C}$ is a pair of maps $f : A^+ \to B^+$ and $f_{\flat} : A^+
  \times A^- \to B^-$. Note that \emph{this is not a lens}.
\end{definition}

\begin{exercise}
\begin{enumerate}
	\item How many lenses are there $\lens{f^\sharp}{f}\colon\lens{\3}{\2}\fromto\lens{\4}{\3}$?
	\item How many charts are there $\lens{f^\flat}{f}\colon\lens{\3}{\2}\tto\lens{\4}{\3}$?
\qedhere
\end{enumerate}
\end{exercise}

\begin{definition}[Behavior of deterministic systems] \label{def.behavior_discrete}
  Let $\Sys{T}$ and $\Sys{S}$ be deterministic systems. Given a chart of interfaces $\lens{f_{\flat}}{f}:
  \lens{\In{T}}{\Out{T}} \rightrightarrows \lens{\In{S}}{\Out{S}}$, a $\lens{f_{\flat}}{f}$-\emph{behavior of
  shape $\Sys{T}$ in $\Sys{S}$}, written
 $\phi :
\Sys{T} \to \Sys{S}$, is a function $\phi : \State{T} \to \State{S}$ sending
states of $\Sys{T}$ to states of $\Sys{S}$ which preserves the dynamics and
exposed variables by satisfying the following equations:
\begin{equation}\label{eqn.behavior_discrete}
\begin{aligned}
  \expose{S}(\phi(t)) &= f(\expose{T}(t)), \\
  \update{S}(\phi(t), f_{\flat}(\expose{T}(t), i)) &= \phi(\update{T}(t, i))
\end{aligned}
\end{equation}
for all $t \in \State{T}$ and $i \in \In{T}$. We say that $\lens{f_{\flat}}{f}$
is the chart of the behavior $\phi$.
\end{definition}

\begin{remark}
  If you prefer commutative diagrams to systems of equations, don't fret. We'll
  reinterpret \cref{eqn.behavior_discrete} in terms of commutative diagrams in \cref{sec.double_cat_arenas}
\end{remark}

\begin{remark}
  Suppose that we have transition diagrams for systems $\Sys{T}$ and $\Sys{S}$.
  Then a behavior of shape $\Sys{T}$ in $\Sys{S}$ will correspond to part of the
  transition diagram of $\Sys{S}$ which is shaped like the transition diagram of
  $\Sys{T}$. See the upcoming examples to see how this looks in practice.
\end{remark}

 Let's make this definition feel real with a few examples.

\begin{example}\label{ex.trajectory_as_behavior_discrete}
  Let $\Sys{Time}$ be the system $\lens{t\mapsto t+1}{\id}\colon\lens{\nn}{\nn}\fromto\lens{\{\fun{tick}\}}{\nn}$, i.e.\ with
\begin{itemize}
\item $\State{Time} \coloneqq \nn$,
\item $\Out{Time} \coloneqq \nn$,
\item $\In{Time} \coloneqq \{\fun{tick}\}$,
\item $\expose{Time} = \id$,
\item $\update{Time}(t, \ast) = t + 1$.
\end{itemize}

As a transition diagram, $\Sys{Time}$ looks like this:

\[
\begin{tikzpicture}
	\node[draw] {
  \begin{tikzcd}[column sep=30pt]
    \LMO{0} \ar[r, thick, "\fun{tick}"] & \LMO{1} \ar[r,  thick, "\fun{tick}"] & \LMO{2} \ar[r, thick, "\fun{tick}"] & \LMO{3} \ar[r, thick, "\fun{tick}"] & \LMO{4} \ar[r, thick, "\fun{tick}"] & \cdots
  \end{tikzcd}
  };
\end{tikzpicture}
\]

Let's see what a behavior of shape $\Sys{Time}$ in an arbitrary system $\Sys{S}$ will be. We will expect the
shape of $\Sys{Time}$ to appear in the transition diagram of $\Sys{S}$, like
this:
\[
\begin{tikzpicture}[baseline=(Center)]
  \coordinate (Center) at (0,0);
	\node[draw] (Diag1) {
  \begin{tikzcd}[column sep=small]
    \LMO{0} \ar[r, red, thick] & \LMO{1} \ar[r, red, thick] & \LMO{2} \ar[r, red, thick] & \LMO{3} \ar[r, red, thick] & \LMO{4} \ar[r, red, thick] & \cdots
  \end{tikzcd}
  };
	\node[draw, right = of Diag1] (Diag2){
\begin{tikzcd}
\vdots                      & \vdots                      & \vdots                      & \vdots                      & \vdots \\
\LMO{} \arrow[r] \arrow[u]  & \LMO{} \arrow[r, red] \arrow[u]  & \LMO{} \arrow[r, red] \arrow[u]  & \LMO{} \arrow[u] \arrow[ru, red] &        \\
\LMO{} \arrow[r] \arrow[u]  & \LMO{} \arrow[r] \arrow[u, red]  & \LMO{} \arrow[u] \arrow[ru] &                             &        \\
\LMO{} \arrow[r] \arrow[u]  & \LMO{} \arrow[ru] \arrow[u, red] &                             &                             &        \\
\LMO{} \arrow[u] \arrow[ru, red] &                             &                             &                             &       
\end{tikzcd}
  };
  \draw[->, shorten <=2pt, shorten >=2pt] (Diag1) to (Diag2);
\end{tikzpicture}
\]

First, we need to know what a chart $\lens{f_{\flat}}{f} :
\lens{\In{Time}}{\Out{Time}} \rightrightarrows \lens{\In{S}}{\Out{S}}$ is like. Since
$\Out{Time} = \nn$ and $\In{Time} \cong \ord{1}$, this means $f : \nn \to \Out{S}$
is a sequence of outputs, and $f_{\flat} : \nn \times \ord{1} \to \In{S}$ is a
sequence of input parameters. We might as well instead call $f$ our sequence of
exposed values $v$, and $f_{\flat}$ our sequence of input parameters $p$, so
that we have a chart $\lens{p}{v} : \lens{\ord{1}}{\nn}
\rightrightarrows \lens{\In{S}}{\Out{S}}$.

Now, let's see what a $\lens{p}{v}$-behavior $\gamma : \Sys{Time} \to \Sys{S}$ is. It
is a function $\gamma
: \State{Time} \to 
\State{S}$ satsifying some properties. But $\State{Time} = \nn$, so $\gamma :
\nn \to \State{S}$ is a sequence of states in $\Sys{S}$. Now,
\cref{eqn.behavior_discrete} becomes the equations:
\begin{align*}
  \expose{S}(\gamma(t)) &= v(t) \\
  \update{S}(\gamma(t), p(t)) &= \gamma(t + 1).
\end{align*}
which are exactly the equations defining a $\lens{p}{v}$-trajectory from \cref{def.trajectory_discrete}!

\end{example}

\begin{example}\label{ex.steady_state_as_behavior_discrete}
  Consider the simple system $\Sys{Fix}$ with:
  \begin{itemize}
  \item $\State{Fix} = \{\ast\}$.
  \item $\Out{Fix} = \{\ast\}$.
  \item $\In{Fix} = \{\ast\}.$
  \item $\expose{Fix} = \id$.
  \item $\update{Fix}(\ast, \ast) = \ast$.
  \end{itemize}

  As a transition diagram, this looks like:
\[
\begin{tikzpicture}
	\node[draw] {
  \begin{tikzcd}[column sep=small]
    \LMO{\ast} \ar[loop left, "\ast"]
  \end{tikzcd}
  };
\end{tikzpicture}
\]

  A behavior $s : \Sys{Fix} \to \Sys{S}$ in an arbitrary system $\Sys{S}$ should be a loop of this shape within the transition
  diagram of $\Sys{S}$: a steady state.
\[
\begin{tikzpicture}
	\node[draw] (Diag1) {
  \begin{tikzcd}[column sep=small]
    \LMO{\ast} \ar[loop left, red, "\ast"]
  \end{tikzcd}
  };
  \node[draw, right = of Diag1]  (Diag2) {
  \begin{tikzcd}[column sep=small]
    \LMO{} \ar[out=120, in=90, loop] \ar[in=210, out=250, loop] \ar[rr, bend left = 10] \ar[rr, leftarrow,bend right= 10] \ar[ddr, leftarrow,  bend right= 10] &  & \LMO{} \ar[loop right] \ar[ddl, bend left= 10] \ar[ddl,leftarrow, bend right= 10]  \\
    & & \\
    & \LMO{} \ar[out=300, in=240, loop, red] & 
  \end{tikzcd}
  };
  \draw[->] (Diag1) to (Diag2); 
\end{tikzpicture}
\]

  Let's check that this works. First, we need to know what a chart $\lens{f_{\flat}}{f} :
  \lens{\In{Fix}}{\Out{Fix}} \rightrightarrows \lens{\In{S}}{\Out{S}}$ is. Since
  $\Out{Fix} = \In{Fix} = \{\ast\}$, we have that $f : \{\ast\} \to \Out{S}$ is
  simply an output value of $\Sys{S}$ and $f_{\flat} : \{\ast\}\times\{\ast\} \to
  \In{S}$ is simply an input parameter. Therefore, we might as well write $o$
  for $f$ and $i$ for $f_{\flat}$, to see that a chart $\lens{i}{o} :
  \lens{\{\ast\}}{\{\ast\}} \rightrightarrows \lens{\In{S}}{\Out{S}}$ is a pair
  of elements $i \in \In{S}$ and $o \in \Out{S}$.

  Now, let's see what a $\lens{i}{o}$-behavior $s : \Sys{Fix} \to \Sys{S}$ is. It is
  a function $s : \State{Fix} \to \State{S}$ satisfying a few properties. But
  $\State{S} = \{\ast\}$ so $s : \{\ast\} \to \State{S}$ is a single state of
  $\Sys{S}$. Then, \cref{eqn.behavior_discrete} becomes the equations
  \begin{align*}
    \expose{S}(s) &= o\\
    \update{S}(s, i) &= s
  \end{align*}
  which are precisely the equations defining a $\lens{i}{o}$-steady state from \cref{def.steady_state_discrete}.
  
\end{example}

\begin{example}\label{ex.periodic_orbit_as_behavior}
  Let $0 < n \in \nn$ be a positive natural number, and consider the system $\Sys{Clock_n}$
  having:
  \begin{itemize}
    \item $\State{Clock_n} = \ord{n} = \{1,\ldots, n\}$.
    \item $\Out{Clock_n} = \ord{n}$.
    \item $\In{Clock_n} = \{\ast\}$.
    \item $\expose{Clock_n} = \id$.
    \item $\update{Clock_n}(t, \ast) = \begin{cases} t + 1 &\mbox{if $t < n$}
        \\ 1 &\mbox{if $t = n$}  \end{cases}$.
  \end{itemize}

This is the clock with $n$ hours. Our example system $\Sys{Clock}$ from
\cref{ex.clock_system} is $\Sys{Clock}_{12}$, a clock with 12 hours. Here's what
$\Sys{Clock}_4$ looks like as a transition diagram:
\[
\begin{tikzpicture}
	\node[draw] {
  \begin{tikzcd}[sep=small]
    \LMO{1} \ar[rr, bend left, "\ast"] & & \LMO{2} \ar[dd, bend left, "\ast"] \\
    & & \\
\LMO{3} \ar[rr, leftarrow, bend right, "\ast"'] \ar[uu, bend left, "\ast"] & & \LMO{4}
  \end{tikzcd}
  };
\end{tikzpicture}
\]

A behavior $\gamma : \Sys{Clock_n} \to \Sys{S}$ should be a cycle like this in the
transition diagram of $\Sys{S}$: a periodic orbit. We can see the $\Sys{Clock_\4}$-behavior inside the system shown right:
\[
\begin{tikzpicture}
	\node[draw] (Diag1) {
  \begin{tikzcd}[sep=small]
    \LMO{1} \ar[rr, bend left, red, "\ast"] & & \LMO{2} \ar[dd, bend left, red, "\ast"] \\
    & & \\
\LMO{3} \ar[rr, leftarrow, bend right, red, "\ast"'] \ar[uu, bend left, red, "\ast"] & & \LMO{4}
  \end{tikzcd}
  };
  \node[draw, right= of Diag1] (Diag2) {\begin{tikzcd}[sep=small]
    \LMO{} \ar[loop left] \ar[rr, bend left, red] \ar[dd, leftarrow, bend right, red] &  & \LMO{} \ar[loop right] \ar[dd, bend left, red ]\\
    & & \\
    \LMO{} \ar[loop left] \ar[rr, leftarrow, bend left] \ar[rr, leftarrow, bend right, red] & & \LMO{}
  \end{tikzcd}
};
\draw[->] (Diag1) to (Diag2);
\end{tikzpicture}
\]

Let's check that this works. First, we need to know what a chart
$\lens{f_{\flat}}{f} : \lens{\In{Clock_n}}{\Out{Clock_n}} \rightrightarrows
\lens{\In{S}}{\Out{S}}$ is. Since $\Out{Clock_n} = \ord{n}$ and $\In{Clock_n} =
\{\ast\}$, $f : \ord{n} \to \Out{S}$ is a sequence of $n$ exposed values of
$\Sys{S}$ while $f_{\flat} : \ord{n} \times \{\ast\} \to \In{S}$ is a sequence
of $n$ parameters. Therefore, we might as well write $v$ for $f$ and $p$ for
$f_{\flat}$ to find that a chart $\lens{p}{v} : \lens{\{\ast\}}{\ord{n}}
\rightrightarrows \lens{\In{S}}{\Out{S}}$ consists of an $n$-length sequence of
parameters and an $n$-length sequence of exposed values. 

A $\lens{p}{v}$-behavior $\gamma : \Sys{Clock_n} \to \Sys{S}$, then, is a function
$\gamma : \State{Clock_n} \to \State{S}$ satisfying a few properties. Since
$\State{Clock_n} = \ord{n}$, $\gamma : \ord{n} \to \State{S}$ is a $n$-length
sequence of states of $S$, and \cref{eqn.behavior_discrete} become the equations
\begin{align*}
  \expose{S}(\gamma(t)) &= v(t) \\
  \update{S}(\gamma(t), p(t)) &= \begin{cases} \gamma(t + 1) &\mbox{if $t < n$} \\
\gamma(1) &\mbox{if $t = n$} \end{cases}.
\end{align*}
As we can see, this determines a sequence of length $n$ of states of $\Sys{S}$
which repeats when it gets to the end. In other words, this is a periodic orbit
with periodic parameters as in \cref{def.periodic_orbit_discrete}!
\end{example}

If we have a certain kind of behavior in mind, and we find a system $\Sys{T}$ so
that behaviors of shape $\Sys{T}$ are precisely this kind of behavior, then we
say that $\Sys{T}$ \emph{represents} that behavior. For example, we have just
seen that:
\begin{itemize}
  \item The system $\Sys{Time} = \lens{\_ + 1}{\id} : \lens{\nn}{\nn} \leftrightarrows
    \lens{\{\ast\}}{\nn}$ represents trajectories.
  \item The system $\Sys{Fix} = \lens{\pi_2}{\id} : \lens{\{\ast\}}{\{\ast\}}
    \leftrightarrows \lens{\{\ast\}}{\{\ast\}}$ represents steady states.
  \item The systems $\Sys{Clock_n} = \lens{\_ + { 1 \mbox{ mod } n }}{\id} :
    \lens{\ord{n}}{\ord{n}} \leftrightarrows \lens{\{\ast\}}{\ord{n}}$
    represents periodic orbits with periodic parameters whose period divides $n$.
\end{itemize}

Note that there is always a particularly simple behavior on a system: the
identity behaviors $\id : \State{T} \to \State{T}$. This says that every system
behaves as itself. In particular, $\Sys{Time}$ has a trajectory behavior given by
$\id : \Sys{Time} \to \Sys{Time}$ (namely, the trajectory $s_t = t$), and $\Sys{Fix}$ has a steady state behavior given by $\id : \Sys{Fix} \to \Sys{Fix}$ (namely,
the steady state $\ast$), etc. We refer to the identity behavior of $\Sys{T}$ as the \emph{generic}
behavior of type $\Sys{T}$.

\begin{exercise}\label{ex.find_representatives_discrete}
  Find a representative system for the following kinds of behavior.
  \begin{enumerate}
    \item An eventually periodic orbit (see \cref{def.eventually_periodic_orbit_discrete}) that takes $n$ steps to get to a period
      of size $m$.
    \item A steady-looking trajectory (see \cref{def.steady_looking_trajectory_discrete}).
    \item A periodic orbit of period at most $n$ whose parameters aren't
      necessarily also periodic (see \cref{def.periodic_orbit_discrete}).
    \item An trajectory which yields the same output value at every
      $10^{\text{th}}$ step.
     \qedhere
  \end{enumerate}
\end{exercise}

\begin{remark}\label{rmk.periodic_parameters_versus_not}
As \cref{ex.find_representatives_discrete} shows, the difference between a
periodic orbit and a periodic orbit with periodic parameters can be surmised
precisely by noting that they are represented by systems with different
interfaces. The dynamics of the systems are the same, but the interfaces (and
accordingly, the exposed variable) are different; this explains how the
difference between a periodic orbit and a periodic orbit with periodic
parameters is all in the chart.
\end{remark}

\begin{exercise}\label{ex.represents_what_discrete}
  What kind of behaviors do the following systems represent? First, figure out
  what kind of charts they have, and then see what a behavior with a given chart
  is. Describe in your own words.

  \begin{enumerate}
  \item The system $\Sys{Plus}$ with:
    \begin{itemize}
    \item $\State{Plus} = \nn$.
    \item $\Out{Plus} = \nn$.
    \item $\In{Plus} = \nn$.
    \item $\expose{Plus} = \id$.
    \item $\update{Plus}(t, j) = t + j$.
    \end{itemize}
    %% A chart is a sequence of output values and an a $\nn \times \nn$ grid of
    %% input values. A behavior is a sequence of states with those output values
    %% where state $s_t$ updates on parameter $(t, j)$ to $s_{t + j}$.
  \item The system $\Sys{T_n}$ with:
    \begin{itemize}
    \item $\State{T_n} = \nn$.
    \item $\Out{T_n} = \{0, \ldots, n - 1\}$.
    \item $\In{T_n} = \{\ast\}$.
    \item $\expose{T_n}(t) = t \mod n$.
    \item $\update{T_n}(t, \ast) = t + 1$.
    \end{itemize}
    %% Its a trajectory that looks periodic, and has periodic parameters.
  \item The system $\Sys{XOR}$ with:
    \begin{itemize}
    \item $\State{XOR} = \Set{Bool} = \{\true,\, \false\}$.
    \item $\Out{XOR} = \Set{Bool}$.
    \item $\In{XOR} = \Set{Bool}$.
    \item $\expose{XOR} = \id$.
    \item \[\begin{aligned}
        &\update{XOR}(\true, \true) &= \false, \\
        &\update{XOR}(\false, \true) &= \true, \\
        &\update{XOR}(\true, \false) &= \true, \\
        &\update{XOR}(\false, \false) &= \false.
          \end{aligned}\]
    \end{itemize}
  \item The system $\Sys{List_C}$ for a set of \emph{choices} $C$ with:
    \begin{itemize}
      \item $\State{List_C} = \Set{List}_C$ is the set of lists of elements in
        $C$.
      \item $\Out{List_C} = \Set{List}_C$.
      \item $\In{List_C} = C$.
      \item $\expose{List_C} = \id$.
      \item $\update{List_C}(\ell, c) = c::\ell$, that is, we update a list by
        appending the character $c \in C$ to the start.
    \end{itemize}
  \end{enumerate}
\end{exercise}

While every system $\Sys{T}$ represents some kind of behavior --- just take the kind of
behavior to be exactly described by behaviors $\Sys{T} \to \Sys{S}$ --- we are
most interested in those simple systems $\Sys{T}$ whose behavior we can fully
understand. 


We have written a behavior of shape $\Sys{T}$ in $\Sys{S}$ with an arrow $\phi :
\Sys{T} \to \Sys{S}$. This suggests that there is a category with deterministic
systems as its objects and behaviors as its morphisms; and there is!

\begin{definition}\label{def.category_of_charts}
  The category $\Cat{Chart}_{\cat{C}}$ of charts in $\cat{C}$ has
  \begin{itemize}
    \item Objects the \emph{arenas} $\lens{A^-}{A^+}$, pairs of objects in $\cat{C}$.
    \item Maps the \emph{charts} $\lens{f_{\flat}}{f} : \lens{A^-}{A^+}
      \tto \lens{B^-}{B^+}$.
    \item Composition the composite of a chart $\lens{f_{\flat}}{f} : \lens{A^-}{A^+}
      \tto \lens{B^-}{B^+}$ with a chart $\lens{g_{\flat}}{g} : \lens{B^-}{B^+}
      \tto \lens{C^-}{C^+}$ is 
$$\lens{g_{\flat}}{g} \circ \lens{f_{\flat}}{f} \coloneqq \lens{(a^+, a^-)
  \mapsto g_{\flat}(f(a^+), f_{\flat}(a^+, a^-))}{g \circ f}.$$
     \item The identity chart is $\lens{\pi_2}{\id} : \lens{A^-}{A^+}
       \rightrightarrows \lens{A^-}{A^+}$.
  \end{itemize}
\end{definition}
\begin{exercise}
  Check that $\Cat{Chart}_{\cat{C}}$ is indeed a category. That is,
  \begin{enumerate}
    \item For charts $\lens{f_{\flat}}{f} : \lens{A^-}{A^+}
      \tto \lens{B^-}{B^+}$, $\lens{g_{\flat}}{g} : \lens{B^-}{B^+}
      \tto \lens{C^-}{C^+}$, and $\lens{h_{\flat}}{h} : \lens{C^-}{C^+}
      \tto \lens{D^-}{D^+}$, show that 
$$\lens{h_{\flat}}{h} \circ \left( \lens{g_{\flat}}{g} \circ \lens{f_{\flat}}{f}
\right) = \left( \lens{h_{\flat}}{h} \circ \lens{g_{\flat}}{g} \right) \circ \lens{f_{\flat}}{f}.$$
    \item For a chart $\lens{f_{\flat}}{f} : \lens{A^-}{A^+}
      \tto \lens{B^-}{B^+}$, show that 
$$ \lens{\pi_2}{\id} \circ \lens{f_{\flat}}{f} = \lens{f_{\flat}}{f} =
\lens{f_{\flat}}{f} \circ \lens{\pi_2}{\id}.$$
  \end{enumerate}
\end{exercise}

\begin{exercise}\label{ex.special_charts}
  What are the charts of the following forms in simpler terms? 
  \begin{enumerate}
    \item $\lens{f_{\flat}}{f} : \lens{\ord{1}}{\ord{1}} \tto \lens{A^-}{A^+}$.
    \item $\lens{f_{\flat}}{f} : \lens{A^-}{A^+} \tto \lens{\ord{1}}{\ord{1}}$.
    \item $\lens{f_{\flat}}{f} : \lens{\ord{1}}{A^+} \tto \lens{B^-}{B^+}$.
   \end{enumerate}
\end{exercise}

\begin{proposition}\label{prop.category_of_systems_discrete}
  There is a category $\Cat{Sys}$ with deterministic systems as its objects and
  where a map $\Sys{T} \to \Sys{S}$ is a pair consisting of a chart
  $\lens{f_{\flat}}{f} : \lens{\In{T}}{\Out{T}} \rightrightarrows
  \lens{\In{S}}{\Out{S}}$ and a $\lens{f_{\flat}}{f}$-behavior $\phi : \Sys{T}
  \to \Sys{S}$. Composition is given by composing both the charts and the
  functions on states, and identities are given by the generic behaviors: the identity chart with the
  identity function $\id : \State{T} \to \State{T}$.
\end{proposition}
\begin{proof}
  We just need to check that the composite $\psi \circ \phi$ of two behaviors $\phi : \Sys{T} \to
  \Sys{S}$ and $\psi : \Sys{S} \to \Sys{U}$ with charts $\lens{f_{\flat}}{f} : \lens{\In{T}}{\Out{T}} \rightrightarrows
  \lens{\In{S}}{\Out{S}}$ and $\lens{g_{\flat}}{g} : \lens{\In{S}}{\Out{S}} \rightrightarrows
  \lens{\In{U}}{\Out{U}}$ is a behavior with chart $\lens{g_{\flat}}{g} \circ
  \lens{f_{\flat}}{f}$. That is, we need to check that
  \cref{eqn.behavior_discrete} is satisfied for $\psi \circ \phi$. We can do
  this using the fact that it is satisfied for both $\psi$ and $\phi$.
  \begin{align*}
    \expose{U}(\psi(\phi(t))) &= \psi(\expose{S}(\phi(t))) \\
                              &= \psi(\phi(\expose{T}(t))).\\
  \end{align*}
  \begin{align*}
    \update{U}(\psi(\phi(t)), &g_{\flat}(f(\expose{T}(t)), f_{\flat}(\expose{T}(t), i)))  \\
                              &=\update{U}(\psi(\phi(t)), g_{\flat}(\expose{S}(\phi(t)), f_{\flat}(\expose{T}(t), i))) \\
                              &= \psi(\update{S}(\phi(t), f_{\flat}(\expose{T}(t), i))) \\
    &= \psi(\phi(\update{T}(t, i))).
    &\qedhere
  \end{align*}
\end{proof}

There are two different ways to understand what composition of behaviors means:
one based on post-composition, and the other based on pre-composition.
\begin{itemize}
  \item We see that any behavior $\Sys{S} \to \Sys{U}$ gives a way of turning
    $\Sys{T}$-shaped behaviors in $\Sys{S}$ to $\Sys{T}$-shaped behaviors in $\Sys{U}$.
  \item We see that any behavior $\Sys{T} \to \Sys{S}$ gives a way of turning
    $\Sys{S}$-shaped behaviors in $\Sys{U}$ into $\Sys{T}$-shaped behaviors in $\Sys{U}$.
\end{itemize}

\begin{example}\label{ex.steady_state_gives_trajectory_discrete}
  Any steady state $s$ can be seen as a particularly simple trajectory: $s_t =
  s$ for all $t$. We have seen in
  \cref{ex.steady_state_as_behavior}
  that steady states are $\Sys{Fix}$-shaped behaviors. We can use composition of behaviors to
  understand how steady states give rise to trajectories.

  The generic steady state $\ast$ of $\Sys{Fix}$ (that is, the identity behavior of
  $\Sys{Fix}$) generates a trajectory $s : \nn \to \State{Fix}$ with input
  parameters $p_t = \ast$ and $s_t = \ast$. This gives us a behavior $s :
  \Sys{Time} \to \Sys{Fix}$.
  
  Now, for every steady state $\gamma : \Sys{Fix} \to \Sys{S}$, we may compose
  to get a trajectory $\gamma \circ s : \Sys{Time} \to \Sys{S}$.
\end{example}

\begin{exercise}\label{ex.behaviors_as_change_of_kind_discrete}
Adapt the argument of \cref{ex.steady_state_gives_trajectory_discrete} to show
that
\begin{enumerate}
  \item Any eventually periodic orbit gives rise to a trajectory.
  \item If $n$ divides $m$, then any orbit of period at most $n$ gives rise to
    an orbit of period of most $m$.
\qedhere
\end{enumerate}
\end{exercise}


\paragraph{Isomorphisms of Systems}

Now that we have a category of systems and behaviors, category theory supplies
us with a definition of isomorphism for systems. 
\begin{definition}\label{def.isomorphism_of_systems_discrete}
  An \emph{isomorphism} of a system $\Sys{T}$ with a system $\Sys{S}$ is a
  a behavior $\phi : \Sys{T} \to \Sys{S}$ for which there is another behavior
  $\phi\inv : \Sys{S} \to \Sys{T}$ such that $\phi \circ \phi \inv =
  \id_{\Sys{S}}$ and $\phi \inv \circ \phi = \id_{\Sys{T}}$.
\end{definition}

Let's see that this is indeed a good notion of sameness for systems.
\begin{proposition}\label{prop.isomorphism_of_systems_discrete}
  A behavior $\phi : \Sys{T} \to \Sys{S}$ is an isomorphism if and only if the
  following conditions hold:
\begin{enumerate}
  \item The map $\phi : \State{T} \to \State{S}$ is an isomorphism of sets --- a
    bijection.
  \item The chart $\lens{f_{\flat}}{f} : \lens{\In{T}}{\Out{T}}
    \rightrightarrows \lens{\In{S}}{\Out{S}}$ of $\phi$ is an isomorphism in
    $\Cat{Chart}_{\smset}$. That is, $f : \Out{T} \to \Out{S}$ is a bijection
    and there is a bijection $f_{\flat}' : \In{T} \to \Out{T}$ such that
    $f_{\flat} = f_{\flat}' \circ \pi_2$.
\end{enumerate}
\end{proposition}
\begin{proof}
  Since composition in the category of systems and behaviors is given by
  composition of the underlying charts and maps, $\phi$ is an isomorphism of
  systems if and only if its action on states is a bijection and its chart is an
  isomorphism in the category of charts. It just remains to see that our
  description of isomorphism of charts is accurate, which we leave to \cref{ex.isomorphism_in_category_of_charts}.
\end{proof}

\begin{exercise}\label{ex.isomorphism_in_category_of_charts}
  Show that a chart $\lens{f_{\flat}}{f} : \lens{A^-}{A^+}
    \rightrightarrows \lens{B^-}{B^+}$ is an isomorphism if and only if $f$ is
    an isomorphisms and there is an isomorphism $f_{\flat}' : A^- \to B^-$ such
    that $f_{\flat} = f_{\flat}' \circ \pi_2$. 
\end{exercise}



\subsection{Internal behaviors}

While we will often be interested in behaviors of systems that change the
interface in the sense of having non-trivial charts, we will also be interested
in behaviors of systems that do not changed the exposed variables at all. These
behaviors play a very different role in the theory of dynamical systems than
behaviors like trajectories and steady states. Because they don't change
observable behavior (since they have identity chart), they say more about how we
\emph{model} the observable behavior than what that behavior is itself.

\begin{definition}\label{def.cat_of_systems_discrete}
  Let $\lens{I}{O}$ be an arena. The category
$$\Cat{Sys}\lens{I}{O}$$
of deterministic $\lens{I}{O}$-systems has as objects the systems
$\lens{\update{S}}{\expose{S}} : \lens{\State{S}}{\State{S}} \fromto
\lens{I}{O}$ with interface $\lens{I}{O}$ and as maps the \emph{internal behaviors} $\phi :
\Sys{T} \to \Sys{S}$, those behaviors whose chart is the identity chart on $\lens{I}{O}$.
\end{definition}

\begin{example}\label{ex.simulation_discrete}
  Recall the $\lens{\{\Green, \Orange\}}{\{a, b\}}$-system $\Sys{S}$ from
  \cref{ex.transition_diagram_discrete}:
 \[
\begin{tikzpicture}
	\node[draw] {
  \begin{tikzcd}[column sep=small]
  	\LMOO{1}{a}\ar[rr, dgreen, thick, bend left]\ar[loop left, thick, orange]&&
  	\LMOO{2}{b}\ar[ll, thick, orange, bend left]\ar[dl, bend left, thick, dgreen]\\&
  	\LMOO{3}{b} \ar[ul, thick, orange, bend left] \ar[loop left, thick, dgreen]
  \end{tikzcd}
  };
\end{tikzpicture}
\]
 If we had built this system as a model of some relationships between input
 colors and output letters we were seeing in the wild, then we have made this
 system a bit redundant. If the output is $a$, and we feed it $\Green$, the
 output will be $b$; if we feed it $\Orange$, the output will be $a$. Similarly,
 if the output is $b$ --- no matter which of states $2$ or $3$ the system
 is actually in --- and we feed it $\Green$, the output will again be $b$, and
 if we feed it $\Orange$, the output will be $a$. And there really isn't much
 else going on in the system.

 We can package this observation into a behavior in $\Cat{Sys}\lens{\{\Green,
   \Orange\}}{\{a, b\}}$. Let $\Sys{U}$ be the system
 \[
\begin{tikzpicture}
	\node[draw] {
  \begin{tikzcd}[column sep=small]
  	\LMOO{1}{a}\ar[rr, dgreen, thick, bend left]\ar[loop left, thick, orange]&&
  	\LMOO{2}{b}\ar[ll, thick, orange, bend left]\ar[loop left, thick, dgreen]
  \end{tikzcd}
  };
\end{tikzpicture}
\]
We can give a behavior $q : \Sys{S} \to \Sys{U}$ with identity chart as follows
defined by
\begin{align*}
  q(1) &= 1 \\
  q(2) &= 2 \\
  q(3) &= 2
\end{align*}
We can check, by cases, that this is indeed a behavior. That it is a behavior in
$\Cat{Sys}\lens{\{\Green, \Orange\}}{\{a, b\}}$ means that it doesn't change the
observable behavior.
\end{example}

\cref{ex.simulation_discrete} also gives us an example of an important relation between
systems: \emph{bisimulation}. We saw what it means for two systems to be
isomorphic: it means they have isomorphic states and the same dynamics and
output relative to those isomorphisms. But this is sometimes too strong a notion
of sameness for systems; we want to know when two systems \emph{look the same}
on the outside. 

Let's see what this notion looks like for deterministic systems; then we will
describe it in a doctrinal way.
\begin{definition}\label{def.bisimulation_discrete}
  In the deterministic doctrine, a \emph{bisimulation} $\sim$ between $\lens{I}{O}$-systems $\Sys{S}$ and
  $\Sys{U}$ is a relation $\sim : \State{S} \times \State{U} \to \{\true,
  \false\}$ between states of these systems such that $s \sim u$ only when $s$
  and $u$ have related dynamics:
  \begin{align*}
    s \sim u &\mbox{ implies } \expose{S}(s) = \expose{U}(u) \\
    s \sim u &\mbox{ implies } \update{S}(s, i) \sim \update{U}(u, i) \mbox{ for all $i \in I$}.
  \end{align*}
  If $\sim$ is a bisimulation, we say that $s$ and $u$ are \emph{bisimilar} when
  $s \sim u$.
  
  A bisimulation $\sim$ is said to be \emph{total} if every $s \in \State{S}$ is
  bisimilar to some $u \in \State{U}$ and vice-versa.
\end{definition}

Bisimilarity is a strong relation between states of systems. For deterministic
systems, this implies that they act the same on any input.
\begin{proposition}\label{prop.bisimulation_discrete}
  Let $\Sys{S}$ and $\Sys{U}$ be deterministic $\lens{I}{O}$-systems, and let $\sim$ be a
  total bisimulation between them. If $s_0 \sim u_0$ are bisimilar, then they
  induce the same transformation on streams of inputs into streams of outputs: 
$$\fun{transform}^{s_0}_{\Sys{S}} = \fun{transform}^{u_0}_{\Sys{U}}.$$
\end{proposition}
\begin{proof}
Let $i : \nn \to I$ be a stream of inputs. Let $s : \nn \to \State{S}$ be the
stream of states generated by $s_0$ and similarly, let $u : \nn \to \State{U}$
be the stream of states generated by $u_0$. 

We first show that $s_n \sim u_n$ for all $n$. Our base case holds by
hypothesis; now suppose that $s_n \sim u_n$ seeking $s_{n+1} \sim u_{n + 1}$.
Well,
\[s_{n +1} = \update{S}(s_n, i_n) \sim \update{U}(u_n, i_n) = u_{n+1}\]
because $\sim$ is a bisimulation.

Finally, 
\[
\fun{transform}_{\Sys{S}}(i)_n = \expose{S}(s_n) = \expose{U}(u_n) = \fun{transform}_{\Sys{U}}(i)_n
\]
because $s_n \sim u_n$. 
\end{proof}

We can talk about bisimilar states without reference to the particular
bisimulation between the systems they are a part of because, as it turns out,
being bisimilar is independent of the particular bisimulation. To see this, we
need to introduce an interesting system: the system of trees.

\begin{definition}
  Let $\lens{I}{O}$ be an arena in the deterministic doctrine. An
  $\lens{I}{O}$-tree $\tau$ (or a $O$-labeled, $I$-branching tree) consists of:
  \begin{itemize}
    \item A \emph{root} $\fun{root}(\tau) \in O$.
    \item For each parameter $i \in I$, a \emph{child} tree $\fun{child}(\tau,
      i)$.
  \end{itemize}
\end{definition}


\begin{definition}
Let $\lens{I}{O}$ be an arena in the deterministic doctrine. The $\lens{I}{O}$-system
$\Sys{Tree}_{\littlelens{I}{O}}$ of $\lens{I}{O}$-trees has
\begin{itemize}
\item $\State{Tree}$ is the set of $\lens{I}{O}$-trees.
\item Each tree exposes its root: $\expose{Tree}(\tau) = \fun{root}(\tau)$.
\item The system updates by following a tree down the $i^{\text{th}}$ branch: $\update{Tree}(\tau, i) = \fun{child}(\tau, i)$
\end{itemize}
\end{definition}

We can think of an $\lens{I}{O}$-tree as a stream of possible outputs of an
$\lens{I}{O}$-system. In the current state, we see the root of the tree. When we
transition to the next state with parameter $i$, we will see the rest of the
output. This observation suggests a universal characterization of the system of $\lens{I}{O}$-trees.

\begin{proposition}
  The $\lens{I}{O}$-system $\Sys{Tree}_{\lens{I}{O}}$ of $\lens{I}{O}$-trees is
  terminal in the category of $\lens{I}{O}$-systems.
\end{proposition}
\begin{proof}
 We will show that there is a unique simulation $!_{\Sys{S}} : \Sys{S} \to
 \Sys{Tree}_{\lens{I}{O}}$ for any $\lens{I}{O}$-system $\Sys{S}$.
 For any $s \in \State{S}$, we will define a tree $!_{\Sys{S}}(s)$ of outputs
 visible from the state $s$. We define this as
 follows:
 \begin{itemize}
   \item The root of $!_{\Sys{S}}(s)$ is the variable exposed by $\Sys{S}$: \[\fun{root}(!_{\Sys{S}}(s)) = \expose{S}(s).\]
   \item The $i^{\text{th}}$ child of $!_{\Sys{S}}(s)$ is the tree of outputs
     visible from the next state $\update{S}(s, i)$: \[\fun{child}(!_{\Sys{S}}(s), i) = !_{\Sys{S}}(\update{S}(s, i)).\]
 \end{itemize}
 Now, we can show that this is a simulation \emph{and} that it is the unique
 such similation by noticing that this definition is precisely what is required
 to satisfy the defining laws of a simulation.
\end{proof}

Now we can express the idea that bisimilarity of states is independent of any
particular bisimulation between their systems with the following theorem.

\begin{theorem}
  Let $\Sys{S}$ and $\Sys{U}$ be $\lens{I}{O}$-systems. A state $s \in
  \State{S}$ is bisimilar to a state $u \in \State{U}$ for some bisimulation
  $\sim$ between $\Sys{S}$ and $\Sys{U}$ if and only if $!_{\Sys{S}}(s) = !_{\Sys{U}}(u)$.
\end{theorem}
\begin{proof}
 First, let's show that if $s$ is bisimilar to $u$ via some bisimulation $\sim$,
 then $!_{\Sys{S}}(s) = !_{\Sys{U}}(u)$. Now, to show that two trees are equal,
 we need to show that they have equal roots and equal children.
 \begin{itemize}
   \item The root of $!_{\Sys{S}}(s)$ is $\expose{S}(s)$, and the root of
     $!_{\Sys{U}}(u)$ is $\expose{U}(u)$. But since $s \sim u$ by hypothesis,
     these are equal.
   \item Similarly, the $i^{\text{th}}$ child of $!_{\Sys{S}}(s)$ is
    $!_{\Sys{S}}(\update{S}(s, i))$, while the $i^{\text{th}}$ child of
    $!_{\Sys{U}}(u)$ is $!_{\Sys{U}}( \update{U}(u, i) )$. But since $\sim$ is a
    bisimulation, we have that $\update{S}(s, i) \sim \update{U}(u, i)$, and so
    by the same argument we are giving, we will find that
    $!_{\Sys{S}}(\update{S}(s, i)) = !_{\Sys{U}}(\update{U}(u,
    i))$.\footnote{This style of proof is called \emph{proof by co-induction}.
      Where induction assumes a base case and then breaks apart the next step
      into a smaller step, co-induction shows that the proof can always be
      continued in a manner which covers all possible options.}
 \end{itemize}

On the other hand, suppose that $!_{\Sys{S}}(s) = !_{\Sys{U}}(u)$. We now need
to defined a bisimulation $\sim$ between $\Sys{S}$ and $\Sys{U}$ for which $s
\sim u$. For any sequence of inputs $i : \ord{n} \to I$, we can evolve a system
in state $s$ by the entire sequence $i$ to yield a state $\update{S}^{\ast}(s,
i)$ in the following way:
\begin{itemize}
  \item If $n = 0$, then $\update{S}^{\ast}(s, i) = s$.
  \item For $n + 1$, then $\update{S}^{\ast}(s, i) =
    \update{S}(\update{S}^{\ast}(s, i|_{\ord{n}}), i_{n + 1})$.
\end{itemize}
We may then define $\sim$ in the following way.

\begin{center}
$x \sim y$ if and only if there is an $n \in \nn$ and $i : \ord{n} \to I$ with
$x = \update{S}^{\ast}(s, i)$ and $y = \update{U}^{\ast}(u, i)$.
\end{center}
It remains to show that this is a bisimulation.

For any
$\lens{I}{O}$-tree $\tau$ and any $n$-length sequence $i : \ord{n} \to I$ of
parameter (for any $n \in \nn$), we can follow the path $i$ through the tree
$\tau$ to get a new tree $\fun{subtree}(\tau, i)$:
\begin{itemize}
  \item If $n = 0$, then $\fun{subtree}(\tau, i) = \tau$.
  \item For $n + 1$, $\fun{subtree}(\tau, i) = \fun{child}(\fun{subtree}(\tau,
    i|_{\ord{n}}))$ is the $i^{\text{th}}$ child of the tree found by following
    $i$ for the first $n$ steps.
\end{itemize}
Note that $\!_{\Sys{S}}(\update{S}^{\ast}(s, i)) = \fun{subtree}(!_{\Sys{S}}(s),
i)$ by a quick inductive argument. Now we can show that $\sim$ is a bisimulation.

\begin{itemize}
  \item Suppose that $x \sim y$, seeking to show that $\expose{S}(x) =
    \expose{U}(y)$. By hypothesis, $x = \update{S}^{\ast}(s, i)$ and $y =
    \update{U}^{\ast}(u, i)$. But then
    \begin{align*}
      \expose{S}(x) &= \fun{root}(!_{\Sys{S}}(x)) \\
                    &= \fun{root}(\fun{subtree}(!_{\Sys{S}}(s), i)) \\
      &= \fun{root}(\fun{subtree}(!_{\Sys{U}}(u), i)) \\
                    &= \fun{root}(!_{\Sys{U}}(y)) \\
      &= \expose{U}(y).
    \end{align*}
  \item Suppose that $x \sim y$, seeking to show that
    $\update{S}(x, j) \sim \update{U}(y, j)$. By hypothesis, $x =
    \update{S}^{\ast}(s, i)$ and same for $y = \update{U}^{\ast}(u,i)$. Then
    letting $i' : \ord{n+1} \to \nn$ be defined by $i'_{n+1} = j$ and $i'_k =
    i_k$ otherwise, we see that $\update{S}(x, j) = \update{S}^{\ast}(s, i')$
    and $\update{U}(y, j) = \update{U}^{\ast}(y, i')$, so that by definition
    they are related by $\sim$.
\end{itemize}
\end{proof}
%% writehere

%---- Section ----%
\section{Dealing with two kinds of composition: Double categories}

In this section, we will introduce the notion of \emph{double category} to help
us deal with our two kinds of composition: the composition of systems, and the
composition of behaviors.

\begin{definition}\label{def.double_category}
  A \emph{double category} $\cat{D}$ has:
  \begin{itemize}
    \item A class $\const{ob}\cat{D}$ of \emph{objects}.
    \item A \emph{horizontal} category $h \cat{D}$ whose objects are those of
      $\cat{D}$. We call the maps in $h\cat{D}$ the \emph{horizontal} maps of $\cat{D}$.
    \item A \emph{vertical} category $v \cat{D}$ whose objects are those of
      $\cat{D}$. We call the maps in $v\cat{D}$ the \emph{vertical} maps of $\cat{D}$.
    \item For vertical maps $j : A \to B$ and $k : C \to D$ and horizontal maps
      $f : A \to C$ and $g : B \to D$, there is a set of
      \emph{squares}
      \[
        \begin{tikzcd}[sep=small]
          A \ar[dd, "j"'] \ar[rr, "f"] & & B \ar[dd, "k"] \\
           & \alpha & \\
          C \ar[rr, "g"'] & & D
        \end{tikzcd}
      \]
    \item Squares can be composed both horizontally and vertically:
        \[
        \begin{tikzcd}[sep=small]
          A_1 \ar[dd, "j"'] \ar[rr, "f_1"] & & A_2 \ar[dd, "k"]  \ar[rr, "f_2"]&
          & A_3 \ar[dd, "\ell"]\\
           & \alpha &  & \beta &\\
          B_1 \ar[rr, "g_1"'] & & B_2 \ar[rr, "f_2"] & & B_3
        \end{tikzcd} \mapsto
        \begin{tikzcd}[sep=small]
          A_1 \ar[dd, "j"'] \ar[rr, "f_2 f_1"] & & A_3 \ar[dd, "\ell"] \\
           & \alpha \mid \beta & \\
          B_1 \ar[rr, "g_2 g_1"'] & & B_3
        \end{tikzcd}
        \]
        \[
        \begin{tikzcd}[sep=small]
          A_1 \ar[dd, "j_1"'] \ar[rr, "f"] & & A_2 \ar[dd, "k_1"]  \\
           & \alpha &  \\
           B_1 \ar[dd, "j_1"'] \ar[rr, "g"] & & B_2 \ar[dd, "k_2"]\\
           & \beta & \\
           C_1 \ar[rr, "h"']& & C_2
        \end{tikzcd} \mapsto
        \begin{tikzcd}[sep=small]
          A_1 \ar[dd, "j_2 j_1"'] \ar[rr, "f"] & & A_3 \ar[dd, "k_2 k_1"] \\
           & \frac{\alpha}{\beta} & \\
          C_1 \ar[rr, "h"'] & & B_3
        \end{tikzcd}
        \]
      \item For every vertical map $j : A \to B$, there is an identity square
        \[
        \begin{tikzcd}[sep=small]
          A \ar[dd, "j"'] \ar[rr, equals] & & A \ar[dd, "j"] \\
           & j & \\
          B \ar[rr, equals] & & B
        \end{tikzcd}
        \]
        which we will also refer to as $j$, for convenience. Similarly, for
        every horizontal map $f : A \to B$, there is an identity square
        \[
        \begin{tikzcd}[sep=small]
          A \ar[rr, "f"] \ar[dd, equals] & & A \ar[dd, equals] \\
           & f & \\
          B \ar[rr, "f"] & & B
        \end{tikzcd}
        \]
        which we will also refer to as $f$, for convenience.
      \item Vertical and horizontal composition is associative and unital, and
        the \emph{interchange law} holds. That is:
        \begin{itemize}
          \item For horizontally composable squares $\alpha$, $\beta$, and
            $\gamma$, $$(\alpha \mid \beta) \mid \gamma = \alpha \mid (\beta \mid
            \gamma).$$
          \item For vertically composable squares $\alpha$, $\beta$, and
            $\gamma$,%
\footnote{If you're seeing this and feeling worried about fractions, you can put your mind at ease; we promise there will be no fractions. Only squares next to squares.}
\[
\begin{array}{c}
\left(  \begin{array}{c}
\alpha \\ \hline
\beta
\end{array}\right)\\ \hline
\gamma 
\end{array}
= 
\begin{array}{c}
  \alpha \\ \hline
\left(  \begin{array}{c}
\beta \\ \hline
\gamma
\end{array}\right)
\end{array}
\]
\item For a square $\alpha$ with left and right vertical edges $j$ and $k$
  respectively, 
$$j \mid \alpha = \alpha = \alpha | k.$$
\item For a square $\alpha$ with top and bottom horizontal edges $f$ and $g$,
$$\frac{f}{\alpha} = \alpha = \frac{\alpha}{g}.%
\footnote{There aren't any fractions here either.}$$
\item For four appropriately composable squares $\alpha$, $\beta$, $\gamma$, and
  $\delta$, the following interchange law holds:
$$\frac{\alpha \mid \beta}{\gamma \mid \delta} = \left.
  \frac{\alpha}{\beta} \middle|  \frac{\gamma}{\delta} \right. .$$
        \end{itemize}
  \end{itemize}
\end{definition}

Phew, that was quite the definition! The reason the definition of a double
category is so much more involved than the definition of a category is that
there is more than twice the data: there's the vertical category and the
horizontal category, but also how they interact through the squares. 

\begin{remark}
  Just like we notate the identity square on a vertical morphism $j$ by $j$ and
  the identity square on a horizontal morphism $f$ by $f$, we will often denote
  composition of vertical morphisms by $\frac{f}{g}$ and of horizontal morphisms
  by $j \mid k$. This notation agrees with the composition of their respective
  identity squares, and will be much more pleasant to look at when writing equations.
\end{remark}


Finally, we are ready to meet the double category of arenas. This is where our dynamical
systems live, and where they behave.

\begin{definition}\label{def.double_category_of_arenas_discrete}
  The \emph{double category of arenas} in the deterministic doctrine is a double
  category which has:
  \begin{itemize}
  \item Its objects are the \emph{arenas}, pairs of sets $\lens{A^-}{A^+}$.
  \item Its horizontal category is the category of charts.
  \item Its vertical category is the category of lenses.
  \item There is a square of the following form
    \begin{equation}\label{eqn.dbl_cat_arena_square}
      \begin{tikzcd}
        \lens{A^-}{A^+} \ar[r, shift left, "\lens{f_{\flat}}{f}"] \ar[r, shift
        right] \ar[d, shift right, "\lens{j^{\sharp}}{j}"'] \ar[d, shift left,
        leftarrow] & \lens{B^-}{B^+} \ar[d, shift left, leftarrow,
        "\lens{k^{\sharp}}{k}"] \ar[d, shift right]\\
        \lens{C^-}{C^+} \ar[r, shift right, "\lens{g^{\sharp}}{g}"'] \ar[r,
        shift left] & \lens{D^-}{D^+}
      \end{tikzcd}
    \end{equation}
    if and only if the following equations hold:
    \begin{align}\label{eqn.dbl_cat_arena_square_commuting}
      g(j(a^+)) &= k(f(a^+)) \\
      k^{\sharp}(f(a^+), g_{\flat}(j(a^+), c^-)) &= f_{\flat}(a^+, j^{\sharp}(a^+, c^-)) 
    \end{align}
    for all $a^+ \in A^+$ and $c^- \in C^-$.
  \end{itemize}
\end{definition}

It's not obvious from this definition that we actually get a double category
with this definition. It's not even clear that we have defined a way to compose
the squares vertically and horizontally.

It turns out we don't need to know anything else to know that we can compose
these squares, at least in principle. This is because there is at most one
square filling any two charts and two lenses that line up as in
\cref{eqn.dbl_cat_arena_square}; to compose these squares just means that if we
have two such squares lining up, the defining equations
\cref{eqn.dbl_cat_arena_square_commuting} hold also for the appropriate
composites. We call double categories with this property \emph{thin}.

\begin{definition}
  A double category is \emph{thin} if there is at most one square of any
  signature.
\end{definition}

So long as composition is well defined in a thin double category, the laws of
associativity and interchange for square composition come for free; there is at
most one square of the appropriate signature, so any two you can write down are
already equal. We do still have to show that composition is well defined in this
way, which we'll do a bit more generally in \cref{sec.groth_double_construction}

Taking for granted that the double category of arenas is indeed a double
category, what does this mean for systems? Well, behaviors are particular
squares in the double category of arenas.

\begin{proposition}\label{prop.behavior_as_square_of_arenas_discrete}
  Let $\Sys{T}$ and $\Sys{S}$ be dynamical systems. A behavior $\phi : \Sys{T}
  \to \Sys{S}$ is equivalently a square of the following form in the double
  category of arenas:
  \begin{equation}\label{eqn.behavior_as_square_of_arenas_discrete}
    \begin{tikzcd}
      \lens{\State{T}}{\State{T}} \ar[r, shift left, "\lens{\phi \circ
        \pi_2}{\phi}"] \ar[r, shift right] \ar[d, shift right,
      "\lens{\update{T}}{\expose{T}}"'] \ar[d, shift left, leftarrow] &
      \lens{\State{S}}{\State{S}} \ar[d, shift left, leftarrow,
      "\lens{\update{S}}{\expose{S}}"] \ar[d, shift right]\\
      \lens{\In{T}}{\Out{T}} \ar[r, shift right, "\lens{f^{\sharp}}{f}"'] \ar[r,
      shift left] & \lens{\In{S}}{\Out{S}}
    \end{tikzcd}
  \end{equation}
\end{proposition}
\begin{proof}
  This is a simple matter of checking the definitions against eachother. The
  defining equations of \cref{def.double_category_of_arenas_discrete} specialize
  to the defining equations of \cref{def.behavior_discrete}.
\end{proof}


\begin{remark}\label{rmk.double_category_direction}
While the definition of double category we gave treated both horizontal and
vertical directions the same, we will often want to see a square
\[
  \begin{tikzcd}[sep=small]
    A \ar[dd, "j"'] \ar[rr, "f"] & & B \ar[dd, "k"] \\
    & \alpha & \\
    C \ar[rr, "g"'] & & D
  \end{tikzcd}
\]
as a sort of map $\alpha : j \to k$ from its left to its right side, or a map
$\alpha : f \to g$ from its top to its bottom side. For example, the systems
themselves are certain lenses (vertical maps), and the behaviors are squares
between them. On the other hand, we can also see a square as a way of wiring
together charts. 
\end{remark}

\begin{example}\label{ex.understanding_squares_in_double_cat_of_arenas}
  A square
  \begin{equation}\label{eqn.understanding_squares_in_double_cat_of_arenas}
    \begin{tikzcd}
      \lens{A^-}{A^+} \ar[r, shift left, "\lens{f_{\flat}}{f}"] \ar[r, shift
      right] \ar[d, shift right, "\lens{j^{\sharp}}{j}"'] \ar[d, shift left,
      leftarrow] & \lens{B^-}{B^+} \ar[d, shift left, leftarrow,
      "\lens{k^{\sharp}}{k}"] \ar[d, shift right]\\
      \lens{C^-}{C^+} \ar[r, shift right, "\lens{g^{\sharp}}{g}"'] \ar[r, shift
      left] & \lens{D^-}{D^+}
    \end{tikzcd}
    \end{equation}
  can be seen as a chart between lenses, that is, two charts which are
  compatible according to the wiring pattern the lenses describe. For example, consider a square of the
  following form where $\lens{w^{\sharp}}{w}$ is a wiring diagram:
  \[
    \begin{tikzcd}
      \lens{\ord{1}}{\ord{1}} \ar[r, shift left, "\lens{b^-}{b^+}"] \ar[r, shift
      right] \ar[d, shift right, equals] \ar[d, shift left, equals] &
      \lens{B^-}{B^+} \ar[d, shift left, leftarrow,
      "\lens{w^{\sharp}}{w}"] \ar[d, shift right]\\
      \lens{\ord{1}}{\ord{1}} \ar[r, shift right, "\lens{d^-}{d^+}"'] \ar[r,
      shift left] & \lens{D^-}{D^+}
    \end{tikzcd}
  \]
  By \cref{ex.special_charts}, we know that the charts in this diagram are pairs
  of elements $\lens{b^-}{b^+}$ and $\lens{d^-}{d^+}$ in the arenas
  $\lens{B^-}{B^+}$ and $\lens{D^-}{D^+}$ respectively. The square then says
  that $\lens{d^-}{d^+}$ are the values you would get if you passed
  $\lens{b^-}{b^+}$ along the wires in the wiring diagram
  $\lens{w^{\sharp}}{w}$.
\end{example}

Let's take a minute to see what composition of squares in the double category of
arenas means for systems. Horizontal composition is familiar because it's what
lets us compose behaviors:
\[
  \begin{tikzcd}
    \lens{\State{T}}{\State{T}} \ar[r, shift left, "\lens{\phi \circ
      \pi_2}{\phi}"] \ar[r, shift right] \ar[d, shift right,
    "\lens{\update{T}}{\expose{T}}"'] \ar[d, shift left, leftarrow] &
    \lens{\State{S}}{\State{S}} \ar[r, shift left, "\lens{\psi \circ
      \pi_2}{\psi}"] \ar[r, shift right] \ar[d, shift left, leftarrow,
    "\lens{\update{S}}{\expose{S}}"] \ar[d, shift right] &
    \lens{\State{U}}{\State{U}}\ar[d, shift left, leftarrow,
    "\lens{\update{U}}{\expose{U}}"] \ar[d, shift right] \\
    \lens{\In{T}}{\Out{T}} \ar[r, shift right, "\lens{f_{\flat}}{f}"'] \ar[r,
    shift left] & \lens{\In{S}}{\Out{S}} \ar[r, shift right,
    "\lens{g_{\flat}}{g}"'] \ar[r, shift left]& \lens{\In{U}}{\Out{U}}
  \end{tikzcd} \xequals{\quad}
  \begin{tikzcd}
    \lens{\State{T}}{\State{T}} \ar[r, shift left, "\lens{\psi\phi \circ
      \pi_2}{\psi\phi}"] \ar[r, shift right] \ar[d, shift right,
    "\lens{\update{T}}{\expose{T}}"'] \ar[d, shift left, leftarrow] &
    \lens{\State{U}}{\State{U}} \ar[d, shift left, leftarrow,
    "\lens{\update{U}}{\expose{U}}"] \ar[d, shift right]\\
    \lens{\In{T}}{\Out{T}} \ar[r, shift right, "\lens{f_{\flat}}{f} \then \lens{g_{\flat}}{g}"']
    \ar[r, shift left] & \lens{\In{U}}{\Out{U}}
  \end{tikzcd}
\]

On the other hand, vertical composition tells us something else interesting: if
you get a chart $\lens{g_{\flat}}{g}$ by wiring together a chart
$\lens{f_{\flat}}{f}$, then a behavior $\phi$ with
chart $\lens{f_{\flat}}{f}$ induces a behavior with chart $\lens{g_{\flat}}{g}$
on the wired together systems.
\[
  \begin{tikzcd}
    \lens{\State{T}}{\State{T}} \ar[r, shift left, "\lens{\phi \circ
      \pi_2}{\phi}"] \ar[r, shift right] \ar[d, shift right,
    "\lens{\update{T}}{\expose{T}}"'] \ar[d, shift left, leftarrow] &
    \lens{\State{S}}{\State{S}} \ar[d, shift left, leftarrow,
    "\lens{\update{S}}{\expose{S}}"] \ar[d, shift right]\\
    \lens{\In{T}}{\Out{T}} \ar[d, shift right, "\lens{j^{\sharp}}{j}"'] \ar[d, shift left,
        leftarrow] \ar[r, shift right, "\lens{f_{\flat}}{f}"']
    \ar[r, shift left] & \lens{\In{S}}{\Out{S}} \ar[d, shift left, leftarrow,
        "\lens{k^{\sharp}}{k}"] \ar[d, shift right]\\
    \lens{I}{O} \ar[r, shift right, "\lens{g_{\flat}}{g}"']
    \ar[r, shift left] & \lens{I'}{O'} \\
  \end{tikzcd} \xequals{\quad}
  \begin{tikzcd}
    \lens{\State{T}}{\State{T}} \ar[r, shift left, "\lens{\phi \circ
      \pi_2}{\psi\phi}"] \ar[r, shift right] \ar[d, shift right,
    "\lens{f^{\sharp}}{f} \circ \lens{\update{T}}{\expose{T}}"'] \ar[d, shift left, leftarrow] &
    \lens{\State{S}}{\State{S}} \ar[d, shift left, leftarrow,
    "\lens{k^{\sharp}}{k} \circ \lens{\update{S}}{\expose{S}}"] \ar[d, shift right]\\
    \lens{I}{O} \ar[r, shift right, "\lens{g_{\flat}}{g}"']
    \ar[r, shift left] & \lens{I}{O'}
  \end{tikzcd}
\]

\begin{example}\label{ex.understanding_squares_in_double_cat_of_arenas2}
  Continuing from \cref{ex.understanding_squares_in_double_cat_of_arenas},
  suppose that we have a $\lens{b^-}{b^+}$-steady state $s$ in a system $\Sys{S}$:
  \begin{equation}\label{eqn.understanding_squares_in_double_cat_of_arenas2}
    \begin{tikzcd}
      \lens{\ord{1}}{\ord{1}} \ar[r, shift left, "\lens{s}{s}"] \ar[r, shift
      right] \ar[d, shift right, equals] \ar[d, shift left, equals] &
      \lens{\State{S}}{\State{S}} \ar[d, shift left, leftarrow,
      "\lens{\update{S}}{\expose{S}}"] \ar[d, shift right]\\
      \lens{\ord{1}}{\ord{1}} \ar[r, shift right, "\lens{b^+}{b^-}"'] \ar[r,
      shift left] & \lens{B^-}{B^+}
    \end{tikzcd}
  \end{equation}
  We can see that $s$ is a $\lens{d^-}{d^+}$-steady state of the wired system by
  vertically composing the square in
  \cref{eqn.understanding_squares_in_double_cat_of_arenas2} with the square in
  \cref{eqn.understanding_squares_in_double_cat_of_arenas}. This basic fact
  underlies our arguments in \cref{sec.steady_states_matrix_arithmetic}. 

We'll return to this idea in \cref{ex.prof_square_from_arena_square}.
\end{example}

Recall the categories $\Cat{Sys}\lens{I}{O}$ of systems with the interface
$\lens{I}{O}$ from
\cref{def.cat_of_systems_discrete}. One thing that vertical composition in the
double category of arenas shows us is that wiring together systems is functorial
with respect to behaviors that don't change the interface.
\begin{proposition}\label{prop.lens_comp_functor_discrete}
  For a lens $\lens{f^{\sharp}}{f} : \lens{I}{O} \leftrightarrows
  \lens{I'}{O'}$, we get a functor
  $$\Cat{Sys}\lens{f^\sharp}{f} : \Cat{Sys}\lens{I}{O} \to
  \Cat{Sys}\lens{I}{O}$$
  Given by composing with $\lens{f^\sharp}{f}$:
  \begin{itemize}
    \item For a system $\Sys{S} = \lens{\update{S}}{\expose{S}} :
      \lens{\State{S}}{\State{S}}\fromto \lens{I}{O}$,
      $$\Cat{Sys}\lens{f^{\sharp}}{f}(\Sys{S}) = \lens{f^{\sharp}}{f} \circ \lens{\update{S}}{\expose{S}}.$$
    \item For a behavior, $\Cat{Sys}\lens{f^{\sharp}}{f}$ acts in the following way:
      \[
  \begin{tikzcd}
    \lens{\State{T}}{\State{T}} \ar[r, shift left, "\lens{\phi \circ
      \pi_2}{\phi}"] \ar[r, shift right] \ar[d, shift right,
    "\lens{\update{T}}{\expose{T}}"'] \ar[d, shift left, leftarrow] &
    \lens{\State{S}}{\State{S}} \ar[d, shift left, leftarrow,
    "\lens{\update{S}}{\expose{S}}"] \ar[d, shift right]\\
    \lens{I}{O} \ar[r, shift right, equals]
    \ar[r, shift left, equals] & \lens{I}{O'}
  \end{tikzcd} \quad \mapsto \quad
  \begin{tikzcd}
    \lens{\State{T}}{\State{T}} \ar[r, shift left, "\lens{\phi \circ
      \pi_2}{\phi}"] \ar[r, shift right] \ar[d, shift right,
    "\lens{\update{T}}{\expose{T}}"'] \ar[d, shift left, leftarrow] &
    \lens{\State{S}}{\State{S}} \ar[d, shift left, leftarrow,
    "\lens{\update{S}}{\expose{S}}"] \ar[d, shift right]\\
    \lens{\In{T}}{\Out{T}} \ar[d, shift right, "\lens{f^{\sharp}}{f}"'] \ar[d, shift left,
        leftarrow] \ar[r, shift right, equals]
    \ar[r, shift left, equals] & \lens{\In{S}}{\Out{S}} \ar[d, shift left, leftarrow,
        "\lens{f^{\sharp}}{f}"] \ar[d, shift right]\\
    \lens{I}{O} \ar[r, shift right, equals]
    \ar[r, shift left, equals] & \lens{I'}{O'} \\
  \end{tikzcd} 
      \]
  \end{itemize}
\end{proposition}
\begin{proof}
  The functoriality of this construction can be seen immediately from the
  interchange law of the double category:
  \begin{align*}
  \frac{\left.\littlelens{\phi \circ \pi_2}{\phi} \middle| \littlelens{psi
    \circ \pi_2}{\psi} \right. }{\littlelens{f^{\sharp}}{f}} &= \frac{\left.\littlelens{\phi \circ \pi_2}{\phi} \middle| \littlelens{psi
    \circ \pi_2}{\psi} \right. }{\left. \littlelens{f^{\sharp}}{f} \middle| \littlelens{f^{\sharp}}{f} \right.} &\mbox{by the horizontal identity law,}\\
    &= \left.
      \frac{\littlelens{\phi \circ \pi_2}{\phi}}{\littlelens{f^{\sharp}}{f}}
      \middle|\frac{\littlelens{\psi \circ
          \pi_2}{\psi}}{\littlelens{f^{\sharp}}{f}} \right. &\mbox{by the interchange law.}
  \end{align*}
\end{proof}


\end{document}
