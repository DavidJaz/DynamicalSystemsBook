\documentclass[DynamicalBook]{subfiles}
\begin{document}
%


\setcounter{chapter}{0}%Just finished 0.


%------------ Chapter ------------%
\chapter{Preface}\label{chapter.0}

This book is a work in progress --- including the acknowledgements below! Use at your own peril!


















Categorical systems theory is an emerging field of study which seeks to apply the methods of category theory to study systems in general. General systems theory has long struggled with settling into an strong formalism, largely because what ``system'' can mean differs so starkly between different fields and different practictioners that it can be hard to find crisp mathematical definitions that connect all these different ways that we model the world. Category theory is uniquely suited to address this problem because it is the mathematics of formal analogy-making. By focusing on the ways that mathematical structures relate to eachother, we can step back and find uniform patterns not in particular definitions but in the relationships these definitions have to eachother.

This focus on relationships rather than definitions is what leads categorical systems theorists to study the way systems can be composed together to become more complicated systems. The ways that systems can be composed together outlines the sorts of relationships they can enter into; and, for the designer of systems, this \emph{modularity} is a useful tool for keeping track of increasingly complex arrangements of components. The main definitions in categorical systems theory therefore describe what we are composing (that is, what a system is) and how we are going to compose them. The main theorems of categorical systems theory are \emph{compositionality} theorems which say that certain properties or behaviors of composite systems may be understood in terms of their components and the ways they were composed

The dream of categorical systems theory is then to provide tools for reasoning about systems in general, with the idea that these general tools will allow for a fruitful cross polination between different disciplines of applied mathematics.

The purpose of this book is to provide an introduction to working with a certain flavor of categorical systems theory --- one which makes central use of the concept of a \emph{lens} which originated in functional programming --- and to gesture at what a foundation of the subject of categorical systems theory might look like. Bill Lawvere describes the goal of a foundation to be ``to concentrate the essense of practice and in turn use the result to guide practice'' in his paper \emph{Foundations and Applications} \cite{Lawvere:Foundations.and.Applications}. This book is the beginnings of a foundation of categorical systems theory of that sort; it introduces a particular formalism for working with dynamical systems that arise as lenses in the first five chapters, and then briefly shows how this formalism can be used to describe other sorts of categorical systems theories in \cref{Chapter.6}.

While this is not a first introduction to applied category theory (for that, see the wonderful \cite{fong2019seven}), it will not assume anything in particular about categorical systems theory and will explain all the important categorical machinery which doesn't appear in the usual introductions to category theory, such as indexed categories and double categories.


\section*{Acknowledgments}

David Spivak has been a friend and mentor to me as I write this book and beyond.
In many ways, I see this book as my take on David's research in lens based
systems in recent years. David and I began writing a book together, of which
this book was to be the first half and David's book on polynomial functors (now
co-authored with Nelson Niu) was to be the second. But as we were writing, we
realized that these weren't two halves of the same book, but rather two books
in the same genre. It was a great pleasure writing with David during the summer
of 2020, and I owe him endless thanks for ideas, suggestions, and great
conversation. This book wouldn't exist without him.



Thanks go to Emily Riehl, tslil clingman, Sophie Libkind, John Baez, Geoff Cruttwell, Brendan
Fong, Christian Williams.

Thanks to Henry Story for pointing out typos.

This book was written with support from the Topos Institute.


\end{document}
