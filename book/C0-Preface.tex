\documentclass[DynamicalBook]{subfiles}
\begin{document}
%


\setcounter{chapter}{0}%Just finished 0.


%------------ Chapter ------------%
\chapter{Preface}\label{chapter.0}

Organization is form: it's the form something takes to carry out its mission. Said another way, a mission, a coherent set of values and techniques, is something around which an organization can be formed. Organizations channel something into the world.

 The world's most beautiful organizational marvels---from cats and dogs to people and effective altruistic organizations---can sometimes be seen to achieve their own life purposes, to be like a dog playing with other dogs on a beach. And then they are a pleasure to behold, a display of the abundance that can be found when things just work.

What is that abundance and play of forms? How do we talk about it in a way that borders between poetic and deeply meaningful on the one hand, and plain and open for investigation on the other? We long to participate together in a conversation about our love: the play that can happen when one is truly at home somewhere. Do you see the dogs playing on the beach?!

We believe a deeper conversation is possible, but it may require the development of appropriate language, good distinctions to have for a sustained consideration of the play of life. How would you create language by which we could see the game of catch being played by two friends? How would you talk about a clock and a counter that could help us track changes? How could you  plug together simpler things and fashion a program for the way they'll interact?

Suppose some people want to create the best conceptual tools for this play. Where would they start? Our answer is to start where we are: two mathematicians who like to play. We invite you to nothing more than a play of forms that we ourselves find both fun and important to the pursuit of  values.

For better or for worse, we start from an advanced level: one of us has a PhD, and the other will likely have one soon. We have studied a language form called category theory, to which we are deeply indebted. Our thinking about reality is steeped in this language form, because we find it so effective in helping us talk about things we care about.

We can never get \emph{at} what we deeply are within a finite language, but we can talk \emph{about} it, dance around it and point like we're touched by something unseen. Here we attempt to invite you in and sketch a vehicle for your own journey, at least for a phase along its way. 


\section*{Acknowledgments}

Thanks go to David Spivak, Emily Riehl, Sophie Libkind, John Baez.


\end{document}
