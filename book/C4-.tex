\documentclass[DynamicalBook]{subfiles}
\begin{document}
%


\setcounter{chapter}{3}%Just finished 3.


%------------ Chapter ------------%
\chapter{Change of Doctrine}

\section{Introduction}

In the last chapter, we saw a general formulation of the notion of behavior of
system and precise definition of the notion of dynamical system doctrine. Let's
recall the definition of dynamical system doctrine.

\begin{definition}\label{def.doctrine}
A \emph{dynamical system doctrine} consists of an indexed category $\cat{A} :
\cat{C}\op \to \Cat{Cat}$ together with a section $T$.
\end{definition}

This concise definition packs a big punch. Describing a dynamical system
doctrine amounts to answering the informal questions about what it means to be a
system:
\begin{informal}
  A \emph{doctrine} of dynamical systems is a particular way to answer the following
  questions about what it means to be a dynamical system:
  \begin{enumerate}
  \item What does it mean to be a state?
  \item How should the output vary with the state --- discretely,
    continuously, linearly?
  \item Can the kinds of input a
    system takes in depend on what it's putting out, and how do they depend on it?
  \item What sorts of changes are possible in a given state?
  \item What does it mean for states to change. 
  \item How should the way the state changes vary with the input?
  \end{enumerate}
\end{informal}

Constructing a doctrine is no small thing. But once we have a doctrine, we have
may work in its double category of arenas to quickly derive a few
compositionality results about systems.

\begin{definition}\label{def.double_cat_of_arenas_general}
  Let $\cat{A} : \cat{C}\op \to \Cat{Cat}$ be an indexed category. Then the
  category of $\cat{A}$-arenas is defined to be the Grothendieck double
  construction of $\cat{A}$:
$$\Cat{Arena}_{\cat{A}} := \sqiint^{C \in \cat{C}} \cat{A}(C).$$

  Note that the horizontal category of $\Cat{Arena}_{\cat{A}}$ is the category
  $\Cat{Chart}_{\cat{A}}$ of $\cat{A}$-charts (\cref{def.chart_general}), and the vertical category of
  $\Cat{Arena}_{\cat{A}}$ is the category $\Cat{Lens}_{\cat{A}}$ of
  $\cat{A}$-lenses (\cref{def.lens_general}).
\end{definition}

We are now in peak category theory
territory: the statements of our propositions are far longer than their proofs,
which amount to trivial calculations in the double category of arenas. As in
much of categorical work, the difficulty is in understanding what to propose;
once that work is done, the proof flows smoothly from the definitions.

Let's see what composition of squares in the double category of
arenas means for systems. Horizontal composition is familiar because it's what
lets us compose behaviors:
\[
  \begin{tikzcd}
    \lens{T\State{T}}{\State{T}} \ar[r, shift left, "\lens{T\phi}{\phi}"] \ar[r, shift right] \ar[d, shift right,
    "\lens{\update{T}}{\expose{T}}"'] \ar[d, shift left, leftarrow] &
    \lens{T\State{S}}{\State{S}} \ar[r, shift left, "\lens{T\psi}{\psi}"] \ar[r, shift right] \ar[d, shift left, leftarrow,
    "\lens{\update{S}}{\expose{S}}"] \ar[d, shift right] &
    \lens{T\State{U}}{\State{U}}\ar[d, shift left, leftarrow,
    "\lens{\update{U}}{\expose{U}}"] \ar[d, shift right] \\
    \lens{\In{T}}{\Out{T}} \ar[r, shift right, "\lens{f_{\flat}}{f}"'] \ar[r,
    shift left] & \lens{\In{S}}{\Out{S}} \ar[r, shift right,
    "\lens{g_{\flat}}{g}"'] \ar[r, shift left]& \lens{\In{U}}{\Out{U}}
  \end{tikzcd} \xequals{\quad}
  \begin{tikzcd}
    \lens{T\State{T}}{\State{T}} \ar[r, shift left, "\lens{T(\psi\phi)}{\psi\phi}"] \ar[r, shift right] \ar[d, shift right,
    "\lens{\update{T}}{\expose{T}}"'] \ar[d, shift left, leftarrow] &
    \lens{T\State{U}}{\State{U}} \ar[d, shift left, leftarrow,
    "\lens{\update{U}}{\expose{U}}"] \ar[d, shift right]\\
    \lens{\In{T}}{\Out{T}} \ar[r, shift right, "\lens{f_{\flat}}{f} \then \lens{g_{\flat}}{g}"']
    \ar[r, shift left] & \lens{\In{U}}{\Out{U}}
  \end{tikzcd}
\]
So, we have a category of systems and behaviors in any doctrine, just as we
defined in the deterministic doctrine.

On the other hand, vertical composition tells us something else interesting: if
you get a chart $\lens{g_{\flat}}{g}$ by wiring together a chart
$\lens{f_{\flat}}{f}$, then a behavior $\phi$ with
chart $\lens{f_{\flat}}{f}$ induces a behavior with chart $\lens{g_{\flat}}{g}$
on the wired together systems.
\[
  \begin{tikzcd}
    \lens{T\State{T}}{\State{T}} \ar[r, shift left, "\lens{T\phi}{\phi}"] \ar[r, shift right] \ar[d, shift right,
    "\lens{\update{T}}{\expose{T}}"'] \ar[d, shift left, leftarrow] &
    \lens{T\State{S}}{\State{S}} \ar[d, shift left, leftarrow,
    "\lens{\update{S}}{\expose{S}}"] \ar[d, shift right]\\
    \lens{\In{T}}{\Out{T}} \ar[d, shift right, "\lens{j^{\sharp}}{j}"'] \ar[d, shift left,
        leftarrow] \ar[r, shift right, "\lens{f_{\flat}}{f}"']
    \ar[r, shift left] & \lens{\In{S}}{\Out{S}} \ar[d, shift left, leftarrow,
        "\lens{k^{\sharp}}{k}"] \ar[d, shift right]\\
    \lens{I}{O} \ar[r, shift right, "\lens{g_{\flat}}{g}"']
    \ar[r, shift left] & \lens{I'}{O'} \\
  \end{tikzcd} \xequals{\quad}
  \begin{tikzcd}
    \lens{T\State{T}}{\State{T}} \ar[r, shift left, "\lens{T\phi}{\psi\phi}"] \ar[r, shift right] \ar[d, shift right,
    "\lens{f^{\sharp}}{f} \circ \lens{\update{T}}{\expose{T}}"'] \ar[d, shift left, leftarrow] &
    \lens{T\State{S}}{\State{S}} \ar[d, shift left, leftarrow,
    "\lens{k^{\sharp}}{k} \circ \lens{\update{S}}{\expose{S}}"] \ar[d, shift right]\\
    \lens{I}{O} \ar[r, shift right, "\lens{g_{\flat}}{g}"']
    \ar[r, shift left] & \lens{I}{O'}
  \end{tikzcd}
\]

The interchange law of the double category of arenas tells us precisely that
these two sorts of composition of behaviors --- composition as maps and wiring --- \emph{commute}. That is, we can
compose two behaviors and then wire them together, or we can wire each
together and then compose them; the end result is the same.

\begin{example}\label{ex.understanding_squares_in_double_cat_of_arenas2}
  Continuing from \cref{ex.understanding_squares_in_double_cat_of_arenas},
  suppose that we have a $\lens{b^-}{b^+}$-steady state $s$ in a system $\Sys{S}$:
  \begin{equation}\label{eqn.understanding_squares_in_double_cat_of_arenas2}
    \begin{tikzcd}
      \lens{\ord{1}}{\ord{1}} \ar[r, shift left, "\lens{s}{s}"] \ar[r, shift
      right] \ar[d, shift right, equals] \ar[d, shift left, equals] &
      \lens{\State{S}}{\State{S}} \ar[d, shift left, leftarrow,
      "\lens{\update{S}}{\expose{S}}"] \ar[d, shift right]\\
      \lens{\ord{1}}{\ord{1}} \ar[r, shift right, "\lens{b^+}{b^-}"'] \ar[r,
      shift left] & \lens{B^-}{B^+}
    \end{tikzcd}
  \end{equation}
  We can see that $s$ is a $\lens{d^-}{d^+}$-steady state of the wired system by
  vertically composing the square in
  \cref{eqn.understanding_squares_in_double_cat_of_arenas2} with the square in
  \cref{eqn.understanding_squares_in_double_cat_of_arenas}. This basic fact
  underlies our arguments in the upcoming \cref{sec.steady_states_matrix_arithmetic}. 

We'll return to this idea in \cref{ex.prof_square_from_arena_square}.
\end{example}

While our results are most smoothly proven in the double category of arenas,
this double category does not capture the way we think of systems and their
behaviors. To think of a behavior, we must first think of its chart; we solve a
differential equation in terms of its parameters, and to get a specific solution
we must first choose specific parameters. Working in the double category of
arenas means treating the chart $\lens{f_{\flat}}{f}$ and the underlying map
$\phi$ of a behavior on equal footing, but we would instead like to say that
$\phi$ is a behavior for the chart $\lens{f_{\flat}}{f}$. 

We would also like to think of the wiring together of systems along a lens
$\lens{w^{\sharp}}{w}$ as an operation performed on systems, and then inquire
into the relationship of this wiring operation with the (horizontal) composition
of behaviors.

What we need is to separate the \emph{interface} of a system from the system
itself. Charts and lenses are best understood as ways of relating interfaces. It just
so happens that systems and their behaviors can also be expressed as certain
sorts of lenses and charts, which drastically facilitates our working with them.
But there is some sense in which this is not essential; the main point is that
for each interface $\lens{I}{O}$ we have a notion of system with interface
$\lens{I}{O}$, for each lens $\lens{w^{\sharp}}{w} : \lens{I}{O} \fromto
\lens{I'}{O'}$ a way of wiring $\lens{I}{O}$-systems into $\lens{I'}{O'}$
systems, and for each chart $\lens{f_{\flat}}{f} : \lens{I}{O} \tto
\lens{I'}{O'}$ a notion of behavior for this chart. It is very convenient that
we can describe wiring and composition of behaviors in the same terms as charts
and lenses, but we shouldn't think that they are the same thing.

In this chapter, we will define the appropriate abstract algebra of systems and
their two sorts of composition keeping in mind the separation between interfaces
and systems. We call this abstract algebra a \emph{doubly indexed category},
since it is a sort of double categorical generalization of an indexed category.
We'll see the definition of this notion in
\cref{sec.indexed_double_category_of_systems}.

Once we have organized our systems into doubly indexed categories, we can
discuss what it means to change our doctrine. A change of doctrine will be a way
of turning one sort of dynamical system into another. This could mean simply
re-interpreting the underlying structure (for example, a deterministic system
where all maps are differentiable is in particular a discrete deterministic
system, just by forgetting the differentiability) or by restricting the use of
certain maps (as in \cref{def.restriction_doctrine}). But it could also mean
approximating one sort of system by another sort of system.

As an example, let's consider the Euler method for approximating a differential
system. Suppose that
\[
\lens{\update{S}}{\expose{S}} : \lens{\rr^{n}}{\rr^n} \fromto \lens{\rr^k}{\rr^m}
\]
is a Euclidean differential system $\Sys{S}$. This represents the differential equation
\[
\frac{ds}{dt} = \update{S}(s, p).
\]
That is, a trajectory is a map $s : \rr \to \rr^n$ satisfying this differential
equation (for a choice of parameters $p : \rr \to \rr^k$). This means that the
direction that the state $s_0$ is tending is given by $\update{S}(s_0, p_0)$. We
could then approximate the solution, given such a starting point, by moving a
small distance in this direction. We could get a whole sequence of states this
way; moving in the direction our dynamics tells us we should go, and then
checking where to go from there.

The result is a deterministic system $\mathcal{E}_{\epsilon}(\Sys{S})$ whose
dynamics is given by
\[
\update{\mathcal{E}_{\epsilon}(\Sys{S})}(s, p) = s + \epsilon \cdot
\update{S}(s, p).
\]
Here, $\epsilon > 0$ is some small increment. We can take $\mathcal{E}_{\epsilon}(\Sys{S})$ to expose the same variable that
$\Sys{S}$ does: $\expose{\mathcal{E}_{\epsilon}(\Sys{S})} = \expose{S}$.

The change of doctrine $\mathcal{E}_{\epsilon}$ is the formula for changing from
the Euclidean differential doctrine to the deterministic doctrine on the
cartesian category of Euclidean spaces. We might wonder: how does changing the
doctrine by using the Euler method affect the wiring together of systems? How
does it affect the behaviors of the systems?

We can answer the question about behaviors here. It is not true that every
behavior of a Euclidean differential system is faithfully represented by its
Euler method approximation. Consider, for example, the simply system
\[
\update{S}(s) = s
\]
having one state variable, and no parameters. The trajectories of this system
are of the form $s(t) = Ce^t$ for some constant $C$. However, if we let
$\epsilon = .1$ and consider the Euler approximation
\[
\update{\mathcal{E}_{.1}(\Sys{S})}(s(0)) = s(0) + .1 \cdot s(0) = 1.01
\cdot C,
\]
This is not the same thing as $s(.1) = Ce^{.1} \approx 1.105 \cdot C$ (though,
as expected, they are rather close). So we see that general behaviors are
\emph{not} preserved!

However, suppose we have a steady state of the system. For example, taking $C =
0$ we get a steady state of the system $\update{S}(s) = s$ above. Then we have
that
\[
\update{\mathcal{E}_{.1}(\Sys{S})}(0) = 0 + .1 \cdot 0 = 0.
\]
In other words, the steady state remains a steady state!

The goal of this chapter will be to introduce the formalism which enables us to
inquire into and prove various compositionality results concerning changes of
doctrine. In the above situation, we will see that the Euler method
$\mathcal{E}_{\epsilon}$ gives a change of doctrine on a \emph{restriction} of
the Euclidean differential doctrine to affine maps. As a result, it will
preserve any behavior whose underlying map is affine (of the form $\phi(v) = Av
+ b$ for a matrix $A$ and vector $b$), which includes all steady states (since
constant maps are affine) but almost no trajectories in general. 

We will introduce the notion of a \emph{doubly indexed functor} to organize the
compositionality results concerning change of doctrine. We will also be using
these doubly indexed functors in the next chapter to organize the
compositionality of behaviors in general.

We will define the notion of change of doctrine formally
(\cref{def.change_of_doctrine}) and show that every change of doctrine gives
rise to a doubly indexed functor between the doubly indexed categories of
systems in the respective doctrines. In particular, we will show that there is a
\emph{2-functor} \jaz{I am starting to think that I should put the
  2-functoriality of this into an appendix so that I don't have to introduce
  2-categories and 2-functors here.}
$$\textbf{Sys} : \textbf{Doctrine} \to \textbf{DblIx}$$
sending a doctrine to the doubly indexed category of systems in it. 



\section{Composing behaviors in general}\label{sec.behaviors_general}

Before we get to this abstract defintion, we will take our time exploring the
sorts of compositionality results one may prove quickly by working in the double
category of arenas.

Recall the categories $\Cat{Sys}\lens{I}{O}$ of systems with the interface
$\lens{I}{O}$ from
\cref{def.cat_of_systems_discrete}. One thing that vertical composition in the
double category of arenas shows us is that wiring together systems is functorial
with respect to simulations --- that is, behaviors that don't change the
interface.

We repeat the definition of $\Cat{Sys}\lens{I}{O}$ for an arbitrary doctrine.
  \begin{definition}\label{def.cat_of_systems}
  Let $\mathbb{D} = (\cat{A}, T)$ be a dynamical system doctrine. For a
    $\cat{A}$-arena $\lens{I}{O}$, the category $\Cat{Sys}\lens{A}{C}$ of
    $\mathbb{D}$-systems with interface $\lens{I}{O}$ is defined by:
\begin{itemize}
  \item Its objects are $\cat{A}$-lenses $\lens{\update{S}}{\expose{S}} :
    \lens{T\State{S}}{\State{S}} \fromto \lens{I}{O}$, which we can call
    \emph{systems} in this general setting.
  \item Its maps are \emph{simulations}, the behaviors which have identity
    chart. That is, the maps are the squares 
\[
    \begin{tikzcd}
      \lens{T\State{T}}{\State{T}} \ar[r, shift left, "\lens{T\phi}{\phi}"] \ar[r, shift right] \ar[d, shift right,
      "\lens{\update{T}}{\expose{T}}"'] \ar[d, shift left, leftarrow] &
      \lens{T\State{S}}{\State{S}} \ar[d, shift left, leftarrow,
      "\lens{\update{S}}{\expose{S}}"] \ar[d, shift right]\\
      \lens{I}{O} \ar[r, shift right, equals] \ar[r,
      shift left, equals] & \lens{I}{O}
    \end{tikzcd}
\]
\item Composition is given by horizontal composition in the double category
  $\Cat{Arena}_{\cat{A}}$ of $\cat{A}$-arenas.
\end{itemize}
  \end{definition}

Now, thanks to the double category of arenas, we can show that every lens
$\lens{f^{\sharp}}{f} : \lens{I}{O} \fromto \lens{I'}{O'}$ gives a functor 
  $$\Cat{Sys}\lens{f^\sharp}{f} : \Cat{Sys}\lens{I}{O} \to
  \Cat{Sys}\lens{I}{O}.$$
We can see this functor as the operation of wiring together our
$\lens{I}{O}$-systems along the lens $\lens{f^{\sharp}}{f}$ to get
$\lens{I'}{O'}$-systems. The functoriality of this operation say that wiring
preserves simulations --- if systems $\Sys{S_i}$ simulate $\Sys{T_i}$ by $\phi_i$,
then the wired together systems $\Sys{S}$ simulate $\Sys{T}$ by $\phi = \prod_i
\phi_i$. 

\begin{proposition}\label{prop.lens_comp_functor_discrete}
  For a lens $\lens{f^{\sharp}}{f} : \lens{I}{O} \leftrightarrows
  \lens{I'}{O'}$, we get a functor
  $$\Cat{Sys}\lens{f^\sharp}{f} : \Cat{Sys}\lens{I}{O} \to
  \Cat{Sys}\lens{I}{O}$$
  Given by composing with $\lens{f^\sharp}{f}$:
  \begin{itemize}
    \item For a system $\Sys{S} = \lens{\update{S}}{\expose{S}} :
      \lens{T\State{S}}{\State{S}}\fromto \lens{I}{O}$,
      $$\Cat{Sys}\lens{f^{\sharp}}{f}(\Sys{S}) = \lens{f^{\sharp}}{f} \circ \lens{\update{S}}{\expose{S}}.$$
    \item For a behavior, $\Cat{Sys}\lens{f^{\sharp}}{f}$ acts in the following way:
      \[
  \begin{tikzcd}
    \lens{T\State{T}}{\State{T}} \ar[r, shift left, "\lens{T\phi}{\phi}"] \ar[r, shift right] \ar[d, shift right,
    "\lens{\update{T}}{\expose{T}}"'] \ar[d, shift left, leftarrow] &
    \lens{T\State{S}}{\State{S}} \ar[d, shift left, leftarrow,
    "\lens{\update{S}}{\expose{S}}"] \ar[d, shift right]\\
    \lens{I}{O} \ar[r, shift right, equals]
    \ar[r, shift left, equals] & \lens{I}{O'}
  \end{tikzcd} \quad \mapsto \quad
  \begin{tikzcd}
    \lens{T\State{T}}{\State{T}} \ar[r, shift left, "\lens{T\phi}{\phi}"] \ar[r, shift right] \ar[d, shift right,
    "\lens{\update{T}}{\expose{T}}"'] \ar[d, shift left, leftarrow] &
    \lens{T\State{S}}{\State{S}} \ar[d, shift left, leftarrow,
    "\lens{\update{S}}{\expose{S}}"] \ar[d, shift right]\\
    \lens{\In{T}}{\Out{T}} \ar[d, shift right, "\lens{f^{\sharp}}{f}"'] \ar[d, shift left,
        leftarrow] \ar[r, shift right, equals]
    \ar[r, shift left, equals] & \lens{\In{S}}{\Out{S}} \ar[d, shift left, leftarrow,
        "\lens{f^{\sharp}}{f}"] \ar[d, shift right]\\
    \lens{I}{O} \ar[r, shift right, equals]
    \ar[r, shift left, equals] & \lens{I'}{O'} \\
  \end{tikzcd} 
      \]
  \end{itemize}
\end{proposition}
\begin{proof}
  The functoriality of this construction can be seen immediately from the
  interchange law of the double category:
  \begin{align*}
  \frac{\left.\littlelens{T\phi}{\phi} \middle| \littlelens{T\psi}{\psi} \right. }{\littlelens{f^{\sharp}}{f}} &= \frac{\left.\littlelens{T\phi }{\phi} \middle| \littlelens{T\psi
    }{\psi} \right. }{\left. \littlelens{f^{\sharp}}{f} \middle| \littlelens{f^{\sharp}}{f} \right.} &\mbox{by the horizontal identity law,}\\
    &= \left.
      \frac{\littlelens{T\phi }{\phi}}{\littlelens{f^{\sharp}}{f}}
      \middle|\frac{\littlelens{T\psi}{\psi}}{\littlelens{f^{\sharp}}{f}} \right. &\mbox{by the interchange law.}
  \end{align*}
  Identities are clearly preserved, since the underlying morphism $\phi :
  \State{T} \to \State{S}$ is not changed. 
\end{proof}


  The notion of profunctor gives us a nice way to understand the relationship
  between a behavior $\phi : \Sys{T} \to \Sys{S}$ and its chart
  $\lens{f_{\flat}}{f} : \lens{I}{O} \tto \lens{I'}{O'}$. When we are using
  behaviors, we usually have the chart $\lens{f_{\flat}}{f}$ in mind first, and
  then look for behaviors with this chart. For example, when finding
  trajectories, we first set the parameters for our system and then solve it. We
  can use profunctors to formalize this relationship.


\begin{proposition}\label{prop.profunctor_from_chart}
  Given a chart $\lens{f_{\flat}}{f} : \lens{I}{O} \tto
  \lens{I'}{O'}$, we get a profunctor
  $$\Cat{Sys}\lens{f_{\flat}}{f} : \Cat{Sys}\lens{I}{O} \tickar
  \Cat{Sys}\lens{I'}{O'}$$

  Defined by:
\begin{align*}
  \Cat{Sys}\lens{f_{\flat}}{f}(\Sys{T}, \Sys{S}) &= \left\{ \phi : \State{T} \to
                                                   \State{S}\, \middle| \mbox{ $\phi$ is a behavior with chart $\lens{f_{\flat}}{f}$} \right\}\\
  &= \left\{  
    {
    \begin{tikzcd}[ampersand replacement = \&]
      \lens{\State{T}}{\State{T}} \ar[r, dashed, shift left, "\lens{\phi \circ
        \pi_2}{\phi}"] \ar[r, dashed, shift right] \ar[d, shift right,
      "\lens{\update{T}}{\expose{T}}"'] \ar[d, shift left, leftarrow] \&
      \lens{\State{S}}{\State{S}} \ar[d, shift left, leftarrow,
      "\lens{\update{S}}{\expose{S}}"] \ar[d, shift right]\\
      \lens{\In{T}}{\Out{T}} \ar[r, shift right, "\lens{f_{\flat}}{f}"'] \ar[r,
      shift left] \& \lens{\In{S}}{\Out{S}}
    \end{tikzcd}
                    }
                    \right\}
\end{align*}

The action of the profunctor $\Cat{Sys}\lens{f_{\flat}}{f}$ on simulations in the
categories $\Cat{Sys}\lens{I}{O}$ and $\Cat{Sys}\lens{I'}{O'}$ is given by
composition on the left and right. That is, for simulations $\phi : \Sys{T'} \to
\Sys{T}$ and $\psi : \Sys{S} \to \Sys{S'}$ and $\lens{f_{\flat}}{f}$-behavior
$\beta \in \Cat{Sys}\lens{f_{\flat}}{f}(\Sys{T}, \Sys{S})$, we define 
\begin{equation}\label{eqn.profunctor_from_chart}
\phi \cdot \beta \cdot \psi := \phi \mid \beta \mid \psi.
\end{equation}
\end{proposition}

\begin{exercise}\label{ex.profunctor_from_chart}
 Prove \cref{prop.profunctor_from_chart}. That is, show that the action defined
 in \cref{eqn.profunctor_from_chart} is functorial, giving a functor 
$$\Cat{Sys}\littlelens{I}{O}\op \times \Cat{Sys}\littlelens{I'}{O'} \to \smset.$$
(Hint: use the double categorical notation. It will be much more concise.)
\end{exercise}


  With a little work in the double category of arenas, we can give a very useful
  example of a square in the double category of profunctors. Consider this
  square in the double category of arenas:
\[ \alpha = 
  \begin{tikzcd}
    \lens{I_1}{O_1} \ar[d, shift right, "\lens{j_{\flat}}{j}"'] \ar[d, shift left,
        leftarrow] \ar[r, shift right]
    \ar[r, shift left, "\lens{f^{\sharp}}{f}"] & \lens{I_2}{O_2} \ar[d, shift left, leftarrow,
        "\lens{k^{\sharp}}{k}"] \ar[d, shift right]\\
    \lens{I_3}{O_3} \ar[r, shift right, "\lens{g_{\flat}}{g}"']
    \ar[r, shift left] & \lens{I_4}{O_4} \\
  \end{tikzcd} 
\]
As we saw in \cref{prop.lens_comp_functor_discrete}, we get functors
$\Cat{Sys}\lens{j^{\sharp}}{j} : \Cat{Sys}\lens{I_1}{O_1} \to
\Cat{Sys}\lens{I_3}{O_3}$ and $\Cat{Sys}\lens{k^{\sharp}}{k} :
\Cat{Sys}\lens{I_2}{O_2} \to \Cat{Sys}\lens{I_4}{O_4}$ given by composing with
these lenses. We also saw in \cref{prop.profunctor_from_chart} that we get
profunctors $\Cat{Sys}\lens{f_{\flat}}{f} : \Cat{Sys}\lens{I_1}{O_1} \tickar
\Cat{Sys}\lens{I_2}{O_2}$ and $\Cat{Sys}\lens{g_{\flat}}{g} :
\Cat{Sys}\lens{I_3}{O_3} \tickar \Cat{Sys}\lens{I_4}{O_4}$ from these charts. We
can also get a square of profunctors 
from the square $\alpha$ in the double category of arenas:
\[
\begin{tikzcd}
  \Cat{Sys}\littlelens{I_1}{O_1} \ar[rr, tick,
  "{\Cat{Sys}\lens{f_{\flat}}{f}}"]\ar[dd,
  "{\Cat{Sys}\littlelens{j^{\sharp}}{j}}"'] & &
  \Cat{Sys}\littlelens{I_2}{O_2} \ar[dd, "{\Cat{Sys}\littlelens{k^{\sharp}}{k}}"]\\
 &\Cat{Sys}(\alpha) & \\
\Cat{Sys}\littlelens{I_3}{O_3} \ar[rr, tick,
"{\Cat{Sys}\lens{g_{\flat}}{g}}"']& & \Cat{Sys}\littlelens{I_4}{O_4}
\end{tikzcd}
\]
That is, a natural transformation of the following signature:
\[
\Cat{Sys}(\alpha) : \Cat{Sys}\lens{f_{\flat}}{f} \to
\Cat{Sys}\lens{g_{\flat}}{g}\left( \Cat{Sys}\littlelens{j^{\sharp}}{j}, \Cat{Sys}\littlelens{k^{\sharp}}{k} \right).\]

To define the natural transformation $\Cat{Sys}(\alpha)$, we need to say what it
does to an element $\phi$ of $\Cat{Sys}\lens{f_{\flat}}{f}\left(\Sys{T},
  \Sys{S}\right)$. Recall that the elements of this profunctor are behaviors with
  chart $\lens{f_{\flat}}{f}$, so really $\phi$ is a square
\[
  \phi =
    \begin{tikzcd}
      \lens{T\State{T}}{\State{T}} \ar[r, shift left, "\lens{T\phi}{\phi}"] \ar[r, shift right] \ar[d, shift right,
      "\lens{\update{T}}{\expose{T}}"'] \ar[d, shift left, leftarrow] &
      \lens{T\State{S}}{\State{S}} \ar[d, shift left, leftarrow,
      "\lens{\update{S}}{\expose{S}}"] \ar[d, shift right]\\
      \lens{I_1}{O_1} \ar[r, shift right, "\lens{f_{\flat}}{f}"'] \ar[r,
      shift left] & \lens{I_2}{O_2}
    \end{tikzcd}
\]
in the double category of arenas. Therefore, we can define
$\Cat{Sys}(\alpha)(\phi)$ to be the vertical composite:
\[
  \begin{tikzcd}
    \lens{T\State{T}}{\State{T}} \ar[r, shift left, "\lens{T\phi}{\phi}"] \ar[r, shift right] \ar[d, shift right,
    "\lens{\update{T}}{\expose{T}}"'] \ar[d, shift left, leftarrow] &
    \lens{T\State{S}}{\State{S}} \ar[d, shift left, leftarrow,
    "\lens{\update{S}}{\expose{S}}"] \ar[d, shift right]\\
    \lens{I_1}{O_1} \ar[d, shift right, "\lens{j^{\sharp}}{j}"'] \ar[d, shift left,
        leftarrow] \ar[r, shift right, "\lens{f_{\flat}}{f}"']
    \ar[r, shift left] & \lens{I_2}{O_2} \ar[d, shift left, leftarrow,
        "\lens{k^{\sharp}}{k}"] \ar[d, shift right]\\
    \lens{I_3}{O_3} \ar[r, shift right, "\lens{g_{\flat}}{g}"']
    \ar[r, shift left] & \lens{I_4}{O_4} \\
  \end{tikzcd}
\]
Or, a little more concisely in double category notation:
$$\Cat{Sys}(\alpha)(\phi) = \frac{\phi}{\alpha}.$$

We record this observation in a proposition.
\begin{proposition}\label{prop.prof_square_from_arena_square}
  Given a square
\[ \alpha = 
  \begin{tikzcd}
    \lens{I_1}{O_1} \ar[d, shift right, "\lens{j_{\flat}}{j}"'] \ar[d, shift left,
        leftarrow] \ar[r, shift right]
    \ar[r, shift left, "\lens{f^{\sharp}}{f}"] & \lens{I_2}{O_2} \ar[d, shift left, leftarrow,
        "\lens{k^{\sharp}}{k}"] \ar[d, shift right]\\
    \lens{I_3}{O_3} \ar[r, shift right, "\lens{g_{\flat}}{g}"']
    \ar[r, shift left] & \lens{I_4}{O_4} \\
  \end{tikzcd} 
\]
in the double category of arenas, we get a square

\[
\begin{tikzcd}
  \Cat{Sys}\littlelens{I_1}{O_1} \ar[rr, tick,
  "{\Cat{Sys}\lens{f_{\flat}}{f}}"]\ar[dd,
  "{\Cat{Sys}\littlelens{j^{\sharp}}{j}}"'] & &
  \Cat{Sys}\littlelens{I_2}{O_2} \ar[dd, "{\Cat{Sys}\littlelens{k^{\sharp}}{k}}"]\\
 &\Cat{Sys}(\alpha) & \\
\Cat{Sys}\littlelens{I_3}{O_3} \ar[rr, tick,
"{\Cat{Sys}\lens{g_{\flat}}{g}}"']& & \Cat{Sys}\littlelens{I_4}{O_4}
\end{tikzcd}
\]
in the double category of categories, functors, and profunctors given by 
$$\Cat{Sys}(\alpha)(\phi) = \frac{\phi}{\alpha}.$$

\end{proposition}

The naturality of this transformation follows from the double category laws. We
leave the particulars as an exercise.
\begin{exercise}\label{ex.prof_square_from_arena_square_naturality}
  Prove that the family of functions 
  \[
\Cat{Sys}(\alpha) : \Cat{Sys}\lens{f_{\flat}}{f} \to
\Cat{Sys}\lens{g_{\flat}}{g}\left( \Cat{Sys}\littlelens{j^{\sharp}}{j}, \Cat{Sys}\littlelens{k^{\sharp}}{k} \right)
\]
defined in \cref{ex.prof_square_from_arena_square} is a natural transformation.
(Hint: use the double category notation, it will be much more concise.)
\end{exercise}



%---- Section ----%
\section{Arranging categories along two kinds of composition: Doubly indexed categories}
\label{sec.indexed_double_category_of_systems}

While we described a category of systems and behaviors in
\cref{prop.category_of_systems_discrete}, we haven't been thinking of systems in
quite this way. We have been organizing our systems a bit more particularly than
just throwing them into one large category. We've made the following observations:
\begin{itemize}
  \item Each system has an interface, and many different systems can have the
    same interface. From this observation, we defined the categories
    $\Cat{Sys}\lens{I}{O}$ of systems with the interface $\lens{I}{O}$ in \cref{def.cat_of_systems_discrete}.
  \item Every wiring diagram, or more generally lens, gives us an operation that
    changes the interface of a system by wiring things together. We formalized
    this observation into a functor $\Cat{Sys}\lens{w^{\sharp}}{w} : \Cat{Sys}\lens{I}{O}
    \to \Cat{Sys}\lens{I'}{O'}$ in \cref{prop.lens_comp_functor_discrete}.
  \item To describe the behavior of a system, first we have to chart out how it
    will look on its interface. We formalized this observation by giving a
    profunctor $\Cat{Sys}\lens{f_{\flat}}{f} : \Cat{Sys}\lens{I}{O} \tickar
    \Cat{Sys}\lens{I'}{O'}$ for each chart in \cref{prop.profunctor_from_chart}.
  \item If we wire together a chart for one interface into a chart for the wired
    interface, then every behavior for that chart gives rise to a behavior for
    the wired together chart. We formalized this observation as a morphism of
    profunctors 
\[
\Cat{Sys}(\alpha) : \Cat{Sys}\lens{f_{\flat}}{f} \to
\Cat{Sys}\lens{g_{\flat}}{g}\left( \Cat{Sys}\littlelens{j^{\sharp}}{j}, \Cat{Sys}\littlelens{k^{\sharp}}{k} \right)
\]
in \cref{prop.prof_square_from_arena_square}.
\end{itemize}

Now comes the time to organize all these observations. In this section, we will
see that collectively, these observations are telling us that there is an
\emph{doubly indexed category} of dynamical systems. We will also see that
matrices of sets give rise to a doubly indexed category which we will call the
doubly indexed category of vectors of sets.

\begin{definition}\label{def.doubly_indexed_category}
A \emph{doubly indexed category} $\cat{A} : \cat{D} \to \Cat{Cat}$ consists of
the following:\footnote{This is what an expert would call a \emph{unital (or
    normal) lax double functor}, but we won't need this concept in any other
  setting.} 
\begin{itemize}
  \item A double category $\cat{D}$ called the \emph{indexing base}.
  \item For every object $D \in \cat{D}$, we have a category $\cat{A}(D)$.
  \item For every vertical arrow $j : D \to D'$, we have a functor $\cat{A}(j) :
    \cat{A}(D)
    \to \cat{A}(D')$.
  \item For every horizontal arrow $f : D \to D'$, we have a profunctor
$\cat{A}(f) : \cat{A}(D) \tickar \cat{A}(D')$.
  \item For every square 
\[
        \begin{tikzcd}[sep=tiny]
          A \ar[dd, "j"'] \ar[rr, "f"] & & B \ar[dd, "k"] \\
           & \alpha & \\
           C \ar[rr, "g"'] & & D
        \end{tikzcd}
\]
in $\cat{D}$, a square
\[
        \begin{tikzcd}[sep=tiny]
          \cat{A}(A) \ar[dd, "\cat{A}(j)"'] \ar[rr, "\cat{A}(f)"] & & \cat{A}(B) \ar[dd, "\cat{A}(k)"] \\
           & \cat{A}( \alpha ) & \\
          \cat{A}( C ) \ar[rr, "\cat{A}( g )"'] & & \cat{A}(D)
        \end{tikzcd}
\]
in $\Cat{Cat}$.
\item For any two horizontal maps $f : A \to B$ and $g : B \to C$ in
  $\cat{D}$, we have a square $\mu_{f, g} : \cat{A}(f) \odot \cat{A}(g) \to
  \cat{A}(f \mid g)$ called the \emph{compositor}:
\begin{equation}\label{eqn.compositor}
        \begin{tikzcd}[sep=tiny]
          \cat{A}(A) \ar[dd, equals] \ar[r, "\cat{A}(f)"] & \cat{A}(B) \ar[r, "\cat{A}(f)"] & \cat{A}(E) \ar[dd, equals] \\
           & \mu_{f, g} & \\
          \cat{A}( A ) \ar[rr, "\cat{A}(f \mid g)"'] & & \cat{A}(C)
        \end{tikzcd}
\end{equation}
\end{itemize}
This data is required to satisfy the following laws:
\begin{itemize}
  \item (Vertical Functoriality) For vertical maps $j : D \to D'$ and $k : D' \to D''$, we have
    that $$\cat{A}\left( \frac{j}{k} \right) = \frac{\cat{A}(j)}{\cat{A}(k)}$$
and that $\cat{A}(\id_D) = \id_{\cat{A}(D)}$.\footnote{Here, we are hiding some
  coherence issues. While our doubly indexed category of deterministic systems
  will satisfy this functoriality condition on the nose, we will soon see a
  doubly indexed category of matrices of sets for which this law only holds up
  to a coherence isomorphism. Again, the issue invovles shuffling parentheses
  around, and we will sweep it under the rug.}
  \item (Horizontal Lax Functoriality) For horizontal maps $f : D_1 \to D_2$, $g :
    D_2 \to D_3$ and $h : D_3 \to D_4$, the compositors $\mu$ satisfy the
    following associativity and unitality conditions:
\begin{itemize}
\item (Associativity) $$\frac{\mu_{f, g} | \cat{A}(h)}{\mu_{(f \mid g), h}} \coheq 
  \frac{\cat{A}(f) | \mu_{g, h}}{\mu_{f, (g \mid h)}}.$$
\item (Unitality) The profunctor $\cat{A}(\id_{D_1}) : \cat{A}(D_1) \tickar
  \cat{A}(D_1)$ is the identity profunctor, $\cat{A}(\id_{D_1}) = \cat{A}(D_1)$.
  Furthermore, $\mu_{\id_{D_1}, f}$ and $\mu_{f, \id_{D_2}}$ are equal to the
  isomorphisms of \cref{ex.identity_profunctor} given by the naturality of
  $\cat{A}(f)$ on the left and right respectively. We may sumarize this may
  saying that 
$$\mu_{\id, f} = \id_{\cat{A}(f)} = \mu_{f, \id}.$$
\end{itemize}
\item (Naturality of Compositors) For any horizontally composable squares
  \[
        \begin{tikzcd}[sep=tiny]
          A \ar[dd, "j"'] \ar[rr, "f_1"] & & B \ar[dd, "k"] \\
           & \alpha & \\
           C \ar[rr, "g_1"'] & & D
        \end{tikzcd}
        \quad\mbox{and}\quad
        \begin{tikzcd}[sep=tiny]
          B \ar[dd, "k"'] \ar[rr, "f_2"] & & E \ar[dd, "l"] \\
           & \alpha & \\
           D \ar[rr, "g_2"'] & & F
        \end{tikzcd}
  \]
  
\[
\frac{\cat{A}( \alpha ) \mid \cat{A}( \beta )}{\mu_{g_1, g_2}} = \frac{\mu_{f_1,f_2}}{\cat{A}(\alpha \mid \beta)}.
\] 
\end{itemize}
\end{definition}

That's another big definition! It seems like it will be a slog to actually ever
prove that something is a doubly indexed category. Luckily, in our cases, these
proofs will go quite smoothly. This is because each of the three laws of a
doubly indexed category has a sort of sister law from the definition of a double
category which will help us prove it.

\begin{itemize}
  \item The Vertical Functoriality law will often involve the vertical
    associativity and unitality of squares in the indexing base.
  \item The Horizontal Lax Functoriality law will often involve the horizontal
    associativity and unitality of squares in the indexing base.
  \item The Naturality of Compositors law will often involve the interchange law
    in the indexing base.
\end{itemize}

We'll see how these sisterhoods play out in practice as we define the doubly
indexed categories of deterministic systems and vectors of sets.

\paragraph{The doubly indexed category of systems}

Let's show that systems in a doctrine $\mathbb{D}$ do indeed form a doubly indexed category
$$\Cat{Sys}_{\mathbb{D}} : \Cat{Arena}_{\mathbb{D}} \to \Cat{Cat}.$$

\begin{definition}\label{def.doubly_indexed_cat_systems}
  The doubly indexed category $\Cat{Sys}_{\mathbb{D}} : \Cat{Arena}_{\mathbb{D}}
  \to \Cat{Cat}$ of systems in the doctrine $\mathbb{D} = (\mathcal{A}, T)$ is defined
  as follows:
\begin{itemize}
\item Our indexing base is the double category $\Cat{Arena}_{\mathbb{D}}$ of arenas, since we
  will arrange our systems according to their interface.
\item To every arena $\lens{I}{O}$, we associate the category $\Cat{Sys}\lens{I}{O}$
of systems with interface $\lens{I}{O}$ and behaviors whose chart is the
identity chart on $\lens{I}{O}$ (\cref{def.cat_of_systems}).
\item To every lens $\lens{w^{\sharp}}{w} : \lens{I}{O} \fromto \lens{I'}{O'}$, we associate the functor
$\Cat{Sys}\lens{w^{\sharp}}{w} : \Cat{Sys}\lens{I}{O} \to \Cat{Sys}\lens{I}{O}$ given by wiring according to
$\lens{w^{\sharp}}{w}$:
$$\Cat{Sys}\lens{w^{\sharp}}{w}(\Sys{S}) =
\frac{\Sys{S}}{\littlelens{w^{\sharp}}{w}}.$$
This is defined in \cref{prop.lens_comp_functor_discrete}.
\item To every chart $\lens{f_{\flat}}{f} : \lens{I}{O} \tto \lens{I'}{O'}$, we
  associate the profunctor $\Cat{Sys}\lens{f_{\flat}}{f} : \Cat{Sys}\lens{I}{O}
  \tickar \Cat{Sys}\lens{I'}{O'}$ which sends the $\lens{I}{O}$-system $\Sys{T}$
  and the $\lens{I'}{O'}$-system $\Sys{S}$ to the set of behaviors $\Sys{T} \to
  \Sys{S}$ with chart $\lens{f_{\flat}}{f}$:
\begin{align*}
  \Cat{Sys}\lens{f_{\flat}}{f}(\Sys{T}, \Sys{S}) &= \left\{ \phi : \State{T} \to
                                                   \State{S}\, \middle| \mbox{ $\phi$ is a behavior with chart $\lens{f_{\flat}}{f}$} \right\}\\
  &= \left\{  
    {
    \begin{tikzcd}[ampersand replacement = \&]
      \lens{T\State{T}}{\State{T}} \ar[r, dashed, shift left, "\lens{T\phi}{\phi}"] \ar[r, dashed, shift right] \ar[d, shift right,
      "\lens{\update{T}}{\expose{T}}"'] \ar[d, shift left, leftarrow] \&
      \lens{T\State{S}}{\State{S}} \ar[d, shift left, leftarrow,
      "\lens{\update{S}}{\expose{S}}"] \ar[d, shift right]\\
      \lens{\In{T}}{\Out{T}} \ar[r, shift right, "\lens{f^{\sharp}}{f}"'] \ar[r,
      shift left] \& \lens{\In{S}}{\Out{S}}
    \end{tikzcd}
                    }
                    \right\}
\end{align*}
We saw this profunctor in \cref{prop.profunctor_from_chart}.
\item To every square $\alpha$, we assign the morphism of profunctors given by
  composing vertically with $\alpha$ in $\Cat{Arena}$:
$$\Cat{Sys}(\alpha)(\phi) = \frac{\phi}{\alpha}.$$
We saw in \cref{prop.prof_square_from_arena_square_naturality} that this was a
natural transformation.
\item The compositor is given by horizontal composition in the double category
  of arenas:
  \begin{align*}
    \mu_{\littlelens{f_{\flat}}{f},\littlelens{g_{\flat}}{g}} : \Cat{Sys}\littlelens{f_{\flat}}{f} \odot \Cat{Sys}\littlelens{g_{\flat}}{g} &\to \Cat{Sys}\left( \littlelens{f_{\flat}}{f} \then \littlelens{g_{\flat}}{g} \right) \\
(\phi, \psi) &\mapsto \phi \mid \psi
  \end{align*}
\end{itemize}
\end{definition}

Let's check now that this does indeed satisfy the laws of a doubly indexed
category. The task may appear to loom over us; there are quite a few laws, and
there is a lot of data involved. But nicely, they all follow quickly from a bit of fiddling in the double
category of arenas.
\begin{itemize}
  \item (Vertical Functoriality) We show that $\Cat{Sys}\left(
      \lens{k^{\sharp}}{k} \circ \lens{j^{\sharp}}{j} \right) =
    \Cat{Sys}\lens{k^{\sharp}}{k} \circ \Cat{Sys}\lens{j^{\sharp}}{j}$ by
    vertical associativity:
\begin{align*}
  \Cat{Sys}\left(\lens{k^{\sharp}}{k} \circ \lens{j^{\sharp}}{j} \right)(\phi) &= \frac{\phi}{\left( \frac{\littlelens{j^{\sharp}}{j}}{\littlelens{k^{\sharp}}{k}} \right)} 
= \frac{\left( \frac{\phi}{\littlelens{j^{\sharp}}{j}} \right)}{\littlelens{k^{\sharp}}{k}} \\
&= \Cat{Sys}\lens{k^{\sharp}}{k} \circ \Cat{Sys}\lens{j^{\sharp}}{j}(\phi).
\end{align*}

\item (Horizontal Lax Functoriality) This law follows from horizontal
  associativity in $\Cat{Arena}$.
\begin{align}
  \mu(\mu(\phi, \psi), \xi) = (\phi \mid \psi ) \mid \xi = \phi \mid (\psi \mid \xi) = \mu(\phi, \mu(\psi, \xi)).
\end{align}
\item (Naturality of Compositor) This law follows from interchange in
  $\Cat{Arena}$.
\begin{align*}
  \left( \frac{\Cat{Sys}(\alpha) \mid \Cat{Sys}(\beta)}{\mu} \right)(\phi, \psi) &= \left. \frac{\phi}{\alpha} \middle| \frac{\psi}{\beta} \right. 
= \frac{\phi \mid \psi}{\alpha \mid \beta} \\
&= \left(  \frac{\mu}{\Cat{Sys}(\alpha \mid \beta)}\right)(\phi, \psi).
\end{align*}
\end{itemize}


\paragraph{The doubly indexed category of vectors of sets}

In addition to our doubly indexed category of systems, we have a doubly indexed
category of ``vectors of sets''. 

Classically, an $m \times n$ matrix $M$ can act on a vector $v$ of length $n$ by multiplication to get
another vector $Mv$ of length $m$. We can generalize this to matrices of sets if
we define a vector of sets of length $A$ to be a dependent set $V : A \to
\smset$. 
\begin{definition}\label{def.linear_functor}
  For a set $A$, we define the category of \emph{vectors of sets of length} $A$
  to be $$\Cat{Vec}(A) \coloneqq \smset^A$$
the category of sets depending on $A$. 

Given a $(B \times A)$-matrix $M$, we can treat a $A$-vector $V$ as a $A \times
\ord{1}$ matrix and form the $B \times \ord{1}$ matrix $MV$. This gives us a
functor
\begin{align*}
\Cat{Vec}(M) : \Cat{Vec}(A) &\to \Cat{Vec}(B)\\
               V &\mapsto (MV)_b = \sum_{a \in A} M_{ba} \times V_a \\
           f : V \to W &\mapsto ( (a, m, v) \mapsto (a, m, f(v)) )
\end{align*}
which we refer to as the linear functor given by $M$.
\end{definition}

\begin{definition}
The doubly indexed category $\Cat{Vec} : \Cat{Matrix} \to \Cat{Cat}$ of vectors
of sets is defined by:
\begin{itemize}
  \item Its indexing base is the double category of matrices of sets.
  \item To every set $A$, we assign the category $\Cat{Vec}(A) = \smset^A$ of
    vectors of length $A$.
  \item To every $(B \times A)$-matrix $M : A \to B$, we assign the linear
    functor $\Cat{Vec}(M) : \Cat{Vec}(A) \to \Cat{Vec}(B)$ given by $M$ (\cref{def.linear_functor}).
  \item To every function $f : A \to B$, we associate the profunctor
    $\Cat{Vec}(f) : \Cat{Vec}(A) \tickar \Cat{Vec}(B)$ defined by
$$\Cat{Vec}(f)(V, W) = \{ F : (a \in A) \to V_a \to W_{f(a)} \}.$$
That is, $F \in \Cat{Vec}(f)(V, W)$ is a family of functions $F(a,-) : V_a \to
W_{f(a)}$ indexed by $a \in A$. This is natural by index-wise composition.
\item To every square
  \[
        \begin{tikzcd}[sep=tiny]
          A \ar[dd, "M"'] \ar[rr, "f"] & & B \ar[dd, "N"] \\
           & \alpha & \\
          C \ar[rr, "g"'] & & D
        \end{tikzcd}
  \]
  that is, family of functions $\alpha_{ca} : M_{ca} \to N_{g(c)f(a)}$, we
  associate the square
  \[
        \begin{tikzcd}[sep=tiny]
          \Cat{Vec}( A ) \ar[dd, "\Cat{Vec}(M)"'] \ar[rr, "\Cat{Vec}(f)"] & & \Cat{Vec}( B ) \ar[dd, "\Cat{Vec}(N)"] \\
           & \Cat{Vec}(\alpha) & \\
          \Cat{Vec}(C) \ar[rr, "\Cat{Vec}(g)"'] & & \Cat{Vec}(D)
        \end{tikzcd}
  \]
  defined by sending a family of functions $F : (a \in A) \to V_{a} \to
  W_{f(a)}$ in $\Cat{Vec}(f)(V, W)$ to the family 
  \begin{align*}
\Cat{Vec}(\alpha)(F) : (c \in C) \to MV_c &\to MW_{g(c)} \\ 
  \Cat{Vec}(\alpha)(F)(c, (a, m, v)) &= (f(a), \alpha(m), F(a, v))
  \end{align*}
  That is, $\Cat{Vec}(\alpha)(F)(c, -)$ takes an element $(a, m, v) \in MV_{c} =
  \sum_{a \in A} M_{ca} \times V_a$ and gives the elements $(f(a), \alpha(m),
  F(a, v))$ of $MW_{g(c)} = \sum_{b \in B} N_{g(c)b} \times W_b$.
\item The compositor is given by componentwise composition: If $f : A \to B$ and
  $g : B \to C$ and $F \in \Cat{Vec}(f)(V, W)$ and $G \in \Cat{Vec}(g)(W, U)$,
  then 
\begin{align*}
  \mu_{f, g}(F, G) : (a \in A) \to V_{a} &\to U_{gf(a)} \\
  \mu_{f, g}(F, G)(a, v) &\coloneqq G(f(a), F(a, v)).
\end{align*}
\end{itemize}
\end{definition}

It might seem like it will turn out to be a big hassle to show that this
definition satisfies all the laws of a doubly indexed category. Like with the
doubly indexed category of arenas, we will find that all the laws follow for
matrices by fiddling around in the double category of matrices.

Let's first rephrase the above definition in terms of the category of matrices.
We note that a vector of sets $V \in \Cat{Vec}(A)$ is equivalently a matrix $V :
\ord{1} \to A$. Then the linear functor $\Cat{Vec}(M) : \Cat{Vec}(A) \to
\Cat{Vec}(B)$ is given by matrix multiplication, or in double category notation:
$$\Cat{Vec}(M)(V) = \frac{V}{M}.$$
This means that the Vertical Functoriality law follows by vertical associativity
in the double category of matrices, which is to say associativity of matrix
multiplication.

Similarly, we can interpret the profunctor $\Cat{Vec}(f)$ for $f : A \to B$ in
terms of the double category $\Cat{Matrix}$. An element $F \in \Cat{Vec}(f)(V, W)$ is
equivalently a square of the following form in $\Cat{Matrix}$:
\[
        \begin{tikzcd}[sep=tiny]
          \ord{1} \ar[dd, "V"'] \ar[rr, equals] & & \ord{1} \ar[dd, "W"] \\
           & F & \\
          A \ar[rr, "f"'] & & B
        \end{tikzcd}
      \]
      Therefore, we can describe $\Cat{Vec}(f)(V, W)$ as the following set:
\[
\Cat{Vec}(f)(V, W) = \left\{ F \,\middle|
        \begin{tikzcd}[sep=tiny]
          \ord{1} \ar[dd, "V"'] \ar[rr, equals] & & \ord{1} \ar[dd, "W"] \\
           & F & \\
          A \ar[rr, "f"'] & & B
        \end{tikzcd}
  \right\}
\]
Then the Horizontal Lax Functoriality laws follow from associativity and unitality of
horizontal composition of squares in $\Cat{Matrix}$! 


Finally, we need to interpret the rather fiddly transformation
$\Cat{Vec}(\alpha)$ in terms of the double category of matrices. Its a matter of
unfolding the definitions to see that
$\Cat{Vec}(\alpha)(F) = \frac{F}{\alpha}$
in $\Cat{Matrix}$, and therefore that the Naturality of Compositors law follows
by the interchange law.

  If this argument seemed wholly too similar to the one we gave for the doubly
  indexed category of systems, your suspicions are not misplaced. These are both are instances of a very general \emph{vertical
    slice construction}, which we turn our attension to now.

  \section{Vertical Slice Construction}\label{sec.vertical_slice}

  In the previous section, we constructed the doubly indexed categories
  $\Cat{Sys}_{\mathbb{D}}$ of systems in a doctrine $\mathbb{D}$ and $\Cat{Vec}$
  of vectors of sets ``by hand''. However, both constructions felt very
  familiar. In this section, we will show that they are both instances of a
  general construction: the \emph{vertical slice construction}.

The main reason for recasting the above constructions in more general terms is
that it will facilitate our main theorem of this chapter: change of doctrine.

  The vertical slice construction will take a \emph{double functor} $F :
  \cat{D}_0 \to \cat{D}_1$ and produce a doubly indexed category $\sigma F :
  \cat{D}_1 \to \Cat{Cat}$ indexed by its codomain. So, in order to describe the
  vertical slice construction, we will need the notion of double functor. We will need the notion of double functor for much of the coming theory as well
  


\subsection{Double Functors}
  
A double functor is the correct sort of functor between double categories. Just
as a double category has a bit more than twice the information involved in a
category, a double functor has a bit more than twice the information involved in
a functor.
 \begin{definition}\label{def.double_functor}
 Let $\cat{D}_0$ and $\cat{D}_1$ be double categories. A \emph{double functor}
 $\Fun{F} : \cat{D}_0 \to \cat{D}_1$ consists of:
\begin{itemize}
  \item An object assignment $F : \Ob \cat{D}_0 \to \Ob \cat{D}_1$ which assigns an object $F D$
    in $\cat{D}_1$ to each object $D$ in $\cat{D}_0$.
  \item A vertical functor $F : v\cat{D}_0 \to v\cat{D}_1$ on the vertical
    categories, which acts the same as the object assignment on objects.
  \item A horizontal functor $F : h\cat{D}_0 \to h\cat{D}_1$ on the horizontal
    categories, which acts the same as the object assignment on objects.
  \item For every square
    \[
        \begin{tikzcd}[sep=tiny]
          A \ar[dd, "j"'] \ar[rr, "f"] & & B \ar[dd, "k"] \\
           & \alpha & \\
          C \ar[rr, "g"'] & & D
        \end{tikzcd}
    \]
    in $\cat{D}_0$, a square
    \[
        \begin{tikzcd}[sep=tiny]
          FA \ar[dd, "Fj"'] \ar[rr, "Ff"] & & FB \ar[dd, "Fk"] \\
           & F\alpha & \\
          FC \ar[rr, "Fg"'] & & FD
        \end{tikzcd}
    \]
    such that the following laws hold:
    \begin{itemize}
    \item $F$ commutes with horizontal compostition: $F(\alpha \mid \beta) = F\alpha \mid F\beta$.
    \item $F$ commutes with vertical comopsition: $F\left( \frac{\alpha}{\beta} \right) = \frac{F\alpha}{F\beta}$.
    \item $F$ sends horizontal identities to horizontal identities, and vertical
      identities to vertical identities.
    \end{itemize}
\end{itemize}
\end{definition}

\begin{remark}
There is, in fact, a double category of double functors $F : \cat{D}_0 \to
\cat{D}_1$, but we won't need to worry about this until we consider the
functoriality of the vertical slice construction in \cref{sec.functoriality_vertical_slice}.
\end{remark}

We will, in time, see many interesting examples of double functors. However, we
will begin with the two simple examples we need to
construct the doubly indexed categories $\Cat{Sys}$ and $\Cat{Vec}$.

\begin{example}\label{ex.double_functor_section}
Let $\mathbb{D} = (\cat{A} : \cat{C}\op \to \Cat{Cat}, T)$ be a doctrine. We
recall that the section $T : \cat{C} \to \int^{C : \cat{C}}\cat{A}(C)$ is a functor to the
Grothendieck construction of $\cat{A}$. We may promote this into a double
functor into the double category of arenas $\Cat{Arena}_{\mathbb{D}}$ in a
rather simple way.

Since the horizontal category of $\Cat{Arena}_{\mathbb{D}}$ is $\int^{C : \cat{C}}
\cat{A}(C)$, the category of charts, we may consider $T$ as a \emph{double}
functor
$$hT : h\cat{C} \to \Cat{Arena}_{\mathbb{D}}$$
from the double category $h\cat{C}$ given by defining its horizontal category to
be $\cat{C}$ and taking its vertical category and its squares to consist only of
identities. Its worth taking a minute to check this trivial observation against
the definition of a double functor.
\end{example}

\begin{example}\label{ex.double_functor_one}
There is a double category $\ord{1}$ with just one object $\ast$ and only identity maps
and squares. A double functor $F : \ord{1} \to \cat{D}$ simply picks out the
object $F(\ast)$; there is no other data involved, since everything else must
get sent to the appropriate identities.

In particular, the one element set $\ord{1}$ is an object of the double category
$\Cat{Matrix}$ of sets, functions, and matrices. Therefore, there is a double
functor $\ord{1} : \ord{1} \to \Cat{Matrix}$ picking out this special element.
\end{example}

Now that we have a notion of double functor, we can define a category
$\Cat{Dbl}$ of double categories.
\begin{definition}\label{def.category_of_double_cats}
The category $\Cat{Dbl}$ of double categories has as its objects the double
categories and as its maps the double functors.
\end{definition}

From any indexed category $\cat{A}$, we can form the double categories of
\emph{arenas} in $\cat{A}$ (\cref{def.double_cat_of_arenas_general}). In
category theory, it is a good habit to inquire into the functoriality of any
construction. Now that we have an appropriate category of double categories, we
can ask if the construction $\cat{A} \mapsto \Cat{Arena}_{\cat{A}}$ is functorial.

\begin{proposition}\label{prop.functoriality_arena_construction}
The assignment $\cat{A} \mapsto \Cat{Arena}_{\cat{A}}$ sending an indexed
category to its Grothendieck double construction (\cref{def.groth_double_construction}) is functorial.
\end{proposition}
\begin{proof}
  Let $\cat{A} : \cat{C}\op \to \Cat{Cat}$ and $\cat{B} : \cat{D}\op \to
  \Cat{Cat}$ be indexed categories, and let $(F, \overline{F}) : \cat{A} \to
  \cat{B}$ be an indexed functor. We will produce a double functor
  \[
\lens{\overline{F}}{F} : \Cat{Arena}_{\cat{A}} \to \Cat{Arena}_{\cat{B}}.
  \]
  
  Recall that the Grothendieck construction is functorial
  (\cref{prop.functoriality_monoidal_groth}). From an indexed functor $(F,
  \overline{F}) : \cat{A} \to \cat{B}$, we get a functor
  \[
\lens{\overline{F}}{F} : \int^{C : \cat{C}} \cat{A}(C) \to \int^{D : \cat{D}} \cat{B}(D).
  \]
  Since the horizontal category of $\Cat{Arena}$ is precisely the Grothendieck
  construction, we can take this to be the horizontal component of
  $\lens{\overline{F}}{F} :\Cat{Arena}_{\cat{A}} \to \Cat{Arena}_{\cat{B}} $.
  Similarly, since the vertical category of $\Cat{Arena}$ is the Grothendieck
  construction of the opposite, we can take the vertical component of
  $\lens{\overline{F}}{F} :\Cat{Arena}_{\cat{A}} \to \Cat{Arena}_{\cat{B}} $ to
  be $\lens{\overline{F}\op}{F} : \int^{C : \cat{C}}\cat{A}(C)\op \to \int^{D :
    \cat{D}} \cat{B}(D)\op$. 

All that remains to check then is that
  $\lens{\overline{F}}{F} ::\Cat{Arena}_{\cat{A}} \to \Cat{Arena}_{\cat{B}}$
  preserves squares. Let 
\[
    \begin{tikzcd}
      \lens{A_1}{C_1} \arrow[r, shift left, "\lens{g_{1\flat}}{g_1}"]\arrow[r, shift right] \arrow[d, leftarrow,  shift left] \arrow[d, shift right, "\lens{f_1^{\sharp}}{f_1}"'] & \lens{A_2}{C_2} \arrow[d, leftarrow, shift left, "\lens{f_2^{\sharp}}{f_2}"] \arrow[d, shift right] \\
      \lens{A_3}{C_3} \arrow[r, shift left]\arrow[r, shift right,
      "\lens{g_{2\flat}}{g_2}"'] & \lens{A_4}{C_4}
    \end{tikzcd}
\]
be a square in $\Cat{Arena}_{\cat{A}}$. We need to show that 
\[
    \begin{tikzcd}
      \lens{\overline{F}A_1}{FC_1} \arrow[r, shift left, "\lens{\overline{F}g_{1\flat}}{Fg_1}"]\arrow[r, shift right] \arrow[d, leftarrow,  shift left] \arrow[d, shift right, "\lens{\overline{F}f_1^{\sharp}}{Ff_1}"'] & \lens{\overline{F}A_2}{FC_2} \arrow[d, leftarrow, shift left, "\lens{\overline{F}f_2^{\sharp}}{Ff_2}"] \arrow[d, shift right] \\
      \lens{\overline{F}A_3}{FC_3} \arrow[r, shift left]\arrow[r, shift right,
      "\lens{\overline{F}g_{2\flat}}{Fg_2}"'] & \lens{\overline{F}A_4}{FC_4}
    \end{tikzcd}
\]
is a square in $\Cat{Arena}_{\cat{B}}$. But this being a square means that the
two diagrams
\[
    \begin{tikzcd}
      FC_1 \arrow[r, "Fg_1"] \arrow[d, "Ff_1"'] & FC_2 \arrow[d, "Ff_2"] &  & (F f_1 )^{\ast}\overline{F}A_3 \arrow[rr, "\overline{F}f_1^{\sharp}"] \arrow[d, "(Ff_1)^{\ast}\overline{F}g_{2\flat}"'] &                                                              & \overline{F}A_1 \arrow[d, "\overline{F}g_{1\flat}"] \\
      FC_3 \arrow[r, "Fg_2"'] & FC_4 & & (Ff_1)^{\ast}(Fg_2)^{\ast}\overline{F}A_4 \arrow[r, equals]
      & (Fg_1)^{\ast}(Ff_2)^{\ast}\overline{F}A_4 \arrow[r, "(Fg_1)^{\ast}\overline{F}f_2^{\sharp}"'] &
      (Fg_1)^{\ast}\overline{F}A_2
    \end{tikzcd}
\]
The left square commutes because $F$ is a functor, and the right square commutes
because $(F, \overline{F})$ is an indexed functor.
\end{proof}

\subsection{The Vertical Slice Construction: Definition}

We are now ready to define the vertical slice construction.
  \begin{definition}[The Vertical Slice Construction]\label{defn.vertical_slice}
Let $F : \cat{D}_0 \to \cat{D}_1$ be a double functor. The \emph{vertical slice
  construction} of $F$ is the doubly indexed category 
$$\sigma F : \cat{D}_1 \to \Cat{Cat}$$
defined as follows:
\begin{itemize}
  \item For $D \in \cat{D}_1$, $\sigma F(D)$ is the category whose objects are
    pairs $(A, j)$ of an object $A \in \cat{D}_0$ and a vertical map $f :
    FA \to D$. A map $(A_1, j_1) \to (A_2, j_2)$ is a pair $(f, \alpha)$ of a
    horizontal $f : A_1 \to A_2$ and a square 
\[
        \begin{tikzcd}[sep=tiny]
          FA_1 \ar[dd, "j_1"'] \ar[rr, "Ff"] & & FA_2 \ar[dd, "j_2"] \\
           & \alpha & \\
          D \ar[rr, equals] & & D
        \end{tikzcd}
\]
in $\cat{D}_1$.
\item For every vertical $j : D \to D'$ in $\cat{D}_1$, we associate the functor
  $\sigma F(j) : \sigma F(D) \to \sigma F(D')$ given by vertical composition
  with $j$:
  \[
        \begin{tikzcd}[sep=tiny]
          FA_1 \ar[dd, "j_1"'] \ar[rr, "Ff"] & & FA_2 \ar[dd, "j_2"] \\
           & \alpha & \\
          D \ar[rr, equals] & & D
        \end{tikzcd}
        \quad\mapsto\quad
        \begin{tikzcd}[sep=tiny]
          FA_1 \ar[dd, "j_1"'] \ar[rr, "Ff"] & & FA_2 \ar[dd, "j_2"] \\
           & \alpha & \\
          D \ar[rr, equals] \ar[dd, "j"'] & & D\ar[dd, "j"]\\
           & \phantom{\alpha} & \\
          D' \ar[rr, equals] & & D'
        \end{tikzcd}
\]
More concisely, this is
$$\sigma F(j)(f, \alpha) = \left(f, \frac{\alpha}{j} \right).$$
\item For every horizontal $g : D \to D'$ in $\cat{D}_1$, we associate the
  profunctor $\sigma F (g) : \sigma F(D) \tickar \sigma F(D')$ given by 
\[
        \begin{tikzcd}[sep=tiny]
          FA_1 \ar[dd, "j_1"'] \\
          \phantom{\alpha} \\
          D 
        \end{tikzcd}\, ,\,
        \begin{tikzcd}[sep=tiny]
          FA_2 \ar[dd, "j_2"'] \\
          \phantom{\alpha} \\
          D'
        \end{tikzcd}
\quad \mapsto \quad
\left\{  \left( f,  
        \begin{tikzcd}[sep=tiny] 
          FA_1 \ar[dd, "j_1"'] \ar[rr, dashed, "Ff"] & & FA_2 \ar[dd, "j_2"] \\
           & \alpha & \\
          D \ar[rr, "g"'] & & D'
        \end{tikzcd}\right).
 \right\}
\]
We note that if $g = \id_D$ is an identity, then this reproduces the hom
profunctor of $\sigma F(D)$.
\item The compositor $\mu$ is given by horizontal composition:
\[
\mu_{g_1, g_2}((f_1, \alpha_1), (f_2, \alpha_2)) = (f_1 \mid f_2, \alpha_1 \mid
\alpha_2).
\]
\end{itemize}
\end{definition}


Let's check now that this does indeed satisfy the laws of a doubly indexed
category. The proof is exactly as it was for $\Cat{Sys}$. 
\begin{itemize}
  \item (Vertical Functoriality) We show that $\sigma F\left(
      \frac{k_1}{k_2} \right) =
    \sigma F(k_2) \circ \sigma F (k_1)$ by
    vertical associativity:
\begin{align*}
  \sigma F\left(\frac{k_1}{k_2}\right)(f, \alpha) &= \left(f,  \frac{\alpha}{\left( \frac{k_1}{k_2} \right)} \right) \\
= \left(f, \frac{\left( \frac{\alpha}{k_1} \right)}{k_2}   \right)\\
&= \sigma F(k_2) \circ \sigma F (k_1)((f, \alpha)).
\end{align*}

\item (Horizontal Lax Functoriality) This law follows from horizontal
  associativity in $\cat{D}_1$.
\begin{align*}
  \mu(\mu((f_1, \alpha_1), (f_2, \alpha_2)), (f_3, \alpha_3)) &= ((f_1 \mid f_2) \mid f_3, (\alpha_1 \mid \alpha_2) \mid \alpha_3) \\
&= (f_1 \mid (f_2 \mid f_3), \alpha_1 \mid (\alpha_2 \mid \alpha_3)) \\
&= \mu((f_1, \alpha_1), \mu((f_2, \alpha_2), (f_3, \alpha_3))).
\end{align*}
\item (Naturality of Compositor) This law follows from interchange in
  $\cat{D}_1$.
\begin{align*}
  \left( \sigma F(\beta_1) \mid \sigma F(\beta_2){\mu} \right)((f_1, \alpha_1), (f_2, \alpha_2)) &= \left(f_1 \mid f_2,   \left. \frac{\phi}{\alpha} \middle| \frac{\psi}{\beta} \right.\right) \\
&= \left(f_1 \mid f_2,  \frac{\phi \mid \psi}{\alpha \mid \beta}\right) \\
&= \left(  \frac{\mu}{\sigma F(\beta_1 \mid \beta_2)}\right)((f_1,\alpha_1),(f_2,\alpha_2)).
\end{align*}
\end{itemize}

We can now see that the vertical slice construction generalizes both the
constructions of $\Cat{Sys}_{\mathbb{D}}$ and $\Cat{Vec}$. 
\begin{proposition}\label{prop.sys_is_vertical_slice}
  The doubly indexed category $\Cat{Sys}_{\mathbb{D}}$ of systems in a doctrine
  $\mathbb{D} = (\cat{A} : \cat{C}\op \to \Cat{Cat}, T)$ is the vertical slice
  construction of the double functor $hT : h\cat{C} \to
  \Cat{Arena}_{\mathbb{D}}$ given by considering the section $T$ as a double
  functor.
$$\Cat{Sys}_{\mathbb{D}} = \sigma (hT : h\cat{C} \to \Cat{Arena}_{\mathbb{D}}).$$
\end{proposition}
\begin{proof}
This is a matter of checking definitions and seeing that they are precisely the same.
\end{proof}

\begin{proposition}
The doubly indexed category $\Cat{Vec}$ of vectors of sets is the vertical slice
construction of the inclusion $\ord{1} : \ord{1} \to \Cat{Matrix}$ of the one
element set into the double category of matrices of sets.
$$\Cat{Vec} = \sigma (\ord{1} : \ord{1} \to \Cat{Matrix}).$$
\end{proposition}
\begin{proof}
This is also a matter of checking that the definitions coincide.
\end{proof}

\subsection{Natural Transformations of Double Functors}
We now turn towards proving the functoriality of the vertical slice construction
as a first step in proving the change of doctrine theorem. In order to express
the functoriality of the vertical slice construction, we will first need learn
about natural transformations between double functors.

Since double categories have two sorts of maps --- vertical and horizontal ---
there are also two sorts of natural transformations between double functors. The
two definitions are symmetric; we may arrive at one by replacing the words
``vertical'' by ``horizontal'' and vice-versa. We will have occasion to use both
of them in this and the coming chapters.

\begin{definition}\label{def.double_natural_transformation}
Let $F$ and $G : \cat{D} \to \cat{E}$ be double functors. A \emph{vertical
  natural transformation} $v : F \Rightarrow G$ consists of the following data:
\begin{itemize}
\item For every object $D \in \cat{D}$, a vertical $v_D : FD \to GD$ in $\cat{E}$.
\item For every horizontal arrow $f : D \to D'$ in $\cat{D}$, a square
\[
        \begin{tikzcd}[sep=tiny]
          FD \ar[dd, "v_D"'] \ar[rr, "Ff"] & & FD'
 \ar[dd, "v_{D'}"] \\
           & v_f & \\
          GD \ar[rr, "Gf"'] & & GD'
        \end{tikzcd}
\]
\end{itemize}
This data must satisfy the following laws:
\begin{itemize}
\item (Vertical Naturality) For any vertical $j : D_1 \to D_2$, we have 
$$\frac{Fj}{v_{D_2}} = \frac{v_{D_1}}{Gj}.$$
\item (Horizontal Naturality) For any horizontal $f_1 : D_1 \to D_2$ and $f_2 :
  D_2 \to D_3$, we have 
$$v_{f_1 \mid f_2} = v_{f_1} \mid v_{f_2}.$$
\item (Horizontal Unity) $v_{\id_D} = \id_{v_D}$. 
\item (Square naturality) For any square
\[
        \begin{tikzcd}[sep=tiny]
          D_1 \ar[dd, "j_1"'] \ar[rr, "f_1"] & & D_2
 \ar[dd, "j_2"] \\
           & \alpha & \\
          D_3 \ar[rr, "f_2"'] & & D_4
        \end{tikzcd}
\]
we have
$$\frac{F\alpha}{v_{f_2}} = \frac{v_{f_1}}{G\alpha}.$$
\end{itemize}

Dually, a \emph{horizontal transformation} $h : F \Rightarrow G$ consists of the
following data:
\begin{itemize}
\item For every object $D \in \cat{D}$ a horizontal morphism $h_D : FD \to GD$.
\item For every vertical $j : D \to D'$ in $\cat{D}$, a square 
\[
        \begin{tikzcd}[sep=tiny]
          FD \ar[dd, "Fj"'] \ar[rr, "h_D"] & & GD
 \ar[dd, "Gj"] \\
           & h_j & \\
          FD' \ar[rr, "h_{D'}"'] & & GD'
        \end{tikzcd}
\]
\end{itemize}
This data is required to satisfy the following laws:
\begin{itemize}
\item (Horizontal Naturality) For horizontal $f : D_1 \to D_2$, we have
$$Ff \mid v_{D_2} = v_{D_1} \mid Gf.$$
\item (Vertical Naturality) For vertical $j_1 : D_1 \to D_2$ and $j_2 : D_2 \to
  D_3$, we have
$$h_{\frac{j_1}{j_2}} = \frac{h_{j_1}}{h_{j_2}}.$$
\item (Vertical Unity) $h_{\id_D} = \id_{h_D}$.
\item (Square Naturality) For any square
\[
        \begin{tikzcd}[sep=tiny]
          D_1 \ar[dd, "j_1"'] \ar[rr, "f_1"] & & D_2
 \ar[dd, "j_2"] \\
           & \alpha & \\
          D_3 \ar[rr, "f_2"'] & & D_4
        \end{tikzcd}
\]
we have 
$$F\alpha \mid h_{j_2} = h_{j_1} \mid G\alpha.$$
\end{itemize}
\end{definition}

\begin{remark}
Note that vertical (resp. horizontal) natural transformations are named for
the direction of arrow they assign to objects. However, a vertical
transformation is defined by its action $v_f$ on \emph{horizontal} maps $f$, and dually a
horizontal transformation $h_j$ by its action on \emph{vertical} maps $j$.
Taking $f$ (resp. $j$) to be an identity $\id_D$ yields the vertical (resp.
horizontal) arrow associated to the object $D$.
\end{remark}

Natural transformations between double functors can be composed in the
appropriate directions.
\begin{lemma}
Suppose that $v_1 : F_1 \Rightarrow F_2$ and $v_2 : F_2 \Rightarrow F_2$ are
vertical transformations. We have a vertical composite $\frac{v_1}{v_2}$ defined
by
$$\left( \frac{v_1}{v_2} \right)_f \coloneqq \frac{(v_1)_f}{(v_2)_f}$$
for horizontal maps $f$. Dually, for horizontal transformations $h_1 : F_1
\Rightarrow F_2$ and $h_2 : F_2 \Rightarrow F_3$, there is a horizontal
composite $h_1 \mid h_2$ defined by
$$(h_1 \mid h_2)_j := (h_1)_j \mid (h_2)_j$$
for every vertical map $j$.
\end{lemma}
\begin{proof}
  We will prove that $\frac{v_1}{v_2}$ is a vertical transformation; the proof
  that $h_1 \mid h_2$ is a horizontal transformation is precisely dual. 
  \begin{itemize}
  \item (Vertical Naturality) This follows by the same argument as for Square
    Naturality below, taking $\alpha = j$ for a vertical$j : D_1 \to D_2$.

\item (Horizontal naturality) For horizontal maps $f_1 :D_1 \to D_2$ and $f_2 :
  D_2 \to D_3$, we have
  \begin{align*}
    \frac{v_1}{v_2}_{f_1 \mid f_2} &= \frac{(v_1)_{f_1 \mid f_2}}{(v_2)_{f_1 \mid f_2}} \\
                                   &= \frac{(v_1)_{f_1} \mid (v_1)_{f_2}}{(v_2)_{f_1} \mid (v_2)_{f_2}} \\
                                   &= \left. \frac{(v_1)_{f_1}}{(v_2)_{f_1}} \middle| \frac{(v_1)_{f_2}}{(v_2)_{f_2}} \right.\\
    &= \left. \left( \frac{v_1}{v_2} \right)_{f_1} \middle| \left( \frac{v_1}{v_2} \right)_{f_2} \right. .
  \end{align*}

    
\item (Horizontal Unity) This holds by definition.

\item (Square Naturality) Consider a square $\alpha$ of the following signature:
\[
        \begin{tikzcd}[sep=tiny]
          D_1 \ar[dd, "j_1"'] \ar[rr, "f_1"] & & D_2
 \ar[dd, "j_2"] \\
           & \alpha & \\
          D_3 \ar[rr, "f_2"'] & & D_4
        \end{tikzcd}
\]
  Then
    \begin{align*}
      \frac{F_1 \alpha}{\left( \frac{v_1}{v_2} \right)_{f_2}} &= \begin{tabular}{c}
        $F_1 \alpha $ \\ \hline
        $(v_1)_{f_2}$ \\ \hline
        $(v_2)_{f_2}$ 
     \end{tabular} \\
&= \begin{tabular}{c}
        $(v_1)_{f_1}$ \\ \hline
        $F_2 \alpha$ \\ \hline
        $(v_2)_{f_2}$ 
     \end{tabular} \\
&= \begin{tabular}{c}
        $(v_1)_{f_1}$ \\ \hline
        $(v_2)_{f_1}$ \\ \hline
        $F_3 \alpha$ 
     \end{tabular} \\
&= \frac{\left( \frac{v_1}{v_2} \right)_{f_1}}{F_3 \alpha}.
    \end{align*}
  \end{itemize}
  
\end{proof}

Amongst double functors we have found two sorts of maps --- vertical and
horizontal --- each with their own sort of composition. This suggests that there
should be a \emph{double category} of double functors $\cat{D} \to \cat{E}$,
just as there is a category of functors between two categories. 

\begin{theorem}
Let $\cat{D}$ and $\cat{E}$ be double categories. There is a double category
$\Cat{Fun}(\cat{D}, \cat{E})$ of double functors from $\cat{D}$ to $\cat{E}$
whose vertical maps are vertical transformations, horizontal maps are horizontal
transformations, and whose squares
\[
        \begin{tikzcd}[sep=tiny]
          F_1 \ar[dd, "v_1"'] \ar[rr, "h_1"] & & F_2
 \ar[dd, "v_2"] \\
           & \alpha & \\
          F_3 \ar[rr, "h_2"'] & & F_4
        \end{tikzcd}
      \]
      are \emph{modifications} defined in the following way. To each object $D
      \in \cat{D}$, we have a square
\[
        \begin{tikzcd}[sep=tiny]
          F_1D \ar[dd, "(h_1)_D"'] \ar[rr, "( v_1 )_D"] & & F_2D
 \ar[dd, "( v_2 )_D"] \\
           & \alpha_D & \\
          F_3 D \ar[rr, "( h_2 )_D"'] & & F_4 D
        \end{tikzcd}
\]
which satisfies the following laws:
\begin{itemize}
  \item (Horizontal Coherence) For every horizontal $f : D_1 \to D_2$, we have
    that
    \[
    (v_1)_f \mid \alpha_{D_2} = \alpha_{D_1} \mid (v_2)_f.
  \]
  We note that this law requires us to use the vertical naturality law of $v_1$
  and $v_2$ so that these composites have the same signature.
  \item (Vertical Coherence) For every vertical $j : D_1 \to D_2$, we have that
    \[
    \frac{\alpha_{D_1}}{(h_2)_j} = \frac{(h_1)_j}{\alpha_{D_2}}.
    \]
  We note that this law requires us to use the horizontal naturality law of $h_1$
  and $h_2$ so that these composites have the same signature.
\end{itemize}
 The compositions $\alpha \mid \beta$ and $\frac{\alpha}{\beta}$ are given
 componentwise by $\alpha_D \mid \beta_D$ and $\frac{\alpha_D}{\beta_D}$.      
\end{theorem}
\begin{proof}
  Since the compositions of modifications are given componentwise, they will
  satisfy associativity and interchange. We just need to show that they are well
  defined, which is to say that they satisfy the laws of a modification. This is
  a straightforward calculation; we'll prove Vertical Coherence for horizontal
  composition since the other cases are similar.

  Let $\alpha$ and $\beta$ be modifications with the following signatures:
  \[
        \begin{tikzcd}[sep=tiny]
          F_1 \ar[dd, "v_1"'] \ar[rr, "h_1"] & & F_2
 \ar[dd, "v_2"] \\
           & \alpha & \\
          F_3 \ar[rr, "h_2"'] & & F_4
        \end{tikzcd}
        \quad\mbox{and}\quad
        \begin{tikzcd}[sep=tiny]
          F_2 \ar[dd, "v_2"'] \ar[rr, "h_3"] & & F_5
 \ar[dd, "v_3"] \\
           & \beta & \\
          F_4 \ar[rr, "h_4"'] & & F_6
        \end{tikzcd}
\]
  Let $j : D_1 \to D_2$ be a vertical map in $\cat{D}$. We calculate:
  \begin{align*}
    \frac{(\alpha \mid \beta)_{D_1}}{(h_2 \mid h_4)_{j}} &= \frac{\alpha_{D_1} \mid \beta_{D_1}}{(h_2)_j \mid (h_4)_j} \\
                                                         &= \left. \frac{\alpha_{D_1}}{(h_2)_{j}} \middle| \frac{\beta_{D_1}}{(h_4)_{j}} \right. \\
                                                         &= \left. \frac{(h_1)_{j}}{\alpha_{D_2}} \middle| \frac{(h_3)_{j}}{\beta_{D_2}} \right. \\
                                                         &= \frac{(h_1 \mid h_3)_{j}}{(\alpha \mid \beta)_{D_2}}.
  \end{align*}
\end{proof}

Before we move on, let's record an important lemma relating modifications to squares.
\begin{lemma}
  Let
  \[
        \begin{tikzcd}[sep=tiny]
          F_1 \ar[dd, "v_1"'] \ar[rr, "h_1"] & & F_2
 \ar[dd, "v_2"] \\
           & \alpha & \\
          F_3 \ar[rr, "h_2"'] & & F_4
        \end{tikzcd}
  \]
  be a modification, and
  \[
        \begin{tikzcd}[sep=tiny]
          D_1 \ar[dd, "j_1"'] \ar[rr, "f_1"] & & D_2
 \ar[dd, "j_2"] \\
           & s & \\
          D_3 \ar[rr, "f_2"'] & & D_4
        \end{tikzcd}
  \]
  be a square in $\cat{D}$. We then have the following four-fold equality in $\cat{E}$:
  \[
    \begin{tikzcd}[ampersand replacement = \&]
  \begin{tabular}{c|c}
    $\alpha_{D_1}$ & $(v_2)_{f_1}$ \\ \hline
    $(h_2)_{j_1}$ & $F_4 s$
  \end{tabular}     \ar[r,
      equals] \ar[d, equals] \& \begin{tabular}{c|c}
    $(v_1)_{f_1}$ & $\alpha_{D_2}$ \\ \hline
    $F_3 s$ & $(h_2)_{j_2}$
  \end{tabular} \ar[d, equals]\\
   \begin{tabular}{c|c}
    $(h_1)_{j_1}$ & $F_2 s$ \\ \hline
  $\alpha_{D_3}$   & $(v_2)_{f_2}$
  \end{tabular}   \ar[r, equals] \&  \begin{tabular}{c|c}
    $F_1 s$ &  $(h_1)_{j_2}$ \\ \hline
    $(v_2)_{f_1}$ & $\alpha_{D_4}$
                                     \end{tabular}
\end{tikzcd}
  \]
 We may refer to the single square given by any of these composites by $\alpha_s$. 
\end{lemma}
\begin{proof}
These all follow by cycling through the square naturality laws of the
transformations and the coherence laws of the modification. 
\end{proof}

\subsection{Vertical Slice Construction: Functoriality}\label{sec.functoriality_vertical_slice}

In this section, we will describe the functoriality of the vertical slice
construction. Since the vertical slice construction takes a double functor $F :
\cat{D}_0 \to \cat{D}_1$ and produces a doubly indexed category $\sigma F :
\cat{D}_{1} \to \Cat{Cat}$, we will need to show that from a certain sort of map
between double functors we get a \emph{doubly indexed functor} between the
resulting vertical slices.

First, we will describe the appropriate notion of map between double functors.
This gives us a category which we will call the \emph{category of double
  functors} $\Cat{DblFun}$ \footnote{Though one could define other categories
  whose objects are double
functors, this is the only such category we will use in
this book.}
\begin{definition}
The category $\Cat{DblFun}$ has objects the double functor $F : \cat{D}_0 \to
\cat{D}_1$. A map $F_1 \to F_2$ is a triple $(v_0, v_1, v)$ where $v_0 :
\cat{D}_{00} \to \cat{D}_{10}$ and $v_1 : \cat{D}_{01} \to \cat{D}_{11}$ are
double functors and $v : F_2 \circ v_0 \Rightarrow v_1 \circ F_1$ is a vertical
transformation.
\[
        \begin{tikzcd}
          \cat{D}_{00} \ar[dd, "F_1"'] \ar[rr, "v_0"] & & \cat{D}_{10}
 \ar[dd, "F_2"] \ar[ddll, Rightarrow, "v"] \\
           &  & \\
          \cat{D}_{01} \ar[rr, "v_1"'] & & \cat{D}_{11}
        \end{tikzcd}
\]
Composition of $(v_0, v_1, v) : F_1 \to F_2$ with $(w_0, w_1, w) : F_2 \to F_3$
is given by $(w_0 \circ v_0, w_1 \circ v_1, v \ast w)$ where $v \ast w$ is the
vertical transformation with horizontal components given by
$$(v \ast w)_f := \frac{w_{v_0 f}}{w_1 v_f}.$$
\end{definition}

It remains to check that this does indeed yield a category. We leave this as an
exercise, since it gives some good practice in using all the various laws for
double functors and double transformations.

\begin{exercise}
Prove that the definition of $\Cat{DblFun}$ does indeed yield a category. That
is:
\begin{enumerate}
  \item Prove that $(\id_{\cat{D}_0}, \id_{\cat{D}_1}, \id_{F})$ provides an
    identity map $F \to F$.
  \item Prove that composition is associative. The key part will be showing that
    $(v \ast w) \ast u = v \ast (w \ast u)$.
\end{enumerate}
\end{exercise}


Next, we need to describe the appropriate category of doubly indexed categories.
There are two sorts of maps of doubly indexed categories which we will need in
this book: \emph{lax} doubly indexed functors, and (\emph{taut}) doubly indexed
functors. In this chapter, we will be using \emph{taut} doubly indexed functors
--- which we may just call doubly indexed functors --- which are a special case
of the more general lax variety.


\begin{definition}\label{def.lax_doubly_indexed_functor}
  Let $\cat{A} : \cat{D}_1 \to \Cat{Cat}$ and $\cat{B} : \cat{D}_2 \to
  \Cat{Cat}$ be doubly indexed categories. A \emph{lax doubly indexed functor}
  $(F^0, F) : \cat{A} \rightharpoonup \cat{B}$ consists of:
  \[
\begin{tikzcd}
\cat{D}_1 \arrow[dd, "{F^0}"'] \arrow[rrd, "{ \cat{A} }", bend left] & {} \arrow[dd, "{F}"', Rightarrow] &      \\
  &   & \Cat{Cat} \\
\cat{D}_2 \arrow[rru, bend right, "{\cat{B}}"']      & {}   &     
\end{tikzcd}
  \]
  
\begin{enumerate}
  \item\label{def:lax.doubly.indexed.functor.0} A double functor $$F^0 : \cat{D}_1 \to \cat{D}_2.$$
  \item \label{def:lax.doubly.indexed.functor.1}For each object $D \in \cat{D}_1$, a functor $$F^D : \cat{A}(D)
    \to \cat{B}(F^0D).$$
  \item\label{def:lax.doubly.indexed.functor.2} For every vertical map $j : D_1 \to D_2$ in $\cat{D}_1$, a natural
    transformation 
\[
\begin{tikzcd}
  \cat{A}(D_1) \ar[d, "{\cat{A}(j)}"'] \ar[r, "F^{D_1}"] & \cat{B}(F^0D_1)
\ar[dl, Rightarrow, "F^j"']  \ar[d, "\cat{B}(F^0j)"] \\
\cat{A}(D_2) \ar[r,"F^{D_2}"'] & \cat{B}(F^0D_2)
\end{tikzcd}
\]
We ask that $F^{\id_{D}} = \id$. We recall (from
    \cref{prop.transformation_as_square_in_cat}) that we may think of such a
    natural transformation as a square 
\[
\begin{tikzcd}[sep=tiny]
\cat{A}(D_1) \ar[d, "F^{D_1}"'] \ar[rr, tick, equals] & & \cat{A}(D_1) \ar[d,
"\cat{A}(j)"] \\
\cat{B}(F^0 D_1) \ar[d, "\cat{B}(F^0 j)"'] & F^j & \cat{A}(D_2) \ar[d,
"F^{D_2}"] \\
\cat{B}(F^0 D_2) \ar[rr, tick, equals]  &  & \cat{B}(F^0 D_2)
\end{tikzcd}
\]
   \item\label{def:lax.doubly.indexed.functor.3} For every horizontal map $f : D_1 \to D_2$, a square
\[
\begin{tikzcd}[sep=tiny]
\cat{A}(D_1) \ar[rr, tick,"\cat{A}(f)"] \ar[dd, "F^{D_1}"'] & & \cat{A}(D_2) \ar[dd, "F^{D_2}"]\\
& F^{j} & \\
\cat{B}(F^0D_1) \ar[rr, tick, "\cat{B}(F^0 f)"'] &  &\cat{B}(F^0 D_2)
\end{tikzcd}
\]
in $\Cat{Cat}$. We ask that $F^{\id_{D}} = \id$.
\end{enumerate}

This data is required to satisfy the following laws:
\begin{itemize}
  \item (Vertical Lax Functoriality) For composable vertical maps $j : D_1 \to D_2$ and $k :
    D_2 \to D_3$, 
\[
F^{\frac{j}{k}} = 
\begin{tikzcd}
  \cat{A}(D_1) \ar[d, "{\cat{A}(j)}"'] \ar[r, "F^{D_1}"] & \cat{B}(F^0D_1)
\ar[dl, Rightarrow, "F^j"']  \ar[d, "\cat{B}(F^0j)"]  \\
\cat{A}(D_2) \ar[d, "{\cat{A}(k)}"']\ar[r,"F^{D_2}"] & \cat{B}(F^0D_2)\ar[d, "\cat{B}(F^0k)"] \ar[dl, Rightarrow, "F^k"'] \\
\cat{A}(D_3) \ar[r,"F^{D_3}"']& \cat{B}(F^0D_3) 
\end{tikzcd}
\]
This is, in terms of squares in $\Cat{Cat}$:
    $$F^{\frac{j}{k}} \coheq \left. \frac{F^j}{\cat{B}(F^0k)} \middle| \frac{\cat{A}(j)}{F^k} \right.$$
  \item (Horizontal functoriality) For composable horizontal arrows $f : D_1 \to
    D_2$ and $g : D_2 \to D_3$, 
$$\frac{\mu^{\cat{A}}_{f, g}}{F^{f \mid g}} = \frac{F^f \mid
  F^g}{\mu^{\cat{B}}_{F^0f, F^0g}}.$$ 
\item (Functorial Interchange) For any square
\[
\begin{tikzcd}[sep=tiny]
D_1 \ar[rr, "f"] \ar[dd, "j"'] & & D_2 \ar[dd, "k"] \\
 & \alpha & \\
D_3 \ar[rr, "g"'] & & D_4
\end{tikzcd}
\]
in $\cat{D}_1$, we have that
\[
  \left. F^j \,\middle|\, \frac{\cat{A}(\alpha)}{F^g} \right. \coheq \left.
    \frac{F^f}{\cat{B}(F^0 \alpha)} \,\middle|\, F^k \right. .
\]

Note the use of ``$\coheq$'' here; the two sides of this equation have
different, but canonically isomorphic boundary. What we are asking is that when these boundaries are made
the same by composing with canonical isomorphisms in any way, they will become
equal

\end{itemize}

A lax doubly indexed functor is \emph{taut} --- which we will in refer to just
as a doubly indexed functor --- if the natural transformations $F^j$ associated
to vertical maps $j : D_1 \to D_2$ in $\cat{D}_1$ are natural isomorphisms.

\end{definition}

The definition of doubly indexed functor involves a lot of data, but this is
because it is a big collection of functoriality results. 

We need to compose lax doubly indexed functors.
\begin{definition}
If $(F^0, F) : \cat{A} \to \cat{B}$ and $(G^0, G) : \cat{B} \to \cat{C}$ are two
doubly indexed functors, we define their composite 
\[
(F^0, F) \then (G^0, G) \coloneqq (F^0 \then G^0, F \then G)
\]
where $F \then G$ is defined by:
\begin{itemize}
  \item We define $(F \then G)^D \coloneqq F^D \then G^{F_0 D}$.
    We note that in $\Cat{Cat}$, where functors are the vertical maps, this can
    be written
\[
(F \then G)^D = \frac{F^D}{G^{F^0 D}}.
\]
  \item For a vertical $j : D_1 \to D_2$, we define 
\[
\begin{tikzcd}
  \cat{A}(D_1) \ar[d, "{\cat{A}(j)}"'] \ar[r, "{ (F \then G) }^{D_1}"] & \cat{C}(G^0F^0D_1)
\ar[dl, Rightarrow, "(F \then G)^j"']  \ar[d, "\cat{C}(G^0F^0j)"] \\
\cat{A}(D_2) \ar[r,"(F \then G)^{D_2}"'] & \cat{C}(G^0F^0D_2)
\end{tikzcd} \coloneqq 
\begin{tikzcd}
  \cat{A}(D_1) \ar[d, "{\cat{A}(j)}"'] \ar[r, "F^{D_1}"] & \cat{B}(F^0D_1)
\ar[dl, Rightarrow, "F^j"']  \ar[d, "\cat{B}(F^0j)"] \ar[r, "G^{F^0D_1}"] & \cat{C}(G^0F^0D_1) \ar[d,
"\cat{C}(G^0F^0 j)"] \ar[dl, Rightarrow, "G^{F^0 j}"']\\
\cat{A}(D_2) \ar[r,"F^{D_2}"'] & \cat{B}(F^0D_2) \ar[r, "G^{F^0 D_2}"'] & \cat{C}(G^0F^0D_2)
\end{tikzcd}
\]
We note that by \cref{lemma.transformation_as_square_vertical}, this corresponds
to the composite of squares: 
\[
(F \then G)^j \coheq \left. \frac{F^{D_1}}{G^{F^0 j}} \middle|
  \frac{F^j}{G^{F^0 D_2}} \right.
\]
\item For a horizontal $f : D_1 \to D_2$, we define
\[
(F \then G)^f \coloneqq \frac{F^f}{G^{F^0 f}}.
\]
\end{itemize}

We refer to the category of doubly indexed categories and lax doubly indexed
functors by $\Cat{LaxDblIx}$ and the category of doubly indexed categories and
(taut) doubly indexed functors by $\Cat{DblIx}$.
\end{definition}

Let's show that this composition operation does indeed produce a lax doubly
indexed functor.

\begin{itemize}
  \item (Vertical Lax Functoriality) For composable vertical maps $j : D_1 \to
    D_2$ and $k : D_2 \to D_3$, consider the following diagram:
\[
\begin{tikzcd}
  \cat{A}(D_1) \ar[d, "{\cat{A}(j)}"'] \ar[r, "F^{D_1}"] & \cat{B}(F^0D_1)
\ar[dl, Rightarrow, "F^j"']  \ar[d, "\cat{B}(F^0j)"] \ar[r, "G^{F^0D_1}"] & \cat{C}(G^0F^0D_1) \ar[d,
"\cat{C}(G^0F^0 j)"] \ar[dl, Rightarrow, "G^{F^0 j}"']\\
\cat{A}(D_2) \ar[d, "{\cat{A}(k)}"']\ar[r,"F^{D_2}"] & \cat{B}(F^0D_2)\ar[d, "\cat{B}(F^0k)"] \ar[dl, Rightarrow, "F^k"'] \ar[r, "G^{F^0 D_2}"] &
\cat{C}(G^0F^0D_2) \ar[dl, Rightarrow, "G^{F^0 k}"']\ar[d, "\cat{C}(G^0F^0 k)"] \\
\cat{A}(D_3) \ar[r,"F^{D_3}"']& \cat{B}(F^0D_3) \ar[r, "G^{F^0 D_2}"'] & \cat{B}(G^0 F^0 D_3)
\end{tikzcd}
\]
There is a single natural transformation given as the composite of this
``pasting diagram''. But, if we read it by composing vertically first, and then
composing horizontally second, we arive at $(F \then G)^{\frac{j}{k}}$, while if
we read it by composing horizontally first and then vertically second, we get
the composite of $(F \then G)^j$ and $(F \then G)^k$ as desired.
\item (Horizontal Functoriality) Let $f : D_1 \to D_2$ and $g : D_2 \to D_3$ be
  horizontal maps. We then calculate:
  \begin{align*}
    \frac{\mu_{f,g}^{\cat{A}}}{(F \then G)^{f \mid g}} &= {\begin{tabular}{c}$\mu_{f, g}^{\cat{A}}$ \\ \hline $F^{f \mid g}$ \\ \hline $G^{F^0(f \mid g)}$ \end{tabular}} \\
    &= {\begin{tabular}{c} $F^f \mid F^g$ \\  \hline $\mu^{\cat{B}}_{F^0f, F^0g}$ \\ \hline  $G^{F^0f \mid F^0 g}$ \end{tabular}} \\
    &= {\begin{tabular}{c} $F^f \mid F^g$ \\ \hline $G^{F^0f} \mid G^{F^0 g}$ \\  \hline $\mu^{\cat{C}}_{G^0F^0f, G^0F^0g}$   \end{tabular}} \\
                                                       &= {\begin{tabular}{c} $\left. \frac{F^f}{G^{F^0f}} \middle| \frac{F^g}{G^{F^0 g}} \right.$ \\  \hline $\mu^{\cat{C}}_{G^0F^0f, G^0F^0g}$   \end{tabular}} \\
    &= \frac{(F\then G)^f \mid (F \then G)^g}{\mu^{\cat{C}}_{G^0F^0f, G^0F^0g}}.
  \end{align*}
\item (Functorial Interchange) Consider a square
\[
\begin{tikzcd}[sep=tiny]
D_1 \ar[rr, "f"] \ar[dd, "j"'] & & D_2 \ar[dd, "k"] \\
 & \alpha & \\
D_3 \ar[rr, "g"'] & & D_4
\end{tikzcd}
\]

We may then calculate:
\[
\begin{tikzcd}
\bullet \ar[r,equals, tick] \ar[d] \ar[rd, "F^D", phantom] & \bullet
\ar[r,equals,tick] \ar[d]                                    & \bullet
\ar[r, tick] \ar[d] \ar[rd, "{ \cat{A} }(\alpha)", phantom] & \bullet \ar[d] \\
\bullet \ar[d] \ar[r, tick]                            & \bullet \ar[d]
\ar[r, "F^j", phantom]                    & \bullet \ar[d] \ar[r, tick] \ar[rd, "F^g", phantom]             & \bullet \ar[d] \\
\bullet \ar[d] \ar[r, "G^{F^0j}", phantom]       & \bullet \ar[d]
\ar[r, equals, tick] \ar[rd, "G^{F^0 D_3}", phantom] & \bullet \ar[d]
\ar[r, tick] \ar[rd, "G^{F^0 g}", phantom]        & \bullet \ar[d] \\
\bullet \ar[r, equals, tick]                                      & \bullet
\ar[r, equals, tick]                                              & \bullet
\ar[r, tick]                                                  & \bullet          
\end{tikzcd}\coheq 
\begin{tikzcd}
\bullet \arrow[r, equals, tick] \arrow[d] \arrow[rd, "F^D", phantom] & \bullet
\arrow[r, tick] \arrow[d] \arrow[rd, "F^f", phantom]                   & \bullet \arrow[r, equals, tick] \arrow[d]                                    & \bullet \arrow[d] \\
\bullet \arrow[d] \arrow[r, equals, tick]                            & \bullet
\arrow[d] \arrow[r, tick] \arrow[rd, "{\cat{B}(F^0 \alpha)}", phantom] & {} \arrow[d] \arrow[r, "F^k", phantom]                         & \bullet \arrow[d] \\
\bullet \arrow[d] \arrow[r, "G^{F^0j}", phantom]       & \bullet \arrow[d]
\arrow[r, tick] \arrow[rd, "G^{F^0 g}", phantom]             & \bullet \arrow[d] \arrow[r, equals, tick] \arrow[rd, "G^{F^0 D_3}", phantom] & \bullet \arrow[d] \\
\bullet \arrow[r, equals, tick]                                      & \bullet
\arrow[r, tick]                                                        & \bullet
\arrow[r, equals, tick]                                              & \bullet          
\end{tikzcd} \coheq
\begin{tikzcd}
\bullet \arrow[r, tick] \arrow[d] \arrow[rd, "F^D", phantom]                      & \bullet \arrow[r, equals, tick] \arrow[d] \arrow[rd, "F^f", phantom] & \bullet \arrow[r, equals, tick] \arrow[d]                                    & \bullet \arrow[d] \\
\bullet \arrow[d] \arrow[r, tick] \arrow[rd, "G^{F^0 f}", phantom]                & \bullet \arrow[d] \arrow[r, equals, tick]                            & {} \arrow[d] \arrow[r, "F^k", phantom]                         & \bullet \arrow[d] \\
\bullet \arrow[d] \arrow[r, tick] \arrow[rd, "{\mbox{\small $\cat{C}(G^0F^0 \alpha)$} }", phantom] & \bullet \arrow[d] \arrow[r, "G^{F^0 k}", phantom]      & \bullet \arrow[d] \arrow[r, equals, tick] \arrow[rd, "G^{F^0 D_3}", phantom] & \bullet \arrow[d] \\
\bullet \arrow[r, tick]                                                           & \bullet \arrow[r, equals, tick]                                      & \bullet \arrow[r, equals, tick]                                              & \bullet          
\end{tikzcd}
\]
\end{itemize}

Now that we have a category of doubly indexed categories, we can state the functoriality result:
\begin{theorem}\label{thm.functoriality_vertical_slice}
  The vertical slice construction (\cref{defn.vertical_slice}) gives a functor
  $$\sigma : \Cat{DblFun} \to \Cat{DblIx}.$$
\end{theorem}

We will spend the rest of this section proving this theorem.
\begin{proposition}\label{prop.vertical_slice_action}
  Let $(v_0, v_1, v) : F_1 \to F_2$ be a map in $\Cat{DblFun}$. Then we have a
  doubly indexed functor 
  \[
\begin{tikzcd}
\cat{D}_{01} \arrow[dd, "{v_1}"'] \arrow[rrd, "{ \sigma F_1}", bend left] & {}
\arrow[dd, "{\sigma v}"', Rightarrow] &      \\
  &   & \Cat{Cat} \\
\cat{D}_{11} \arrow[rru, bend right, "{\sigma F_2}"']      & {}   &     
\end{tikzcd}
  \]
\end{proposition}
\begin{proof}
We define $\sigma v$ as follows:
  \begin{itemize}
  \item We have $\sigma v^D : \sigma F_1(D) \to \sigma F_2(v_1 D)$ given by the
    following action on maps:
    \[
        \begin{tikzcd}[sep=tiny]
          F_1A_1 \ar[dd, "{ j_1 }"'] \ar[rr, "{ F_1f }"] & & F_1A_2 \ar[dd, "{ j_2 }"] \\
           & \alpha & \\
          D \ar[rr, equals] & & D
        \end{tikzcd}
        \quad\mapsto\quad
        \begin{tikzcd}[sep=tiny]
          F_2v_0A_1 \ar[dd, "v_{A_1}"'] \ar[rr, "F_2 v_0f"] & & F_2 v_0 A_2 \ar[dd, "v_{A_2}"] \\
          & v_f & \\
          v_1F_1A_1 \ar[dd, "v_1j_1"'] \ar[rr, "v_1F_1f"] & & FA_2 \ar[dd, "v_1j_2"] \\
           & v_1\alpha & \\
          v_1D \ar[rr, equals] & & v_1D
        \end{tikzcd}
    \]
    In short: 
\[
  \sigma v^D(f, \alpha) \coloneqq \left( v_0f,\, \frac{v_f}{v_1 \alpha} \right).
\]
\item For any vertical map $j : D_1 \to D_2$ in $\cat{D}_{01}$, we will show
  that 
$$\sigma F_2(v_1 j) \circ \sigma v^{D_1} = \sigma v^{D_2} \circ \sigma F_1 (j)$$
so that we may take $\sigma v^j$ to be the identity natural transformation.
\begin{align*}
  \sigma F_2(v_1 j) \circ \sigma v^{D_1}(f, \alpha) &= \sigma F_2(v_1 j)\left( v_0 f, \frac{v_f}{v_1 \alpha} \right) \\
&= \left( v_0 f, {\begin{tabular}{c} $v_f$ \\ \hline $v_1 \alpha$ \\ \hline $v_1 j$ \end{tabular}} \right) \\
&= \left( v_0 f, \frac{v_f}{v_1 \left( \frac{\alpha}{j} \right)} \right) \\
  &= \sigma v^{D_2}\left( f, \frac{\alpha}{j} \right)\\
&= \sigma v^{D_2} \circ \sigma F_1 (j) (f, \alpha)
\end{align*}

\item For a horizontal map $\varphi : D_1 \to D_2$, we give the square 
\[
\begin{tikzcd}[sep=tiny]
\sigma F_1(D_1) \ar[rr, tick,"\sigma F_1(f)"] \ar[dd, "\sigma v^{D_1}"'] & &
\sigma F_1(D_2) \ar[dd, "\sigma v^{D_2}"]\\
& \sigma v^{\varphi} & \\
\sigma F_2(v_1 D_2) \ar[rr, tick, "\sigma F_2 (v_1 f)"'] &  &\sigma F_2(v_1 D_2)
\end{tikzcd}
\]
defined by 
    \[
        \begin{tikzcd}[sep=tiny]
          F_1A_1 \ar[dd, "{ j_1 }"'] \ar[rr, "{ F_1f }"] & & F_1A_2 \ar[dd, "{ j_2 }"] \\
           & \alpha & \\
          D \ar[rr, "{ \varphi }"'] & & D
        \end{tikzcd}
        \quad\mapsto\quad
        \begin{tikzcd}[sep=tiny]
          F_2v_0A_1 \ar[dd, "v_{A_1}"'] \ar[rr, "F_2 v_0f"] & & F_2 v_0 A_2 \ar[dd, "v_{A_2}"] \\
          & v_f & \\
          v_1F_1A_1 \ar[dd, "v_1j_1"'] \ar[rr, "v_1F_1f"] & & FA_2 \ar[dd, "v_1j_2"] \\
           & v_1\alpha & \\
          v_1D \ar[rr, "v_1 {\varphi}"'] & & v_1D
        \end{tikzcd}
    \]
    In short:
\[
  \sigma v^{\varphi}(f, \alpha) \coloneqq \left( v_0f,\, \frac{v_f}{v_1 \alpha} \right).
\]
  \end{itemize}

We will show that this data satisfies the laws of a doubly indexed functor.
\begin{itemize}
  \item (Vertical Lax Functoriality)As we've taken the natural transformations $\sigma v^j$ to be
    identities, they are functorial since composites of identities are identities.
  \item (Horizontal functoriality) For composable horizontal maps $\varphi_1 : D_1 \to
    D_2$ and $\varphi_2 : D_2 \to D_3$, we may calculate:
\begin{align*}
\left( \frac{\mu^{\sigma F_1}_{\varphi_1, \varphi_2}}{\sigma v^{\varphi_1 \mid \varphi_2}} \right)((f_1, \alpha_1), (f_2, \alpha_2)) &= \left( v_1(f_1 \mid f_2),\, \frac{v_{f_1 \mid f_2}}{v_1(\alpha_1 \mid \alpha_2)}\right) \\
&= \left( v_1 f_1 \mid v_1 f_2,\, \frac{v_{f_1} \mid v_{f_2}}{v_1 \alpha_1 \mid v_1 \alpha_2} \right) \\
&= \left( v_1 f_1 \mid v_1 f_2,\, \left. \frac{v_{f_1}}{v_1 \alpha_1} \middle| \frac{v_{f_2}}{v_1 \alpha_2}\right. \right) \\
&= \left( \frac{v^{\varphi_1} \mid v^{\varphi_2}}{\mu^{\sigma F_2}_{v_1 \varphi_1, v_2 \varphi_2}} \right)((f_1, \alpha_1), (f_2, \alpha_2)).
\end{align*}

\item (Functorial Interchange) Consider a square
\[
\begin{tikzcd}[sep=tiny]
D_1 \ar[rr, "\varphi_1"] \ar[dd, "j_1"'] & & D_2 \ar[dd, "f_2"] \\
 & \beta & \\
D_3 \ar[rr, "\varphi_2"'] & & D_4
\end{tikzcd}
\]

Since $\sigma v^{j_1}$ and $\sigma v^{j_2}$ are identities, we just need to show
that 
$$\frac{\sigma F_1 (\beta)}{\sigma v^{\varphi_2}} = \frac{\sigma
  v^{\varphi_1}}{\sigma F_2 (v_1 \beta)}.$$
To that end, we calculate:
\begin{align*}
\left( \frac{\sigma F_1 (\beta)}{\sigma v^{\varphi_2}} \right)(f, \alpha) &= \sigma v^{\varphi_2}\left( f,\, \frac{\alpha}{\beta} \right) \\
&= \left( v_1 f,\, \frac{v_f}{v_1\left( \frac{\alpha}{\beta} \right)} \right) \\
&= \left( v_1 f,\, {\begin{tabular}{c} $v_f$ \\ \hline $v_1 \alpha$ \\ \hline $v_1 \beta$ \end{tabular}} \right) \\
&= \sigma F_2(v_1 \beta)\left( v_1 f,\,  \frac{v_f}{v_1 \alpha} \right) \\
&= \left( \frac{\sigma v^{\varphi_1}}{F_2(v_1 \beta)} \right)(f, \alpha).
\end{align*}
\end{itemize}

\end{proof}

We now finish the proof of \cref{thm.functoriality_vertical_slice}.
\begin{lemma}
The assignment $(v_0, v_1, v) \mapsto (v_1, \sigma v)$ defined in
\cref{prop.vertical_slice_action} is functorial.
\end{lemma}
\begin{proof}
Let $(v_0, v_1, v) : F_1 \to F_2$ and $(w_0, w_1, w) : F_2 \to F_3$ be maps in
$\Cat{DblFun}$. We will show that
$$(v_1 \then w_1,\, \sigma(v \ast w)) = (v_1, \sigma v) \then (w_1, \sigma w).$$
The first components of these pairs are equal by definition, so we just need to
show that $\sigma(v \ast w) = \sigma v \then \sigma w$.
\begin{itemize}
\item This calculation is the same as for a general horizontal. 
  \item For a vertical $j : D_1 \to D_2$, we calculate, we note that both sides
    are the same identity natural transformation.
  \item For a horizontal $\varphi : D_1 \to D_2$, we calculate:
  \begin{align*}
\sigma(v \ast w)^{\varphi}(f, \alpha) &\coloneqq \left( w_0v_0f, \frac{(v \ast w)_f}{w_1
    v_1 \alpha} \right)  \\ 
    &= \left( w_0 v_0 f, {\begin{tabular}{c} $w_{v_0 f}$\\ \hline $w_1 v_f$ \\ \hline $w_1 v_1 \alpha$ \end{tabular}} \right) \\
                              &= \left( w_0v_0 f, \frac{w_{v_0 f}}{w_1 \left( \frac{v_f}{v_1 \alpha} \right)} \right) \\
                              &= \sigma w^{v_1 \varphi} \left( v_0 f, \frac{v_f}{v_1 \alpha} \right) \\
    &= (\sigma v^{\varphi}) \then (\sigma w^{v_1 \varphi})(f, \alpha).
    \end{align*}
\end{itemize}
\end{proof}

\section{Change of Doctrine}

We have learned about a variety of doctrines through the course of this book:
\begin{itemize}
  \item There are the deterministic doctrines (\cref{def.deterministic_doctrines})
\[
(\smctx{-}: \cat{C}\op \to \Cat{Cat},\, \phi \mapsto \phi \circ \pi_2)
\]
which may be defined for any cartesian category $\cat{C}$. While we have focused
so far on the case $\cat{C} = \smset$, many other cartesian categories are of
interest in the study of deterministic dynamical systems. For example, in
ergodic theory we most often use the category of measureable spaces and
measurable functions.\footnote{We most often consider maps which preserve a
  specific measure on a space as well, but the category of such measure
  preserving maps is not cartesian. Often one needs to go and twiddle these
  general definitions of doctrines in particular cases to suit the particular
  needs of a subject.} We often assume the dynamics of the systems are not arbitary
set maps, but are furthermore continuous or differentiable; this means working
in the cartesian categories of topological spaces or differentiable manifolds.
\item There are also the differential doctrines
  (\cref{def.euclidean_diff_doctrine,def.toric_diff_doctrine,def.general_diff_doctrine})
  where the tangent bundle plays an important role. \jaz{ There are also non-standard
  differential doctrines arising from cartesian differential categories and
  tangent categories with display maps.
 }
\item There are the non-deterministic doctrines for any commutative monad $M$ on
  a cartesian category $\cat{C}$. As we saw in \cref{Chapter.2}, by varying the
  monad $M$ we can achieve a huge variety of flavors of non-determinism. This
  includes possibilistic and stochastic non-determinism, but also other variants
  like systems with cost-sensitive transitions and (\cref{def.system_with_costs}). 
\end{itemize}
These are just large classes of doctrines that have been easy to describe in
generality. Different particular situations will require different particular
doctrines. For example, we may decide to restrict the sorts of maps appearing in
our doctrines by changing the base $\cat{C}$ as in
\cref{sec.restriction_of_doctrines}. There may also be doctrines constructed by
hand for particular purposes, such as ergodic theory.

These doctrines are not isolated from each other. We have seen already in
\cref{sec.restriction_of_doctrines} that some doctrines may be formed by
restricting others. There are also some apparent inclusions of doctrines that
are not explained by restriction; for example, the Euclidean differential
doctrine is a special case of the general differential doctrine. We should be
able to think of Euclidean differential systems and general differential systems
without too much hassle, and we should be able to apply theorems that pertain to
general differential systems to Euclidean ones. Another example of inclusion of
doctrines is of deterministic systems into non-deterministic systems of any flavor.

There are also more drastic ways to change doctrine. Any map of commutative
monads $\phi : M \to N$ gives us a way of changing an $M$-system into an
$N$-system, changing the flavor of non-determinism. We may also
\emph{approximate} a differential system by a deterministic system. 

These are all ways of \emph{changing our doctrine}, and it is these changes of
doctrine that we will attend to in this section. 

We will begin by defining a \emph{change of doctrine}, which will give us a
category of doctrines.\footnote{\jaz{Actually a 2-category, but this will likely
    go into an appendix.}} We will then show that forming the doubly indexed
category of systems $\Cat{Sys}(\dd)$ is functorial in the doctrine $\dd$.

\subsection{Definition}

Let's recall the informal and formal definitions of dynamical system doctrines. 

The informal definition is that a doctrine is a way to answer a series of
questions about what it means to be a dynamical system.
\begin{informal}
  A \emph{doctrine} of dynamical systems is a particular way to answer the following
  questions about what it means to be a dynamical system:
  \begin{enumerate}
  \item What does it mean to be a state?
  \item How should the output vary with the state --- discretely,
    continuously, linearly?
  \item Can the kinds of input a
    system takes in depend on what it's putting out, and how do they depend on it?
  \item What sorts of changes are possible in a given state?
  \item What does it mean for states to change. 
  \item How should the way the state changes vary with the input?
  \end{enumerate}
\end{informal}

This informal definition is captured by the sparse, formal definition that a
doctrine is a pair consisting of an indexed category $\cat{A} : \cat{C}\op \to
\Cat{Cat}$ together with a section $T$. The various questions correspond to the
choices one can make when defining such a pair.

To change a doctrine, then, means to change our answers to these questions. We
want to enact this change by some formulated process. For example, if what it
means to be a state is a to be a vector in Euclidean space, and we would like to
change this to instead answer that to be a state means to be an element of an
abstract set, then we want a way of taking Euclidean spaces and producing an
abstract set.

Now, we can't just fiddle arbitrarily with the answers to our questions; they
all have to hang together in a coherent way. The formal definition can guide us
to what sort of changes we can make that cohere in just this way. We can change
what it means to be a state, how the output varies with the state, and the way
the inputs vary by changing the indexed category $\cat{A}$. 

Suppose that $(\cat{A}, T_1)$ and $(\cat{B}, T_2)$ are dynamical system
doctrines. If we have an indexed functor (\cref{def.indexed_functor}) $(F, \overline{F}) :
\cat{A} \to \cat{B}$ between indexed categories, then from a dynamical system
$\lens{\update{S}}{\expose{S}} : \lens{T_1\State{S}}{\State{S}} \fromto
\lens{\In{S}}{\Out{S}}$ we can get a lens 
\[
\lens{\overline{F}\update{S}}{F\expose{S}} : \lens{\overline{F}T_1
  \State{S}}{F\State{S}} \fromto \lens{\overline{F}\In{S}}{F\Out{S}}
\]
That is, we have changed what it means to be a state ($F\State{S}$), how the
output varies with state ($F\expose{S}$), and how the inputs vary with output
($\overline{F}\In{S}$).
This is not quite a dynamical system, however, since since its domain is not
$\lens{T_2 F\State{S}}{F\State{S}}$. In order for us to get a $(\cat{B},
T_2)$-system, we need to say how to change the what it means for a state to
change. 

The most direct way to produce a $(\cat{B}, T_2)$-system would be to compose
with a map $\phi : \overline{F}T_1 \State{S} \to T_2 F\State{S}$ which
tells us how to take a $T_1$ change (re-interpreted already by $\overline{F}$),
and get a $T_2$ change (for the re-interpretation of state by $F$). Indeed, if
we considered this map $\phi$ as a lens $\lens{\phi}{\id} : \lens{T_2 F \State{S}}{F \State{S}} \fromto \lens{\overline{F}T_1 \State{S}}{F
  \State{S}}$, we may form the composite
\[
\lens{\overline{F}\update{S}}{F\expose{S}} \then \lens{\phi}{\id} : \lens{T_2 F
  \State{S}}{F \State{S}} \fromto \lens{\overline{F}\In{S}}{F\Out{S}}.
\]
This is a $(\cat{B}, T_2)$-system, and this process is how we may use a change
of doctrine to turn $(\cat{A}, T_1)$-systems into $(\cat{B},T_2)$-systems.

We therefore arrive at the following formal defintion of change of doctrine.
\begin{definition}\label{def.change_of_doctrine}
  Let $(\cat{A} : \cat{C} \to \Cat{Cat}, T_1)$ and $(\cat{B} : \cat{D} \to
  \Cat{Cat}, T_2)$ be dynamical system doctrines. A \emph{change of doctrine}
  $((F, \overline{F}), \phi) : (\cat{A}, T_1) \to (\cat{B}, T_2)$ consists of:
\begin{itemize}
  \item An indexed functor $(F, \overline{F}) : \cat{A} \to \cat{B}$. 
  \item A \emph{transformation of sections} $\phi : \overline{F}T_1 \to T_2 F$,
    which consists of a family of maps $\phi_C : \overline{F}T_1 C \to T_2 FC$
    for each $C$ in $C$, satisfying the following naturality condition:
    \begin{itemize}
      \item For any $f : C \to C'$, we have that the following square commutes
        in $\cat{B}(FC)$:
        \begin{equation}\label{eqn.transformation_of_sections}
\begin{tikzcd}
\overline{F}T_1 C  \ar[d, "\overline{F}T_1f"'] \ar[r, "\phi_C"]& T_2 F C \ar[d, "T_2Ff"] \\
\overline{F}f^{\ast}T_1 C' \ar[r,"(Ff)^{\ast}\phi_{C'}"']& (Ff)^{\ast}T_2 FC'
\end{tikzcd}
\end{equation}
\end{itemize}

\end{itemize} 
\end{definition}

We can package the transformation of sections into a natural transformation,
which will make it easier to work with theoretically.
\begin{proposition}\label{prop.transform_section_as_nat_transform}
The data of a transformation of sections as in \cref{def.change_of_doctrine} is equivalent to the data
of a natural transformation $\lens{\phi}{\id} : \lens{\overline{F}}{F} \circ T_1
    \Rightarrow \lens{T_2(-)}{(-)} \circ F$ which acts as the identity on $F$ on
    its bottom component. We can express this condition with the following equation on
    diagrams of natural transformations: 
    \[
\begin{tikzcd}
  \cat{C} \ar[d,"{\lens{T_1(-)}{(-)}}"'] \ar[r, "F"] & \cat{D} \ar[d,
  "{\lens{T_2(-)}{(-)}}"] \\
\int^{C : \cat{C}}\cat{A}(C) \ar[r, "\lens{\overline{F}}{F}"'] \ar[ur,
Rightarrow, "{\lens{\phi}{\id}}"] \ar[d,"\pi"'] & \int^{D :
  \cat{D}} \cat{B}(D) \ar[d, "\pi"]\\
\cat{C} \ar[r, "F"'] & \cat{D}
\end{tikzcd} 
=
\begin{tikzcd}
  \cat{C} \ar[dd, equals] \ar[r, "F"] & \cat{D} \ar[dd, equals] \\
  & \\
\cat{C} \ar[r, "F"'] \ar[uur, Rightarrow, "\id_F"] & \cat{D}
\end{tikzcd}
\]
\end{proposition}
\begin{remark}
  We note that the components of the natural transformation $\lens{\phi}{\id}$
  here are \emph{charts} and not lenses. We will, however, exploit the duality
  between lenses and charts whose lower component are identities.
\end{remark}
\begin{proof}
  That the transformation $\lens{\phi}{\id}$ acts as the identity on $F$ means
  that it is determined by its top map $\phi$. We can then see that the
  naturality square for $\phi$ is precisely the square given in \cref{def.change_of_doctrine}.
\end{proof}


Every restriction (from \cref{sec.restriction_of_doctrines}) is a change of doctrine.
\begin{proposition}
Let $\dd = (\cat{A} : \cat{C}\op \to \Cat{Cat}, T)$ be a doctrine, and let $F :
\cat{D} \to \cat{C}$ be a functor. Then there is a change of doctrine $((F,
\id), \id) : \dd_{|F} \to \dd$ from the restriction $\dd_F = (\cat{A} \circ
F\op, T \circ F)$ (\cref{def.restriction_doctrine}) of $\dd$ by $F$ to $\dd$.
\end{proposition}
\begin{proof}
By definition, $(F, \id) : \cat{A} \to (\cat{A} \circ F\op)$ is an indexed
functor. Since, by \cref{prop.transform_section_as_nat_transform}, the data of a
transformation of sections is the same as a natural transformation of a certain
sort, we may take that transformation to be the identity.
\end{proof}

There are, however, more interesting changes of doctrine. For example, every
morphism of commutative monads gives rise to a change of doctrine.

\begin{proposition}\label{prop.morphism_commutative_monoid_change_doctrine}
  Let $\phi : M \to N$ be a morphism of commutative monads on a cartesian
  category $\cat{C}$. Then there is a change of doctrine given by
  \[
((\id, \phi_{\ast}), \id) : \nondet_{M} \to \nondet_{N}.
  \]
\end{proposition}
\begin{proof}
We constructed the indexed functor $(\id, \phi_{\ast}) : \smctx{-}^M \to
\smctx{-}^N$ in \cref{prop.commutative_monad_indexed_functor}. It remains to
show that the following square of functors commutes, so that we may take the
transformation of sections to be the identity:
\[
  \begin{tikzcd}
\cat{C} \ar[r, equals]\ar[d,"T^M"'] & \cat{C}  \ar[d,"T^N"]\\
  \int^{C : \cat{C}}\smctx{C}^M \ar[r, "{\littlelens{\phi_{\ast}}{\id}}"']  & \int^{C : \cat{C}}
  \smctx{C}^N  
  \end{tikzcd}
\]
Let $f : C' \to C$ be a map in $\cat{C}$. Then $T^Mf$ is $\pi_2 \then f \then \eta^M
: C' \times C' \to MC$ and $T^N f$ is $\pi_2 \then f \then \eta^N
: C' \times C' \to NC$. Now, $\phi_{\ast}T^M f$ is $\pi_2 \then f \then \eta^M
\then \phi_C$, but by the unit law for morphisms of commutative monads, $\eta^M
\then \phi_C = \eta^N$. So the square commutes and we can take the
transformation of sections to be the identity.
\end{proof}
  
We may also describe changes of doctrines between various sorts of deterministic
doctrine. 
\begin{proposition}\label{prop.change_of_doctrine_cartesian}
Let $F : \cat{C} \to \cat{D}$ be a cartesian functor between cartesian
categories. Then there is a change of doctrine
\[
((F, \overline{F}), \id) : \determ_{\cat{C}} \to \determ_{\cat{D}} 
\]
from the deterministic doctrine in $\cat{C}$ to the cartesian doctrine in $\cat{D}$.
\end{proposition}
\begin{proof}
We need to construct the indexed functor $(F, \overline{F})$, and then prove
that the square 
\[
  \begin{tikzcd}
\cat{C} \ar[r, "F"]\ar[d,"T^{\cat{C}}"'] & \cat{C}  \ar[d,"T^{\cat{D}}"]\\
  \int^{C : \cat{C}}\smctx{C} \ar[r, "{\littlelens{\overline{F}}{F}}"']  & \int^{D : \cat{D}}
  \smctx{D}  
  \end{tikzcd}
\]
commutes, so that we may take the transformation of sections to be the identity.

We begin first by constructing $\overline{F}$. We note that since $F$ is
cartesian, it extends to a functor
$$\overline{F}_C : \smctx{C} \to \smctx{FC}$$
by sending $f : C \times X \to Y$ to $Ff : FC \times FX \to FY$. It is routine
to check that this makes $(F, \overline{F})$ into an indexed functor. In
particular, for a map
$r : C' \to C$ in $\cat{C}$, we see that
\[
\overline{F}_{C'}(r^{\ast}f) = \overline{F}((r \times \id) \then f) = F((r
\times \id) \then f) = (Fr \times \id) \then Ff = (Fr)^{\ast}\overline{F}(f)
\]

Next we check that the square commutes. Let $f : C' \to C$ be a map in
$\cat{C}$. Then $T^{\cat{D}} \circ F (f) = \lens{\pi_2 \then Ff}{Ff}$, while
$\lens{\overline{F}}{F}\lens{Tf}{f} = \lens{\overline{F}(\pi_2 \then f)}{Ff}$.
But since $F$ is cartesian, $F(\pi_2) = \pi_2$, so these are equal.
\end{proof}

\begin{example}
  \cref{prop.change_of_doctrine_cartesian} gives us a number of trivial ways to
  change the flavor of our deterministic systems.

For example, it is obvious that any deterministic dynamical system whose
$\update{}$ and $\expose{}$ maps are continuous gives rise to a deterministic
dynamcial system without the constraint of continuity, simply by forgetting that
the maps are continuous. We formalize this observation by applying
\cref{prop.change_of_doctrine_cartesian} to the forgetful functor $\Fun{U} : \Cat{Top}
\to \smset$ which sends a topological space to its underlying set of points.

Similarly, any deterministic dynamical system gives rise to a continuous
deterministic dynamical system if we equip all sets involved with the discrete
topology. This is formalized by applying
\cref{prop.change_of_doctrine_cartesian} to the functor $\Fun{disc} : \smset \to
\Cat{Top}$ which equips a set with the discrete topology.
\end{example}

The most interesting examples of changes of doctrine are the ones which move
between different sorts of doctrine, such as from differential to deterministic.
An example of this is the Euler approximation, which takes a Eulidean
differential system to a deterministic system.

Let's take a minute to recall the Euler method. If $\lens{u}{r} : \lens{\rr^n}{\rr^n} \fromto \lens{\rr^k}{\rr^m}$
is a differential system representing the differential equation
$$\frac{ds}{dt} = u(s, i),$$
then for a sufficiently small $\varepsilon > 0$, the state at time $t + \varepsilon$
will be roughly
$$s(t + \varepsilon) \approx s(t) + \varepsilon \cdot u(s(t), i(t)).$$
Choosing a specific $\varepsilon$ as a time increment, we can define a discrete
time, deterministic system by
\begin{equation}\label{eqn:euler.method.equation}
\mathcal{E}_{\varepsilon}u(s, i) = s + \varepsilon \cdot u(s, i).
\end{equation}
This simple method of approximating the solution of a differential equation is
called the Euler method. We can see the Euler method as a change of doctrine
from a differential doctrine to a deterministic doctrine.

\begin{proposition}
  For any $\varepsilon > 0$ The Euler method gives rise to a change of doctrine
  \[
\mathcal{E}_{\varepsilon} : \eucdiff_{|\textbf{Aff}} \to \determ_{\Cat{Euc}}.
  \]
  This is given by
  $$((\id, \id), \phi) : (\smctx{|\Cat{Aff}} : \Cat{Aff}\op \to \Cat{Cat}, T) \to
  (\smctx{|\Cat{Aff}} : \Cat{Aff}\op \to \Cat{Cat}, \rr^n \mapsto \rr^n)$$
  where $\phi : \rr^n \times \rr^n \to \rr^n$ is defined by
  $$\phi(s, v) = s + \varepsilon \cdot v.$$
\end{proposition}
\begin{proof}
  We note, first of all, that composing with $\phi$ gives us the correct formula
  for the Euler approximation. Explicitly,
  $$\phi \circ u(s, i) = s + \varepsilon  \cdot u(s, i),$$
which was the definition for $\mathcal{E}_{\varepsilon}u$ in \cref{eqn:euler.method.equation}.

  All that we need to show is that $\phi$ is a
  transformation of sections. This means that the following square commutes for
  any \emph{affine} $f : \rr^n \to \rr^m$: 
  \[
  \begin{tikzcd}
    \lens{\rr^n}{\rr^n} \ar[r, shift left, "{\lens{\pi_2 \then f}{f}}"] \ar[r, shift right] \ar[d, shift right,
    "{\lens{\phi}{\id}}"'] \ar[d, shift left, leftarrow] &
    \lens{\rr^m}{\rr^m} \ar[d, shift left, leftarrow,
    "{\lens{\phi}{\id}}"] \ar[d, shift right]\\
    \lens{\rr^n}{\rr^n} \ar[r, shift right, "{\lens{Tf}{f}}"']
    \ar[r, shift left] & \lens{\rr^m}{\rr^m}
  \end{tikzcd}
  \]
  The bottom component of this square commutes trivially. The top component
  comes down to the equation 
  \begin{equation}\label{eqn:euler.affine.map}
f(s + \varepsilon \cdot v) = f(s) + \varepsilon Tf(s, v)
\end{equation}
which says that incrementing $s$ by $\varepsilon$ in the $v$ direction in $f$ is
the same as incrementing $f(s)$ by the $\varepsilon$ times the directional
derivative of $f$ in the $v$ direction. This is true for affine function; even
more, it characterizes affine functions, so that we see that we must assume that
$f$ is affine for this square to commute.
\end{proof}

\begin{remark}
It would be very interesting to have a theory which allowed us to speak of
``approximate'' changes of doctrine. If we plug a function $f : \rr^n \to \rr^m$
into the above formulas for the Euler method, then we find that \cref{eqn:euler.affine.map} 
only holds up to $O(\varepsilon^2)$. For affine functions, this means that it does
hold, which is why we restrict to affine functions. But it would be interesting
to have a theory which could account for how these approximate equalities
affected the various compositionality results all the way down.
\end{remark}

%writehere

In the upcoming \cref{sec:functoriality.of.sys}, we will see what knowing that
the Euler method is a change of doctrine lets us conclude about the behaviors
and compositionality of Euler method approximations. 

Considering doctrines together with their changes gives us a category
$\Cat{Doctrine}$. 

\begin{definition}\label{def.category_of_doctrines}
  The category $\Cat{Doctrine}$ has as objects the dynamical systems doctrines
  and as morphisms the changes of doctrine.

  If $((F_1, \overline{F}_1), \phi_1) : (\cat{A}_1, T_1) \to (\cat{A}_2, T_2)$
  and $((F_2, \overline{F}_2), \phi_2) : (\cat{A}_2, T_2) \to (\cat{A}_3, T_3)$
  are changes of doctrine, then their composite is defined to be
  \[
((F_1, \overline{F}_1), \phi_1) \then ((F_2, \overline{F}_2), \phi_2) \coloneqq
((F_1, \overline{F}_1)\then(F_2, \overline{F}_2), \phi_1 \ast \phi_2)
\]
where $\phi_1 \ast \phi_2$ is the transformation of sections given by
\[
(\phi_1 \ast \phi_2)_C \coloneqq \overline{F_2}\overline{F}_1 T_1 C
\xto{\overline{F}_2 \phi_1} \overline{F_2} T_2 F_1 C \xto{(\phi_2)_{F_1 C}} T_3
F_2 F_1 C.
\]

In terms of natural transformations (see
\cref{prop.transform_section_as_nat_transform}), this is the diagram
\[
  \begin{tikzcd}[row sep = large]
    \cat{C}_1 \ar[r, "F_1"]\ar[d, "T_1"'] & \cat{C}_2\ar[r, "F_2"]\ar[d, "T_2"] & \cat{C}_3\ar[d, "T_3"] \\
    \int^{C : \cat{C}_1}\cat{A}_1(C)\ar[r, "{\lens{\overline{F}_1}{F_1}}"']
    \ar[ur, Rightarrow, "{\littlelens{\phi_1}{\id}}"] &\int^{C : \cat{C}_2}\cat{A}_2(C)\ar[r, "{\lens{\overline{F}_2}{F_2}}"']\ar[ur, Rightarrow, "{\littlelens{\phi_2}{\id}}"] &\int^{C : \cat{C}_3}\cat{A}_3(C) 
  \end{tikzcd}
\]
\end{definition}



\subsection{Functoriality}\label{sec:functoriality.of.sys}

We use changes of doctrines to turn a system of one sort into a system of
another sort. We sketched how this process goes above, but for good measure
let's revisit it.
\begin{definition}\label{def.change_of_doctrine_action_systems}
Let $\mathbb{F} = ((F, \overline{F}), \phi) : (\cat{A}, T_1) \to (\cat{B}, T_2)$ be a change of doctrine, and
let
\[
\Sys{S} = \lens{\update{S}}{\expose{S}} : \lens{T_1\State{S}}{\State{S}} \fromto \lens{\In{S}}{\Out{S}}
\]
be a $(\cat{A}, T_1)$-system. Then we have a $(\cat{B}, T_2)$-system $\mathbb{F}
\Sys{S}$ defined to
be the composite 
\[
\lens{\phi}{\id} \then \lens{\overline{F}\update{S}}{F\expose{S}} : \lens{T_2F\State{S}}{F\State{S}} \fromto \lens{\overline{F}\In{S}}{F\Out{S}}.
\]
Explicitly, this system has update $\overline{F} \update{S} \then \phi$ and
expose $F\expose{S}$.
\end{definition}

The goal of this section will be to provide a number of compositionality results
concerning how changing the doctrine of a system relates to wiring systems
together and to behaviors. Specifically, we will prove the following theorem:
\begin{theorem}\label{thm.functoriality_system_construction}
  There is a functor 
\[
\Cat{Sys} : \Cat{Doctrine} \to \Cat{DblIx}
\]
sending a dynamical system doctrine $\dd$ to the doubly indexed category
$\Cat{Sys}_{\dd}$ (\cref{def.doubly_indexed_cat_systems}) of systems in it. 

This functor sends a change of doctrine $\mathbb{F} : \dd_1 \to \dd_2$ to the doubly indexed functor $\Cat{Sys}(\dd_1)
\to \Cat{Sys}(\dd_2)$ which sends a $\dd_1$-system $\Sys{S}$ to the
$\dd_2$-system $\mathbb{F}\Sys{S}$ from \cref{def.change_of_doctrine_action_systems}.
\end{theorem}

We will prove this theorem using the vertical slice construction. Recall that
the doubly indexed category $\Cat{Sys}(\dd)$ is the vertical slice construction
of the section $T$ considered as a double functor $(hT : h\cat{C} \to
\Cat{Arena}_{\dd})$ (\cref{prop.sys_is_vertical_slice}). This means that if we
can show that the assignment 
\[
(\cat{A} : \cat{C}\op \to \Cat{Cat}, T) \mapsto (hT : h\cat{C} \to
\Cat{Arena}_{(\cat{A}, T)})
\]
gives a functor $\Cat{Doctrine} \to \Cat{DblFun}$, then we can compose this with
the vertical slice construction $\sigma : \Cat{DblFun} \to \Cat{DblIx}$. This is
what we will focus on.

\begin{lemma}
The assignment 
\[
(\cat{A} : \cat{C}\op \to \Cat{Cat}, T) \mapsto (hT : h\cat{C} \to
\Cat{Arena}_{(\cat{A}, T)})
\]
gives a functor $\iota : \Cat{Doctrine} \to \Cat{DblFun}$. This functor sends a
change of doctrines
\begin{equation}\label{eqn.change_of_doctrines}
  \begin{tikzcd}[row sep = large]
    \cat{C} \ar[r, "F"]\ar[d, "T_1"'] & \cat{D} \ar[d, "T_2"] \\
    \int^{C : \cat{C}}\cat{A}(C)\ar[r, "{\lens{\overline{F}}{F}}"']
    \ar[ur, Rightarrow, "{\littlelens{\phi}{\id}}"] &\int^{C : \cat{D}}\cat{B}(C)
  \end{tikzcd}
\end{equation}
to the morphism double functors
\begin{equation}\label{eqn.change_of_doctrines_dblfun}
  \begin{tikzcd}[row sep = large]
    h\cat{C} \ar[r, "F"]\ar[d, "hT_1"'] & h\cat{D} \ar[d, "hT_2"]\ar[dl, Rightarrow, "{\littlelens{\phi}{\id}}"'] \\
    \Cat{Arena}_{(\cat{A}, T_1)}\ar[r, "{\lens{\overline{F}}{F}}"']
     & \Cat{Arena}_{(\cat{B}, T_2)}
  \end{tikzcd}
\end{equation}
\end{lemma}
\begin{proof}
  With all that we have set up, there is not too much to prove here. We first
  note that the the functoriality of the assignment $\cat{A} \mapsto
  \Cat{Arena}_{\cat{A}}$ was proven in
  \cref{prop.functoriality_arena_construction}. We only need to focus on the
  vertical transformation.

  We need to
show that $\lens{\phi}{\id}$ may be interpreted as a vertical transformation
$hT_2 \circ F \to \lens{\overline{F}}{F} \circ hT_1$. There is some subtlety
here; in \cref{eqn.change_of_doctrines}, $\lens{\phi}{\id}$ is interpreted as a
natural transformation taking place in the category of \emph{$\cat{B}$-charts}, while in
\cref{eqn.change_of_doctrines_dblfun} we have a vertical transformation in the
double category of arenas. But the vertical arrows in $\Cat{Arena}_{(\cat{B},
  T_2)}$ are \emph{$\cat{B}$-lenses}, not $\cat{B}$-charts. This explains the
change of direction: we can consider the \emph{chart} $\lens{\phi}{\id} :\lens{\overline{F}}{F} \circ hT_1  \to  hT_2
\circ F$ as a \emph{lens}
$\lens{\phi}{\id} : hT_2 \circ F \to \lens{\overline{F}}{F} \circ hT_1$ by the
duality between pure charts and lenses --- those having an identity in the
bottom component.

Let's describe precisely how $\lens{\phi}{\id}$ becomes a vertical
transformation.
\begin{itemize}
  \item For every $C \in h\cat{C}$, we have the lens $\lens{\phi}{\id} :
    \lens{T_2 FC}{F X}  \to \lens{\overline{F}T_1 C}{F F}$.
  \item For every horizontal arrow $f : C' \to C$ in $h\cat{C}$ (which is to
    say, any map in $\cat{C}$), we have the square
    \begin{equation}\label{eqn.nat_trans_proof_square}
  \begin{tikzcd}
    \lens{T_2FC'}{FC'} \ar[r, shift left, "{\lens{T_2 Ff}{Ff}}"] \ar[r, shift right] \ar[d, shift right,
    "{\lens{\phi}{\id}}"'] \ar[d, shift left, leftarrow] &
    \lens{T_2FC}{FC} \ar[d, shift left, leftarrow,
    "{\lens{\phi}{\id}}"] \ar[d, shift right]\\
    \lens{\overline{F}T_1 C'}{FC'} \ar[r, shift right, "{\lens{\overline{F}T_1f}{Ff}}"']
    \ar[r, shift left] & \lens{\overline{F}T_1C}{FC}
  \end{tikzcd}
  \end{equation}
in $\Cat{Arena}_{(\cat{B}, T_2)}$. This is a square because both its top and
bottom component squares commute; the bottom one trivially, and the top one by
the defining \cref{eqn.transformation_of_sections} of $\phi$.
\end{itemize}
We now check that this satisfies the laws of a vertical transformation. It is
largely trivial, since the double categories are particularly simple (as double categories).
\begin{itemize}
  \item (Vertical Naturality) By construction, the only vertical arrows in
    $h\cat{C}$ are identities, so there is nothing to check.
  \item (Horizontal Naturality) Since $\Cat{Arena}$ is thin
    (\cref{def.thin_double_cat}), any two squares with the same signature are
    equal, so there is nothing to check.
  \item (Horizontal Unity) This is true since all the functors involved in
    defining the top and bottom of the square \cref{eqn.nat_trans_proof_square}
    preserve identities.
  \item (Square Naturality) This again follows trivially by the thin-ness of $\Cat{Arena}$.
\end{itemize}

The proof of functoriality itself follows from a straightforward comparison of
the two definitions of composition. They simply give the same formula on
objects, and on horizontal morphisms we get squares of the same signature in a
thin double category so there is nothing more to check.
\end{proof}

We can therefore define
\[
\Cat{Doctrine} \xto{\Cat{Sys}} \Cat{DblIx} \coloneqq \Cat{Doctrine} \xto{\iota}
\Cat{DblFun} \xto{\sigma} \Cat{DblIx}.
\]
Let's take a moment to understand this definition in full. Suppose we have a
change of doctrine $((F, \overline{F}), \phi) : (\cat{A}, T_1) \to (\cat{B},
T_2)$. Then $\iota((F, \overline{F}), \phi)$ is a map of double functors:
\[
  \begin{tikzcd}[row sep = large]
    h\cat{C} \ar[r, "F"]\ar[d, "hT_1"'] & h\cat{D} \ar[d, "hT_2"]\ar[dl, Rightarrow, "{\littlelens{\phi}{\id}}"'] \\
    \Cat{Arena}_{(\cat{A}, T_1)}\ar[r, "{\lens{\overline{F}}{F}}"']
     & \Cat{Arena}_{(\cat{B}, T_2)}
  \end{tikzcd}
\]
Then, by \cref{prop.vertical_slice_action}, we get a doubly indexed functor
\[
\begin{tikzcd}
\Cat{Arena}_{\dd_1} \arrow[dd, "{\littlelens{\overline{F}}{F}}"'] \arrow[rrd, "{
  \Cat{Sys}_{\dd_1}}", bend left] & {}
\arrow[dd, "{\sigma \littlelens{\phi}{\id}}"', Rightarrow] &      \\
  &   & \Cat{Cat} \\
\Cat{Arena}_{\dd_2} \arrow[rru, bend right, "{\Cat{Sys}_{\dd_2}}"']      & {}   &     
\end{tikzcd}
\]
In this diagram, $\sigma\littlelens{\phi}{\id}$ is defined as follows:
\begin{itemize}
\item (\cref{def.lax_doubly_indexed_functor}: \cref{def:lax.doubly.indexed.functor.1}) For a $\dd_1$-arena $\lens{A^-}{A^+}$,
  we have the functor 
  \[
    \sigma\littlelens{\phi}{\id}^{\littlelens{A^-}{A^+}} :
    \Cat{Sys}_{\dd_1}\littlelens{A^-}{A^+} \to \Cat{Sys}_{\dd_2}\littlelens{\overline{F}A^-}{FA^+}.
  \]
given by sending a simulation $\psi : \Sys{T} \to \Sys{S}$ to the composite:
\[
    \begin{tikzcd}
      \lens{T_1\State{T}}{\State{T}} \ar[r, shift left, "\lens{T_1\psi}{\psi}"] \ar[r, shift right] \ar[d, shift right,
      "\lens{\update{T}}{\expose{T}}"'] \ar[d, shift left, leftarrow] &
      \lens{T_1\State{S}}{\State{S}} \ar[d, shift left, leftarrow,
      "\lens{\update{S}}{\expose{S}}"] \ar[d, shift right]\\
      \lens{A^-}{A^+} \ar[r, shift right, equals] \ar[r,
      shift left, equals] & \lens{A^-}{A^+}
    \end{tikzcd}
    \quad\mapsto\quad
    \begin{tikzcd}
      \lens{T_2F\State{T}}{F\State{T}} \ar[r, shift left,
      "\lens{T_2F\psi}{F\psi}"] \ar[r, shift right] \ar[d, shift right, "\lens{\phi}{\id}"'] \ar[d, shift left, leftarrow] &
      \lens{T_2F\State{S}}{F\State{S}} \ar[d, shift left, leftarrow,
      "\lens{\phi}{\id}"] \ar[d, shift right]\\
      \lens{\overline{F}T_1\State{T}}{F\State{T}} \ar[r, shift left, "\lens{\overline{F}T_1\psi}{F\psi}"] \ar[r, shift right] \ar[d, shift right,
      "\lens{\overline{F}\update{T}}{F\expose{T}}"'] \ar[d, shift left, leftarrow] &
      \lens{\overline{F}T_1\State{S}}{F\State{S}} \ar[d, shift left, leftarrow,
      "\lens{\overline{F}\update{S}}{F\expose{S}}"] \ar[d, shift right]\\
      \lens{\overline{F}A^-}{FA^+} \ar[r, shift right, equals] \ar[r,
      shift left, equals] & \lens{\overline{F}A^-}{FA^+}
    \end{tikzcd}
\]

\item (\cref{def.lax_doubly_indexed_functor}:
  \cref{def:lax.doubly.indexed.functor.2}) Since the doubly indexed functor is
  \emph{taut}, for any lens $\lens{j^{\sharp}}{j} : \lens{A^{-}}{A^+} \fromto \lens{B^{-}}{B^+}$ have a commuting square
  \begin{equation}
  \begin{tikzcd}
    \Cat{Sys}_{\dd_1}\littlelens{A^{-}}{A^+} \ar[d, "{\Cat{Sys}_{\dd_1}\littlelens{j^{\sharp}}{j}}"'] \ar[r, "{\sigma\littlelens{\phi}{\id}^{\littlelens{A^-}{A^+}}}"] & \Cat{Sys}_{\dd_2}\littlelens{\overline{F}A^{-}}{FA^+}  \ar[d, "{\Cat{Sys}_{\dd_2}\littlelens{\overline{F}j^{\sharp}}{Fj}}"]\\
    \Cat{Sys}_{\dd_1}\littlelens{B^{-}}{B^+} \ar[r, "{\sigma\littlelens{\phi}{\id}^{\littlelens{B^-}{B^+}}}"'] & \Cat{Sys}_{\dd_2}\littlelens{\overline{F}B^{-}}{FB^+} 
  \end{tikzcd}
  \end{equation}
  
This tells us that changing doctrine and then wiring together systems gives the
same result as wiring together the systems first and then changing doctrine.

\item (\cref{def.lax_doubly_indexed_functor}:
  \cref{def:lax.doubly.indexed.functor.3}) For a $\dd_1$-chart $\lens{f_{\flat}}{f} : \lens{A^-}{A^+} \tto \lens{B^-}{B^+}$,
  we have the square in $\Cat{Cat}$
  \begin{equation}\label{eqn:change.of.doctrine.prof.diag}
\begin{tikzcd}
  \Cat{Sys}_{\dd_1}\littlelens{A^-}{A^+} \ar[rr, tick,
  "{\Cat{Sys}\littlelens{f_{\flat}}{f}}"] \ar[dd, "{
    \sigma\littlelens{\phi}{\id}}^{\littlelens{A^-}{A^+}}"'] & & \Cat{Sys}_{\dd_1}\littlelens{B^-}{B^+}
  \ar[dd, "{ \sigma\littlelens{\phi}{\id}^{\littlelens{B^-}{B^+}}}"] \\
& \sigma\littlelens{\phi}{\id}^{\littlelens{f_{\flat}}{f}} & \\
\Cat{Sys}_{\dd_2}\littlelens{\overline{F}A^-}{FA^+} \ar[rr, tick,
"{\Cat{Sys}\littlelens{\overline{F}f_{\flat}}{Ff}}"'] & & \Cat{Sys}_{\dd_2}\littlelens{\overline{F}B^-}{FB^+}
\end{tikzcd}
\end{equation}
given by sending a $\lens{f_{\flat}}{f}$-behavior to the composite:
\[
    \begin{tikzcd}
      \lens{T_1\State{T}}{\State{T}} \ar[r, shift left, "\lens{T_1\psi}{\psi}"] \ar[r, shift right] \ar[d, shift right,
      "\lens{\update{T}}{\expose{T}}"'] \ar[d, shift left, leftarrow] &
      \lens{T_1\State{S}}{\State{S}} \ar[d, shift left, leftarrow,
      "\lens{\update{S}}{\expose{S}}"] \ar[d, shift right]\\
      \lens{A^-}{A^+} \ar[r, shift right, "\lens{f_{\flat}}{f}"'] \ar[r,
      shift left] & \lens{B^-}{B^+}
    \end{tikzcd}
    \quad\mapsto\quad
    \begin{tikzcd}
      \lens{T_2F\State{T}}{F\State{T}} \ar[r, shift left,
      "\lens{T_2F\psi}{F\psi}"] \ar[r, shift right] \ar[d, shift right, "\lens{\phi}{\id}"'] \ar[d, shift left, leftarrow] &
      \lens{T_2F\State{S}}{F\State{S}} \ar[d, shift left, leftarrow,
      "\lens{\phi}{\id}"] \ar[d, shift right]\\
      \lens{\overline{F}T_1\State{T}}{F\State{T}} \ar[r, shift left, "\lens{\overline{F}T_1\psi}{F\psi}"] \ar[r, shift right] \ar[d, shift right,
      "\lens{\overline{F}\update{T}}{F\expose{T}}"'] \ar[d, shift left, leftarrow] &
      \lens{\overline{F}T_1\State{S}}{F\State{S}} \ar[d, shift left, leftarrow,
      "\lens{\overline{F}\update{S}}{F\expose{S}}"] \ar[d, shift right]\\
      \lens{\overline{F}A^-}{FA^+} \ar[r, shift right, "\lens{\overline{F}f_{\flat}}{Ff}"'] \ar[r,
      shift left] & \lens{\overline{F}B^-}{FB^+}
    \end{tikzcd}
\]

In other words, changes of doctrine preserve behavior in the sense that if
$\psi$ is a $\lens{f_{\flat}}{f}$-behavior then $F\psi$ is a $\len{\overline{F}f_{\flat}}{f}$-behavior.
\end{itemize}

\begin{example}
For Euler approximatation 
  \[
\mathcal{E}_{\varepsilon} : \eucdiff_{|\textbf{Aff}} \to \determ_{\Cat{Euc}},
  \]
  we get a doubly indexed functor
  $$(\littlelens{\id}{\id}, \sigma\littlelens{\phi}{\id}) : \Sys(\eucdiff_{|\textbf{Aff}}) \to \determ_{\Cat{Euc}}$$
  by the functoriality of $\Cat{Sys}$, where $\phi : \rr^n \times \rr^m \to
  \rr^n$ is $\phi(p, v) = p + \varepsilon \cdot v$. Let's see it means for this
  to be a doubly indexed functor.

  First, we have a functor
  \[
    \sigma\littlelens{\phi}{\id}^{\littlelens{A^-}{A^+}} :
    \Cat{Sys}_{\eucdiff_{\textbf{Aff}}}\littlelens{A^-}{A^+} \to \Cat{Sys}_{\determ_{\Cat{Euc}}}\littlelens{A^-}{A^+}.
  \]
  This says that the Euler method preserves simulations. Second, we have a
  square like \cref{eqn:change.of.doctrine.prof.diag} which says that the Euler
  method preserves behaviors. However, we have to be careful here; the behaviors
  $\left( \varphi, \lens{f_{\flat}}{f} \right)$ which are preserved must have
$\varphi$ and $\lens{f_{\flat}}{f}$ in the appropriate double category of arenas, and here
  we had to restrict to those for which $\varphi$ and $f$ are affine maps so
  that \cref{eqn:euler.affine.map} can hold. In other words, we see that the
  Euler method will preserve any \emph{affine} behaviors of differential systems.

  Most solutions to a system of differential equations --- most trajectories ---
  are not affine. This is to say that
  there aren't many behaviors of shape $\Sys{Time}$ (from
  \cref{ex:diff.doc.time.system}). There is, however, an important class of
  affine solutions: steady states. These are the behaviors of shape $\Sys{Fix}$
  from \cref{ex:diff.doctrine.fix}. So, in particular, we see that the Euler
  method preserves steady states.

  That the Euler method preserves steady states is of course evident from the
  formula: if $u(s, i) = 0$, then $\mathcal{E}_{\varepsilon}u(s, i) = s +
  \varepsilon \cdot u(s, i) = s$. But we deduced this fact from our general
  definition of change of doctrine. This sort of analysis can tell us precisely
  which sorts of behaviors are preserved even in situtations where it may not be
  so obvious from looking at a defining formula.

  The fact that $\Cat{Sys}(\mathcal{E}_{\varepsilon})$ is a doubly indexed
  functor gives us a litany of compositionality checks. In particular, the
  commuting square (\cref{def.lax_doubly_indexed_functor}:
  \cref{def:lax.doubly.indexed.functor.2}) shows
  that if we are to wire together a family of differential systems and then
  approximate the result with the Euler method, we could have approximated each
  one and then wired together the result with the same wiring pattern.


  
%writehere
\end{example}

\section{Summary and Further Reading}

In this chapter, we organized the systems in a doctrine into doubly indexed
categories. While all the action takes place within the double category of
arenas, the doubly indexed category of systems separates the systems from their
interfaces and the behaviors from their charts. This let's us describe the
various sorts of composition --- of systems and of behaviors --- and their
relationships. We then saw how this construction varied as we changed doctrine.

\end{document}
