\documentclass[DynamicalBook]{subfiles}
\begin{document}
%


\setcounter{chapter}{3}%Just finished 3.


%------------ Chapter ------------%
\chapter{Doubly Indexed Categories of Systems}

\section{Introduction}

In the last chapter, we saw a general formulation of the notion of behavior of
system and precise definition of the notion of dynamical system doctrine. Let's
recall the definition of dynamical system doctrine.

\begin{definition}\label{def.doctrine}
A \emph{dynamical system doctrine} consists of an indexed category $\cat{A} :
\cat{C}\op \to \Cat{Cat}$ together with a section $T$.
\end{definition}

This concise definition packs a big punch. Describing a dynamical system
doctrine amounts to answering the informal questions about what it means to be a
system:
\begin{informal}
  A \emph{doctrine} of dynamical systems is a particular way to answer the following
  questions about what it means to be a dynamical system:
  \begin{enumerate}
  \item What does it mean to be a state?
  \item How should the output vary with the state --- discretely,
    continuously, linearly?
  \item Can the kinds of input a
    system takes in depend on what it's putting out, and how do they depend on it?
  \item What sorts of changes are possible in a given state?
  \item What does it mean for states to change. 
  \item How should the way the state changes vary with the input?
  \end{enumerate}
\end{informal}

Constructing a doctrine is no small thing. But once we have a doctrine, we have
may work in its double category of arenas to quickly derive a few
compositionality results about systems.

\begin{definition}\label{def.double_cat_of_arenas_general}
  Let $\cat{A} : \cat{C}\op \to \Cat{Cat}$ be an indexed category. Then the
  category of $\cat{A}$-arenas is defined to be the Grothendieck double
  construction of $\cat{A}$:
$$\Cat{Arena}_{\cat{A}} := \sqiint^{C \in \cat{C}} \cat{A}(C).$$

  Note that the horizontal category of $\Cat{Arena}_{\cat{A}}$ is the category
  $\Cat{Chart}_{\cat{A}}$ of $\cat{A}$-charts (\cref{def.chart_general}), and the vertical category of
  $\Cat{Arena}_{\cat{A}}$ is the category $\Cat{Lens}_{\cat{A}}$ of
  $\cat{A}$-lenses (\cref{def.lens_general}).
\end{definition}

We are now in peak category theory
territory: the statements of our propositions are far longer than their proofs,
which amount to trivial calculations in the double category of arenas. As in
much of categorical work, the difficulty is in understanding what to propose;
once that work is done, the proof flows smoothly from the definitions.

Let's see what composition of squares in the double category of
arenas means for systems. Horizontal composition is familiar because it's what
lets us compose behaviors:
\[
  \begin{tikzcd}
    \lens{T\State{T}}{\State{T}} \ar[r, shift left, "\lens{T\phi}{\phi}"] \ar[r, shift right] \ar[d, shift right,
    "\lens{\update{T}}{\expose{T}}"'] \ar[d, shift left, leftarrow] &
    \lens{T\State{S}}{\State{S}} \ar[r, shift left, "\lens{T\psi}{\psi}"] \ar[r, shift right] \ar[d, shift left, leftarrow,
    "\lens{\update{S}}{\expose{S}}"] \ar[d, shift right] &
    \lens{T\State{U}}{\State{U}}\ar[d, shift left, leftarrow,
    "\lens{\update{U}}{\expose{U}}"] \ar[d, shift right] \\
    \lens{\In{T}}{\Out{T}} \ar[r, shift right, "\lens{f_{\flat}}{f}"'] \ar[r,
    shift left] & \lens{\In{S}}{\Out{S}} \ar[r, shift right,
    "\lens{g_{\flat}}{g}"'] \ar[r, shift left]& \lens{\In{U}}{\Out{U}}
  \end{tikzcd} \xequals{\quad}
  \begin{tikzcd}
    \lens{T\State{T}}{\State{T}} \ar[r, shift left, "\lens{T(\psi\phi)}{\psi\phi}"] \ar[r, shift right] \ar[d, shift right,
    "\lens{\update{T}}{\expose{T}}"'] \ar[d, shift left, leftarrow] &
    \lens{T\State{U}}{\State{U}} \ar[d, shift left, leftarrow,
    "\lens{\update{U}}{\expose{U}}"] \ar[d, shift right]\\
    \lens{\In{T}}{\Out{T}} \ar[r, shift right, "\lens{f_{\flat}}{f} \then \lens{g_{\flat}}{g}"']
    \ar[r, shift left] & \lens{\In{U}}{\Out{U}}
  \end{tikzcd}
\]
So, we have a category of systems and behaviors in any doctrine, just as we
defined in the deterministic doctrine.

On the other hand, vertical composition tells us something else interesting: if
you get a chart $\lens{g_{\flat}}{g}$ by wiring together a chart
$\lens{f_{\flat}}{f}$, then a behavior $\phi$ with
chart $\lens{f_{\flat}}{f}$ induces a behavior with chart $\lens{g_{\flat}}{g}$
on the wired together systems.
\[
  \begin{tikzcd}
    \lens{T\State{T}}{\State{T}} \ar[r, shift left, "\lens{T\phi}{\phi}"] \ar[r, shift right] \ar[d, shift right,
    "\lens{\update{T}}{\expose{T}}"'] \ar[d, shift left, leftarrow] &
    \lens{T\State{S}}{\State{S}} \ar[d, shift left, leftarrow,
    "\lens{\update{S}}{\expose{S}}"] \ar[d, shift right]\\
    \lens{\In{T}}{\Out{T}} \ar[d, shift right, "\lens{j^{\sharp}}{j}"'] \ar[d, shift left,
        leftarrow] \ar[r, shift right, "\lens{f_{\flat}}{f}"']
    \ar[r, shift left] & \lens{\In{S}}{\Out{S}} \ar[d, shift left, leftarrow,
        "\lens{k^{\sharp}}{k}"] \ar[d, shift right]\\
    \lens{I}{O} \ar[r, shift right, "\lens{g_{\flat}}{g}"']
    \ar[r, shift left] & \lens{I'}{O'} \\
  \end{tikzcd} \xequals{\quad}
  \begin{tikzcd}
    \lens{T\State{T}}{\State{T}} \ar[r, shift left, "\lens{T\phi}{\psi\phi}"] \ar[r, shift right] \ar[d, shift right,
    "\lens{f^{\sharp}}{f} \circ \lens{\update{T}}{\expose{T}}"'] \ar[d, shift left, leftarrow] &
    \lens{T\State{S}}{\State{S}} \ar[d, shift left, leftarrow,
    "\lens{k^{\sharp}}{k} \circ \lens{\update{S}}{\expose{S}}"] \ar[d, shift right]\\
    \lens{I}{O} \ar[r, shift right, "\lens{g_{\flat}}{g}"']
    \ar[r, shift left] & \lens{I}{O'}
  \end{tikzcd}
\]

The interchange law of the double category of arenas tells us precisely that
these two sorts of composition of behaviors --- composition as maps and wiring --- \emph{commute}. That is, we can
compose two behaviors and then wire them together, or we can wire each
together and then compose them; the end result is the same.

\begin{example}\label{ex.understanding_squares_in_double_cat_of_arenas2}
  Continuing from \cref{ex.understanding_squares_in_double_cat_of_arenas},
  suppose that we have a $\lens{b^-}{b^+}$-steady state $s$ in a system $\Sys{S}$:
  \begin{equation}\label{eqn.understanding_squares_in_double_cat_of_arenas2}
    \begin{tikzcd}
      \lens{\ord{1}}{\ord{1}} \ar[r, shift left, "\lens{s}{s}"] \ar[r, shift
      right] \ar[d, shift right, equals] \ar[d, shift left, equals] &
      \lens{\State{S}}{\State{S}} \ar[d, shift left, leftarrow,
      "\lens{\update{S}}{\expose{S}}"] \ar[d, shift right]\\
      \lens{\ord{1}}{\ord{1}} \ar[r, shift right, "\lens{b^+}{b^-}"'] \ar[r,
      shift left] & \lens{B^-}{B^+}
    \end{tikzcd}
  \end{equation}
  We can see that $s$ is a $\lens{d^-}{d^+}$-steady state of the wired system by
  vertically composing the square in
  \cref{eqn.understanding_squares_in_double_cat_of_arenas2} with the square in
  \cref{eqn.understanding_squares_in_double_cat_of_arenas}. This basic fact
  underlies our arguments in the upcoming \cref{sec.steady_states_matrix_arithmetic}. 

We'll return to this idea in \cref{ex.prof_square_from_arena_square}.
\end{example}

While our results are most smoothly proven in the double category of arenas,
this double category does not capture the way we think of systems and their
behaviors. To think of a behavior, we must first think of its chart; we solve a
differential equation in terms of its parameters, and to get a specific solution
we must first choose specific parameters. Working in the double category of
arenas means treating the chart $\lens{f_{\flat}}{f}$ and the underlying map
$\phi$ of a behavior on equal footing, but we would instead like to say that
$\phi$ is a behavior for the chart $\lens{f_{\flat}}{f}$. 

We would also like to think of the wiring together of systems along a lens
$\lens{w^{\sharp}}{w}$ as an operation performed on systems, and then inquire
into the relationship of this wiring operation with the (horizontal) composition
of behaviors.

What we need is to separate the \emph{interface} of a system from the system
itself. Charts and lenses are best understood as ways of relating interfaces. It just
so happens that systems and their behaviors can also be expressed as certain
sorts of lenses and charts, which drastically facilitates our working with them.
But there is some sense in which this is not essential; the main point is that
for each interface $\lens{I}{O}$ we have a notion of system with interface
$\lens{I}{O}$, for each lens $\lens{w^{\sharp}}{w} : \lens{I}{O} \fromto
\lens{I'}{O'}$ a way of wiring $\lens{I}{O}$-systems into $\lens{I'}{O'}$
systems, and for each chart $\lens{f_{\flat}}{f} : \lens{I}{O} \tto
\lens{I'}{O'}$ a notion of behavior for this chart. It is very convenient that
we can describe wiring and composition of behaviors in the same terms as charts
and lenses, but we shouldn't think that they are the same thing.

In this chapter, we will define the appropriate abstract algebra of systems and
their two sorts of composition keeping in mind the separation between interfaces
and systems. We call this abstract algebra a \emph{doubly indexed category},
since it is a sort of double categorical generalization of an indexed category.
We'll see the definition of this notion in
\cref{sec.indexed_double_category_of_systems}.

The main reason for introducing doubly indexed categories is that doubly indexed
\emph{functors} will organize the variety of compositionality results we will
prove in this and the next chapter. 

We will show how doubly
indexed categories of systems can be constructed systematically from their
doctrine. In particular, we will show that there is a
\emph{2-functor} \jaz{I am starting to think that I should put the
  2-functoriality of this into an appendix so that I don't have to introduce
  2-categories and 2-functors here.}
$$\textbf{Sys} : \textbf{Doctrine} \to \textbf{DblIx}$$
sending a doctrine to the doubly indexed category of systems in it. The
2-functoriality of this construction will let us explore \emph{change of
  doctrine}. Changes of doctrine will include the changes in flavors of
non-determinism given by commutative monad morphisms described in
\cref{chapter.2}, and various sorts of \emph{approximation} of differential
systems by discrete ones. 




\section{Composing behaviors in general}\label{sec.behaviors_general}

Before we get to this abstract defintion, we will take our time exploring the
sorts of compositionality results one may prove quickly by working in the double
category of arenas.

Recall the categories $\Cat{Sys}\lens{I}{O}$ of systems with the interface
$\lens{I}{O}$ from
\cref{def.cat_of_systems_discrete}. One thing that vertical composition in the
double category of arenas shows us is that wiring together systems is functorial
with respect to simulations --- that is, behaviors that don't change the
interface.

We repeat the definition of $\Cat{Sys}\lens{I}{O}$ for an arbitrary doctrine.
  \begin{definition}\label{def.cat_of_systems}
  Let $\mathbb{D} = (\cat{A}, T)$ be a dynamical system doctrine. For a
    $\cat{A}$-arena $\lens{I}{O}$, the category $\Cat{Sys}\lens{A}{C}$ of
    $\mathbb{D}$-systems with interface $\lens{I}{O}$ is defined by:
\begin{itemize}
  \item Its objects are $\cat{A}$-lenses $\lens{\update{S}}{\expose{S}} :
    \lens{T\State{S}}{\State{S}} \fromto \lens{I}{O}$, which we can call
    \emph{systems} in this general setting.
  \item Its maps are \emph{simulations}, the behaviors which have identity
    chart. That is, the maps are the squares 
\[
    \begin{tikzcd}
      \lens{T\State{T}}{\State{T}} \ar[r, shift left, "\lens{T\phi}{\phi}"] \ar[r, shift right] \ar[d, shift right,
      "\lens{\update{T}}{\expose{T}}"'] \ar[d, shift left, leftarrow] &
      \lens{T\State{S}}{\State{S}} \ar[d, shift left, leftarrow,
      "\lens{\update{S}}{\expose{S}}"] \ar[d, shift right]\\
      \lens{I}{O} \ar[r, shift right, equals] \ar[r,
      shift left, equals] & \lens{I}{O}
    \end{tikzcd}
\]
\item Composition is given by horizontal composition in the double category
  $\Cat{Arena}_{\cat{A}}$ of $\cat{A}$-arenas.
\end{itemize}
  \end{definition}

Now, thanks to the double category of arenas, we can show that every lens
$\lens{f^{\sharp}}{f} : \lens{I}{O} \fromto \lens{I'}{O'}$ gives a functor 
  $$\Cat{Sys}\lens{f^\sharp}{f} : \Cat{Sys}\lens{I}{O} \to
  \Cat{Sys}\lens{I}{O}.$$
We can see this functor as the operation of wiring together our
$\lens{I}{O}$-systems along the lens $\lens{f^{\sharp}}{f}$ to get
$\lens{I'}{O'}$-systems. The functoriality of this operation say that wiring
preserves simulations --- if systems $\Sys{S_i}$ simulate $\Sys{T_i}$ by $\phi_i$,
then the wired together systems $\Sys{S}$ simulate $\Sys{T}$ by $\phi = \prod_i
\phi_i$. 

\begin{proposition}\label{prop.lens_comp_functor_discrete}
  For a lens $\lens{f^{\sharp}}{f} : \lens{I}{O} \leftrightarrows
  \lens{I'}{O'}$, we get a functor
  $$\Cat{Sys}\lens{f^\sharp}{f} : \Cat{Sys}\lens{I}{O} \to
  \Cat{Sys}\lens{I}{O}$$
  Given by composing with $\lens{f^\sharp}{f}$:
  \begin{itemize}
    \item For a system $\Sys{S} = \lens{\update{S}}{\expose{S}} :
      \lens{T\State{S}}{\State{S}}\fromto \lens{I}{O}$,
      $$\Cat{Sys}\lens{f^{\sharp}}{f}(\Sys{S}) = \lens{f^{\sharp}}{f} \circ \lens{\update{S}}{\expose{S}}.$$
    \item For a behavior, $\Cat{Sys}\lens{f^{\sharp}}{f}$ acts in the following way:
      \[
  \begin{tikzcd}
    \lens{T\State{T}}{\State{T}} \ar[r, shift left, "\lens{T\phi}{\phi}"] \ar[r, shift right] \ar[d, shift right,
    "\lens{\update{T}}{\expose{T}}"'] \ar[d, shift left, leftarrow] &
    \lens{T\State{S}}{\State{S}} \ar[d, shift left, leftarrow,
    "\lens{\update{S}}{\expose{S}}"] \ar[d, shift right]\\
    \lens{I}{O} \ar[r, shift right, equals]
    \ar[r, shift left, equals] & \lens{I}{O'}
  \end{tikzcd} \quad \mapsto \quad
  \begin{tikzcd}
    \lens{T\State{T}}{\State{T}} \ar[r, shift left, "\lens{T\phi}{\phi}"] \ar[r, shift right] \ar[d, shift right,
    "\lens{\update{T}}{\expose{T}}"'] \ar[d, shift left, leftarrow] &
    \lens{T\State{S}}{\State{S}} \ar[d, shift left, leftarrow,
    "\lens{\update{S}}{\expose{S}}"] \ar[d, shift right]\\
    \lens{\In{T}}{\Out{T}} \ar[d, shift right, "\lens{f^{\sharp}}{f}"'] \ar[d, shift left,
        leftarrow] \ar[r, shift right, equals]
    \ar[r, shift left, equals] & \lens{\In{S}}{\Out{S}} \ar[d, shift left, leftarrow,
        "\lens{f^{\sharp}}{f}"] \ar[d, shift right]\\
    \lens{I}{O} \ar[r, shift right, equals]
    \ar[r, shift left, equals] & \lens{I'}{O'} \\
  \end{tikzcd} 
      \]
  \end{itemize}
\end{proposition}
\begin{proof}
  The functoriality of this construction can be seen immediately from the
  interchange law of the double category:
  \begin{align*}
  \frac{\left.\littlelens{T\phi}{\phi} \middle| \littlelens{T\psi}{\psi} \right. }{\littlelens{f^{\sharp}}{f}} &= \frac{\left.\littlelens{T\phi }{\phi} \middle| \littlelens{T\psi
    }{\psi} \right. }{\left. \littlelens{f^{\sharp}}{f} \middle| \littlelens{f^{\sharp}}{f} \right.} &\mbox{by the horizontal identity law,}\\
    &= \left.
      \frac{\littlelens{T\phi }{\phi}}{\littlelens{f^{\sharp}}{f}}
      \middle|\frac{\littlelens{T\psi}{\psi}}{\littlelens{f^{\sharp}}{f}} \right. &\mbox{by the interchange law.}
  \end{align*}
  Identities are clearly preserved, since the underlying morphism $\phi :
  \State{T} \to \State{S}$ is not changed. 
\end{proof}


  The notion of profunctor gives us a nice way to understand the relationship
  between a behavior $\phi : \Sys{T} \to \Sys{S}$ and its chart
  $\lens{f_{\flat}}{f} : \lens{I}{O} \tto \lens{I'}{O'}$. When we are using
  behaviors, we usually have the chart $\lens{f_{\flat}}{f}$ in mind first, and
  then look for behaviors with this chart. For example, when finding
  trajectories, we first set the parameters for our system and then solve it. We
  can use profunctors to formalize this relationship.


\begin{proposition}\label{prop.profunctor_from_chart}
  Given a chart $\lens{f_{\flat}}{f} : \lens{I}{O} \tto
  \lens{I'}{O'}$, we get a profunctor
  $$\Cat{Sys}\lens{f_{\flat}}{f} : \Cat{Sys}\lens{I}{O} \tickar
  \Cat{Sys}\lens{I'}{O'}$$

  Defined by:
\begin{align*}
  \Cat{Sys}\lens{f_{\flat}}{f}(\Sys{T}, \Sys{S}) &= \left\{ \phi : \State{T} \to
                                                   \State{S}\, \middle| \mbox{ $\phi$ is a behavior with chart $\lens{f_{\flat}}{f}$} \right\}\\
  &= \left\{  
    {
    \begin{tikzcd}[ampersand replacement = \&]
      \lens{\State{T}}{\State{T}} \ar[r, dashed, shift left, "\lens{\phi \circ
        \pi_2}{\phi}"] \ar[r, dashed, shift right] \ar[d, shift right,
      "\lens{\update{T}}{\expose{T}}"'] \ar[d, shift left, leftarrow] \&
      \lens{\State{S}}{\State{S}} \ar[d, shift left, leftarrow,
      "\lens{\update{S}}{\expose{S}}"] \ar[d, shift right]\\
      \lens{\In{T}}{\Out{T}} \ar[r, shift right, "\lens{f^{\sharp}}{f}"'] \ar[r,
      shift left] \& \lens{\In{S}}{\Out{S}}
    \end{tikzcd}
                    }
                    \right\}
\end{align*}

The action of the profunctor $\Cat{Sys}\lens{f_{\flat}}{f}$ on simulations in the
categories $\Cat{Sys}\lens{I}{O}$ and $\Cat{Sys}\lens{I'}{O'}$ is given by
composition on the left and right. That is, for simulations $\phi : \Sys{T'} \to
\Sys{T}$ and $\psi : \Sys{S} \to \Sys{S'}$ and $\lens{f_{\flat}}{f}$-behavior
$\beta \in \Cat{Sys}\lens{f_{\flat}}{f}(\Sys{T}, \Sys{S})$, we define 
\begin{equation}\label{eqn.profunctor_from_chart}
\phi \cdot \beta \cdot \psi := \phi \mid \beta \mid \psi.
\end{equation}
\end{proposition}

\begin{exercise}\label{ex.profunctor_from_chart}
 Prove \cref{prop.profunctor_from_chart}. That is, show that the action defined
 in \cref{eqn.profunctor_from_chart} is functorial, giving a functor 
$$\Cat{Sys}\littlelens{I}{O}\op \times \Cat{Sys}\littlelens{I'}{O'} \to \smset.$$
(Hint: use the double categorical notation. It will be much more concise.)
\end{exercise}


  With a little work in the double category of arenas, we can give a very useful
  example of a square in the double category of profunctors. Consider this
  square in the double category of arenas:
\[ \alpha = 
  \begin{tikzcd}
    \lens{I_1}{O_1} \ar[d, shift right, "\lens{j_{\flat}}{j}"'] \ar[d, shift left,
        leftarrow] \ar[r, shift right]
    \ar[r, shift left, "\lens{f^{\sharp}}{f}"] & \lens{I_2}{O_2} \ar[d, shift left, leftarrow,
        "\lens{k^{\sharp}}{k}"] \ar[d, shift right]\\
    \lens{I_3}{O_3} \ar[r, shift right, "\lens{g_{\flat}}{g}"']
    \ar[r, shift left] & \lens{I_4}{O_4} \\
  \end{tikzcd} 
\]
As we saw in \cref{prop.lens_comp_functor_discrete}, we get functors
$\Cat{Sys}\lens{j^{\sharp}}{j} : \Cat{Sys}\lens{I_1}{O_1} \to
\Cat{Sys}\lens{I_3}{O_3}$ and $\Cat{Sys}\lens{k^{\sharp}}{k} :
\Cat{Sys}\lens{I_2}{O_2} \to \Cat{Sys}\lens{I_4}{O_4}$ given by composing with
these lenses. We also saw in \cref{prop.profunctor_from_chart} that we get
profunctors $\Cat{Sys}\lens{f_{\flat}}{f} : \Cat{Sys}\lens{I_1}{O_1} \tickar
\Cat{Sys}\lens{I_2}{O_2}$ and $\Cat{Sys}\lens{g_{\flat}}{g} :
\Cat{Sys}\lens{I_3}{O_3} \tickar \Cat{Sys}\lens{I_4}{O_4}$ from these charts. We
can also get a square of profunctors 
from the square $\alpha$ in the double category of arenas:
\[
\begin{tikzcd}
  \Cat{Sys}\littlelens{I_1}{O_1} \ar[rr, tick,
  "{\Cat{Sys}\lens{f_{\flat}}{f}}"]\ar[dd,
  "{\Cat{Sys}\littlelens{j^{\sharp}}{j}}"'] & &
  \Cat{Sys}\littlelens{I_2}{O_2} \ar[dd, "{\Cat{Sys}\littlelens{k^{\sharp}}{k}}"]\\
 &\Cat{Sys}(\alpha) & \\
\Cat{Sys}\littlelens{I_3}{O_3} \ar[rr, tick,
"{\Cat{Sys}\lens{g_{\flat}}{g}}"']& & \Cat{Sys}\littlelens{I_4}{O_4}
\end{tikzcd}
\]
That is, a natural transformation of the following signature:
\[
\Cat{Sys}(\alpha) : \Cat{Sys}\lens{f_{\flat}}{f} \to
\Cat{Sys}\lens{g_{\flat}}{g}\left( \Cat{Sys}\littlelens{j^{\sharp}}{j}, \Cat{Sys}\littlelens{k^{\sharp}}{k} \right).\]

To define the natural transformation $\Cat{Sys}(\alpha)$, we need to say what it
does to an element $\phi$ of $\Cat{Sys}\lens{f_{\flat}}{f}\left(\Sys{T},
  \Sys{S}\right)$. Recall that the elements of this profunctor are behaviors with
  chart $\lens{f_{\flat}}{f}$, so really $\phi$ is a square
\[
  \phi =
    \begin{tikzcd}
      \lens{T\State{T}}{\State{T}} \ar[r, shift left, "\lens{T\phi}{\phi}"] \ar[r, shift right] \ar[d, shift right,
      "\lens{\update{T}}{\expose{T}}"'] \ar[d, shift left, leftarrow] &
      \lens{T\State{S}}{\State{S}} \ar[d, shift left, leftarrow,
      "\lens{\update{S}}{\expose{S}}"] \ar[d, shift right]\\
      \lens{I_1}{O_1} \ar[r, shift right, "\lens{f^{\sharp}}{f}"'] \ar[r,
      shift left] & \lens{I_2}{O_2}
    \end{tikzcd}
\]
in the double category of arenas. Therefore, we can define
$\Cat{Sys}(\alpha)(\phi)$ to be the vertical composite:
\[
  \begin{tikzcd}
    \lens{T\State{T}}{\State{T}} \ar[r, shift left, "\lens{T\phi}{\phi}"] \ar[r, shift right] \ar[d, shift right,
    "\lens{\update{T}}{\expose{T}}"'] \ar[d, shift left, leftarrow] &
    \lens{T\State{S}}{\State{S}} \ar[d, shift left, leftarrow,
    "\lens{\update{S}}{\expose{S}}"] \ar[d, shift right]\\
    \lens{I_1}{O_1} \ar[d, shift right, "\lens{j^{\sharp}}{j}"'] \ar[d, shift left,
        leftarrow] \ar[r, shift right, "\lens{f^{\sharp}}{f}"']
    \ar[r, shift left] & \lens{I_2}{O_2} \ar[d, shift left, leftarrow,
        "\lens{k^{\sharp}}{k}"] \ar[d, shift right]\\
    \lens{I_3}{O_3} \ar[r, shift right, "\lens{g^{\sharp}}{g}"']
    \ar[r, shift left] & \lens{I_4}{O_4} \\
  \end{tikzcd}
\]
Or, a little more concisely in double category notation:
$$\Cat{Sys}(\alpha)(\phi) = \frac{\phi}{\alpha}.$$

We record this observation in a proposition.
\begin{proposition}\label{prop.prof_square_from_arena_square}
  Given a square
\[ \alpha = 
  \begin{tikzcd}
    \lens{I_1}{O_1} \ar[d, shift right, "\lens{j_{\flat}}{j}"'] \ar[d, shift left,
        leftarrow] \ar[r, shift right]
    \ar[r, shift left, "\lens{f^{\sharp}}{f}"] & \lens{I_2}{O_2} \ar[d, shift left, leftarrow,
        "\lens{k^{\sharp}}{k}"] \ar[d, shift right]\\
    \lens{I_3}{O_3} \ar[r, shift right, "\lens{g_{\flat}}{g}"']
    \ar[r, shift left] & \lens{I_4}{O_4} \\
  \end{tikzcd} 
\]
in the double category of arenas, we get a square

\[
\begin{tikzcd}
  \Cat{Sys}\littlelens{I_1}{O_1} \ar[rr, tick,
  "{\Cat{Sys}\lens{f_{\flat}}{f}}"]\ar[dd,
  "{\Cat{Sys}\littlelens{j^{\sharp}}{j}}"'] & &
  \Cat{Sys}\littlelens{I_2}{O_2} \ar[dd, "{\Cat{Sys}\littlelens{k^{\sharp}}{k}}"]\\
 &\Cat{Sys}(\alpha) & \\
\Cat{Sys}\littlelens{I_3}{O_3} \ar[rr, tick,
"{\Cat{Sys}\lens{g_{\flat}}{g}}"']& & \Cat{Sys}\littlelens{I_4}{O_4}
\end{tikzcd}
\]
in the double category of categories, functors, and profunctors given by 
$$\Cat{Sys}(\alpha)(\phi) = \frac{\phi}{\alpha}.$$

\end{proposition}

The naturality of this transformation follows from the double category laws. We
leave the particulars as an exercise.
\begin{exercise}\label{ex.prof_square_from_arena_square_naturality}
  Prove that the family of functions 
  \[
\Cat{Sys}(\alpha) : \Cat{Sys}\lens{f_{\flat}}{f} \to
\Cat{Sys}\lens{g_{\flat}}{g}\left( \Cat{Sys}\littlelens{j^{\sharp}}{j}, \Cat{Sys}\littlelens{k^{\sharp}}{k} \right)
\]
defined in \cref{ex.prof_square_from_arena_square} is a natural transformation.
(Hint: use the double category notation, it will be much more concise.)
\end{exercise}



%---- Section ----%
\section{Arranging categories along two kinds of composition: Doubly indexed categories}
\label{sec.indexed_double_category_of_systems}

While we described a category of systems and behaviors in
\cref{prop.category_of_systems_discrete}, we haven't been thinking of systems in
quite this way. We have been organizing our systems a bit more particularly than
just throwing them into one large category. We've made the following observations:
\begin{itemize}
  \item Each system has an interface, and many different systems can have the
    same interface. From this observation, we defined the categories
    $\Cat{Sys}\lens{I}{O}$ of systems with the interface $\lens{I}{O}$ in \cref{def.cat_of_systems_discrete}.
  \item Every wiring diagram, or more generally lens, gives us an operation that
    changes the interface of a system by wiring things together. We formalized
    this observation into a functor $\lens{w^{\sharp}}{w} : \Cat{Sys}\lens{I}{O}
    \to \Cat{Sys}\lens{I'}{O'}$ in \cref{prop.lens_comp_functor_discrete}.
  \item To describe the behavior of a system, first we have to chart out how it
    will look on its interface. We formalized this observation by giving a
    profunctor $\Cat{Sys}\lens{f_{\flat}}{f} : \Cat{Sys}\lens{I}{O} \tickar
    \Cat{Sys}\lens{I'}{O'}$ for each chart in \cref{prop.profunctor_from_chart}.
  \item If we wire together a chart for one interface into a chart for the wired
    interface, then every behavior for that chart gives rise to a behavior for
    the wired together chart. We formalized this observation as a morphism of
    profunctors 
\[
\Cat{Sys}(\alpha) : \Cat{Sys}\lens{f_{\flat}}{f} \to
\Cat{Sys}\lens{g_{\flat}}{g}\left( \Cat{Sys}\littlelens{j^{\sharp}}{j}, \Cat{Sys}\littlelens{k^{\sharp}}{k} \right)
\]
in \cref{ex.prof_square_from_arena_square}.
\end{itemize}

Now comes the time to organize all these observations. In this section, we will
see that collectively, these observations are telling us that there is an
\emph{doubly indexed category} of dynamical systems. We will also see that
matrices of sets give rise to a doubly indexed category which we will call the
doubly indexed category of vectors of sets.

\begin{definition}\label{def.doubly_indexed_category}
A \emph{doubly indexed category} $\cat{A} : \cat{D} \to \Cat{Cat}$ consists of
the following:\footnote{This is what an expert would call a \emph{unital (or
    normal) lax double functor}, but we won't need this concept in any other
  setting.} 
\begin{itemize}
  \item A double category $\cat{D}$ called the \emph{indexing base}.
  \item For every object $D \in \cat{D}$, we have a category $\cat{A}(D)$.
  \item For every vertical arrow $j : D \to D'$, we have a functor $\cat{A}(j) :
    \cat{A}(D)
    \to \cat{A}(D')$.
  \item For every horizontal arrow $f : D \to D'$, we have a profunctor
$\cat{A}(f) : \cat{A}(D) \tickar \cat{A}(D')$.
  \item For every square 
\[
        \begin{tikzcd}[sep=tiny]
          A \ar[dd, "j"'] \ar[rr, "f"] & & B \ar[dd, "k"] \\
           & \alpha & \\
           C \ar[rr, "g"'] & & D
        \end{tikzcd}
\]
in $\cat{D}$, a square
\[
        \begin{tikzcd}[sep=tiny]
          \cat{A}(A) \ar[dd, "\cat{A}(j)"'] \ar[rr, "\cat{A}(f)"] & & \cat{A}(B) \ar[dd, "\cat{A}(k)"] \\
           & \cat{A}( \alpha ) & \\
          \cat{A}( C ) \ar[rr, "\cat{A}( g )"'] & & \cat{A}(D)
        \end{tikzcd}
\]
in $\Cat{Cat}$.
\item For any two horizontal maps $f : A \to B$ and $g : B \to E$ in
  $\cat{D}$, we have a square $\mu_{f, g} : \cat{A}(f) \odot \cat{A}(g) \to
  \cat{A}(f \mid g)$ called the \emph{compositor}:
\begin{equation}\label{eqn.compositor}
        \begin{tikzcd}[sep=tiny]
          \cat{A}(A) \ar[dd, equals] \ar[r, "\cat{A}(f)"] & \cat{A}(B) \ar[r, "\cat{A}(f)"] & \cat{A}(E) \ar[dd, equals] \\
           & \mu_{f, g} & \\
          \cat{A}( C ) \ar[rr, "\cat{A}(f \mid g)"'] & & \cat{A}(F)
        \end{tikzcd}
\end{equation}
\end{itemize}
This data is required to satisfy the following laws:
\begin{itemize}
  \item (Vertical Functoriality) For vertical maps $j : D \to D'$ and $k : D' \to D''$, we have
    that $$\cat{A}\left( \frac{j}{k} \right) = \cat{A}(k) \circ \cat{A}(j)$$
and that $\cat{A}(\id_D) = \id_{\cat{A}(D)}$.\footnote{Here, we are hiding some
  coherence issues. While our doubly indexed category of deterministic systems
  will satisfy this functoriality condition on the nose, we will soon see a
  doubly indexed category of matrices of sets for which this law only holds up
  to a coherence isomorphism. Again, the issue invovles shuffling parentheses
  around, and we will sweep it under the rug.}
  \item (Horizontal Lax Functoriality) For horizontal maps $f : D_1 \to D_2$, $g :
    D_2 \to D_3$ and $h : D_3 \to D_4$, the compositors $\mu$ satisfy the
    following associativity and unitality conditions:
\begin{itemize}
\item (Associativity) $$\frac{\mu_{f, g} | \cat{A}(h)}{\mu_{(f \mid g), h}} =
  \frac{\cat{A}(f) | \mu_{g, h}}{\mu_{f, (g \mid h)}}.$$
\item (Unitality) The profunctor $\cat{A}(\id_{D_1}) : \cat{A}(D_1) \tickar
  \cat{A}(D_1)$ is the identity profunctor, $\cat{A}(\id_{D_1}) = \cat{A}(D_1)$.
  Furthermore, $\mu_{\id_{D_1}, f}$ and $\mu_{f, \id_{D_2}}$ are equal to the
  isomorphisms of \cref{ex.identity_profunctor} given by the naturality of
  $\cat{A}(f)$ on the left and right respectively. We may sumarize this may
  saying that 
$$\mu_{\id, f} = \id_{\cat{A}(f)} = \mu_{f, \id}.$$
\end{itemize}
\item (Naturality of Compositors) For any horizontally composable squares
  $\alpha$ and $\beta$ with bottom horizontal maps $f$ and $g$ respectively, 
\[
\frac{\cat{A}( \alpha ) \mid \cat{A}( \beta )}{\mu_{f, g}} = \frac{\mu_{f} \mid \mu_g}{\cat{A}(\alpha \mid \beta)}.
\] 
\end{itemize}
\end{definition}

That's another big definition! It seems like it will be a slog to actually ever
prove that something is a doubly indexed category. Luckily, in our cases, these
proofs will go quite smoothly. This is because each of the three laws of a
doubly indexed category has a sort of sister law from the definition of a double
category which will help us prove it.

\begin{itemize}
  \item The Vertical Functoriality law will often involve the vertical
    associativity and unitality of squares in the indexing base.
  \item The Horizontal Lax Functoriality law will often involve the horizontal
    associativity and unitality of squares in the indexing base.
  \item The Naturality of Compositors law will often involve the interchange law
    in the indexing base.
\end{itemize}

We'll see how these sisterhoods play out in practice as we define the doubly
indexed categories of deterministic systems and vectors of sets.

\paragraph{The doubly indexed category of systems}

Let's show that systems in a doctrine $\mathbb{D}$ do indeed form a doubly indexed category
$$\Cat{Sys}_{\mathbb{D}} : \Cat{Arena}_{\mathbb{D}} \to \Cat{Cat}.$$

\begin{definition}
  The doubly indexed category $\Cat{Sys}_{\mathbb{D}} : \Cat{Arena}_{\mathbb{D}}
  \to \Cat{Cat}$ of systems in the doctrine $\mathbb{D} = (\mathcal{A}, T)$ is defined
  as follows:
\begin{itemize}
\item Our indexing base is the double category $\Cat{Arena}_{\mathbb{D}}$ of arenas, since we
  will arrange our systems according to their interface.
\item To every arena $\lens{I}{O}$, we associate the category $\Cat{Sys}\lens{I}{O}$
of systems with interface $\lens{I}{O}$ and behaviors whose chart is the
identity chart on $\lens{I}{O}$ (\cref{def.cat_of_systems}).
\item To every lens $\lens{w^{\sharp}}{w} : \lens{I}{O} \fromto \lens{I'}{O'}$, we associate the functor
$\Cat{Sys}\lens{w^{\sharp}}{w} : \Cat{Sys}\lens{I}{O} \to \Cat{Sys}\lens{I}{O}$ given by wiring according to
$\lens{w^{\sharp}}{w}$:
$$\Cat{Sys}\lens{w^{\sharp}}{w}(\Sys{S}) =
\frac{\Sys{S}}{\littlelens{w^{\sharp}}{w}}.$$
This is defined in \cref{prop.lens_comp_functor_discrete}.
\item To every chart $\lens{f_{\flat}}{f} : \lens{I}{O} \tto \lens{I'}{O'}$, we
  associate the profunctor $\Cat{Sys}\lens{f_{\flat}}{f} : \Cat{Sys}\lens{I}{O}
  \tickar \Cat{Sys}\lens{I'}{O'}$ which sends the $\lens{I}{O}$-system $\Sys{T}$
  and the $\lens{I'}{O'}$-system $\Sys{S}$ to the set of behaviors $\Sys{T} \to
  \Sys{S}$ with chart $\lens{f_{\flat}}{f}$:
\begin{align*}
  \Cat{Sys}\lens{f_{\flat}}{f}(\Sys{T}, \Sys{S}) &= \left\{ \phi : \State{T} \to
                                                   \State{S}\, \middle| \mbox{ $\phi$ is a behavior with chart $\lens{f_{\flat}}{f}$} \right\}\\
  &= \left\{  
    {
    \begin{tikzcd}[ampersand replacement = \&]
      \lens{T\State{T}}{\State{T}} \ar[r, dashed, shift left, "\lens{T\phi}{\phi}"] \ar[r, dashed, shift right] \ar[d, shift right,
      "\lens{\update{T}}{\expose{T}}"'] \ar[d, shift left, leftarrow] \&
      \lens{T\State{S}}{\State{S}} \ar[d, shift left, leftarrow,
      "\lens{\update{S}}{\expose{S}}"] \ar[d, shift right]\\
      \lens{\In{T}}{\Out{T}} \ar[r, shift right, "\lens{f^{\sharp}}{f}"'] \ar[r,
      shift left] \& \lens{\In{S}}{\Out{S}}
    \end{tikzcd}
                    }
                    \right\}
\end{align*}
We saw this profunctor in \cref{prop.profunctor_from_chart}.
\item To every square $\alpha$, we assign the morphism of profunctors given by
  composing vertically with $\alpha$ in $\Cat{Arena}$:
$$\Cat{Sys}(\alpha)(\phi) = \frac{\phi}{\alpha}.$$
We saw in \cref{prop.prof_square_from_arena_square_naturality} that this was a
natural transformation.
\item The compositor is given by horizontal composition in the double category
  of arenas:
  \begin{align*}
    \mu_{\littlelens{f_{\flat}}{f},\littlelens{g_{\flat}}{g}} : \Cat{Sys}\littlelens{f_{\flat}}{f} \odot \Cat{Sys}\littlelens{g_{\flat}}{g} &\to \Cat{Sys}\left( \littlelens{f_{\flat}}{f} \then \littlelens{g_{\flat}}{g} \right) \\
(\phi, \psi) &\mapsto \phi \mid \psi
  \end{align*}
\end{itemize}
\end{definition}

Let's check now that this does indeed satisfy the laws of a doubly indexed
category. The task may appear to loom over us; there are quite a few laws, and
there is a lot of data involved. But nicely, they all follow quickly from a bit of fiddling in the double
category of arenas.
\begin{itemize}
  \item (Vertical Functoriality) We show that $\Cat{Sys}\left(
      \lens{k^{\sharp}}{k} \circ \lens{j^{\sharp}}{j} \right) =
    \Cat{Sys}\lens{k^{\sharp}}{k} \circ \Cat{Sys}\lens{j^{\sharp}}{j}$ by
    vertical associativity:
\begin{align*}
  \Cat{Sys}\left(\lens{k^{\sharp}}{k} \circ \lens{j^{\sharp}}{j} \right)(\phi) &= \frac{\phi}{\left( \frac{\littlelens{j^{\sharp}}{j}}{\littlelens{k^{\sharp}}{k}} \right)} 
= \frac{\left( \frac{\phi}{\littlelens{j^{\sharp}}{j}} \right)}{\littlelens{k^{\sharp}}{k}} \\
&= \Cat{Sys}\lens{k^{\sharp}}{k} \circ \Cat{Sys}\lens{j^{\sharp}}{j}(\phi).
\end{align*}

\item (Horizontal Lax Functoriality) This law follows from horizontal
  associativity in $\Cat{Arena}$.
\begin{align}
  \mu(\mu(\phi, \psi), \xi) = (\phi \mid \psi ) \mid \xi = \phi \mid (\psi \mid \xi) = \mu(\phi, \mu(\psi, \xi)).
\end{align}
\item (Naturality of Compositor) This law follows from interchange in
  $\Cat{Arena}$.
\begin{align*}
  \left( \frac{\Cat{Sys}(\alpha) \mid \Cat{Sys}(\beta)}{\mu} \right)(\phi, \psi) &= \left. \frac{\phi}{\alpha} \middle| \frac{\psi}{\beta} \right. 
= \frac{\phi \mid \psi}{\alpha \mid \beta} \\
&= \left(  \frac{\mu}{\Cat{Sys}(\alpha \mid \beta)}\right)(\phi, \psi).
\end{align*}
\end{itemize}


\paragraph{The doubly indexed category of vectors of sets}

In addition to our doubly indexed category of systems, we have a doubly indexed
category of ``vectors of sets''. 

Classically, an $m \times n$ matrix $M$ can act on a vector $v$ of length $n$ by multiplication to get
another vector $Mv$ of length $m$. We can generalize this to matrices of sets if
we define a vector of sets of length $A$ to be a dependent set $V : A \to
\smset$. 
\begin{definition}\label{def.linear_functor}
  For a set $A$, we define the category of \emph{vectors of sets of length} $A$
  to be $$\Cat{Vec}(A) \coloneqq \smset^A$$
the category of sets depending on $A$. 

Given a $(B \times A)$-matrix $M$, we can treat a $A$-vector $V$ as a $A \times
\ord{1}$ matrix and form the $B \times \ord{1}$ matrix $MV$. This gives us a
functor
\begin{align*}
\Cat{Vec}(M) : \Cat{Vec}(A) &\to \Cat{Vec}(B)\\
               V &\mapsto (MV)_b = \sum_{a \in A} M_{ba} \times V_a \\
           f : V \to W &\mapsto ( (a, m, v) \mapsto (a, m, f(v)) )
\end{align*}
which we refer to as the linear functor given by $M$.
\end{definition}

\begin{definition}
The doubly indexed category $\Cat{Vec} : \Cat{Matrix} \to \Cat{Cat}$ of vectors
of sets is defined by:
\begin{itemize}
  \item Its indexing base is the double category of matrices of sets.
  \item To every set $A$, we assign the category $\Cat{Vec}(A) = \smset^A$ of
    vectors of length $A$.
  \item To every $(B \times A)$-matrix $M : A \to B$, we assign the linear
    functor $\Cat{Vec}(M) : \Cat{Vec}(A) \to \Cat{Vec}(B)$ given by $M$ (\cref{def.linear_functor}).
  \item To every function $f : A \to B$, we associate the profunctor
    $\Cat{Vec}(f) : \Cat{Vec}(A) \tickar \Cat{Vec}(B)$ defined by
$$\Cat{Vec}(f)(V, W) = \{ F : (a \in A) \to V_a \to W_{f(a)} \}.$$
That is, $F \in \Cat{Vec}(f)(V, W)$ is a family of functions $F(a,-) : V_a \to
W_{f(a)}$ indexed by $a \in A$. This is natural by index-wise composition.
\item To every square
  \[
        \begin{tikzcd}[sep=tiny]
          A \ar[dd, "M"'] \ar[rr, "f"] & & B \ar[dd, "N"] \\
           & \alpha & \\
          C \ar[rr, "g"'] & & D
        \end{tikzcd}
  \]
  that is, family of functions $\alpha_{ca} : M_{ca} \to N_{g(c)f(a)}$, we
  associate the square
  \[
        \begin{tikzcd}[sep=tiny]
          \Cat{Vec}( A ) \ar[dd, "\Cat{Vec}(M)"'] \ar[rr, "\Cat{Vec}(f)"] & & \Cat{Vec}( B ) \ar[dd, "\Cat{Vec}(N)"] \\
           & \Cat{Vec}(\alpha) & \\
          \Cat{Vec}(C) \ar[rr, "\Cat{Vec}(g)"'] & & \Cat{Vec}(D)
        \end{tikzcd}
  \]
  defined by sending a family of functions $F : (a \in A) \to V_{a} \to
  W_{f(a)}$ in $\Cat{Vec}(f)(V, W)$ to the family 
  \begin{align*}
\Cat{Vec}(\alpha)(F) : (c \in C) \to MV_c &\to MW_{g(c)} \\ 
  \Cat{Vec}(\alpha)(F)(c, (a, m, v)) &= (f(a), \alpha(m), F(a, v))
  \end{align*}
  That is, $\Cat{Vec}(\alpha)(F)(c, -)$ takes an element $(a, m, v) \in MV_{c} =
  \sum_{a \in A} M_{ca} \times V_a$ and gives the elements $(f(a), \alpha(m),
  F(a, v))$ of $MW_{g(c)} = \sum_{b \in B} N_{g(c)b} \times W_b$.
\item The compositor is given by componentwise composition: If $f : A \to B$ and
  $g : B \to C$ and $F \in \Cat{Vec}(f)(V, W)$ and $G \in \Cat{Vec}(g)(W, U)$,
  then 
\begin{align*}
  \mu_{f, g}(F, G) : (a \in A) \to V_{a} &\to U_{gf(a)} \\
  \mu_{f, g}(F, G)(a, v) &\coloneqq G(f(a), F(a, v)).
\end{align*}
\end{itemize}
\end{definition}

It might seem like it will turn out to be a big hassle to show that this
definition satisfies all the laws of a doubly indexed category. Like with the
doubly indexed category of arenas, we will find that all the laws follow for
matrices by fiddling around in the double category of matrices.

Let's first rephrase the above definition in terms of the category of matrices.
We note that a vector of sets $V \in \Cat{Vec}(A)$ is equivalently a matrix $V :
\ord{1} \to A$. Then the linear functor $\Cat{Vec}(M) : \Cat{Vec}(A) \to
\Cat{Vec}(B)$ is given by matrix multiplication, or in double category notation:
$$\Cat{Vec}(M)(V) = \frac{V}{M}.$$
This means that the Vertical Functoriality law follows by vertical associativity
in the double category of matrices, which is to say associativity of matrix
multiplication.

Similarly, we can interpret the profunctor $\Cat{Vec}(f)$ for $f : A \to B$ in
terms of the double category $\Cat{Matrix}$. An element $F \in \Cat{Vec}(f)(V, W)$ is
equivalently a square of the following form in $\Cat{Matrix}$:
\[
        \begin{tikzcd}[sep=tiny]
          \ord{1} \ar[dd, "V"'] \ar[rr, equals] & & \ord{1} \ar[dd, "W"] \\
           & F & \\
          A \ar[rr, "f"'] & & B
        \end{tikzcd}
      \]
      Therefore, we can describe $\Cat{Vec}(f)(V, W)$ as the following set:
\[
\Cat{Vec}(f)(V, W) = \left\{ F \,\middle|
        \begin{tikzcd}[sep=tiny]
          \ord{1} \ar[dd, "V"'] \ar[rr, equals] & & \ord{1} \ar[dd, "W"] \\
           & F & \\
          A \ar[rr, "f"'] & & B
        \end{tikzcd}
  \right\}
\]
Then the Horizontal Lax Functoriality laws follow from associativity and unitality of
horizontal composition of squares in $\Cat{Matrix}$! 


Finally, we need to interpret the rather fiddly transformation
$\Cat{Vec}(\alpha)$ in terms of the double category of matrices. Its a matter of
unfolding the definitions to see that
$\Cat{Vec}(\alpha)(F) = \frac{F}{\alpha}$
in $\Cat{Matrix}$, and therefore that the Naturality of Compositors law follows
by the interchange law.

  If this argument seemed wholly too similar to the one we gave for the doubly
  indexed category of systems, your suspicions are not misplaced. These are both are instances of a very general \emph{vertical
    slice construction}, which we turn our attension to now.

  \section{Vertical Slice Construction}\label{sec.vertical_slice}

  In the previous section, we constructed the doubly indexed categories
  $\Cat{Sys}_{\mathbb{D}}$ of systems in a doctrine $\mathbb{D}$ and $\Cat{Vec}$
  of vectors of sets ``by hand''. However, both constructions felt very
  familiar. In this section, we will show that they are both instances of a
  general construction: the \emph{vertical slice construction}.

The main reason for recasting the above constructions in more general terms is
that it will facilitate our main theorem of this chapter: change of doctrine.

  The vertical slice construction will take a \emph{double functor} $F :
  \cat{D}_0 \to \cat{D}_1$ and produce a doubly indexed category $\sigma F :
  \cat{D}_1 \to \Cat{Cat}$ indexed by its codomain. So, in order to describe the
  vertical slice construction, we will need the notion of double functor. We will need the notion of double functor for much of the coming theory as well
  


\subsection{Double Functors}
  
A double functor is, unsurprisingly, a functor between double categories. Just
as a double category has a bit more than twice the information involved in a
category, a double functor has a bit more than twice the information involved in
a functor.
 \begin{definition}\label{def.double_functor}
 Let $\cat{D}_0$ and $\cat{D}_1$ be double categories. A \emph{double functor}
 $\Fun{F} : \cat{D}_0 \to \cat{D}_1$ consists of:
\begin{itemize}
  \item An object assignment $F : \Ob \cat{D}_0 \to \Ob \cat{D}_1$ which assigns an object $F D$
    in $\cat{D}_1$ to each object $D$ in $\cat{D}_0$.
  \item A vertical functor $F : v\cat{D}_0 \to v\cat{D}_1$ on the vertical
    categories, which acts the same as the object assignment on objects.
  \item A horizontal functor $F : h\cat{D}_0 \to h\cat{D}_1$ on the horizontal
    categories, which acts the same as the object assignment on objects.
  \item For every square
    \[
        \begin{tikzcd}[sep=tiny]
          A \ar[dd, "j"'] \ar[rr, "f"] & & B \ar[dd, "k"] \\
           & \alpha & \\
          C \ar[rr, "g"'] & & D
        \end{tikzcd}
    \]
    in $\cat{D}_0$, a square
    \[
        \begin{tikzcd}[sep=tiny]
          FA \ar[dd, "Fj"'] \ar[rr, "Ff"] & & FB \ar[dd, "Fk"] \\
           & F\alpha & \\
          FC \ar[rr, "Fg"'] & & FD
        \end{tikzcd}
    \]
    such that the following laws hold:
    \begin{itemize}
    \item $F$ commutes with horizontal compostition: $F(\alpha \mid \beta) = F\alpha \mid F\beta$.
    \item $F$ commutes with vertical comopsition: $F\left( \frac{\alpha}{\beta} \right) = \frac{F\alpha}{F\beta}$.
    \item $F$ sends horizontal identities to horizontal identities, and vertical
      identities to vertical identities.
    \end{itemize}
\end{itemize}
\end{definition}

\begin{remark}
There is, in fact, a double category of double functors $F : \cat{D}_0 \to
\cat{D}_1$, but we won't need to worry about this until we consider the
functoriality of the vertical slice construction in \cref{sec.functoriality_vertical_slice}.
\end{remark}

We will, in time, see many interesting examples of double functors. For now,
however, we will content ourselves with the two simple examples we need to
construct the doubly indexed categories $\Cat{Sys}$ and $\Cat{Vec}$.

\begin{example}\label{ex.double_functor_section}
Let $\mathbb{D} = (\cat{A} : \cat{C}\op \to \Cat{Cat}, T)$ be a doctrine. We
recall that the section $T : \cat{C} \to \int^{C : \cat{C}}\cat{A}(C)$ is a functor to the
Grothendieck construction of $\cat{A}$. We may promote this into a double
functor into the double category of arenas $\Cat{Arena}_{\mathbb{D}}$ in a
rather simple way.

Since the horizontal category of $\Cat{Arena}_{\mathbb{D}}$ is $\int^{C : \cat{C}}
\cat{A}(C)$, the category of charts, we may consider $T$ as a \emph{double}
functor
$$hT : h\cat{C} \to \Cat{Arena}_{\mathbb{D}}$$
from the double category $h\cat{C}$ given by defining its horizontal category to
be $\cat{C}$ and taking its vertical category and its squares to consist only of
identities. Its worth taking a minute to check this trivial observation against
the definition of a double functor.
\end{example}

\begin{example}\label{ex.double_functor_one}
There is a double category $\ord{1}$ with just one object $\ast$ and only identity maps
and squares. A double functor $F : \ord{1} \to \cat{D}$ simply picks out the
object $F(\ast)$; there is no other data involved, since everything else must
get sent to the appropriate identities.

In particular, the one element set $\ord{1}$ is an object of the double category
$\Cat{Matrix}$ of sets, functions, and matrices. Therefore, there is a double
functor $\ord{1} : \ord{1} \to \Cat{Matrix}$ picking out this special element.
\end{example}


\subsection{The Vertical Slice Construction: Definition}

We are now ready to define the vertical slice construction.
  \begin{definition}[The Vertical Slice Construction]
Let $F : \cat{D}_0 \to \cat{D}_1$ be a double functor. The \emph{vertical slice
  construction} of $F$ is the doubly indexed category 
$$\sigma F : \cat{D}_1 \to \Cat{Cat}$$
defined as follows:
\begin{itemize}
  \item For $D \in \cat{D}_1$, $\sigma F(D)$ is the category whose objects are
    pairs $(A, j)$ of an object $A \in \cat{D}_0$ and a vertical map $f :
    FA \to D$. A map $(A_1, j_1) \to (A_2, j_2)$ is a pair $(f, \alpha)$ of a
    horizontal $f : A_1 \to A_2$ and a square 
\[
        \begin{tikzcd}[sep=tiny]
          FA_1 \ar[dd, "j_1"'] \ar[rr, "Ff"] & & FA_2 \ar[dd, "j_2"] \\
           & \alpha & \\
          D \ar[rr, equals] & & D
        \end{tikzcd}
\]
in $\cat{D}_1$.
\item For every vertical $j : D \to D'$ in $\cat{D}_1$, we associate the functor
  $\sigma F(j) : \sigma F(D) \to \sigma F(D')$ given by vertical composition
  with $j$:
  \[
        \begin{tikzcd}[sep=tiny]
          FA_1 \ar[dd, "j_1"'] \ar[rr, "Ff"] & & FA_2 \ar[dd, "j_2"] \\
           & \alpha & \\
          D \ar[rr, equals] & & D
        \end{tikzcd}
        \quad\mapsto\quad
        \begin{tikzcd}[sep=tiny]
          FA_1 \ar[dd, "j_1"'] \ar[rr, "Ff"] & & FA_2 \ar[dd, "j_2"] \\
           & \alpha & \\
          D \ar[rr, equals] \ar[dd, "j"'] & & D\ar[dd, "j"]\\
           & \phantom{\alpha} & \\
          D' \ar[rr, equals] & & D'
        \end{tikzcd}
\]
More concisely, this is
$$\sigma F(j)(f, \alpha) = \left(f, \frac{\alpha}{j} \right).$$
\item For every horizontal $g : D \to D'$ in $\cat{D}_1$, we associate the
  profunctor $\sigma F (g) : \sigma F(D) \tickar \sigma F(D')$ given by 
\[
        \begin{tikzcd}[sep=tiny]
          FA_1 \ar[dd, "j_1"'] \\
          \phantom{\alpha} \\
          D 
        \end{tikzcd}\, ,\,
        \begin{tikzcd}[sep=tiny]
          FA_2 \ar[dd, "j_2"'] \\
          \phantom{\alpha} \\
          D'
        \end{tikzcd}
\quad \mapsto \quad
\left\{  \left( f,  
        \begin{tikzcd}[sep=tiny] 
          FA_1 \ar[dd, "j_1"'] \ar[rr, dashed, "Ff"] & & FA_2 \ar[dd, "j_2"] \\
           & \alpha & \\
          D \ar[rr, "g"'] & & D'
        \end{tikzcd}\right).
 \right\}
\]
We note that if $g = \id_D$ is an identity, then this reproduces the hom
profunctor of $\sigma F(D)$.
\item The compositor $\mu$ is given by horizontal composition:
\[
\mu_{g_1, g_2}((f_1, \alpha_1), (f_2, \alpha_2)) = (f_1 \mid f_2, \alpha_1 \mid
\alpha_2).
\]
\end{itemize}
\end{definition}


Let's check now that this does indeed satisfy the laws of a doubly indexed
category. The proof is exactly as it was for $\Cat{Sys}$. 
\begin{itemize}
  \item (Vertical Functoriality) We show that $\sigma F\left(
      \frac{k_1}{k_2} \right) =
    \sigma F(k_2) \circ \sigma F (k_1)$ by
    vertical associativity:
\begin{align*}
  \sigma F\left(\frac{k_1}{k_2}\right)(f, \alpha) &= \left(f,  \frac{\alpha}{\left( \frac{k_1}{k_2} \right)} \right) \\
= \left(f, \frac{\left( \frac{\alpha}{k_1} \right)}{k_2}   \right)\\
&= \sigma F(k_2) \circ \sigma F (k_1)((f, \alpha)).
\end{align*}

\item (Horizontal Lax Functoriality) This law follows from horizontal
  associativity in $\cat{D}_1$.
\begin{align*}
  \mu(\mu((f_1, \alpha_1), (f_2, \alpha_2)), (f_3, \alpha_3)) &= ((f_1 \mid f_2) \mid f_3, (\alpha_1 \mid \alpha_2) \mid \alpha_3) \\
&= (f_1 \mid (f_2 \mid f_3), \alpha_1 \mid (\alpha_2 \mid \alpha_3)) \\
&= \mu((f_1, \alpha_1), \mu((f_2, \alpha_2), (f_3, \alpha_3))).
\end{align*}
\item (Naturality of Compositor) This law follows from interchange in
  $\cat{D}_1$.
\begin{align*}
  \left( \sigma F(\beta_1) \mid \sigma F(\beta_2){\mu} \right)((f_1, \alpha_1), (f_2, \alpha_2)) &= \left(f_1 \mid f_2,   \left. \frac{\phi}{\alpha} \middle| \frac{\psi}{\beta} \right.\right) \\
&= \left(f_1 \mid f_2,  \frac{\phi \mid \psi}{\alpha \mid \beta}\right) \\
&= \left(  \frac{\mu}{\sigma F(\beta_1 \mid \beta_2)}\right)((f_1,\alpha_1),(f_2,\alpha_2)).
\end{align*}
\end{itemize}

We can now see that the vertical slice construction generalizes both the
constructions of $\Cat{Sys}_{\mathbb{D}}$ and $\Cat{Vec}$. 
\begin{proposition}
  The doubly indexed category $\Cat{Sys}_{\mathbb{D}}$ of systems in a doctrine
  $\mathbb{D} = (\cat{A} : \cat{C}\op \to \Cat{Cat}, T)$ is the vertical slice
  construction of the double functor $hT : h\cat{C} \to
  \Cat{Arena}_{\mathbb{D}}$ given by considering the section $T$ as a double
  functor.
$$\Cat{Sys}_{\mathbb{D}} = \sigma (hT : h\cat{C} \to \Cat{Arena}_{\mathbb{D}}).$$
\end{proposition}
\begin{proof}
This is a matter of checking definitions and seeing that they are precisely the same.
\end{proof}

\begin{proposition}
The doubly indexed category $\Cat{Vec}$ of vectors of sets is the vertical slice
construction of the inclusion $\ord{1} : \ord{1} \to \Cat{Matrix}$ of the one
element set into the double category of matrices of sets.
$$\Cat{Vec} = \sigma (\ord{1} : \ord{1} \to \Cat{Matrix}).$$
\end{proposition}
\begin{proof}
This is also a matter of checking that the definitions coincide.
\end{proof}

\subsection{Natural Transformations of Double Functors}
We now turn towards proving the functoriality of the vertical slice construction
as a first step in proving the change of doctrine theorem. In order to express
the functoriality of the vertical slice construction, we will first need learn
about natural transformations between double functors.

Since double categories have two sorts of maps --- vertical and horizontal ---
there are also two sorts of natural transformations between double functors. The
two definitions are symmetric; we may arrive at one by replacing the words
``vertical'' by ``horizontal'' and vice-versa. We will have occasion to use both
of them in this and the coming chapters.

\begin{definition}\label{def.double_natural_transformation}
Let $F$ and $G : \cat{D} \to \cat{E}$ be double functors. A \emph{vertical
  natural transformation} $v : F \Rightarrow G$ consists of the following data:
\begin{itemize}
\item For every object $D \in \cat{D}$, a vertical $v_D : FD \to GD$ in $\cat{E}$.
\item For every horizontal arrow $f : D \to D'$ in $\cat{D}$, a square
\[
        \begin{tikzcd}[sep=tiny]
          FD \ar[dd, "v_D"'] \ar[rr, "Ff"] & & FD'
 \ar[dd, "v_{D'}"] \\
           & v_f & \\
          GD \ar[rr, "Gf"'] & & GD'
        \end{tikzcd}
\]
\end{itemize}
This data must satisfy the following laws:
\begin{itemize}
\item (Vertical Naturality) For any vertical $j : D_1 \to D_2$, we have 
$$\frac{Fj}{v_{D_2}} = \frac{v_{D_1}}{Gj}.$$
\item (Horizontal Naturality) For any horizontal $f_1 : D_1 \to D_2$ and $f_2 :
  D_2 \to D_3$, we have 
$$v_{f_1 \mid f_2} = v_{f_1} \mid v_{f_2}.$$
\item (Horizontal Unity) $v_{\id_D} = \id_{v_D}$. 
\item (Square naturality) For any square
\[
        \begin{tikzcd}[sep=tiny]
          D_1 \ar[dd, "j_1"'] \ar[rr, "f_1"] & & D_2
 \ar[dd, "j_2"] \\
           & \alpha & \\
          D_3 \ar[rr, "f_2"'] & & D_4
        \end{tikzcd}
\]
we have
$$\frac{F\alpha}{v_{f_2}} = \frac{v_{f_1}}{G\alpha}.$$
\end{itemize}

Dually, a \emph{horizontal transformation} $h : F \Rightarrow G$ consists of the
following data:
\begin{itemize}
\item For every object $D \in \cat{D}$ a horizontal morphism $h_D : FD \to GD$.
\item For every vertical $j : D \to D'$ in $\cat{D}$, a square 
\[
        \begin{tikzcd}[sep=tiny]
          FD \ar[dd, "Fj"'] \ar[rr, "h_D"] & & GD
 \ar[dd, "Gj"] \\
           & h_j & \\
          FD' \ar[rr, "h_{D'}"'] & & GD'
        \end{tikzcd}
\]
\end{itemize}
This data is required to satisfy the following laws:
\begin{itemize}
\item (Horizontal Naturality) For horizontal $f : D_1 \to D_2$, we have
$$Ff \mid v_{D_2} = v_{D_1} \mid Gf.$$
\item (Vertical Naturality) For vertical $j_1 : D_1 \to D_2$ and $j_2 : D_2 \to
  D_3$, we have
$$h_{\frac{j_1}{j_2}} = \frac{h_{j_1}}{h_{j_2}}.$$
\item (Vertical Unity) $h_{\id_D} = \id_{h_D}$.
\item (Square naturality) For any square
\[
        \begin{tikzcd}[sep=tiny]
          D_1 \ar[dd, "j_1"'] \ar[rr, "f_1"] & & D_2
 \ar[dd, "j_2"] \\
           & \alpha & \\
          D_3 \ar[rr, "f_2"'] & & D_4
        \end{tikzcd}
\]
we have 
$$F\alpha \mid h_{j_2} = h_{j_1} \mid G\alpha.$$
\end{itemize}
\end{definition}

\begin{remark}
Note that vertical (resp. horizontal) natural transformations are named for
the direction of arrow they assign to objects. However, a vertical
transformation is defined by its action $v_f$ on \emph{horizontal} maps $f$, and dually a
horizontal transformation $h_j$ by its action on \emph{vertical} maps $j$.
Taking $f$ (resp. $j$) to be an identity $\id_D$ yields the vertical (resp.
horizontal) arrow associated to the object $D$.
\end{remark}

Natural transformations between double functors can be composed in the
appropriate directions.
\begin{lemma}
Suppose that $v_1 : F_1 \Rightarrow F_2$ and $v_2 : F_2 \Rightarrow F_2$ are
vertical transformations. We have a vertical composite $\frac{v_1}{v_2}$ defined
by
$$\left( \frac{v_1}{v_2} \right)_f \coloneqq \frac{(v_1)_f}{(v_2)_f}$$
for horizontal maps $f$. Dually, for horizontal transformations $h_1 : F_1
\Rightarrow F_2$ and $h_2 : F_2 \Rightarrow F_3$, there is a horizontal
composite $h_1 \mid h_2$ defined by
$$(h_1 \mid h_2)_j := (h_1)_j \mid (h_2)_j$$
for every vertical map $j$.
\end{lemma}
\begin{proof}
  We will prove that $\frac{v_1}{v_2}$ is a vertical transformation; the proof
  that $h_1 \mid h_2$ is a horizontal transformation is precisely dual. 
  \begin{itemize}
  \item (Vertical Naturality) This follows by the same argument as for Square
    Naturality below, taking $\alpha = j$ for a vertical$j : D_1 \to D_2$.

\item (Horizontal naturality) For horizontal maps $f_1 :D_1 \to D_2$ and $f_2 :
  D_2 \to D_3$, we have
  \begin{align*}
    \frac{v_1}{v_2}_{f_1 \mid f_2} &= \frac{(v_1)_{f_1 \mid f_2}}{(v_2)_{f_1 \mid f_2}} \\
                                   &= \frac{(v_1)_{f_1} \mid (v_1)_{f_2}}{(v_2)_{f_1} \mid (v_2)_{f_2}} \\
                                   &= \left. \frac{(v_1)_{f_1}}{(v_2)_{f_1}} \middle| \frac{(v_1)_{f_2}}{(v_2)_{f_2}} \right.\\
    &= \left. \left( \frac{v_1}{v_2} \right)_{f_1} \middle| \left( \frac{v_1}{v_2} \right)_{f_2} \right. .
  \end{align*}

    
\item (Horizontal Unity) This holds by definition.

\item (Square Naturality) Consider a square $\alpha$ of the following signature:
\[
        \begin{tikzcd}[sep=tiny]
          D_1 \ar[dd, "j_1"'] \ar[rr, "f_1"] & & D_2
 \ar[dd, "j_2"] \\
           & \alpha & \\
          D_3 \ar[rr, "f_2"'] & & D_4
        \end{tikzcd}
\]
  Then
    \begin{align*}
      \frac{F_1 \alpha}{\left( \frac{v_1}{v_2} \right)_{f_2}} &= \begin{tabular}{c}
        $F_1 \alpha $ \\ \hline
        $(v_1)_{f_2}$ \\ \hline
        $(v_2)_{f_2}$ 
     \end{tabular} \\
&= \begin{tabular}{c}
        $(v_1)_{f_1}$ \\ \hline
        $F_2 \alpha$ \\ \hline
        $(v_2)_{f_2}$ 
     \end{tabular} \\
&= \begin{tabular}{c}
        $(v_1)_{f_1}$ \\ \hline
        $(v_2)_{f_1}$ \\ \hline
        $F_3 \alpha$ 
     \end{tabular} \\
&= \frac{\left( \frac{v_1}{v_2} \right)_{f_1}}{F_3 \alpha}.
    \end{align*}
  \end{itemize}
  
\end{proof}

Amongst double functors we have found two sorts of maps --- vertical and
horizontal --- each with their own sort of composition. This suggests that there
should be a \emph{double category} of double functors $\cat{D} \to \cat{E}$,
just as there is a category of functors between two categories. 

\begin{theorem}
Let $\cat{D}$ and $\cat{E}$ be double categories. There is a double category
$\Cat{Fun}(\cat{D}, \cat{E})$ of double functors from $\cat{D}$ to $\cat{E}$
whose vertical maps are vertical transformations, horizontal maps are horizontal
transformations, and whose squares
\[
        \begin{tikzcd}[sep=tiny]
          F_1 \ar[dd, "v_1"'] \ar[rr, "h_1"] & & F_2
 \ar[dd, "v_2"] \\
           & \alpha & \\
          F_3 \ar[rr, "h_2"'] & & F_4
        \end{tikzcd}
      \]
      are \emph{modifications} defined in the following way. To each object $D
      \in \cat{D}$, we have a square
\[
        \begin{tikzcd}[sep=tiny]
          F_1D \ar[dd, "(h_1)_D"'] \ar[rr, "( v_1 )_D"] & & F_2D
 \ar[dd, "( v_2 )_D"] \\
           & \alpha_D & \\
          F_3 D \ar[rr, "( h_2 )_D"'] & & F_4 D
        \end{tikzcd}
\]
which satisfies the following laws:
\begin{itemize}
  \item (Horizontal Coherence) For every horizontal $f : D_1 \to D_2$, we have
    that
    \[
    (v_1)_f \mid \alpha_{D_2} = \alpha_{D_1} \mid (v_2)_f.
  \]
  We note that this law requires us to use the vertical naturality law of $v_1$
  and $v_2$ so that these composites have the same signature.
  \item (Vertical Coherence) For every vertical $j : D_1 \to D_2$, we have that
    \[
    \frac{\alpha_{D_1}}{(h_2)_j} = \frac{(h_1)_j}{\alpha_{D_2}}.
    \]
  We note that this law requires us to use the horizontal naturality law of $h_1$
  and $h_2$ so that these composites have the same signature.
\end{itemize}
 The compositions $\alpha \mid \beta$ and $\frac{\alpha}{\beta}$ are given
 componentwise by $\alpha_D \mid \beta_D$ and $\frac{\alpha_D}{\beta_D}$.      
\end{theorem}
\begin{proof}
  Since the compositions of modifications are given componentwise, they will
  satisfy associativity and interchange. We just need to show that they are well
  defined, which is to say that they satisfy the laws of a modification. This is
  a straightforward calculation; we'll prove Vertical Coherence for horizontal
  composition since the other cases are similar.

  Let $\alpha$ and $\beta$ be modifications with the following signatures:
  \[
        \begin{tikzcd}[sep=tiny]
          F_1 \ar[dd, "v_1"'] \ar[rr, "h_1"] & & F_2
 \ar[dd, "v_2"] \\
           & \alpha & \\
          F_3 \ar[rr, "h_2"'] & & F_4
        \end{tikzcd}
        \quad\mbox{and}\quad
        \begin{tikzcd}[sep=tiny]
          F_2 \ar[dd, "v_2"'] \ar[rr, "h_3"] & & F_5
 \ar[dd, "v_3"] \\
           & \beta & \\
          F_4 \ar[rr, "h_4"'] & & F_6
        \end{tikzcd}
\]
  Let $j : D_1 \to D_2$ be a vertical map in $\cat{D}$. We calculate:
  \begin{align*}
    \frac{(\alpha \mid \beta)_{D_1}}{(h_2 \mid h_4)_{j}} &= \frac{\alpha_{D_1} \mid \beta_{D_1}}{(h_2)_j \mid (h_4)_j} \\
                                                         &= \left. \frac{\alpha_{D_1}}{(h_2)_{j}} \middle| \frac{\beta_{D_1}}{(h_4)_{j}} \right. \\
                                                         &= \left. \frac{(h_1)_{j}}{\alpha_{D_2}} \middle| \frac{(h_3)_{j}}{\beta_{D_2}} \right. \\
                                                         &= \frac{(h_1 \mid h_3)_{j}}{(\alpha \mid \beta)_{D_2}}.
  \end{align*}
\end{proof}

Before we move on, let's record an important lemma relating modifications to squares.
\begin{lemma}
  Let
  \[
        \begin{tikzcd}[sep=tiny]
          F_1 \ar[dd, "v_1"'] \ar[rr, "h_1"] & & F_2
 \ar[dd, "v_2"] \\
           & \alpha & \\
          F_3 \ar[rr, "h_2"'] & & F_4
        \end{tikzcd}
  \]
  be a modification, and
  \[
        \begin{tikzcd}[sep=tiny]
          D_1 \ar[dd, "j_1"'] \ar[rr, "f_1"] & & D_2
 \ar[dd, "j_2"] \\
           & s & \\
          D_3 \ar[rr, "f_2"'] & & D_4
        \end{tikzcd}
  \]
  be a square in $\cat{D}$. We then have the following four-fold equality in $\cat{E}$:
  \[
    \begin{tikzcd}[ampersand replacement = \&]
  \begin{tabular}{c|c}
    $\alpha_{D_1}$ & $(v_2)_{f_1}$ \\ \hline
    $(h_2)_{j_1}$ & $F_4 s$
  \end{tabular}     \ar[r,
      equals] \ar[d, equals] \& \begin{tabular}{c|c}
    $(v_1)_{f_1}$ & $\alpha_{D_2}$ \\ \hline
    $F_3 s$ & $(h_2)_{j_2}$
  \end{tabular} \ar[d, equals]\\
   \begin{tabular}{c|c}
    $(h_1)_{j_1}$ & $F_2 s$ \\ \hline
  $\alpha_{D_3}$   & $(v_2)_{f_2}$
  \end{tabular}   \ar[r, equals] \&  \begin{tabular}{c|c}
    $F_1 s$ &  $(h_1)_{j_2}$ \\ \hline
    $(v_2)_{f_1}$ & $\alpha_{D_3}$
                                     \end{tabular}
\end{tikzcd}
  \]
 We may refer to the single square given by any of these composites by $\alpha_s$. 
\end{lemma}
\begin{proof}
These all follow by cycling through the square naturality laws of the
transformations and the coherence laws of the modification. 
\end{proof}

\subsection{Vertical Slice Construction: Functoriality}\label{sec.functoriality_vertical_slice}

In this section, we will describe the functoriality of the vertical slice
construction. Since the vertical slice construction takes a double functor $F :
\cat{D}_0 \to \cat{D}_1$ and produces a doubly indexed category $\sigma F :
\cat{D}_{1} \to \Cat{Cat}$, we will need to show that from a certain sort of map
between double functors we get a \emph{doubly indexed functor} between the
resulting vertical slices.

First, we will describe the appropriate notion of map between double functors.
This gives us a category which we will call the \emph{category of double
  functors} $\Cat{DblFun}$ \footnote{Though one could define other categories
  whose objects are double
functors, this is the only such category we will use in
this book.}
\begin{definition}
The category $\Cat{DblFun}$ has objects the double functor $F : \cat{D}_0 \to
\cat{D}_1$. A map $F_1 \to F_2$ is a triple $(v_0, v_1, v)$ where $v_0 :
\cat{D}_{00} \to \cat{D}_{10}$ and $v_1 : \cat{D}_{01} \to \cat{D}_{11}$ are
double functors and $v : F_2 \circ v_0 \Rightarrow v_1 \circ F_1$ is a vertical
transformation.
\[
        \begin{tikzcd}
          \cat{D}_{00} \ar[dd, "F_1"'] \ar[rr, "v_0"] & & \cat{D}_{10}
 \ar[dd, "F_2"] \ar[ddll, Rightarrow, "v"] \\
           &  & \\
          \cat{D}_{01} \ar[rr, "v_1"'] & & \cat{D}_{11}
        \end{tikzcd}
\]
Composition of $(v_0, v_1, v) : F_1 \to F_2$ with $(w_0, w_1, w) : F_2 \to F_3$
is given by $(w_0 \circ v_0, w_1 \circ v_1, v \ast w)$ where $v \ast w$ is the
vertical transformation with horizontal components given by
$$(v \ast w)_f := \frac{w_{v_0 f}}{w_1 v_f}.$$
\end{definition}

\section{Change of Doctrine}

\section{Approximation: Euler and Runge-Kutta}

\end{document}
