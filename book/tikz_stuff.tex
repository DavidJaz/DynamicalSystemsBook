\pgfdeclarelayer{edgelayer}
\pgfdeclarelayer{nodelayer}
\pgfsetlayers{edgelayer,nodelayer,main}



\tikzset{
	tick/.style={postaction={
  	decorate,
    decoration={markings, mark=at position 0.5 with
    	{\draw[-] (0,.4ex) -- (0,-.4ex);}}}
  }
} 

\tikzset{
biml/.tip={Glyph[glyph math command=triangleleft, glyph length=.87ex]},
bimr/.tip={Glyph[glyph math command=triangleright, glyph length=.92ex]},
}
  
\tikzset{trees/.style={
	inner sep=0, 
	minimum width=0, 
	minimum height=0,
	level distance=.5cm, 
	sibling distance=.5cm,
%	every child/.style={fill},
	edge from parent/.style={shorten <= -2pt, draw, ->},
	grow'=up,
	decoration={markings, mark=at position 0.75 with \arrow{stealth}}
	}
}
\newcommand{\idchild}{edge from parent[double, -]}
  	
  \tikzset{
     oriented WD/.style={%everything after equals replaces "oriented WD" in key.
        every to/.style={out=0,in=180,draw},
        label/.style={
           font=\everymath\expandafter{\the\everymath\scriptstyle},
           inner sep=0pt,
           node distance=2pt and -2pt},
        semithick,
        node distance=1 and 1,
        decoration={markings, mark=at position \stringdecpos with \stringdec},
        ar/.style={postaction={decorate}},
        execute at begin picture={\tikzset{
           x=\bbx, y=\bby,
           }}
        },
     string decoration/.store in=\stringdec,
     string decoration={\arrow{stealth};},
     string decoration pos/.store in=\stringdecpos,
     string decoration pos=.7,
	 	 dot size/.store in=\dotsize,
	   dot size=3pt,
	 	 dot/.style={
			 circle, draw, thick, inner sep=0, fill=black, minimum width=\dotsize
	   },
     bbx/.store in=\bbx,
     bbx = 1.5cm,
     bby/.store in=\bby,
     bby = 1.5ex,
     bb port sep/.store in=\bbportsep,
     bb port sep=1.5,
     % bb wire sep/.store in=\bbwiresep,
     % bb wire sep=1.75ex,
     bb port length/.store in=\bbportlen,
     bb port length=4pt,
     bb penetrate/.store in=\bbpenetrate,
     bb penetrate=0,
     bb min width/.store in=\bbminwidth,
     bb min width=1cm,
     bb rounded corners/.store in=\bbcorners,
     bb rounded corners=2pt,
     bb small/.style={bb port sep=1, bb port length=2.5pt, bbx=.4cm, bb min width=.4cm, 
bby=.7ex},
		 bb medium/.style={bb port sep=1, bb port length=2.5pt, bbx=.4cm, bb min width=.4cm, 
bby=.9ex},
     bb/.code 2 args={%When you see this key, run the code below:
        \pgfmathsetlengthmacro{\bbheight}{\bbportsep * (max(#1,#2)+1) * \bby}
        \pgfkeysalso{draw,minimum height=\bbheight,minimum width=\bbminwidth,outer 
sep=0pt,
           rounded corners=\bbcorners,thick,
           prefix after command={\pgfextra{\let\fixname\tikzlastnode}},
           append after command={\pgfextra{\draw
              \ifnum #1=0{} \else foreach \i in {1,...,#1} {
                 ($(\fixname.north west)!{\i/(#1+1)}!(\fixname.south west)$) +(-
\bbportlen,0) 
  coordinate (\fixname_in\i) -- +(\bbpenetrate,0) coordinate (\fixname_in\i')}\fi 
  %Define the endpoints of tickmarks
              \ifnum #2=0{} \else foreach \i in {1,...,#2} {
                 ($(\fixname.north east)!{\i/(#2+1)}!(\fixname.south east)$) +(-
\bbpenetrate,0) 
  coordinate (\fixname_out\i') -- +(\bbportlen,0) coordinate (\fixname_out\i)}\fi;
           }}}
     },
     bb name/.style={append after command={\pgfextra{\node[anchor=north] at 
(\fixname.north) {#1};}}}
  }
  

\newcommand{\boxofbullets}[6][1]{% [y-spacing] {number of bullets} {left endpoint} {label} {text} {position}
	\foreach \i [evaluate=\i as \y using {#1/2*((#2-1)/2-\i)}] in {1,...,#2}{
		\node (pt_#4_\i) at ($(0,\y)+#3$) {$\bullet$};
		\node[#6, font=\tiny] (lab_#4_\i) at ($(0,\y)+#3$) {#5$_\i$};
	}
	\node[draw, rounded corners, inner xsep=2pt, inner ysep=0pt, fit={(pt_#4_1) (pt_#4_#2) (lab_#4_1) (lab_#4_#2)}] (box_#4) {};
}

\tikzset{polybox/.style={
	polyb/.style ={
  	rectangle split, 
  	rectangle split parts=2, 
		rectangle split part align={bottom},
  	draw, 
  	anchor=south, 
  	inner ysep=4.5,
	  prefix after command={\pgfextra{\let\fixname=\tikzlastnode}},
	},
	poly/.style={
		polyb,
		append after command={\pgfextra{
			\node[inner xsep=0, inner ysep=0, 
				fit=(\fixname.north west) (\fixname.two split east)] 
				(\fixname_dir) {};
			\node[inner xsep=0, inner ysep=0, 
				fit=(\fixname.south west) (\fixname.two split east)] 
				(\fixname_pos) {};
			}}
	},
	poly left/.style={
		polyb,
  	rectangle split horizontal,
		append after command={\pgfextra{
			\node[inner xsep=0, inner ysep=0, 
				fit=(\fixname.north west) (\fixname.two split south)] 
				(\fixname_dir) {};
			\node[inner xsep=0, inner ysep=0, 
				fit=(\fixname.north east) (\fixname.two split south)] 
				(\fixname_pos) {};
			}}
	},
	poly right/.style={
		polyb,
  	rectangle split horizontal,
		append after command={\pgfextra{
			\node[inner xsep=0, inner ysep=0, 
				fit=(\fixname.north west) (\fixname.two split south)] 
				(\fixname_pos) {};
			\node[inner xsep=0, inner ysep=0, 
				fit=(\fixname.north east) (\fixname.two split south)] 
				(\fixname_dir) {};
			}}
	},
	poly down/.style={
		polyb,
		append after command={\pgfextra{
			\node[inner xsep=0, inner ysep=0, 
				fit=(\fixname.north west) (\fixname.two split east)] 
				(\fixname_pos) {};
			\node[inner xsep=0, inner ysep=0, 
				fit=(\fixname.south west) (\fixname.two split east)] 
				(\fixname_dir) {};
			}}
	},
	dom/.style={
		rectangle split part fill={none, blue!10}
	}, 
	cod/.style={
		rectangle split part fill={blue!10, none}
	},
	constant/.style={
		rectangle split part fill={red!50, none}
	},
	identity/.style={
		rectangle split part fill={gray, gray}
	},
	linear/.style={
		rectangle split part fill={gray, none}
	},
	pure/.style={
		rectangle split part fill={none, gray}
	},
	pure dom/.style={
		rectangle split part fill={none, blue!10!gray}
	},
	shorten <=3pt, shorten >=3pt,
	first/.style={out=0, in=180},
	climb/.style={out=180, in=180, looseness=2},
	last/.style={out=180, in=0},
	mapstos/.style={arrows={|->}},
	tos/.style={arrows={->}},
	font=\footnotesize,
	node distance=2ex and 1.5cm
}
}




\newcommand{\earpic}{
\begin{tikzpicture}[polybox, tos]
	\node[poly, dom, "$m$" left] (m) {};
	\node[poly, cod, right=of m, "$m$" right] (mm) {};
	\node[poly, cod, above=of mm, "$C$" right] (C) {};
	\node[poly, cod, below=of mm, "$D$" right] (D) {};
%
	\draw (m_pos) to[out=0, in=180] (D_pos);
	\draw (D_dir) to[climb] (mm_pos);
	\draw (mm_dir) to[climb] (C_pos);
	\draw (C_dir) to[last] (m_dir);
\end{tikzpicture}
}

\newcommand{\treepic}{
\begin{tikzpicture}[trees, scale=1.5]
\begin{scope}[
  level 1/.style={sibling distance=20mm},
  level 2/.style={sibling distance=10mm},
  level 3/.style={sibling distance=5mm},
  level 4/.style={sibling distance=2.5mm},
  level 5/.style={sibling distance=1.25mm}]
  \node[dgreen] (a) {$\bullet$}
    child {node[dgreen] {$\bullet$}
    	child {node[dgreen] {$\bullet$}
    		child {node[dgreen] {$\bullet$}
  				child {node[dgreen] {$\bullet$}
    				child {}
    				child {}
    			}
  				child {node[dyellow] {$\bullet$}
    				child {}
    				child {}
    			}
  			}
    		child {node[dyellow] {$\bullet$}
					child {node[dgreen] {$\bullet$}
      			child {}
      			child {}
     			}
    			child  {node[red] {$\bullet$}}
  			}
    	}
    	child {node[dyellow] {$\bullet$}
    		child {node[dgreen] {$\bullet$}
  				child {node[dgreen] {$\bullet$}
    				child {}
    				child {}
    			}
  				child {node[dyellow] {$\bullet$}
    				child {}
    				child {}
    			}
  			}
    		child  {node[red] {$\bullet$}}
    	}
    }
    child {node[dyellow] {$\bullet$}
    	child {node[dgreen] {$\bullet$}
    		child {node[dgreen] {$\bullet$}
  				child {node[dgreen] {$\bullet$}
    				child {}
    				child {}
    			}
  				child {node[dyellow] {$\bullet$}
    				child {}
    				child {}
    			}
  			}
    		child {node[dyellow] {$\bullet$}
					child {node[dgreen] {$\bullet$}
      			child {}
      			child {}
     			}
    			child  {node[red] {$\bullet$}}
  			}
  		}
  		child {node[red] {$\bullet$}
  		}
  	}
  ;
\end{scope}
\draw[blue, densely dotted] (current bounding box.south west) rectangle
(current bounding box.north east);
\end{tikzpicture}
}

\newcommand{\coverpic}{\treepic}%{\earpic}



















